 \chapter{非线性粘弹性流体的守恒-耗散理论}
%   \documentclass{article}
% \usepackage{ctex}
% \usepackage{amsfonts}
% \usepackage{amsmath}
% \usepackage{amsthm}
% \usepackage{cite}
% %\usepackage{showkeys}
% \theoremstyle{plain}
%  \newtheorem{theorem}{定理}
%  \newtheorem{remark}{注释}
%  \newtheorem{thm}{Theorem}
%  \newtheorem{lemma}{Lemma}
%  \theoremstyle{definition}
%  \theoremstyle{Remark}
% \newtheorem{rem}{Remark}
% \newtheorem{defn}{Definition}

% \begin{document}
上一章我们利用守恒-耗散理论讨论了线性粘弹性流体力学的模型。然而,线性粘弹性模型由于违背粘弹性流体建模的一个重要法则——客观性原理,因而无法很好地描述粘弹性流体的行为\cite{oldroyd1950formulation,dimitrienko2010nonlinear,edwards1990remarks}。为了在模型中考虑这一原理,我们需要包含张量的客观导数。但是由于这一导数并不是守恒形式的,所以守恒-耗散理论对方程守恒形式的假设并不成立。
% 为了发展守恒-耗散理论以纳入包含客观导数的非线性粘弹性流体力学模型,
本章通过放松对方程守恒形式的要求而推广守恒-耗散理论,以对非线性粘弹性进行建模。

第一节我们将首先讨论客观性原理。第二节提出推广的守恒-耗散理论。第三节利用推广的守恒-耗散理论发展了非线性粘弹性流体力学模型,推广了不可压上对流Maxwell方程和FENE-P模型,并提出了非等温可压上对流Maxwell模型。最后我们利用Yong的双曲平衡率方程组小解整体存在性理论和双曲方程松弛极限理论分析了一维等温可压上对流Maxwell方程的数学性质。

\section{粘弹性流体建模的客观性原理}
1949年J. G. Oldroyd发表了题为《流变学状态方程的构建》(On the formulation of rheological equations of state)一文\cite{oldroyd1950formulation}。这篇文章提出了一个重要的概念——客观性原理(Principle of Objectivity),或称为物质坐标不变性原理(Material Indifference Principle)。根据这一原理,描述物质运动的方程不依赖于坐标系的选取。
%物质附着的随体坐标系的运动方程在转换到固定参考坐标系时需要满足客观性原理。
如果我们假设连续介质的物质坐标系(Lagrange坐标系)为$X$,运动到时间$t$时的空间坐标系(Euler坐标系)为$x$,则物质的运动方程的可以采用下面的函数来描述
\begin{equation*}
	x = x (t,X).
\end{equation*}
假设物质的速度为$v = v (t,X)$,那么速度的向量场相伴着下面的自治微分方程
\begin{equation*}
	\frac{d}{dt} x(t) = v (x(t)).
\end{equation*}
其以$x|_{t=0}=X$为初始条件的积分曲线(解)可以记做
\begin{equation*}
	x = \mathcal{F}_t (X) = x (t,X).
\end{equation*}
这样我们可以建立微分同胚的单参数局部群和速度向量场的对应关系。即采用上面的方法可以定义局部单参数微分同胚群$\mathcal{F}_t$,且如果有$\mathcal{F}_t$我们可以采用下面的公式求得速度向量场
\begin{equation*}
	v = \frac{d}{dt} \mathcal{F}_t \big|_{t=0}.
\end{equation*}
根据这一定义,我们可以定义张量在Lagrange坐标系$X$和Euler坐标系$x$之间的转换关系。定义在$x$上的张量$T^{i_1,\cdots,i_p}_{j_1, \cdots,j_q}$转移到$X$上的坐标表示为
\begin{equation*}
	(\mathcal{F}_t T)^{i_1,\cdots,i_p}_{j_1, \cdots,j_q} = T^{l_1,\cdots,l_p}_{k_1, \cdots,k_q} \frac{\partial x^{k_1}}{\partial x_0^{j_1}} \cdots \frac{\partial x^{k_q}}{\partial x_0^{j_q}} \frac{\partial x_0^{i_1}}{\partial x^{l_1}} \cdots \frac{\partial x_0^{i_p}}{\partial x^{l_p}},  
\end{equation*}
其中的上下标分别表示协变坐标和共变坐标。张量$T$沿速度场的李导数定义为
\begin{equation*}
	{\mathcal{L}_v T^{i_1 \cdots i_p}_{j_1 \cdots j_q} =\frac{d}{dt} (\mathcal{F}_t T)} |_{t=0}.
\end{equation*}
李导数给出了张量$T$沿速度向量场$v$的变化。通过计算我们可以得到上面的李导数的坐标表示如下
\begin{eqnarray} \label{eq:Lie}
	 {\mathcal{L}_v T^{i_1 \cdots i_p}_{j_1 \cdots j_q} = \frac{d}{dt} (\mathcal{F}_t T)} |_{t=0}  = && v^s \frac{T^{i_1 \cdots i_p}_{j_1\cdots j_q}}{\partial x^s} + T^{i_1 \cdots i_p}_{k j_2 \cdots j_q}\frac{\partial v^k}{\partial x^{j_1}} + \cdots + T^{i_1 \cdots i_p}_{j_1\cdots j_{q-1}k}\frac{\partial v^k}{\partial x^{j_q}} \nonumber
	\\ && - T^{li_2 \cdots i_p}_{j_1\cdots j_q}\frac{\partial v^{i_1}}{\partial x^l}- \cdots - T^{i_1 \cdots i_{p-1}l}_{j_1\cdots j_q}\frac{\partial v^{i_p}}{\partial x^l}. 
\end{eqnarray}
在连续介质力学中我们还要考虑应力张量$T$本身随时间的变化,所以通常采用的是全导数,定义为
\begin{equation*}
	\frac{\mathcal{D} T}{\mathcal{D}t} = \frac{\partial T}{\partial t} + \mathcal{L}_v T. 
\end{equation*}
这样我们就得到了张量演化的全导数,这一导数很好地描述了速度向量场$v$对应的单参数微分同胚群$\mathcal{F}_t$引起的空间形变\cite{dubrovinmodern}。

J. G. Oldroyd定义的导数实际上就是张量场的李导数。李导数的定义是同坐标系的选取无关的,不同坐标系下张量坐标的转换满足张量坐标变换的一般原理。如果采用共变基底描述Cauchy应力张量$\sigma$,那么应力张量可以写作
\begin{equation*}{}
	\sigma^{ij} e_i \otimes e_j.
\end{equation*}
从而根据\eqref{eq:Lie},我们可以得到
\begin{equation}
	\left( \frac{\mathcal{D} \sigma}{\mathcal{D}t}  \right)_{ij} = \partial_t \sigma_{ij} + v^s \frac{\partial \sigma^{ij}}{\partial x^s} - \sigma^{lj} \frac{\partial v_i}{\partial x_l} -  \sigma^{il} \frac{\partial v_j}{\partial x_l}. 
\end{equation}
该导数即为Oldroyd论文中定义的右不变导数,又称作上对流Maxwell导数,记做$\stackrel{\nabla} \sigma$。如果采用反变坐标描述,则应力张量为$\sigma_{ij} e^i \otimes e^j$,其李导数为
\begin{equation}
	\left( \frac{\mathcal{D} \sigma}{\mathcal{D}t} \right)_{ij} = \partial_t \sigma_{ij} + v^s \frac{\partial \sigma_{ij}}{\partial x^s} + \sigma_{kj} \frac{\partial v_k}{\partial x_i} +  \sigma_{ik} \frac{\partial v_k}{\partial x_j}. 
\end{equation}
我们称这一导数为下对流Maxwell导数(或称作Cotter-Rivlin导数),记做$\stackrel{\Delta} \sigma$\cite{oldroyd1950formulation,dimitrienko2010nonlinear}。当然也可以采用混合基底来描述应力张量$\sigma$,从而可以定义混合导数。在连续介质力学中还存在其他客观导数,如Jaumann导数等。由于大部分粘弹性流体模型采用上对流Maxwell导数,我们在本章中仅考虑含上对流导数的模型,对于含其他导数的模型可以采用类似的方法得到。

另外,J. G. Oldroyd定义的张量变换关系考虑了空间压缩性的影响,此时张量的变换关系变为
\begin{equation*}
	(\mathcal{F}_t T)^{i_1,\cdots,i_p}_{j_1, \cdots,j_q} = {\det}^\gamma({\frac{\partial X}{\partial x}}) T^{l_1,\cdots,l_p}_{k_1, \cdots,k_q} \frac{\partial x^{k_1}}{\partial x_0^{j_1}} \cdots \frac{\partial x^{k_q}}{\partial x_0^{j_q}} \frac{\partial x_0^{i_1}}{\partial x^{l_1}} \cdots \frac{\partial x_0^{i_p}}{\partial x^{l_p}},  
\end{equation*}
% 定义形变张量
% \begin{equation*}
% 	F = \frac{\partial x}{\partial X}.
% \end{equation*}
从而我们可以得到$T$的全导数(考虑压缩性的导数记为$\frac{\mathcal{d}}{\mathcal{d} t}$,以上对流Maxwell导数为例)为
\begin{equation*}
	\frac{\mathcal{d} T}{\mathcal{d} t} =  T_t +v \cdot \nabla T + \gamma T \nabla \cdot v - (\nabla v) T - T(\nabla v)^T.   
\end{equation*}

由于坐标系的变形对张量演化的影响,张量的客观导数一般不是守恒形式的。这也意味着守恒-耗散理论对方程守恒形式的假设不适用于张量。对于第二章考虑的线性粘弹性流体力学模型实际上不满足客观性原理,从而应用受到了很大的局限。通过直接将张量的导数$\partial_t \sigma  + v \cdot \nabla \sigma$写成客观导数的形式,可以推广线性Maxwell模型,例如下面的不可压粘弹性流体的上对流Maxwell模型就是通过推广线性的Maxwell模型得出的\cite{oldroyd1950formulation}:
\begin{subequations} \label{eq:UCM}
	\begin{align}
		\nabla \cdot v &= 0, \\
		v_t + v \cdot \nabla v + \nabla p  &= \nabla \cdot \sigma, \\
		\stackrel{\nabla} \sigma - \frac{2\eta_p}{\lambda} D &= - \frac{\sigma}{\lambda}
	\end{align}
\end{subequations}
其中$\eta_p$为高分子引起的粘性系数。
另外通过微观的方法推导出的粘弹性模型也经常包含客观导数,例如Giesekus模型、Phan-Thien-Tanner模型和FENE-P模型等。这些模型均具有下面的形式
\begin{equation*}
	\lambda \stackrel{\nabla} \tau  -2 \eta_p D = - \mathcal{G}(\tau,D),
\end{equation*}
其中$\mathcal{G}$的表达式为\cite{le2009multiscale}:
\begin{itemize}
\item Giesekus 模型:$\mathcal{G} = \tau_p + \alpha \frac{\lambda}{\eta_p} \tau_p\tau_p$。
\item PTT 模型:$\mathcal{G} = e^{\phi \lambda \frac{\mbox{Tr}(\tau_p)}{\eta_p}}\tau_p + \frac{\xi}{2} \lambda (D \tau_p + \tau_p D)$。
\item FENE-P 模型:$\mathcal{G}  = Z(\mbox{Tr}(\tau_p))\tau_p  -\lambda (\tau_p + \frac{\eta_p}{\lambda} I)\left( (\partial_t + v \cdot \nabla) \ln \mbox{Tr}(\tau_p) \right), Z(\mbox{Tr}(\tau_p)) = 1 + \frac{d}{b} (1+ \lambda \frac{\mbox{Tr}{(\tau_p)}}{d \eta_p})$。
\end{itemize}
% 另外,对于高分子熔融液,需要考虑高分子纠缠效应,例如Doi-Edwards模型的一个简单形式(Larson近似)
% \begin{equation}
% 	\stackrel{\nabla} \sigma + \frac{2}{3G} D:\sigma \sigma + \frac{1}{\lambda}(\sigma - G I) = 0
% \end{equation}
% 也包含上对流Maxwell导数\cite{larson2015modeling}。

上面提到的模型均为不可压模型,且没有考虑温度的影响。而由于张量客观导数的非守恒性,应力张量方程的形式不是守恒的,从而不满足守恒-耗散理论耗散变量演化方程为守恒形式的假设。为了克服这些困难,我们将推广守恒-耗散理论以纳入含客观导数的非线性粘弹性流体模型。

%see http://imechanica.org/files/Cauchy_Sress_Tensor.pdf

\section{推广的守恒-耗散理论}
我们仍然认为热力学过程可以采用守恒过程和耗散过程来描述。与经典守恒-耗散理论不同的是,我们在这里不再假设耗散变量$U_d$满足守恒形式的方程。我们仍然假设守恒变量$U_c$满足
\begin{equation*}
	\partial_t U_c + \sum_{j=1}^n \partial_{x_j} f_j(U) = 0.
\end{equation*}
耗散变量$U_d$的方程假设有形式(其中$\frac{\mathcal{D} }{\mathcal{D} t}$为客观导数)
\begin{eqnarray*}
	\frac{\mathcal{D} U_d}{\mathcal{D} t} + \sum_{j=1}^n B_{j}(U)U_{x_j} = \mathcal{q}(U).
\end{eqnarray*}
为使热力学第二定律成立,我们仍然假设熵函数的存在性。然而由于耗散变量$U_d$方程的非守恒性质,我们不能假设$\eta_{UU}F_{jU} = 0$,而假设熵函数满足类似守恒-耗散理论中熵函数的方程。对于源项,我们采用同样的方式描述。推广的守恒-耗散理论的基本假设如下:
\begin{enumerate}
		\item 存在严格上凸函数$\eta = \eta (U)$,称$\eta$为系统的熵函数,使得存在$J=J(U),\Delta = \Delta(U)$,成立
		\begin{equation}
			\eta_t + \nabla \cdot J = \Delta.
		\end{equation}
		\item 存在正定矩阵$M = M(U)$,称为耗散矩阵,使得$\mathcal{q}(U) = M \eta_{U_d}$。
	\end{enumerate}
	由这两条假设我们得到
	\begin{equation*}
		\Delta = \eta_{U_d}^T M \eta_{U_d} \ge 0,
	\end{equation*}
从而熵增大于0,即热力学第二定律成立。推广的守恒-耗散理论仅仅通过改变耗散变量方程的形式来推广守恒-耗散理论,而守恒-耗散理论的核心假设仍然不变。守恒变量的守恒方程保证了物理守恒律,例如质量、动量、能量守恒的成立,熵的存在性保证了热力学第二定律的成立。另外,与经典守恒-耗散理论相同,我们一般假设$M$对称。

% 然而与经典守恒-耗散理论不同的是,熵函数的存在无法保证得到的方程是对称双曲组,从而其适定性需要单独考虑。另外形式上$\mathcal{Q}(U)=0$可以得到平衡态时耗散变量的值$U_d = U_{de}(U_c)$。此时可以得到对应的平衡态方程
% \begin{equation*}
% 	\partial_t U_c + \sum_{j=1}^d \partial_{x_j} f_j(U_c,U_{de}(U_c))=0.
% \end{equation*}
% 如何说明这一模型在$\mathcal{Q}$很小时对原模型的近似性也是需要单独考虑的一个问题。

% 我们期待类似第二章定理1的结论成立。needproof

% 另外,第二章提到的一样,我们可以令$\mathcal{q}=0$而得到下面的方程
% \begin{eqnarray*}
% 	\partial_t U_c + \sum_{j=1}^d f_j(U) = 0, \\
% 	\mathcal{D}_t U_d + \sum_{j=1}^d B_j(U) U_{x_j} = 0.	
% \end{eqnarray*}
% 这一模型亦可视为另一个平衡态体系的方程,对应的熵函数$\eta$满足$\eta_t + \nabla J = 0$。即我们可以认为推广的守恒-耗散理论所描述的体系是介于两个平衡态之间的。由一个平衡态到另一个平衡态的演化过程是非平衡态过程\cite{lieb2000fresh,lieb2013entropy,lieb2014entropy}。

下面我们将利用推广的守恒-耗散理论建立非线性粘弹性模型。

\section{推广的守恒-耗散理论在非线性粘弹性流体建模中的应用}
与上一章相同,我们取$U_c = \{ \rho,\rho v,\rho e\}$,满足方程\eqref{eq:fluid}。耗散变量仍取作$U_d =\{\rho w, \rho c\}$,假设熵函数$\eta =\eta(\rho,\rho v,\rho e,\rho w,\rho c)$可以表示成
\begin{equation*}
	\eta = \rho s(\nu,u,w,c).
\end{equation*}
其中$s$为比熵,$\nu = \frac{1}{\rho}$,$u$为系统的内能。根据Gibbs关系
\begin{equation*}
		\theta^{-1} = s_u, \quad \theta^{-1} p = s_{\nu}.
\end{equation*}
下面计算熵的产生率
\begin{eqnarray*}
		&&\eta_t + \nabla \cdot (\eta v) \\
		&=& \rho (s_t + v \cdot \nabla s) \\
		&=& -\nabla \cdot (\theta^{-1} q) + s_w \cdot [\rho (w_t + v \cdot \nabla w) + \nabla \theta^{-1}] \\
		&&+ (s_c:[\rho (c_t + v \cdot \nabla c - (\nabla v) c - c (\nabla v)^T) + \rho (\nabla v c + c (\nabla v)^T)] - \theta^{-1} \tau : D) \\
		&=& -\nabla \cdot (\theta^{-1} q) + s_w \cdot [\rho (w_t + v \cdot \nabla w) + \nabla \theta^{-1}] + [ s_c:\rho \stackrel{\nabla}{c} + (\rho s_c c - \frac{1}{2} \theta^{-1} \tau): 2 D]   \\
		&=& -\nabla \cdot J + \Delta.
	\end{eqnarray*}	
	这里我们假设了$q=s_w$,$c$为对称的。$w$的方程同经典的守恒-耗散理论相同。为了得到$c$的方程,有两个选择,一是令
	\begin{equation*}
		\theta^{-1} \tau = 2 \rho s_c c,
	\end{equation*}
	得到
	\begin{equation} \label{eq:ECDFgeneral1}
		\left( \begin{array}{c} 
			(\rho w)_t +  \nabla \cdot (\rho w \otimes v)  + \nabla \theta^{-1} \\
			(\rho c)_t +  \nabla \cdot (\rho c \otimes v) - (\nabla v) \rho c - (\rho c) (\nabla v)^T 
		\end{array} \right) = M \cdot
		\left( \begin{array}{c} 
			q \\ s_c
		\end{array}\right).
	\end{equation}
	我们称这样的选择得到的模型为第一类模型。
	另一个选择为令$\theta^{-1} \tau = 2 s_c + 2 \rho s_c c$,此时本构方程可以写为
	\begin{equation}
	\left( \begin{array}{c} 
			(\rho w)_t +  \nabla \cdot (\rho w \otimes v)  + \nabla \theta^{-1} \\
			(\rho c)_t +  \nabla \cdot (\rho c \otimes v) - (\nabla v) \rho c - (\rho c) (\nabla v)^T - 2 D 
		\end{array} \right) = M \cdot
		\left( \begin{array}{c} 
			q \\ s_c
		\end{array}\right).	
	\end{equation}
	我们称这样的选择得到的模型为第二类模型。

	\subsection{第一类模型}
	% 许多经典的模型都可以归为此类。
	这里我们将利用推广的守恒-耗散理论推广经典的不可压上对流Maxwell模型和FENE-P模型。
	\subsubsection{上对流Maxwell模型}
	首先我们推导上对流导数的Maxwell模型。取熵函数的形式为
	\begin{equation*}
			s = s_0(\nu,u)  - \frac{1}{2  \alpha_0} w^2 - \frac{1}{2  \alpha_1} (\mbox{Tr}(c) - \ln \det c).
	\end{equation*}
	其中$\alpha_0,\alpha_1$大于$0$,且$s_0(\nu,u)$为严格上凸函数,显然熵函数$\eta = \rho s$此时为上凸的。
	

	取$q=s_w=-\frac{ w}{\alpha_0},\tau = 2\theta \rho s_c c = \frac{1}{\alpha_1} \rho \theta (I-c)$,$M$为
	\begin{equation*}
		M = \left( \begin{array}{ccc} 
			\frac{1}{\theta^2 \lambda} & 0 \\
			0 &  \frac{2 \rho^2 \theta c \otimes I}{\xi}   
		\end{array} \right).
	\end{equation*}
	我们得到
	\begin{eqnarray*}
		\alpha_0 [\partial_t q +  \nabla \cdot (q \otimes v)] - \nabla \theta^{-1} = -\frac{q}{\theta^2 \lambda}, \\
		\alpha_1[\partial_t (\theta^{-1} {\tau}) + \nabla \cdot (\theta^{-1} {\tau} \otimes v) - \nabla v \theta^{-1} \tau - \theta^{-1}\tau (\nabla v)^T] + 2 \rho D = -\frac{{\rho \tau}}{\xi}.
	\end{eqnarray*}
	
	形式上当$\alpha_0, \alpha_1$趋于0时,可以得到
	\begin{equation*}
		q = -\lambda \nabla \theta, \quad \tau = - 2 \xi {D} .
	\end{equation*}
	从而可知形式上这一模型与Fourier定律和Newton定律一致。
	
	下面我们来推导压力$p$的表达式。设$\pi= s_{0\nu}$,由$\theta^{-1} p = s_\nu$,我们可以得到
	\begin{equation*}
		p = \theta s_\nu = \theta \pi . 
	\end{equation*}
	我们得到的最终方程为 
% \newpage
	\begin{subequations} \label{eq:genUCM}
		\begin{align}
			\rho_t + \nabla \cdot (\rho v) = 0 ,\\
			(\rho v)_t + \nabla \cdot (\rho v \otimes v) + \nabla (\theta \pi)  + \nabla \cdot ( \frac{\rho \theta(I-c)}{\alpha_1}) =0 ,\\
			(\rho e)_t + \nabla \cdot q + \nabla \cdot (P \cdot v) = 0, \\
			(\rho w)_t + \nabla \cdot (\rho w \otimes v) + \nabla \theta^{-1} = -\frac{1}{\lambda \theta^2 \alpha_0} w, \\
			(\rho c)_t +  \nabla \cdot (\rho c \otimes v) - (\nabla v) \rho c - (\rho c) (\nabla v)^T  = -\frac{2 \rho^2 \theta (c-I) }{\xi \alpha_1}.
		\end{align}
	\end{subequations}
	其中$P = pI + \tau,q =-\frac{w}{\alpha_0}$。

	在不可压缩($\rho=1$)和忽略温度的情形($\theta=1$),$\nabla \cdot v = 0$,应力$\sigma = -\tau$的演化方程为
	\begin{equation*}
		\partial_t \sigma + v \cdot \nabla \sigma - \nabla v \sigma - \sigma (\nabla v)^T - 2 D = -\frac{\sigma}{\xi}. 
	\end{equation*}
	此即上对流Maxwell粘弹性流体力学模型\eqref{eq:UCM}.	

	考虑温度的不可压Maxwell模型为
	\begin{eqnarray*}% \label{eq:Tmaxwell}
		v_t + v \cdot \nabla v + \nabla \cdot (-\theta \frac{c-I}{\alpha_1}) = 0, \\
		c_t + v \cdot \nabla c - (\nabla v) c - c (\nabla v)^T - 2 D = -\frac{1}{\xi} c.
	\end{eqnarray*}
	取$\alpha_1 = \frac{2}{\eta_p k}, T = \theta,\xi = \frac{\zeta}{2 kT \beta}$,即可得到由微观理论推出的含温度上对流Maxwell模型\eqref{eq:MicroUCM}。	实际上,上对流Maxwell模型微观上可以通过稀疏高分子溶液的Hooke弹簧模型得到\cite{larson1999structure,le2009multiscale}。弹簧受到的力$F_s$一般写作
	\begin{equation*}
		F_s = H R = 2 k_B T \beta^2 R.
	\end{equation*}
	其中$H$为弹性常数,$R$为位移。实际上可以推出$\langle RR \rangle$满足的方程同$c$一样(其中$\langle \cdot \rangle$表示对$R$的积分)\cite{larson1999structure},这样$c$实际上代表了微观量的统计量$c = \langle RR \rangle$。

	
	% 向第二章一样,如果假设熵函数的形式为
	% \begin{equation*}
	% 		s = s_0(\nu,u)  - \frac{1}{2 \nu \alpha_0} w^2 - \frac{1}{2 \nu \alpha_1} (\mbox{Tr}(c) - \ln \det c).
	% \end{equation*}
	% 其中$\alpha_0,\alpha_1$为常数。这样我们可以得到
	% \begin{equation*}
	% 	q = s_w = -\frac{\rho w}{\alpha_0} ,\quad \theta^{-1} \tau = \rho s_c c = \frac{1}{\alpha_1} \rho^2  (I-c).
	% \end{equation*}
	% 由压力$p$的定义,可以得到
	% \begin{equation*}
	% 	\theta^{-1} p = s_\nu = \pi +\frac{\rho^2}{2} w^2 + \frac{\rho}{2} c:c.
	% \end{equation*}
	% 取\begin{equation*}
	% 	M = \left( \begin{array}{ccc} 
	% 		\frac{1}{\theta^2 \lambda} & 0 \\
	% 		0 &  \frac{2 \rho^2 \theta c \otimes I}{\xi}   
	% 	\end{array} \right),
	% \end{equation*}	
	% 可以得到体系的方程为
	% \begin{subequations}\label{eq:generalizedUCM}
	% 	\begin{align}
	% 		\rho_t + \nabla \cdot (\rho v) = 0 ,\\
	% 		(\rho v)_t + \nabla \cdot (\rho v \otimes v) + \nabla (\theta \pi + \frac{\rho^2}{2} w^2 + \frac{\rho}{2} c:c)  + \nabla \cdot ( \frac{\rho^2 \theta(I-c)}{\alpha_1}) =0 ,\\
	% 		(\rho e)_t + \nabla \cdot q + \nabla \cdot (P \cdot v) = 0, \\
	% 		(\rho w)_t + \nabla \cdot (\rho w \otimes v) + \nabla \theta^{-1} = -\frac{1}{\lambda \theta^2} w, \\
	% 		(\rho c)_t +  \nabla \cdot (\rho c \otimes v) - (\nabla v) \rho c - (\rho c) (\nabla v)^T  = -\frac{\rho (c-I) }{\xi}.
	% 	\end{align}
	% \end{subequations}

	\subsubsection{FENE-P模型}
	由于粘弹性流体的弹簧模型中的Hooke弹簧是可以无限延展的,对于大分子这一假设并不成立,所以人们发展了FENE(Finitely Extensible, Nonlinear Elastic)模型。FENE模型假设弹簧受力满足
	\begin{equation*}
		F_s = \frac{2 \beta^2 k_B T}{1-(|R|/L)^2} R = H(R^2)R.
	\end{equation*}
	由于FENE模型无法写成宏观方程的形式\cite{le2009multiscale},人们将$H$对$R^2$的依赖关系改为对$R^2$的空间平均的依赖,从而得到FENE-P模型(这里$P$代表Peterlin矩封闭\cite{larson1999structure,le2009multiscale,masmoudi2011global})。FENE-P模型的应力张量的表达式如下\cite{larson1999structure}
	\begin{equation} \label{eq:FENEP}
		\sigma = \mu k_B T [ \beta^2 (1- \frac{\text{\footnotesize $\mbox{Tr}$}(S)}{2L^2})^{-1}S - I],  \quad \stackrel{\nabla} S + \frac{1}{\lambda} (\frac{S}{1-\frac{\text{\footnotesize $\mbox{Tr}$}(S)}{2L^2}} - \beta^{-2}I) = 0. 
	\end{equation}

	下面我们利用推广的守恒-耗散理论推广这一模型到可压缩、包含温度的情形。根据\cite{masmoudi2011global,hu2007new},我们取熵函数有形式
	\begin{equation*}
		s = s_0(\nu,u)  - \frac{1}{2  \alpha_0} w^2 + \frac{1}{2 \alpha_1} (a \ln \det c + b \ln(1-\frac{\text{\footnotesize $\mbox{Tr}$} (c)}{b})).
	\end{equation*}
	从而$q=s_w=-\frac{ w}{\alpha_0},\tau =  2\theta \rho s_c c = \frac{1}{\alpha_1} \rho \theta (a I-\frac{c}{1-{\text{\footnotesize $\mbox{Tr}$} (c)}/{b}})$。取$M$为
	\begin{equation*}
		M = \left( \begin{array}{ccc} 
			\frac{1}{\theta^2 \lambda} & 0 \\
			0 &  \frac{2 \rho  c \otimes I}{\xi}   
		\end{array} \right).
	\end{equation*}
	我们得到
	\begin{subequations}\label{eq:generalizedFENEP}
		\begin{align}
			(\rho w)_t + \nabla \cdot (\rho w \otimes v) + \nabla \theta^{-1} = -\frac{1}{\lambda \theta^2 \alpha_0} w, \\
			(\rho c)_t +  \nabla \cdot (\rho c \otimes v) - (\nabla v) \rho c - (\rho c) (\nabla v)^T  = -\frac{\rho (\frac{c}{1-	{\text{\footnotesize $\mbox{Tr}$}(c)}/{b}}-a I) }{\xi}.
		\end{align}
	\end{subequations}
	取$\alpha_1 = \frac{\beta^2}{\mu k_B},a=\beta^{-2},b=2L^2,\xi = \lambda$,假设流体不可压($\rho=1$),那么我们可以由$c=S,\tau = -\sigma$得到\eqref{eq:FENEP},即我们得到了FENE-P模型。

	这一节我们利用了守恒-耗散理论推广了经典的不可压上对流Maxwell模型和FENE-P模型,通过本节的分析可以看出,推广的守恒-耗散理论可以很方便地对经典模型进行推广。

	\subsection{第二类模型:非等温可压上对流Maxwell模型}
	\subsubsection{模型的推导}
	下面我们将采用第二种方法推导非等温可压上对流Maxwell模型。
% 	首先选取熵函数为
% 	\begin{equation*}
% 			s = s_0(\nu,u)  - \frac{1}{2  \alpha_0} w^2 - \frac{1}{2  \alpha_1} c:c.
% 	\end{equation*}
% 	从而$q=s_w=-\frac{ w}{\alpha_0},\tau = - \frac{2}{\alpha_1}  \theta ( c + \rho c \cdot c)$。取$M$为
% 	\begin{equation*}
% 		M = \left( \begin{array}{ccc} 
% 			\frac{1}{\theta^2 \lambda} & 0 \\
% 			0 &  \frac{2 \rho^2 \theta c \otimes I}{\xi}   
% 		\end{array} \right).
% 	\end{equation*}
% 	我们得到
% \begin{eqnarray*}
% 			(\rho w)_t +  \nabla \cdot (\rho w \otimes v)  + \nabla \theta^{-1} = -\frac{1}{\lambda \theta^2}  w \\
% 			(\rho c)_t +  \nabla \cdot (\rho c \otimes v) - (\nabla v) \rho c - (\rho c) (\nabla v)^T - 2 D = - \frac{c}{\xi} .
% \end{eqnarray*}
% 从而流体的方程可以写为\begin{subequations}
% 		\begin{align*}
% 			\rho_t + \nabla \cdot (\rho v) = 0 ,\\
% 			(\rho v)_t + \nabla \cdot (\rho v \otimes v) + \nabla (\theta \pi)  - \nabla \cdot ( \frac{1}{\alpha_1}  \theta ( c + \rho c \cdot c)) =0 ,\\
% 			(\rho e)_t + \nabla \cdot q + \nabla \cdot (P \cdot v) = 0, \\
% 			(\rho w)_t + \nabla \cdot (\rho w \otimes v) + \nabla \theta^{-1} = -\frac{1}{\lambda \theta^2} w, \\
% 			(\rho c)_t +  \nabla \cdot (\rho c \otimes v) - (\nabla v) \rho c - (\rho c) (\nabla v)^T - 2 D = - \frac{c}{\xi} .
% 		\end{align*}
% 	\end{subequations}

	首先我们假设熵函数有形式(为了方便取$\alpha_0=\alpha_1=1$)
	\begin{equation*}
			s = s_0(\nu,u)  - \frac{1}{2 \nu } w^2 - \frac{1}{2  \nu } c:c.
	\end{equation*}
	从而$q=s_w=-{\rho w},\tau = \theta(2s_c + 2 \rho s_c c)= - \theta (2 \rho c + 2 \rho c \cdot \rho c)$。取$M$为
	\begin{equation*}
		M = \left( \begin{array}{ccc} 
			\frac{1}{\theta^2 \lambda} & 0 \\
			0 &  (\frac{1}{\kappa}\dot{\mathcal{T}}  +  \frac{1}{\xi}\mathring{\mathcal{T}}  )
		\end{array} \right).
	\end{equation*}
	我们得到
\begin{eqnarray*}
			(\rho w)_t +  \nabla \cdot (\rho w \otimes v)  + \nabla \theta^{-1} = -\frac{1}{\lambda \theta^2}  \rho w, \\
			(\rho c)_t +  \nabla \cdot (\rho c \otimes v) - (\nabla v) \rho c - (\rho c) (\nabla v)^T - D = - \frac{\rho \dot{c}I}{\kappa} -  \frac{\rho \mathring{c}}{\xi} .
\end{eqnarray*}
假设$\pi=\pi(\nu,u) = s_{0\nu}(\nu,u)$,可以得到静压力$p$的表达式为
\begin{equation*}
			\theta^{-1} p = s_\nu = \pi +\frac{\rho^2}{2} w^2 + \frac{1}{2} \rho c: \rho c.
\end{equation*}
从而流体的方程可以写为
\begin{subequations} \label{eq:ECDFsecond}
		\begin{align}
			\rho_t + \nabla \cdot (\rho v) = 0 ,\\
			(\rho v)_t + \nabla \cdot (\rho v \otimes v) + \nabla [\theta(\pi +\frac{1}{2} (\rho w)^2 + \frac{1}{2} \rho c: \rho c)] \nonumber \\
			 - \nabla \cdot (\theta (2 \rho c + 2 \rho c \cdot \rho c)) =0 ,\\
			(\rho e)_t + \nabla \cdot (\rho e v) + \nabla \cdot q + \nabla \cdot (P \cdot v) = 0, \\
			(\rho w)_t + \nabla \cdot (\rho w \otimes v) + \nabla \theta^{-1} = -\frac{1}{\lambda \theta^2} w, \\
			(\rho c)_t +  \nabla \cdot (\rho c \otimes v) - (\nabla v) \rho c - (\rho c) (\nabla v)^T - 2 D = - \frac{\rho \dot{c}I}{\kappa} -  \frac{\rho \mathring{c}}{\xi}  .
		\end{align}
	\end{subequations}
	我们称这一模型为非等温可压上对流Maxwell模型。
	%本文中我们将这一模型称为粘弹性流体第二模型。我们将在后文讨论这一模型的数学性质。

	\subsubsection{与\"Ottinger模型的对比}
	上面的模型是受GENERIC启发提出的。文献\cite{ottinger2005beyond}第五章讨论了下对流导数的一个模型,我们在这里称其为\"Ottinger模型。下面我们将其中的Possion矩阵的下对流导数换为上对流导数,并选取简单的熵函数来采用GENERIC理论推导类似\eqref{eq:ECDFsecond}的模型。首先我们根据GENERIC理论选取描述系统的状态变量为$X=(\rho,\rho v, s ,w ,c)$,其中$\rho,v,s$分别代表密度、速度和熵。$w,c$与守恒-耗散理论中的选取相同,分别用来描述热传导和粘弹性流动的不可逆性。假设能量函数有下面的形式
	\begin{equation*}
		E(X) =	\int \frac{1}{2} \rho(x) v(x)^2 + \epsilon(x) dx,
	\end{equation*}
	其中$\epsilon$是内能密度。根据文献\cite{ottinger2005beyond}中5.1.1节的假设
	\begin{equation*}
		\epsilon = \epsilon (\rho,s,\mbox{Tr}(c^2),w^2) .
	\end{equation*}
	假设$\epsilon$为$\mbox{Tr}(c^2),w^2$的线性函数,即
	\begin{equation*}
		\epsilon = \epsilon_0(\rho,s) + \frac{1}{2} \phi_0 w^2  + \frac{1}{4} \phi_1 \mbox{Tr}(c^2) .
	\end{equation*}
	从而我们有
	\begin{equation*}
		\mu = \frac{\epsilon}{\rho},\quad T = \frac{\partial \epsilon}{\partial s}, \phi_0 = 2 \frac{\partial \epsilon}{\partial w}, \quad \phi_1 = 4 \frac{\partial \epsilon}{\partial \mbox{Tr}(c^2)}.
	\end{equation*}
	于是我们可以得到
	\begin{equation*}
		\frac{\delta E(X)}{ \delta X} = \left( \begin{array}{c} \mu - \frac{1}{2} v^2 \\ v \\ T \\ \phi_0 w \\ \frac{1}{2} \phi_1 c \end{array} \right).
	\end{equation*}
	Possion矩阵选为
	\begin{equation*}
		L(X) = -\left( \begin{array}{ccccc}
		0 & \frac{\partial}{\partial {x_{i'}}} \rho & 0 & 0 & 0 \\
		\rho \frac{\partial}{\partial {x_i}} & \frac{\partial}{\partial {x_{i'}}} ({\rho v_i}) + \rho v_{i'} \frac{\partial}{\partial {x_i}} & s \frac{\partial}{\partial {x_i}} & -[\frac{\partial w_m'}{\partial x_i}] + (\frac{\partial}{\partial {x_m'}} w_i) & L_{25} \\
		0 & \frac{\partial}{\partial x_{i'}} s & 0 & \frac{\partial }{\partial x_{m'}} & 0 \\
		0 & [\frac{\partial w_m}{\partial x_{i'}}] + w_{i'} \frac{\partial}{\partial x_{m}} & \frac{\partial}{\partial {x_m}} & 0 & 0  
		\end{array}\right)
	\end{equation*}
	其中
	\begin{eqnarray*}
		(L_{52})_{kl,i'} = [\frac{\partial {c_{kl}}}{\partial x_{i'}}] - (c_{ki'} + \delta_{ki'}) \frac{\partial}{\partial x_l} - ( c_{li'} + \delta_{li'}) \frac{\partial}{\partial x_{k}}, \\
		(L_{52})_{i,k'l'} = -[\frac{\partial {c_{k'l'}}}{\partial x_{i}}] - \frac{\partial}{\partial x_{k'}} (c_{il'} + \delta_{il'})  - \frac{\partial}{\partial x_{l'}} ( c_{ik'} + \delta_{ik'}).
	\end{eqnarray*}
	注意这里$[\cdot]$表示该项作为一个量,$(\cdot)$表示作为一个算子,
	可以验证$L(X)$满足Possion矩阵的性质。

	% 选取熵函数$s$为二次的。
	% \begin{equation*}
	% 	s = s_0(\rho, \epsilon) - \frac{1}{2} ( w^2 +  \mbox{Tr} (c^2) ).
	% \end{equation*}
	熵
	\begin{equation*}
		S(X) = \int s(x) dx
	\end{equation*}
	的导数为
	\begin{equation*}
		\frac{\delta S(X)}{ \delta X} = \left( \begin{array}{c} 0 \\ 0 \\ 1 \\  0 \\ 0 \end{array} \right).
	\end{equation*}
	这里我们假设了平衡态熵满足
	\begin{equation*}
		\frac{\delta \int s_0(x) dx}{ \delta X} = 0. 
	\end{equation*}
	假设耗散矩阵取值$M$如下
	\begin{equation*}
		M = \left( \begin{array}{ccccc} 
		0 & & & & \\
		& 0 & & &  \\
		& & \frac{1}{T \tau_1} \phi_0 w^2 + \frac{\phi_1 \dot{c}^2}{2 T \tau_0} + \frac{\phi_1 \mathring{c}:\mathring{c}}{2 T \tau_2} & -\frac{w}{\tau_1} & -\frac{1}{\tau_0} \dot{c} I - \frac{1}{\tau_2} \mathring{c} \\  
		& &  -\frac{w}{\tau_1} & \frac{T}{\tau_1} \phi_0^{-1} & 0  \\
	    & &  -\frac{1}{\tau_0} \dot{c} I - \frac{1}{\tau_2} \mathring{c} & 0 & \frac{2T}{\tau_0} \dot{\mathcal{T}}   + \frac{2T}{\tau_2}  \mathring{\mathcal{T}}   
		\end{array} \right).
	\end{equation*}
	我们假设参数$\tau_0,\tau_1,\tau_2,\phi_0,\phi_1$均为正数,从而$M$为半正定的。
	根据GENERIC理论,系统的方程可以写为
	\begin{equation*}
		\frac{dX}{dt} = L(X) \cdot \frac{\delta E(X)}{\delta X} + M(X) \cdot \frac{\delta S(X)}{\delta X} .
	\end{equation*}
	具体形式为
	
	\begin{subequations}\label{eq:Ottinger}
		\begin{align} 
			\partial_t \rho + \nabla \cdot (\rho v) = 0, \\
			\partial_t (\rho v) + \nabla \cdot (\rho v v ) + \nabla \cdot P + s \nabla T = 0, \\
			\partial_t s + \nabla \cdot (s v) + \phi_0  \nabla w = \frac{\phi_0 w^2}{T \tau_1} + \frac{1}{2} \frac{\phi_1 \dot{c}^2}{T \tau_0} + \frac{1}{2} \frac{\phi_1 \mathring{c}:\mathring{c}}{T \tau_2}, \\
			\partial_t w + v \cdot \nabla w + (\nabla v)^T \cdot w + \nabla T  = -\frac{w}{\tau_1}, \\
			\partial_t c + v \cdot \nabla c - (\nabla v) c - c (\nabla v)^T - (\nabla v + (\nabla v)^T) = -\frac{\dot{c}I}{\tau_0} - \frac{\mathring{c}}{\tau_1}.
		\end{align}
	\end{subequations}
	其中
	\begin{equation*}
		P = (\rho \mu + sT - \epsilon) I  - \phi_1(c^2 + c) + \phi_0 w^2 I.
	\end{equation*}
	采用\"Ottinger的理论得到的模型\eqref{eq:Ottinger}与模型\eqref{eq:ECDFsecond}的相同之处在于$w$和$c$的方程类似,且应力的表达式均出现了$c^2+c$和$w^2$。不同之处在于\eqref{eq:Ottinger}引入了熵函数作为系统的状态变量,在$w$的方程中包含了$(\nabla v)^T \cdot w$一项,并且$c$的演化方程并不包含密度$\rho$。
	%而实际上,$w$作为一个向量,其客观导数就是其Lagrange导数。$(\nabla v)^T \cdot w$一项的引入似乎没有意义。另一方面,$c$的方程中采用了上对流Maxwell导数,且在应力张量中含有$c$的项也不含有密度,
	\"Ottinger提出的模型抓住了系统的守恒律并采用耗散矩阵对不可逆过程进行了描述。但是我们认为这一模型无法很好地描述压缩性对应力张量的影响,也不符合J. G. Oldroyd提出的应力张量转换原理\cite{oldroyd1950formulation}。在耗散矩阵$M$的选取中,右下角$2\times 2$的子矩阵对于方程没有影响,但是对$M$的正定性影响很大。我们采用推广的守恒-耗散理论避免了GENERIC中对Possion矩阵、耗散矩阵选取的复杂性,并且仅仅假设耗散矩阵的正定性,抓住了不可逆过程建模最关键的两个原理:热力学第一和第二定律,并且这里得到的模型\eqref{eq:ECDFsecond}也具有类似GENERIC的耗散Hamilton结构。%下面我们将会看出,采用推广的守恒-耗散理论导出的粘弹性流体第二模型具有很好的数学结构。

%	\subsection{推广的守恒-耗散理论应用于热传导}
%	由于Cattaneo定律和热质理论均采用向量来描述热流,采用经典的守恒-耗散理论就可以很好地处理。对于推广的Guyer-Krumhansl理论\eqref{eq:EGK},我们采用了张量$Q$来描述热传导。从而我们也需要$Q$的导数满足客观性原理。

\section{一维等温可压上对流Maxwell模型的数学分析}
由于推广的守恒-耗散理论破坏了方程的守恒结构,因而熵函数的存在无法保证方程的对称双曲性质。对于一般情况下由推广的守恒-耗散理论得到的模型的数学分析目前尚缺乏一般的结果。我们在本节将考虑采用第二种方法得到的非等温可压上对流Maxwell模型\eqref{eq:ECDFsecond}在等温时的情形。此情形下($\theta=1$),$s_0$只依赖于$\nu$,从而$\pi = s_{0\nu}$只依赖于密度$\rho$,此时我们得到
	\begin{subequations} \label{eq:ECDFsecondisothermal}
		\begin{align}
			\rho_t + \nabla \cdot (\rho v) = 0 ,\\
			(\rho v)_t + \nabla \cdot (\rho v \otimes v) + \nabla (\pi + \frac{1}{2} \rho c: \rho c)  - \nabla \cdot ( 2 \rho c + 2 \rho c \cdot \rho c) =0 ,\\
			(\rho c)_t +  \nabla \cdot (\rho c \otimes v) - (\nabla v) \rho c - (\rho c) (\nabla v)^T - 2 D = - \frac{\rho \dot{c}I}{\kappa} -  \frac{\rho \mathring{c}}{\xi}  .
		\end{align}
	\end{subequations}
	在本文中,我们称这一模型为等温可压上对流Maxwell模型。下面我们将会验证这一模型的熵函数的Hessian矩阵无法对称化方程组\eqref{eq:ECDFsecondisothermal},但是对于一维的情况
% 由于推广的守恒-耗散理论破坏了方程的守恒结构,因而熵函数的存在无法保证方程的对称双曲性质。对于一般情况下由推广的守恒-耗散理论得到的模型的数学分析目前尚缺乏一般的结果。我们在本节将考虑采用第二种方法得到的等温可压上对流Maxwell模型\eqref{eq:ECDFsecondisothermal}一维情形时的数学性质。在一维时,对应的方程为
\begin{subequations} \label{eq:ECDFsecond1D}
		\begin{align}
			\rho_t + \partial_x (\rho v) = 0 ,\\
			(\rho v)_t + \partial_x (\rho v^2) + \partial_x (\pi)   -  (2+ 3 \rho c) \partial_x (  \rho c) =0 ,\\
			(\rho c)_t +  \partial_x (\rho c  v) - 2 \rho c \partial_x  v  - 2 \partial_x v = - \frac{\rho {c}}{\kappa}  .
		\end{align}
\end{subequations}
对称子可以找到,从而可以采用双曲方程的相关理论进行分析。本节我们将利用Yong的双曲平衡率方程组小解整体存在性理论和双曲方程松弛极限理论证明方程组\eqref{eq:ECDFsecond1D}平衡态附近的整体存在性和与一维可压Navier-Stokes方程组的兼容性。

下面我们首先给出其对称子,然后利用Yong的双曲平衡率方程组小解整体存在性理论\cite{yong2004entropy}证明其平衡态附近解的整体存在性,最后我们利用Yong、Yang提出的Chapman-Enskog展开的数学理论\cite{yang2015validity}证明当$\kappa$趋于$0$时方程组\eqref{eq:ECDFsecond1D}与由Maxwell迭代得到的一维Navier-Stokes方程组的兼容性。由于方程的非守恒形式,这里的分析无法直接利用\cite{yong2004entropy,yang2015validity}的结果,但其中对方程的估计方法仍然适用。

% draft of the paper, built on Sep.29 by Huo Xiaokai, Email: hxk12@mails.tsinghua.edu.cn
% Chang on Dec. 3
% Revised on Dec. 7
% this is a combination of Oldroydnew1 and Oldroyddecay
% \documentclass{article}
% \usepackage{ctex}
% \usepackage{amsfonts}
% \usepackage{amsmath}
% \usepackage{amsthm}
% \usepackage{cite}
% %\usepackage{showkeys}
% \theoremstyle{plain}
% \begin{document}
%  \newtheorem{theorem}{定理}
%  \newtheorem{remark}{注释}
%  \newtheorem{thm}{Theorem}
%  \newtheorem{lemma}{Lemma}
%  \theoremstyle{definition}
%  \theoremstyle{Remark}
% \newtheorem{rem}{Remark}
% \newtheorem{defn}{Definition}
% \begin{subequations} \label{eq:ECDFsecondisothermal}
% 		\begin{align}
% 			\rho_t + \nabla \cdot (\rho v) = 0 ,\\
% 			(\rho v)_t + \nabla \cdot (\rho v \otimes v) + \nabla (p + \frac{1}{2} \rho c: \rho c)  - \nabla \cdot ( (2 \rho c + 2 \rho c \cdot \rho c)) =0 ,\\
% 			(\rho c)_t +  \nabla \cdot (\rho c \otimes v) - (\nabla v) \rho c - (\rho c) (\nabla v)^T - 2 D = - \frac{\rho \dot{c}I}{\kappa} -  \frac{\rho \mathring{c}}{\xi}  .
% 		\end{align}
% \end{subequations}
% 在一维时,对应的方程为
% \begin{subequations} \label{eq:ECDFsecond1D}
% 		\begin{align}
% 			\rho_t + \partial_x (\rho v) = 0 ,\\
% 			(\rho v)_t + \partial_x (\rho v^2) + \partial_x (p)   -  (2+ 3 \rho c) \partial_x (  \rho c) =0 ,\\
% 			(\rho c)_t +  \partial_x (\rho c  v) - 2 \rho c \partial_x  v  - 2 \partial_x v = - \frac{\rho \dot{c}}{\kappa}  .
% 		\end{align}
% \end{subequations}

虽然模型\eqref{eq:ECDFsecond1D}的熵函数存在,但是由于方程不是守恒形式,熵函数的Hessian矩阵无法对称化这个一维方程组。我们在本小节中将会首先给出这一方程组的一个对称子。然后验证其满足Kawashima条件,从而证明平衡态附近解的整体存在性定理。最后我们讨论了源项含有小参数时的松弛极限,并给出了松弛参数趋于0时其近似一维Navier-Stokes方程的严格分析。

\subsection{熵函数和对称性}
首先由前面的分析,我们可以得到方程\eqref{eq:ECDFsecondisothermal}的比熵为
\begin{equation*}
	s (\nu, c ) =  -\int_{\rho_0}^{1/\nu} \frac{\pi(z)}{z^2} dx - \frac{1}{2 \nu}  \mbox{Tr} (c^2).
\end{equation*}
令$U = (\rho, \rho v, \rho c)^T$。我们可以得到方程的熵函数
\begin{equation*}
	\eta = \eta(U) = \frac{(\rho v)^2}{2\rho}+\rho \int_{\rho_0}^{\rho} \frac{\pi(z)}{z^2} dz + \frac{1}{2}  \mbox{Tr} \left((\rho c)^2\right).
\end{equation*}
注意我们在这里采用的熵函数$\eta$的符号与之前正好相反。%实际上这里$\eta$代表的是Hemholtz自由能。
$\eta$为其变量的下凸函数。这是因为
\begin{equation*}
	\eta_{U_{i''k''l''}U_{ikl}} = \left( \begin{array}{ccc} 
		\frac{p_\rho}{\rho} + \frac{v^2}{\rho} & -\frac{v_i}{\rho} & 0 \\
		-\frac{v_{i''}}{\rho} & \frac{1}{\rho} & 0 \\
		0 & 0 & \delta_{k''k}\delta_{l''l}
	\end{array}\right)
\end{equation*}
当$\pi_\rho>0$时是正定的。而因为静压力$\pi$为$\rho$的单调增函数$\pi_\rho>0$成立。

通过计算,我们可以得到熵的演化方程
\begin{eqnarray}\label{23}
	\eta_t &=&  - \nabla \cdot ( \eta v+ \pi v - (2\rho c + 2 \rho c \cdot \rho c) \cdot v) - \frac{1}{\kappa} \rho c: \rho c \nonumber\\
  &\equiv& -\nabla \cdot J(U) + \Delta.	
\end{eqnarray}
其中$\Delta \ge 0$。

对于含有熵函数的守恒律方程组\cite{friedrichs1971systems},我们知道熵函数的Hessian矩阵可以对称化该方程组。然而由于方程组\eqref{eq:ECDFsecondisothermal}不是守恒形式,熵函数的Hessian矩阵不能对称化这个方程组,下面我们来说明这一点。

首先我们将$U$的方程表示成下面的形式
\begin{equation*}
	U_t + \sum_{j=1}^n A_j(U) U_{x_j} = \mathcal{Q}(U).
\end{equation*}
其中$A_j$和$\mathcal{Q}$的表达式如下
\begin{eqnarray*}
	&&\mathcal{Q} (U)= \left( \begin{array}{c}
		0 \\ 0 \\-\frac{1}{\kappa} \rho c
	\end{array} \right), \quad A_j(U)_{ikl,i'k'l'} = \\
	 &&\left( \begin{array}{ccc}
		0 & \delta_{i'j} & 0 \\
		p_\rho -v_i v_j & v_i \delta_{i'j}+v_j \delta_{i'i} & (23) \\
		- c_{kl} v_j + c_{jl} v_k + c_{kj} v_l + \frac{1}{\rho}({v_k} \delta_{jl} + v_l \delta_{jk})  & (32) & v_j \delta_{kk'} \delta_{ll'}
	\end{array} \right)
\end{eqnarray*}
其中$(23)=\rho (c_{k'l'} \delta_{ij} - 2\delta_{ik'} c_{jl'} - 2\delta_{il'} c_{jk'} ) - (\delta_{ik'} \delta_{jl'} + \delta_{il'} \delta_{jk'}),(32)= c_{kl} \delta_{i'j}- \frac{1}{\rho} ( \delta_{i'k} c_{jl} + \delta_{i'l} c_{jk} + \delta_{ik} \delta_{jl} + \delta_{il} \delta_{jk})$
而熵函数的Hessian矩阵$\eta_{UU}$与$A_j$的乘积为
\begin{eqnarray*}
	\eta_{UU}  A_j(U) = \\
		 \left( \begin{array}{ccc}
		* & \frac{\pi_\rho -v_{i} v_j}{\rho}  & (13)  \\
		\frac{\pi_\rho -v_i v_j}{\rho} & * &  (23) \\
		(31) & (32) & *
	\end{array} \right)
\end{eqnarray*}
其中$(13)=- c_{k'l'} v_j + 2 c_{jl'} v_{k'} + 2 c_{k'j} v_{l'} + \frac{1}{\rho}({v_{k'}} \delta_{j{l'}} + v_{l'} \delta_{jk'}), (31)=- c_{k''l''} v_j + c_{jl''} v_{k''} + c_{k''j} v_{l''} + \frac{1}{\rho}({v_{k''}} \delta_{j{l''}} + v_{l''} \delta_{jk''}),(23)=c_{k'l'} \delta_{ij} -( \delta_{ik'} c_{jl'} + \delta_{il'} c_{jk'} +\delta_{ik'} \delta_{jl'} + \delta_{il'} \delta_{jk'}),(32) =c_{k''l''} \delta_{i'j} -( \delta_{i'k''} c_{jl''} + \delta_{i'l''} c_{jk''} + \delta_{ik''} \delta_{jl''} + \delta_{il''} \delta_{jk''}) $。
注意到该矩阵的$(1,3)$和$(3,1)$项中$c_{jl''} v_{k''}+ c_{k''j} v_{l''}$与$c_{jl'} v_{k'}+ c_{k'j} v_{l'}$前的系数不同。
%由于上对流Maxwell导数中包含非守恒形式的项$(\nabla v) \rho c$和$\rho c (\nabla v)^T$,
从而熵的Hessian矩阵无法对称化方程组\eqref{eq:ECDFsecondisothermal}。

然而,对于一维的情况,
\begin{equation*}
	A(U) = \left( \begin{array}{ccc}
		0 & 1 & 0 \\
		\pi_\rho - v^2 & 2v & -3 \rho c - 2 \\
		\frac{ (\rho c + 2) v}{\rho} & -  \frac{\rho c + 2}{\rho} & v 
	\end{array}\right).
\end{equation*}
我们可以找到下面的对称子
\begin{eqnarray}\label{31}
A_0(U) = \frac{1}{\rho} \left( \begin{array}{ccc}
	 \pi_\rho  +v^2 & -v & 0 \\ [2mm]
	-v & 1 & 0 \\[2mm]
	0 & 0 & \frac{3\rho c+2}{\rho c+2}\rho  \end{array} \right),
\end{eqnarray}
使得$A_0(U)$和
\begin{eqnarray*}
A_0(U) A(U) = \frac{1}{\rho} \left( \begin{array}{ccc}
		-(\pi_{\rho}-v^2)v & {p_\rho -v^2} & {(3\rho c + 2)v} \\[2mm]
		{\pi_\rho-v^2} & {v} & -{(3 \rho c+2)} \\[2mm]
		{(3 \rho c+ 2)v} & -{(3 \rho c+2)} & \frac{3\rho c+2}{\rho c + 2}\rho v
		\end{array} \right)
\end{eqnarray*}
都是对称的。并且当
\begin{eqnarray*}
\pi_\rho > 0 \quad \mbox{and} \quad \rho c > -\frac{2}{3} \quad \mbox{or} \quad  \rho c < -2 
\end{eqnarray*}
时,$A_0(U)$为正定的。在这里,我们取
\begin{equation}\label{37}
G := \{(\rho,\rho v, \rho c): \rho>0, \quad \rho c> - \frac{2}{3}\}
% \ \mbox{or} \ > 1
\end{equation}
作为1维模型\eqref{eq:ECDFsecond1D}的定义域。

另外,我们可以得到对于平衡态$U=U_e = (\rho_e,0,0)$,有
\begin{equation}\label{32}
A_0(U_e)Q_U(U_e) + Q_U^T(U_e)A_0(U_e) =-\frac{2}{\kappa}\mbox{diag}(0, 0, 1)
\end{equation}
从而\eqref{eq:ECDFsecond1D}在文献\cite{yong1992singular,yong1999singular}中定义的意义下是耗散的。

\subsection{Kawashima条件}
对于存在熵函数的带松弛源项的双曲守恒律方程组,其平衡态附近的整体存在性已经在文献\cite{yong2004entropy,hanouzet2003global}中给。其整体存在性的证明是基于Kawashima理论。对于非守恒形式含有熵的方程\eqref{eq:ECDFsecond1D}的整体存在性,无法采用一般的Kawashima理论直接得到。在这里我们将验证方程组\eqref{eq:ECDFsecond1D}满足Kawashima条件并给出补偿矩阵(Compensating Matrix)$K$\cite{kawashima1985systems},然后利用Kawashima理论来做估计得到整体存在性定理的证明。

令
\begin{eqnarray}\label{33}
K=\left( \begin{array}{ccc}
	0 & 1 & 0 \\
	-1 & 0 & -\rho_e \\
	0 & \rho_e & 0
	\end{array} \right),
\end{eqnarray}
和$\bar{L} = \eta\mbox{diag}(0, 0, 1)$(\eqref{32}的对角矩阵)。计算可得
\begin{eqnarray*}
K A(U_e) + (K A(U_e))^T + \bar{L} =
	\left( \begin{array}{ccc}
	2 \pi_\rho(\rho_e) & 0 & \rho_e \pi_\rho(\rho_e) -2  \\
	0 & 2 & 0 \\
	\rho_e \pi_\rho(\rho_e) - 2 & 0 & \eta -4 	
	\end{array} \right).
\end{eqnarray*}
在下面的条件下该矩阵为正定的。
\begin{eqnarray*}
\pi_\rho  > 0, \ \eta > 4 \rho, \ 2\pi_\rho(\eta - 4)-(\pi_{\rho}-2)^2 = 2 \pi_\rho \eta  - (2 +  \rho \pi_\rho)^2 >0.
\end{eqnarray*}
第一个条件是物理上对压力的要求。后两个在
$$
\eta \ge 2 \rho_e +\frac{(2 + \rho_e \pi_\rho(\rho_e))^2}{2 \pi_\rho(\rho_e)}.
$$
时成立。
从而对于足够大的$\eta$,我们有
\begin{eqnarray}\label{35}
K A(U_e) + (K A(U_e))^T + \bar{L} \ge C_s I
\end{eqnarray}
其中$C_s$为仅依赖于$\rho_e$的常数。

\subsection{平衡态附近解的整体存在性}
本小节我们考虑方程组\eqref{eq:ECDFsecond1D}在平衡态$U_e$附近解的整体存在性。我们将采用\cite{yong2004entropy,kawashima2009decay}中对于双曲守恒系统的相关估计得到\eqref{eq:ECDFsecond1D}的整体解存在的证明。虽然这里的方程是非守恒形式的,然而由于熵函数的存在、耗散性条件的成立\eqref{32}以及Kawashima条件的成立,我们仍然可以得到相关的估计及整体存在性定理。本小节的主要结果如下。
% 我们期望方程组\eqref{eq:ECDFsecond1D}在平衡点$U_e$附近存在整体光滑解。这里主要的问题是处理上对流导数中非守恒形式的项。本小节的主要结果为下面的定理。

\begin{theorem} \label{theoremglobal}
令整数$s \ge 2$。假设$U_0=U_0(x) \in H^s(\mathbb{R})$且$\|U_0 -U_e\|_{H^s}$足够小。那么方程组\eqref{eq:ECDFsecond1D}以$U_0$为初值的Cauchy问题存在唯一的整体解$U=U(x,t)$。且对于任意的$T>0$,$U$满足
$$
U-U_e \in C([0,+\infty),H^s(\mathbb{R}))
$$
与
\begin{eqnarray}\label{41}
\|U(T)-U_e\|^2_{H^s} + \int_0^T \left[ \|c(t)\|^2_{H^s} + \|\partial_x U(t)\|^2_{H^{s-1}} \right] dt \le C \|U_0 -U_e\|^2_{H^s}
\end{eqnarray}
其中$C$为与时间$T$无关的正常数。
\end{theorem}

\begin{proof}
由于方程组\eqref{eq:ECDFsecond1D}是对称双曲组,根据相关理论\cite{majda1984compressible, kato1975cauchy},其存在唯一的局部解$U \in H^s$。为了证明解的整体存在性,我们只需要证明先验估计(a priori estimate)\eqref{41},然后解的整体存在性可以通过标准的分析得到\cite{yong2004entropy}。下面我们证明\eqref{41}成立。我们分三步证明。

\emph{第一步:}我们首先利用熵函数得到$U-U_e$的$L^2$估计。

定义
\begin{eqnarray*}
E=E(U,U_e) = \eta(U)-\eta(U_e)-\eta_U(U_e)(U-U_e).
\end{eqnarray*}
因为$\eta=\eta(U)$是严格下凸的,所以存在正常数$c_1$ 和 $C_1$,使得对于充分靠近$U_e$的$U$,有
\begin{eqnarray}\label{42}
c_1 |U-U_e|^2 \le E(U,U_e) \le C_1 |U-U|^2
\end{eqnarray}
下面我们计算$E$的演化方程。根据\eqref{eq:ECDFsecond1D}和\eqref{23},
\begin{eqnarray}\label{43}
E_t &= \eta_t - \eta_U(U_e)U_t = -\nabla \cdot \left(J(U) +( \int_{\rho_0}^{\rho_e} \frac{\pi(z)}{z^2} dz + \frac{\pi(\rho_e)}{\rho_e})\rho v \right) - \frac{(\rho c)^2}{\kappa} \nonumber \\
& - S_U(U_e)(\mathcal{Q}(U) - A(U)U_x).
\end{eqnarray}
这里我们用到了
\begin{eqnarray*}
\eta_U(U_e) A(U) U_x &=& ( \int_{\rho_0}^{\rho_e} \frac{\pi(z)}{z^2} dz + \frac{\pi(\rho_e)}{\rho_e}, 0, 0)
\left( \begin{array}{ccc}
		0 & 1 & 0 \\
		\pi_\rho - v^2 & 2v & -3 \rho c - 2 \\
		cv + \frac{2 v}{\rho} & -  \rho c - 2 & v 
	\end{array}\right)
\left( \begin{array}{ccc}
\rho \\ \rho v \\ \rho c \end{array} \right)_x \\
&=& \left( \int_{\rho_0}^{\rho_e} \frac{\pi(z)}{z^2} dz + \frac{\pi(\rho_e)}{\rho_e}\right) (\rho v )_x.
\end{eqnarray*}
在$(x, t)\in(-\infty, + \infty)\times[0,T]$对\eqref{43}积分可以得到
\begin{eqnarray*}
\int_\mathbb{R}E(U(x, T), U_e)dx - \int_\mathbb{R} E(U_0, U_e)dx = - \int_0^T\int_\mathbb{R} \frac{c^2(x, t)}{\kappa} dx dt.
\end{eqnarray*}
结合\eqref{42},我们最后可以得到下面的$L^2$估计。
\begin{eqnarray}\label{44}
\|U(\cdot, T)-U_0\|^2_{L^2} + \int_0^T \|c(\cdot, t)\|^2_{L^2}dt \le C\|U_0 - U_e\|^2_{L^2},
\end{eqnarray}

\emph{第二步:}下面我们对高阶导数的$L^2$范数进行估计。

首先对方程\eqref{eq:ECDFsecond1D}的两端求$l\le s$($l$为整数)阶导数。 
\begin{eqnarray*}
\partial^l_x U_t + A(U) \partial^l_x U_x = \partial^l_x Q(U) + [A(U),\partial^l_x]U_x.
\end{eqnarray*}
其中$[X,Y]=XY-YX$为交换子。由于$A_0(U)$和$A_0(U)A(U)$对称,对上面的式子与$A_0(U)\partial^l_x U$取$L^2$内积得到
\begin{multline}\label{45}
  (A_0(U)\partial^l_x U,\partial^l_x U)_t + \int (\partial^l_x U^T A_0(U)A(U)\partial^l_x U)_x dx \\
  = ((\partial_t A_0(U)+\partial_x(A_0(U)A(U)))\partial^l_x U,\partial^l_x U) \\
  + 2(A_0(U)[A(U),\partial^l_x]U_x,\partial^l_x U) + 2(A_0(U) \partial^l_x Q(U), \partial^l_x U).
\end{multline}
对任意的时间$t$,$U(\cdot, t)\in H^s(\mathbb{R})$,所以上式左端第二项为0。右端三项估计如下。
%It follows from the postive definite property of $A_0(U)$ that

对第一项,有
\begin{eqnarray}\label{HsEST:1}
  && ((\partial_t A_0(U) + \partial_x (A_0(U)A(U)))\partial^l_x U, \partial^l_x U) \nonumber\\
  &\le&|\partial_t A_0(U) + \partial_x (A_0(U)A(U))|_{L^\infty}\|\partial^l_x U\|_{L^2}^2 \nonumber\\
  &\le& C (|\partial_t U|_{L^\infty}+|\partial_x U|_{L^{\infty}})\|\partial^l_x U\|_{L^2}^2  \\
  &\le& C (|\partial_x U|_{L^\infty}+|Q(U)|_{L^\infty})\|\partial^l_x U\|_{L^2}^2\nonumber\\
  &\le& C (|\partial_x U|_{L^\infty} + |c|_{L^\infty}) \|\partial^l_x U\|_{L^2}^2\nonumber \\
  &\le& C \|U-U_e\|_{H^s} \|\partial^l_x U\|_{L^2}^2. \nonumber
\end{eqnarray}
第二项可以估计如下
\begin{eqnarray*}
  2(A_0(U) [A,\partial^l_x] U_x, \partial^l_x U) \le C |A_0(U)|_{L^\infty} \|[A(U),\partial^l_x] U_x\|_{L^2} \|\partial^l_x U\|_{L^2}.
\end{eqnarray*}
对于含交换子的项我们可以采用Sobolev空间中的演算不等式\cite{majda1984compressible}。
\begin{eqnarray*}
  \|[A(U),\partial^l_x] U_x\|_{L^2} &\le& C (\|\partial_x^s A(U)\|_{L^2}|U_x|_{L^\infty} + |\partial_x A(U)|_{L^{\infty}} \|\partial_x^{s-1} U_x\|_{L^2})  \\
  &\le& C|U_x|_{L^\infty} \|\partial_x^s U\|_{L^2}.
\end{eqnarray*}
于是我们得到
\begin{eqnarray}\label{HsEST:2}
  2(A_0(U) [A,\partial_x^l] U_x,\partial^l_x U) \le C \|U-U_e\|_{H^s}\|\partial_x U\|_{H^{s-1}}^2.
\end{eqnarray}
对于第三项,回忆$Q(U)$和$A_0(U)$的表达式,我们得到
\begin{eqnarray}\label{HsEST:3}
&& 2(A_0(U)\partial^l_x Q(U),\partial^l_x U) \nonumber \\
&=& 2( A_0(U_e)\partial^l_x Q(U),\partial^l_x U) + 2((A_0(U)-A_0(U_e)) \partial^l_x Q(U),\partial^l_x U) \nonumber \\
&\le&  -2 \|\partial^l_x c\|_{L^2}^2 + C|U-U_e|_{L^\infty} \|\partial^l_x U\|_{L^2}^2 \nonumber \\
&\le& -2 \|\partial^l_x c\|_{L^2}^2 + C\|U-U_e\|_{H^s} \|\partial^l_x U\|_{L^2}^2.
\end{eqnarray}

由估计\eqref{HsEST:1}、\eqref{HsEST:2}和\eqref{HsEST:3},我们在$[0,T]$对\eqref{45}积分得到
\begin{eqnarray}\label{49}
  \|\partial^l_x U(T)\|_{L^2}^2&  + & \int_0^T \|\partial^l_x c(t)\|^2_{L^2}dt \le C\|\partial^l_x U(0)\|_{L^2}^2  \nonumber\\[3mm]
 & +& C\sup_{t \in [0,T]}\|U(t)-U_e\|_{H^s} \int_0^T \|\partial_x U(t)\|^2_{H^{s-1}}dt .
\end{eqnarray}
这里我们用到了
\begin{eqnarray*}
   C^{-1} \|\partial^l_x U\|_{L^2}^2 \le  (A_0(U) \partial^l_x U,\partial^l_x U) \le C \|\partial^l_x U\|_{L^2}^2.
\end{eqnarray*}
将\eqref{44}和\eqref{49}($1 \le l \le s$)相加可以得到
  \begin{eqnarray}\label{210}
    && \|U(T)-U_e\|^2_{H^{s}}  +  \int_0^T \|c(t)\|^2_{H^s} dt \nonumber \\
    & \le & C \|U_0-U_e\|_{H^{s}}^2 + C \sup_{t \in [0,T]} \|U(t) - U_e\|_{H^s} \int_0^T \|\partial_x U\|_{H^{s-1}}^2dt.
%  \\  \le C \|\partial_x U_0\|^2_{H^{s-1}} + C M(t)D_0(t)^2
\end{eqnarray}

\emph{第三步:}下面我们利用Kawashima条件来得到上式\eqref{210}中最后一项的估计。

首先将\eqref{eq:ECDFsecond1D}写作
\begin{eqnarray*}
  U_t + A(U_e) U_x  = (A(U_e) -A(U))U_x + Q(U).
\end{eqnarray*}
对两边求$l$阶导数并用$K$乘以方程两边得到
\begin{eqnarray*}
  K \partial^l_x U_t + K A(U_e) \partial^l_x U_x  = K \partial^l_x ((A(U_e)-A(U))U_x) + K\partial^l_x Q(U).
\end{eqnarray*}
然后与$\partial^l_x U_x$ 取$L^2$内积得到
\begin{eqnarray}\label{211}
  (K \partial^l_x U_t, \partial^l_x U_x) + (KA(U_e) \partial^l_x U_x, \partial^l_x U_x) \nonumber \\ = (K \partial^l_x((A(U_e)-A(U))U_x),\partial^l_x U_x) + (\partial^l_x( K Q(U) ),\partial^l_x U_x).
\end{eqnarray}
由$K$反对称,左端第一项可以处理如下。
\begin{eqnarray}\label{212}
  (K\partial^l_x U_t, \partial^l_x U_x) &=& \frac{1}{2} \int_\mathbb{R} \left[(\partial^l_x U^T_x K \partial^l_x U )_t -  (\partial^l_x U_t^T K\partial^l_x U  )_x \right]dx  \nonumber \\
  &=& \frac{1}{2}(K \partial^l_x U,\partial^l_x U_x)_t .
\end{eqnarray}
对第二项,利用\eqref{35}得到
\begin{eqnarray}\label{213}
  2(KA(U_e) \partial^l_x U_x,\partial^l_x U_x) &=& ( (KA(U_e)+ (K A(U_e))^T +\bar{L}) \partial^l_x U_x, \partial^l_x U_x) - (\bar{L}\partial^l_x U_x, \partial^l_x U_x) \nonumber\\
  &\ge& C_s \|\partial^l_x U_x \|^2_{L^2} -\eta \|\partial^l_x c\|^2_{L^2},
\end{eqnarray}

\eqref{211}的右端可以估计如下
\begin{eqnarray}\label{214}
  (K \partial^l_x Q(U),\partial^l_x U_x) \le \epsilon \|\partial^l_x U_x\|_{L^2}^2 + \frac{C}{\epsilon} \|\partial^l_x c\|_{L^2}^2
\end{eqnarray}
以及
\begin{eqnarray}\label{215}
  && (K\partial^l_x ((A(U_e)-A(U))U_x),\partial^l_x U_x) \le \epsilon\|\partial^l_x U_x\|_{L^2}^2 + \frac{C}{\epsilon} \|\partial^l_x( (A(U_e)-A(U))U_x)\|_{L^2}^2 \nonumber \\
  &\le& \epsilon \|\partial^l_x U_x \|_{L^2}^2 + C(\epsilon)( |A(U_e)-A(U )|_{L^\infty}^2 \|\partial^l_x U_x\|^2_{L^2}+\|\partial^l_x A(U)\|_{L^2}^2|U_x|_{L^\infty}^2) \nonumber \\
  &\le& \epsilon \|\partial^l_x U_x\|_{L^2}^2 + C(\epsilon) (|U-U_e|_{L^\infty}^2 \|\partial^l_x U_x\|_{L^2}^2 + |U_x|_{L^\infty}^2 \|\partial^l_x U\|_{L^2}^2).
\end{eqnarray}
这里我们也用到了交换子的估计。

利用估计\eqref{212}—\eqref{215},我们从\eqref{211}可以得到
\begin{multline*}
  C_s \|\partial^l_x U_x\|^2_{L^2} \le (\eta+\frac{C}{\epsilon}) \|\partial^l_x c\|_{L^2}^2-\frac{1}{2} (K \partial^l_x U,\partial^l_x U_x)_t  + 2\epsilon \|\partial^l_x U_x\|^2_{L^2} \\
  +C(\epsilon) |U-U_e|_{L^\infty}^2 \|\partial^l_x U_x\|_{L^2}^2 + C(\epsilon) |U_x|_{L^\infty}^2 \|\partial^l_x U\|_{L^2}^2.
\end{multline*}
选取足够小的$\epsilon$(例如$\epsilon = C_s/4$)。在$[0,T]$上对上述不等式积分,我们得到对于$1\le l \le s-1$,估计
\begin{multline*}
  \int_0^T \|\partial^l_x U_x(t)\|^2_{L^2} dt \le C \int_0^T \|\partial^l_x c(t)\|^2_{L^2} dt
  +C\|\partial^l_x U(T)\|_{H^1}^2 + C\|\partial^l_x U_0\|_{H^1}^2 \\
  + C \sup_{t\in [0,T]} (|U(t)-U_e|_{L^\infty}^2 + |U_x(t)|_{L^\infty}^2) \int_0^T \|\partial^l_x U(t)\|_{L^2}^2 dt
\end{multline*}
成立。对$1 \le l \le s-1$求和可以得到
\begin{multline}\label{216}
  \int_0^T \|\partial_x U(t)\|^2_{H^{s-1}} dt \le C \int_0^T \|c(t)\|^2_{H^s} dt + C\|U_0-U_e\|^2_{H^s} + C\|U(T)-U_e\|^2_{H^s} \\+C \sup_{0 \le t \le T}(|U(t)-U_e|_{L^\infty}^2 + |U_x(t)|_{L^\infty}^2) \int_0^T \|\partial_x U(t)\|_{H^{s-1}}^2 dt \\
  \le C \int_0^T \|c(t)\|_{H^s}^2 dt + C \|U_0-U_e\|_{H^s}^2 + C\|U(T) -U_e\|_{H^s}^2  \\ + C \sup_{t \in [0,T] }\|U(t)-U_e\|_{H^s}^2 \int_0^T \|\partial_x U(t)\|_{H^{s-1}}^2 dt.
\end{multline}

令$\alpha>0$,用$\alpha$乘式子\eqref{216} 并将结果与\eqref{210}相加可以得到
\begin{eqnarray*}
  \|U(T)-U_e\|_{H^s}^2 + \int_0^T \|c(t)\|_{H^s}^2 dt + \alpha \int_0^T \|\partial_x U(t)\|_{H^{s-1}}^2 dt \\
  \le C \alpha \int_0^T \|c(t)\|_{H^s}^2 dt +C(1+\alpha)\|U_0 -U_e\|_{H^s}^2 + C \alpha \|U(T)-U_e\|_{H^s}^2  \\ + C(\sup_{t \in [0,T]} \|U(t)-U_e\|_{H^s} + \alpha \sup_{t \in [0,T]} \|U(t)-U_e\|_{H^s}^2)  \int_0^T \|\partial_x U\|_{H^{s-1}}^2 dt.
\end{eqnarray*}
$\alpha$足够小时,由上述不等式可以导出
\begin{eqnarray*}
  \|U(T)-U_e\|_{H^s}^2 + \int_0^T \|c(t)\|_{H^s}^2 dt + \int_0^T \|\partial_x U(t)\|_{H^{s-1}}^2 dt \\
  \le C \|U_0-U_e\|_{H^s}^2 + C \sup_{t \in [0,T]} \|U(t)-U_e\|_{H^s} \int_0^T \|\partial_x U(t)\|_{H^{s-1}}^2 dt \\
  +C\alpha  \sup_{t \in [0,T]} \|U(t)-U_e\|_{H^s}^2 \int_0^T \|\partial_x U(t)\|_{H^{s-1}}^2 dt.
\end{eqnarray*}
由此定理\ref{theoremglobal}中的先验估计对于足够小的$\|U_0 -U_e\|_{H^s}$成立,这样我们证明了定理\ref{theoremglobal}。
\end{proof}

\subsection{与一维Navier-Stokes方程组的一致性}

当$\kappa$足够小时,我们可以认为粘弹性的效应足够小,这时我们有理由相信仅考虑粘性效应足以对粘弹性流体做很好的近似描述。在本小节中,我们将说明一维的等温可压上对流导数Maxwell模型\eqref{eq:ECDFsecond1D}与经典的可压Navier-Stokes方程在$\kappa$很小时是一致的,即$\kappa$很小时Navier-Stokes方程可以很好地近似等温可压上对流导数Maxwell模型。

首先采用Maxwell迭代可以得到方程组\eqref{eq:ECDFsecond1D}形式上的近似Navier-Stokes方程组。将\eqref{eq:ECDFsecond1D}写为
\begin{eqnarray*}
 \rho c= -\kappa(( \rho c)_t + v \partial_x (\rho c) -  \rho c \partial_x v + 2 \partial_x v.
\end{eqnarray*}
迭代一次得到
\begin{eqnarray*}
  \rho c = 2 \kappa \partial_x v + O(\kappa^2)。
\end{eqnarray*}
带入\eqref{eq:ECDFsecond1D}中的动量方程,并忽略高阶项,可以得到下面的一维Navier-Stokes方程组。
\begin{align}\label{51}
  \partial_t \rho + \partial_x (\rho v ) = 0, \nonumber \\
  \partial_t (\rho v) + \partial_x( \rho v^2 + \pi) = 4 \kappa \partial^2_x v
\end{align}
其中$4\kappa$为粘性系数。这一形式近似的合理性由下面的定理保证。

% 由\eqref{32},方程组\eqref{eq:ECDFsecond1D}满足文献\cite{yong1999singular}中的条件。从而可以得知其一阶摄动展开方程组的解可以很好地近似原方程组的解。这样我们只需证明Navier-Stokes方程组可以很好地近似其摄动展开方程组。

% 我们的分析基于\cite{yang2015validity},但是我们需要处理\eqref{eq:ECDFsecond1D}中的非守恒项。本小节的主要结果如下。
\begin{theorem}\label{theoremCE}
令整数$s \ge 2$, ${\bar u} =({\bar \rho}(x),\bar{\rho}(x){\bar v}(x))$满足
  \begin{eqnarray*}
    \bar{u}\in H^{s+2},\ \inf_{x} \bar{\rho}(x)>0.
 \end{eqnarray*}
那么存在与松弛参数$\kappa$无关的时间$T_*>0$,使得方程组\eqref{eq:ECDFsecond1D}以$(\bar{u},0)$为初值的Cauchy问题和Navier-Stokes方程组\eqref{51}以${\bar u}$为初值的Cauchy问题分别有唯一解$(u^\kappa=(\rho^\kappa,\rho^\kappa v^\kappa), c^\kappa)(x,t),u^\kappa_p=(\rho^\kappa_p,\rho^\kappa_p v^\kappa_p)(x,t) \in C([0,T_*], H^s)$。且对于充分小的$\kappa$,它们满足
  \begin{equation}\label{52}
    \sup_{t \in [0, T_*]} \|(u^\kappa-u^\kappa_p)(\cdot,t)\|_{H^s} \le C(T_*) \kappa^2
  \end{equation}
 其中$C(T_*)>0$不依赖于$\kappa$.
\end{theorem}

% \begin{remark}
% 简单起见,定理\ref{theoremCE}中方程组\eqref{eq:ECDFsecond1D}和\eqref{51}的初值相同。但是对于一般的初值,对\eqref{51}取初值${\tilde u}_p =({\tilde \rho}_p(x, \kappa),\tilde{\rho}_p(x,\kappa){\tilde v}_p(x, \kappa))$,对$\eqref{eq:ECDFsecond1D}$取$({\tilde u} =({\tilde \rho}(x, \kappa),\tilde{\rho}(x,\kappa){\tilde v}(x, \kappa)),\tilde{\rho} \tilde {c}(x, \kappa))$,那么只要初值只要满足
% \begin{eqnarray*}
%     \tilde{u}_p(\cdot,\kappa),\tilde{u}(\cdot,\kappa) \in H^{s+2}, \quad  \inf_{x, \kappa} \tilde{\rho}_p(x,\kappa), \inf_{x, \kappa} \tilde{\rho}(x,\kappa)>0, \quad \inf_{x, \kappa} \tilde{c}(x,\kappa)> -2/3, \nonumber  \\
%     \|\tilde{\rho}(\cdot,\kappa)-\tilde{\rho}_p(\cdot,\kappa)\|_{H^s}, \| \tilde{\rho} \tilde{v}(\cdot,\kappa)-\tilde{\rho}_p\tilde{v}_p(\cdot,\kappa)- \frac{3}{2} \kappa \bar{c}_{0x} \bar{c}_0 +2 \kappa \bar{c}_{0x}\|_{H^s}  = O(\kappa^2)
%  \end{eqnarray*}
%   其中$\bar{c}_0=\tilde{c}(x,0)$。那么定理\ref{theoremCE}仍然成立,只是估计\eqref{52}变为
% \begin{equation*}\label{eq:Result}
%     \sup_{t \in [-\kappa\ln\kappa, T_*]} \|(u^\kappa-
%     %u'_\kappa-
%     u^\kappa_p)(\cdot,t)\|_{H^s} \le C(T_*) \kappa^2,
%   \end{equation*}
%   这是因为对一般的初始值边界层可能存在。只需要对下面的证明做微小改动就可以得到一般情况的证明。
% \end{remark}

为了证明上述定理,我们注意到方程组\eqref{eq:ECDFsecond1D}满足文献\cite{yong1992singular,yong1999singular}中的结构条件。从而,其中的一阶双曲偏微分方程的奇异摄动理论可以用于本方程组。于是我们得到\eqref{eq:ECDFsecond1D}以$(\bar{u},0)$为初值的解$U^\kappa = (\rho^\kappa, \rho^\kappa v^\kappa, c^\kappa)^T$和一阶摄动展开对应的方程组的解$U_\kappa^1=(\rho_\kappa^1,\rho_\kappa^1 v_\kappa^1, c^1_\kappa)^T$在$[0,T_*]$上满足
\begin{eqnarray}\label{53}
  \sup_{t \in [0, T_*]} \|U^\kappa(\cdot, t) - U_\kappa^1(\cdot, t)\|_{H^s} \le K\kappa^2,
\end{eqnarray}
其中$K$为不依赖与$\kappa$的正常数。从而$U^\kappa$在$[0,T_*]$的解亦存在。

$U^\kappa_1$的表达式可由\cite{yong1992singular,yong1999singular}中的方法得到。假设其形式为
\begin{eqnarray*}
U^1_\kappa  = U_0(x,t) + \kappa U_1(x,t) + U'_0(x,t') + \kappa U'_1(x,t'),
\end{eqnarray*}
其中$t'=t/\kappa$。
对于外展开$U_0=(u_0, (\rho c)_0)$和$U_1=(u_1, (\rho c)_1)$满足
\begin{eqnarray}
 (\rho c)_0 = 0, \quad
  \partial_t u_0+ \partial_x f(u_0,0)=0 , \nonumber \\
  (\rho c)_1 = 2\partial_x v_0, \quad
  \partial_t u_1 + \partial_x (f_u(u_0,0) u_1 + f_{\rho c}(u_0,0)(\rho c)_1)  = 0 \label{54},
	\end{eqnarray}
其中
\begin{eqnarray*}
  f(\rho,\rho v,\rho c) = \left( \begin{array}{cc} \rho v \\ \rho v^2 + \pi - \frac{3}{2} (\rho c)^2 - 2\rho c \end{array} \right);
\end{eqnarray*}
由随后一个方程可得
\begin{equation*}
	  \partial_t u_1 + \partial_x (f_u(u_0,0) u_1   = \left( \begin{matrix}
	  		0 \\ 4 \partial_x^2 v_0
	  \end{matrix} \right).
\end{equation*}
内展开$U_0'=(u_0', (\rho c)_0')$和$U_1'=(u_1', (\rho c)_1')$满足
\begin{eqnarray*}
 \partial_{t'} u'_0 = 0,  \quad \partial_{t'} (\rho c)'_0 = -(\rho c)'_0 , \\
  \partial_{t'} u'_1 = \partial_x(f(\bar u_0,0) - f(\bar u_0,(\rho c)'_0)), \\
  \partial_{t'} (\rho c)'_1 = - (\rho c)'_1 - \bar v_0 \partial_x (\rho c)_0' -2 (\rho c)_0'\partial_x \bar v_0.
\end{eqnarray*}
其初值由下面的匹配条件决定。
\begin{eqnarray*}
  \lim_{t' \to \infty} U'_0(x,t') = 0 , \quad \lim_{t' \to \infty} U'_1(x,t') = 0 .
\end{eqnarray*}
从而有
\begin{eqnarray*}
 u'_0 =0, \quad u_0(x,0) = \bar{u}(x), \quad  (\rho c)'_0 = (\rho c)'_0(x,0)e^{-t/\kappa}=0 , \\[4mm]
 u'_1(x,0) = -\partial_x\int_0^{+\infty} (f(\bar u_0,0) - f(\bar u_0,(\rho c)'_0)) dt'=0\\
% = \partial_x\int_0^{+\infty} \left( \begin{array}{cc} 0 \\  \frac{3}{2} (c_0')^2 - 2c'_0 \end{array} \right) dt' =
 %\left( \begin{array}{cc} 0 \\  \frac{3}{2} \bar c_{0x}\bar c_0 - 2\bar c_{0x} \end{array} \right), \\[4mm]
 u_1(x,0) =  - u'_1(x,0)=0, \quad (\rho c)'_1(x,0) = - 2\partial_x\bar v_0(x).
\end{eqnarray*}
这与式子\eqref{53}给出下面的估计
\begin{eqnarray*}
  \|u^\kappa(\cdot, t) - u_0 (\cdot, t) - \kappa u_1(\cdot, t) \|_{H^s} \le K \kappa^2
\end{eqnarray*}
对$t \in [0,T_*]$成立。从而为了证明定理\ref{theoremCE},只需证明下面的结果。



\begin{lemma}\label{lemmaCE}
在定理\ref{theoremCE}的条件下,方程组\eqref{51}以$\bar{u}$为初值的Cauchy问题有唯一解$u_p^\kappa \in C([0,T_*],H^s)$,且对充分小的$\kappa$其满足
\begin{eqnarray}\label{eq:wcediff}
  \sup_{t \in [0,T_*]} \| u^\kappa_p(\cdot,t) - u_0 (\cdot, t) - \kappa u_1(\cdot, t)\|_{H^s} \le C \kappa^2
\end{eqnarray}
这里$C=C(T_*)$与$\kappa$无关。
\end{lemma}

\begin{proof}
由Navier-Stokes方程的存在性定理\cite{kawashima1984systems},方程组\eqref{51}存在唯一解$u^\kappa_p=u^\kappa_p(x,t)$,且其满足$u^\kappa_p \in C([0,T],H^s)$。对集合$G_1\subset\subset G= \{(\rho, \rho v, \rho c): \rho>0,\ \rho c > -2/3\}$,我们定义其解的最大存在时间为
\begin{eqnarray*}
  T^\kappa :=\sup \{T>0 : u_p^\kappa  \in C([0,T],H^s), u^\kappa_p(x,t) \in G_1 \}.
\end{eqnarray*}
由\cite{yong2001basic}中结果可知只要下面的估计成立就有$\kappa$趋于$0$时
$T^\kappa > T_*$
成立。
\begin{eqnarray*}
\sup_{x,t} |u_p^\kappa(x,t) - u_0 (x, t) - \kappa u_1(x,t)|=o(1), \\
\sup_t \| u_p^\kappa(\cdot ,t) - u_0 (\cdot, t) - \kappa u_1(\cdot, t) \|_{H^s} = O(1)
\end{eqnarray*}
对$t \in [0,\min\{T_*,T^\kappa\})$成立。这样我们只需证明误差估计\eqref{eq:wcediff}(其中$T_*$由$\min\{T_*,T^\kappa\}$代替)。

方便起见,我们估计$w=(\rho,v)$从而得到$u=(\rho,\rho v)$的估计。这里我们将$u^\kappa_p$记做$u^\kappa$。由Sobolev演算不等式\cite{majda1984compressible},$w$的范数与$u$的范数等价。

令
\begin{eqnarray*}
  a(w) = \left( \begin{array}{cc} v & \rho \\ \frac{\pi_\rho}{\rho} & v \end{array} \right) \ \ \mbox{与} \ \
  a_0(w) = \left( \begin{array}{cc} \frac{\pi_\rho}{\rho^2} & 0 \\ 0 & 1 \end{array} \right).
\end{eqnarray*}
这里$a_0(w)$是对称正定的且$a_0(w)a(w)=a^T(w)a_0(w)$。
方程组\eqref{51}可以写作
\begin{eqnarray}\label{57}
  \partial_t w^\kappa + a(w^\kappa) \partial_x w^\kappa = \left( \begin{array}{cc} 0 \\ \frac{4 \kappa}{\rho^\kappa} \partial^2_x v^\kappa \end{array} \right) .
\end{eqnarray}

另外,由\eqref{54}得到$u_\kappa \equiv u_0 +\kappa u_1$满足下面的方程
\begin{eqnarray*}
  &&\partial_t u_\kappa + f(u_\kappa,0)_x \nonumber \\
  &=&  (f(u_\kappa,0) - \kappa f_u(u_0,0)u_1 - f(u_0,0))_x + \left( \begin{array}{c} 0 \\ 4 \kappa \partial^2_x v_0 \end{array} \right) \nonumber \\
	&=& \left( \begin{array}{c} 0 \\ 4 \kappa \partial^2_x v_\kappa \end{array} \right)   + R
\end{eqnarray*}
其中
$$
R=(f(u_\kappa,0) - \kappa f_u(u_0,0)u_1 - f(u_0,0))_x -\left( \begin{array}{c} 0 \\ 4 \kappa^2 \partial^2_x v_1 \end{array} \right).
$$
易知$w_\kappa$满足
\begin{eqnarray}\label{580}
  \partial_t w_\kappa + a(w_\kappa) \partial_x w_\kappa = \left( \begin{array}{cc} 0 \\ \frac{4 \kappa}{\rho_\kappa} \partial^2_x v_\kappa \end{array} \right) + \hat R
\end{eqnarray}
其中
\begin{eqnarray*}
\hat{R} = \left( \begin{array}{cc} 1 & 0 \\ 0 & 1/\rho_\kappa \end{array} \right)R .
\end{eqnarray*}
由Sobolev演算不等式\cite{majda1984compressible}与$u_\kappa \in H^{s+2}$得到
\begin{eqnarray*}
  \|\partial^2_x v_1 \|_{H^s} &\le&  \|v_1\|_{H^{s+2}}
\end{eqnarray*}
与
\begin{eqnarray*}
  \|(f(u_\kappa,0) &-& \kappa f_u(u_0,0)u_1 - f(u_0,0))_x\|_{H^s} \\
  &=& \kappa^2 \| u_1^Tf_{uu}(u_0+\kappa \theta_2 \theta_1  u_1,0) \theta_1 u_1\|_{H^{s+1}} \\
  &\le&  C \kappa^2 \|u_1\|_{H^{s+1}}^2 \|u_0+\kappa \theta_2 \theta_1  u_1\|_{H^{s+1}},
\end{eqnarray*}
其中$\theta_1,\theta_2 \in (0,1)$,从而$\|\hat{R}\|_{H^s} =O(\|R\|_{H^s})= O(\kappa^2)$。


取$E = w^\kappa - w_\kappa$,由\eqref{57}和\eqref{57}得到
\begin{eqnarray*}
  \partial_t E + a(w^\kappa) \partial_x E = (a(w_\kappa) - a(w^\kappa)) \partial_x w_\kappa + \left( \begin{array}{cc} 0 \\ \frac{4\kappa}{\rho^\kappa} \partial^2_x v^\kappa - \frac{4\kappa}{\rho_\kappa} \partial^2_x v_\kappa \end{array} \right) - \hat{R}.
\end{eqnarray*}
这可以重写为
\begin{eqnarray*}
   \partial_t E + a(w^\kappa) \partial_x E = (a(w_\kappa) - a(w^\kappa)) \partial_x w_\kappa + \left( \begin{array}{cc} 0 \\ \frac{4\kappa}{\rho^\kappa} \partial^2_x (v^\kappa - v_\kappa) \end{array} \right) \\
   + \left( \begin{array}{cc} 0 \\ 4\kappa(\frac{1}{\rho^\kappa} -\frac{1}{ \rho_\kappa})\partial^2_x v_\kappa \end{array} \right)- \hat{R}.
\end{eqnarray*}
两边求$l$阶导数($0 \le l \le s$)得到
\begin{eqnarray*}
  \partial_t \partial^l_x E + a(w^\kappa) \partial^l_x E_x  = [a(w^\kappa),\partial^l_x] E_x  + \left( \begin{array}{cc} 0 \\4\kappa \partial^l_x(\frac{1}{\rho^\kappa} \partial^2_x (v^\kappa-v_\kappa)) \end{array} \right) \\  + \left( \begin{array}{cc} 0 \\ 4 \kappa \partial^l_x ((\frac{1}{\rho^\kappa} - \frac{1}{\rho_\kappa}) \partial^2_x v_\kappa) \end{array} \right) -\partial^l_x \hat{R} + \partial^l_x ((a(w_\kappa)-a(w^\kappa))\partial_x w_\kappa).
\end{eqnarray*}
将上述等式乘以$\partial^l_x E^Ta_0(w^\kappa)$并对空间$x$积分得到
\begin{eqnarray}\label{58}
&& (a_0(w^\kappa) \partial^l_x E,\partial^l_x E)_t + \int ( \partial^l_x E^T a_0(w^\kappa) a(w^\kappa) \partial^l_x E )_x dx \nonumber \\
&=&  ((a_0(w^\kappa)_t + (a_0(w^\kappa)a(w^\kappa))_x) \partial^l_x E,\partial^l_x E)
+ 2(a_0(w^\kappa)[a(w^\kappa),\partial^l_x] E_x,\partial^l_x E)\nonumber \\
&& + 2 (4\kappa \partial^l_x (\frac{1}{\rho^\kappa} \partial^2_x (v^\kappa-v_\kappa)),\partial^l_x (v^\kappa-v_\kappa)) + 2(4\kappa \partial^l_x ( (\frac{1}{\rho^\kappa}-\frac{1}{\rho_\kappa})\partial^2 v_\kappa),\partial^l_x (v^\kappa-v_\kappa)) \nonumber\\
&& - 2(a_0(w^\kappa)\partial^l_x\hat{R},\partial^l_x E)+ 2(a_0(w^\kappa)\partial^l_x((a(w_\kappa)-a(w^\kappa))\partial_x w_\kappa),\partial^l_x E) \nonumber \\
& \equiv& I_1 +I_2 +I_3 +I_4+I_5 + I_6.
\end{eqnarray}

下面我们估计上述方程中的项。首先我们注意到左端第二项为$0$。并且$w^\kappa$在紧集$G_1$中取值,$\|w_\kappa(\cdot, t)\|_{H^{s+2}}$对$\kappa$一致有界。 利用Sobolev演算不等式\cite{majda1984compressible},$I_2, I_4, I_5$和$I_6$可以估计如下。
\begin{eqnarray*}
  I_2 &\le& 2 |a_0(w^\kappa)|_{L^\infty} \|[a(w^\kappa),\partial^l_x]E_x\|_{L^2} \|\partial^l_x E\|_{L^2} \nonumber \\
  &\le& C \|\partial^l_x E\|_{L^2} (|\partial_x a(w^\kappa)|_{L^\infty}\|\partial^{s-1}_x E_x\|_{L^2} + |E_x|_{L^\infty} \|\partial^s_x a(w^\kappa)\|_{L^2})  \nonumber \\
  &\le& C \|w^\kappa\|_{H^s} \|E\|_{H^s}^2 \le C (1+\|E\|_{H^s}) \|E\|_{H^s}^2, \nonumber \\
  I_4 &\le&  C \kappa \|\partial^l_x (v^\kappa-v_\kappa)\|_{L^2} \|\frac{1}{\rho^\kappa}-\frac{1}{\rho_\kappa}\|_{H^s} \|\partial^2_x v_\kappa\|_{H^s}
  \le C \kappa \|E\|_{H^s}^2, \nonumber\\
  I_5 &\le& C \|\partial^l_x \hat{R}\|_{L^2} \|\partial^l_x E\|_{L^2} \le C\kappa^2\|\partial^l_x E\|_{L^2} \le C \kappa^4 + C \|\partial^l_x E\|_{L^2}^2, \nonumber \\
  I_6 &\le& C\|\partial^l_x ((a(w_\kappa)-a(w^\kappa))\partial_x w_\kappa)\|_{L^2} \|\partial^l_x E\|_{L^2}\nonumber\\
      &\le&  C \|w_\kappa-w^\kappa\|_{H^s} \|w_\kappa\|_{H^{s+1}} \|\partial^l_x E\|_{L^2} \nonumber\\
      &\le&  C \|E\|_{H^s}^2. \label{515}
\end{eqnarray*}

对第一项,利用\eqref{51}得到
\begin{eqnarray*}
  I_1 &\le&  |a_0(w^\kappa)_t + (a_0(w^\kappa) a(w^\kappa))_x|_{L^\infty} \|\partial^l_x E\|_{L^2}^2 \nonumber\\
      &\le& C (|w^\kappa_t|_{L^\infty} + |w^\kappa_x|_{L^\infty}) \|\partial^l_x E\|_{L^2}^2 \nonumber\\
      &\le& C (|w^\kappa_x|_{L^\infty} + \kappa |\partial^2_x v^\kappa|_{L^\infty}) \|\partial^l_x E\|_{L^2}^2 \nonumber\\
      &\le& C (\|w^\kappa\|_{H^s} + \kappa \|\partial_x(v^\kappa-v_\kappa)\|_{H^s} + \kappa \|\partial_x v_\kappa\|_{H^s} )\|\partial^l_x E\|_{L^2}^2 \nonumber\\
      &\le& C (\|w^\kappa-w_\kappa\|_{H^s}+\|w_\kappa\|_{H^s}+\kappa\|v_\kappa\|_{H^{s+1}} + \kappa \|\partial_x(v^\kappa-v_\kappa)\|_{H^s}) \|\partial^l_x E\|_{L^2}^2 \nonumber\\
      &\le& C (1+\|E\|_{H^s} + \kappa \|\partial_x (v^\kappa-v_\kappa)\|_{H^s}) \|E\|_{H^s}^2. \label{516}
\end{eqnarray*}
第三项$I_3$的处理需要采用分部积分。
\begin{eqnarray*}
  I_3 &=& 8 \kappa ([\partial^l_x,\frac{1}{\rho^\kappa}] \partial^2_x(v^\kappa-v_\kappa) + \frac{1}{\rho^\kappa} \partial^{l+2}_x(v^\kappa-v_\kappa),\partial^l_x (v^\kappa-v_\kappa) )\\
  &=& 8 \kappa ([\partial^l_x,\frac{1}{\rho^\kappa}] \partial^2_x(v^\kappa-v_\kappa),\partial^l_x (v^\kappa-v_\kappa))  -8 \kappa(\partial_x (\frac{1}{\rho^\kappa}) \partial^l_x(v^\kappa-v_\kappa),\partial^{l+1}_x (v^\kappa-v_\kappa)) \\ && -8\kappa ( \frac{1}{\rho^\kappa} \partial^{l+1}_x(v^\kappa-v_\kappa), \partial^{l+1}_x (v^\kappa-v_\kappa)) \\
  &\le& 8\kappa \|[\partial^l_x,\frac{1}{\rho^\kappa}]\partial^2_x(v^\kappa-v_\kappa)\|_{L^2} \|\partial^l_x (v^\kappa-v_\kappa)\|_{L^2} \\ & &+8\kappa |\partial_x \frac{1}{\rho^\kappa}|_{L^\infty} \|\partial^l_x (v^\kappa-v_\kappa)\|_{L^2} \|\partial^{l+1}_x(v^\kappa-v_\kappa)\|_{L^2}  -C_0 \kappa \|\partial^{l+1}_x (v^\kappa-v_\kappa)\|_{L^2}^2\\
  &\le& C\kappa \|\partial^l_x (v^\kappa-v_\kappa)\|_{L^2} (|\partial_x \frac{1}{\rho^\kappa}|_{L^\infty} \|\partial_x^{s-1} \partial^2_x(v^\kappa-v_\kappa)\|_{L^2} + |\partial^2_x (v^\kappa-v_\kappa)|_{L^\infty} \|\partial^s_x \frac{1}{\rho^\kappa}\|_{L^2})  \\ & & + C\kappa \|\rho^\kappa\|_{H^s} \|v^\kappa-v_\kappa\|_{H^s}\|\partial_x (v^\kappa-v_\kappa)\|_{H^s} -C_0 \kappa\|\partial^{l+1}_x (v^\kappa-v_\kappa)\|_{L^2}^2\\
  &\le& C\kappa (1+ \|E\|_{H^s})\|E\|_{H^s} \|\partial_x (v^\kappa-v_\kappa)\|_{H^s}-C_0 \kappa \|\partial^{l+1}_x (v^\kappa-v_\kappa)\|_{L^2}^2 ,
\end{eqnarray*}
其中$C_0 =\frac{ 8}{{\max_{x,t,\kappa} \{\rho^\kappa(x,t) \}}}$.

结合\eqref{58}和上面$I_k$的估计,我们得到
\begin{eqnarray*}
 && \sum_{l=0}^s  (a_0(w^\kappa) \partial^l_x E,\partial^l_x E)_t + C_0 \kappa \|\partial_x (v^\kappa-v_\kappa)\|_{H^s}^2 \\
 &\le& C (1+\|E\|_{H^s})\|E\|_{H^s}^2 + C \kappa(1+\|E\|_{H^s})\|E\|_{H^s} \|\partial_x(v^\kappa-v_\kappa)\|_{H^s} + C \kappa^4  \\
 &\le& C (1+\|E\|_{H^s})\|E\|_{H^s}^2 + \frac{C_0\kappa}{2} \|\partial_x(v^\kappa-v_\kappa)\|_{H^s}^2 + C \kappa (1+\|E\|_{H^s})^2 \|E\|_{H^s}^2 + C \kappa^4.
\end{eqnarray*}
从而
\begin{eqnarray*}
  \sum_{l=0}^s  (a_0(w^\kappa) \partial^l_x E,\partial^l_x E)_t + \frac{C_0}{2} \kappa \|\partial_x (v^\kappa-v_\kappa)\|_{H^s}^2 \le   C (1+\|E\|_{H^s}^2)\|E\|_{H^s}^2 + C \kappa^4.
\end{eqnarray*}
回忆$a_0(w^\kappa)$是正定的。我们将上述不等式在$t \in [0,T],\ T \le \min \{T_*,T^\kappa\}$上积分得到
\begin{eqnarray} \label{59}
 && \|E(T)\|_{H^s}^2 + \kappa \int_0^T \|\partial_x(v^\kappa(t)-v_\kappa(t))\|_{H^s}^2 dt \nonumber \\
& \le &  C\int_0^T (1+\|E(t)\|_{H^s}^2)\|E(t)\|_{H^s}^2 dt +CT_* \kappa^4.
\end{eqnarray}
这里我们用到了$E(x, 0)=0$。

令
\begin{eqnarray*}
  \phi(T) =  \int_0^T (1+\|E(t)\|_{H^s}^2)\|E(t)\|_{H^s}^2 dt + T_* \kappa^4.
\end{eqnarray*}
由\eqref{59}得到
\begin{eqnarray*}
  \phi' \le C  \phi(1+C\phi)
\end{eqnarray*}
或
\begin{eqnarray*}
  -(\frac{1}{\phi})_t \le -C (-\frac{1}{\phi})+C^2.
\end{eqnarray*}
采用Gronwall不等式我们得到
\begin{eqnarray*}
  \phi(T) \le \frac{e^{CT} \phi_0}{1-C e^{CT}  \phi_0}\le\frac{e^{CT_*} \phi_0}{1-C e^{CT_*} \phi_0} .
\end{eqnarray*}
选取充分小的$\kappa$以使$C e^{CT_*} \phi_0 \le \frac{1}{2}$,我们最终得到
\begin{eqnarray*}
  \phi(T) \le 2T_*e^{CT_*}\kappa^4.
\end{eqnarray*}
于是
\begin{eqnarray*}
  \|E(T)\|_{H^s} \le C(T_*) \kappa^2
\end{eqnarray*}
对$T \in [0,\min\{T_*,T^\kappa\})$成立。我们证明了引理\ref{lemmaCE},从而定理\ref{theoremCE}成立。
\end{proof}

% \bibliography{ref}
% \bibliographystyle{unsrt}

 % \end{document}







\section{本章小结}
本章通过推广守恒-耗散理论将含有客观导数的非线性粘弹性流体模型纳入了守恒-耗散理论的框架中。利用推广的守恒-耗散理论我们推广了上对流Maxwell模型和FENE-P模型,包含了温度和压缩性的影响。另外,我们还导出了非等温可压上对流Maxwell模型(模型\eqref{eq:ECDFsecond})。最后利用Yong的双曲平衡率方程组小解的整体存在性理论和双曲方程松弛极限理论证明了一维等温可压上对流Maxwell模型光滑解在平衡态附近的整体存在性(定理\ref{theoremglobal})以及当松弛参数$\kappa$趋于$0$时与Navier-Stokes方程组的兼容性(定理\ref{theoremCE})。
%应力张量的方程由于满足客观性原理不可以写成守恒形式,然而物理上的守恒律和熵增原理仍然成立。通过假设熵函数的存在性而放松对方程守恒形式的要求,推广的守恒-耗散理论很好地纳入了客观导数。本章利用推广的守恒-耗散理论推广了上对流Maxwell模型\eqref{eq:generalizedUCM}和FENE-P模型\eqref{eq:generalizedFENEP},这些模型包含了温度和压缩性的影响。另外,我们还利用守恒-耗散理论导出了类似\"Ottinger的一个模型——粘弹性流体力学第二模型\eqref{eq:ECDFsecond}。这一模型有很好的Hamilton结构。另外我们证明了其在一维时的光滑解在平衡态附近的整体存在性以及严格分析了在右端松弛参数趋于$0$时与Navier-Stokes方程组的一致性。


 % \bibliography{ref}
 % \bibliographystyle{unsrt}

% \end{document}