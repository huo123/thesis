% \chapter{非线性粘弹性流体的守恒-耗散理论}
 \documentclass{article}
 \usepackage{ctex}
 \usepackage{amsmath}
 \usepackage{amsthm}
 \newtheorem{theorem}{定理}

\begin{document}
上一章我们讨论了线性粘弹性流体力学的模型。守恒-耗散理论可以很好地用来建立线性粘弹性流体力学模型。然而,粘弹性流体的建模中存在一个重要法则——客观性原理,线性粘弹性模型由于违背这一原理所以无法很好地描述粘弹性流体的行为\cite{}。为了在模型中考虑这一原理,我们需要包含应力张量的客观导数。而由于这一导数并不是守恒形式的,所以守恒-耗散理论的假设并不成立。

在本章中,我们将首先讨论客观性原理,然后推广守恒-耗散理论以包含客观导数。并利用守恒-耗散理论发展了非线性粘弹性流体力学模型。最后我们给出了这些模型的一些数学分析结果。

\subsection{客观性原理与粘弹性流体模型}
1949年J. G. Oldroyd发表了题为《流变学状态方程的构建》一文\cite{}。这篇文章提出了一个重要的概念-客观性原理(Principle of Objectivity),或称为物质坐标不变性原理(Material Indifference Principle)。

\end{document}