% \chapter{非线性粘弹性流体的守恒-耗散理论}
 \documentclass{article}
 \usepackage{ctex}
 \usepackage{amsmath}
 \usepackage{amsthm}
 \newtheorem{theorem}{定理}

\begin{document}
上一章我们讨论了线性粘弹性流体力学的模型。守恒-耗散理论可以很好地用来建立线性粘弹性流体力学模型。然而,粘弹性流体的建模中存在一个重要法则——客观性原理,线性粘弹性模型由于违背这一原理所以无法很好地描述粘弹性流体的行为\cite{}。为了在模型中考虑这一原理,我们需要包含应力张量的客观导数。而由于这一导数并不是守恒形式的,所以守恒-耗散理论的假设并不成立。

在本章中,我们将首先讨论客观性原理,然后推广守恒-耗散理论以包含客观导数。并利用守恒-耗散理论发展了非线性粘弹性流体力学模型。最后我们给出了这些模型的一些数学分析结果。

\subsection{客观性原理与粘弹性流体模型}
1949年J. G. Oldroyd发表了题为《流变学状态方程的构建》一文\cite{}。这篇文章提出了一个重要的概念-客观性原理(Principle of Objectivity),或称为物质坐标不变性原理(Material Indifference Principle)。根据这一原理,如果物质附着的随体坐标系的力学描述在转换到固定坐标系时需要满足客观性原理。如果我们假设连续介质的物质坐标系(Lagrange坐标系)为$X$。运动到时间$t$时的空间坐标系(Euler坐标系)为$x(t)$。物质的运动方程的可以采用下面的函数来描述
\begin{equation*}
	x = x (t,X).
\end{equation*}
假设物质的速度为$v = v (t,X)$。那么速度的向量场相伴着下面的自治微分方程组
\begin{equation*}
	\frac{d}{dt} x(t) = v (x(t)).
\end{equation*}
其以$x|_{t=0}=X$为初始条件的积分曲线(解)可以记做
\begin{equation*}
	x = \mathcal{F}_t (X) = x (t,X).
\end{equation*}
这样我们可以建立微分同胚的单参数局部群和速度向量场的对应关系。即采用上面的方法可以定义局部单参数微分同胚群$\mathcal{F}_t$。且如果有$\mathcal{F}_t$我们可以采用下面的公式求得速度向量场
\begin{equation*}
	v = \frac{d}{dt} \mathcal{F}_t \big|_{t=0}。
\end{equation*}
根据这一定义,我们可以定义张量在Lagrange坐标系$X$和Euler坐标系$x$之间的转换关系。定义在$x$上的张量$T^{i_1,\cdots,i_p}_{j_1, \cdots,j_q}$转移到$X$点的坐标表示为
\begin{equation*}
	(\mathcal{F}_t T)^{i_1,\cdots,i_p}_{j_1, \cdots,j_q} = T^{l_1,\cdots,l_p}_{k_1, \cdots,k_q} \frac{\partial x^{k_1}}{\partial x_0^{j_1}} \cdots \frac{\partial x^{k_q}}{\partial x_0^{j_q}} \frac{\partial x_0^{i_1}}{\partial x^{l_1}} \cdots \frac{\partial x_0^{i_p}}{\partial x^{l_p}},  
\end{equation*}
其中的上下标分别表示协变坐标和共变坐标。从而可以定义张量$T$沿速度场的李导数定义为
\begin{equation*}
	{\mathcal{L}_v T^{i_1 \cdots i_p}_{j_1 \cdots j_q} =\frac{d}{dt} (\mathcal{F}_t T)} |_{t=0} 
\end{equation*}


连续介质的Cauchy应力张量的时间演化需要采用客观导数来描述。


\end{document}