 \chapter{非线性粘弹性流体的守恒-耗散理论}
%   \documentclass{article}
% \usepackage{ctex}
% \usepackage{amsfonts}
% \usepackage{amsmath}
% \usepackage{amsthm}
% \usepackage{cite}
% %\usepackage{showkeys}
% \theoremstyle{plain}
%  \newtheorem{theorem}{定理}
%  \newtheorem{remark}{注释}
%  \newtheorem{thm}{Theorem}
%  \newtheorem{lemma}{Lemma}
%  \theoremstyle{definition}
%  \theoremstyle{Remark}
% \newtheorem{rem}{Remark}
% \newtheorem{defn}{Definition}

% \begin{document}
上一章我们利用守恒-耗散理论讨论了线性粘弹性流体力学的模型。然而,线性粘弹性模型由于违背粘弹性流体建模的一个重要法则——客观性原理,所以无法很好地描述粘弹性流体的行为\cite{oldroyd1950formulation,dimitrienko2010nonlinear,edwards1990remarks}。为了在模型中考虑这一原理,我们需要包含张量的客观导数。但是由于这一导数并不是守恒形式的,所以守恒-耗散理论对方程守恒形式的假设并不成立。
% 为了发展守恒-耗散理论以纳入包含客观导数的非线性粘弹性流体力学模型,
本章通过放松对方程守恒形式的要求而推广守恒-耗散理论,以对非线性粘弹性进行建模。

第一节我们将首先讨论客观性原理。第二节提出推广的守恒-耗散理论。第三节利用推广的守恒-耗散理论发展了非线性粘弹性流体力学模型,推广了不可压上对流Maxwell方程和FENE-P模型,并提出了非等温可压上对流Maxwell模型。最后我们利用Yong的双曲平衡率方程组小解整体存在性理论和双曲方程松弛极限理论分析了一维等温可压上对流Maxwell方程的数学性质。

\section{粘弹性流体建模的客观性原理}
1949年J. G. Oldroyd发表了题为《流变学状态方程的构建》(On the formulation of rheological equations of state)一文\cite{oldroyd1950formulation}。这篇文章提出了一个重要的概念——客观性原理(Principle of Objectivity),或称为物质坐标不变性原理(Material Indifference Principle)。根据这一原理,描述物质运动的方程不依赖于坐标系的选取。
%物质附着的随体坐标系的运动方程在转换到固定参考坐标系时需要满足客观性原理。
如果我们假设连续介质的物质坐标系(Lagrange坐标系)为$X$,运动到时间$t$时的空间坐标系(Euler坐标系)为$x$,则物质的运动方程的可以采用下面的函数来描述
\begin{equation*}
	x = x (t,X).
\end{equation*}
假设物质的速度为$v = v (t,X)$,那么速度的向量场相伴着下面的自治微分方程
\begin{equation*}
	\frac{d}{dt} x(t) = v (x(t)).
\end{equation*}
其以$x|_{t=0}=X$为初始条件的积分曲线(解)可以记做
\begin{equation*}
	x = \mathcal{F}_t (X) = x (t,X).
\end{equation*}
这样我们可以建立微分同胚的单参数局部群和速度向量场的对应关系。即采用上面的方法可以定义局部单参数微分同胚群$\mathcal{F}_t$,且如果有$\mathcal{F}_t$我们可以采用下面的公式求得速度向量场
\begin{equation*}
	v = \frac{d}{dt} \mathcal{F}_t \big|_{t=0}.
\end{equation*}
根据这一定义,我们可以定义张量在Lagrange坐标系$X$和Euler坐标系$x$之间的转换关系。定义在$x$上的张量$T^{i_1,\cdots,i_p}_{j_1, \cdots,j_q}$转移到$X$上的坐标表示为
\begin{equation*}
	(\mathcal{F}_t T)^{i_1,\cdots,i_p}_{j_1, \cdots,j_q} = T^{l_1,\cdots,l_p}_{k_1, \cdots,k_q} \frac{\partial x^{k_1}}{\partial x_0^{j_1}} \cdots \frac{\partial x^{k_q}}{\partial x_0^{j_q}} \frac{\partial x_0^{i_1}}{\partial x^{l_1}} \cdots \frac{\partial x_0^{i_p}}{\partial x^{l_p}},  
\end{equation*}
其中的上下标分别表示协变坐标和共变坐标。张量$T$沿速度场的李导数定义为
\begin{equation*}
	{\mathcal{L}_v T^{i_1 \cdots i_p}_{j_1 \cdots j_q} =\frac{d}{dt} (\mathcal{F}_t T)} |_{t=0}.
\end{equation*}
李导数给出了张量$T$沿速度向量场$v$的变化。通过计算我们可以得到上面的李导数的坐标表示如下
\begin{eqnarray} \label{eq:Lie}
	 {\mathcal{L}_v T^{i_1 \cdots i_p}_{j_1 \cdots j_q} = \frac{d}{dt} (\mathcal{F}_t T)} |_{t=0}  = && v^s \frac{T^{i_1 \cdots i_p}_{j_1\cdots j_q}}{\partial x^s} + T^{i_1 \cdots i_p}_{k j_2 \cdots j_q}\frac{\partial v^k}{\partial x^{j_1}} + \cdots + T^{i_1 \cdots i_p}_{j_1\cdots j_{q-1}k}\frac{\partial v^k}{\partial x^{j_q}} \nonumber
	\\ && - T^{li_2 \cdots i_p}_{j_1\cdots j_q}\frac{\partial v^{i_1}}{\partial x^l}- \cdots - T^{i_1 \cdots i_{p-1}l}_{j_1\cdots j_q}\frac{\partial v^{i_p}}{\partial x^l}. 
\end{eqnarray}
在连续介质力学中我们还要考虑应力张量$T$本身随时间的变化,所以通常采用的是全导数,定义为
\begin{equation*}
	\frac{\mathcal{D} T}{\mathcal{D}t} = \frac{\partial T}{\partial t} + \mathcal{L}_v T. 
\end{equation*}
这样我们就得到了张量演化的全导数,这一导数很好地描述了速度向量场$v$对应的单参数微分同胚群$\mathcal{F}_t$引起的空间形变\cite{dubrovinmodern}。

J. G. Oldroyd定义的导数实际上就是张量场的李导数。李导数的定义是同坐标系的选取无关的,不同坐标系下张量坐标的转换满足张量坐标变换的一般原理。如果采用共变基底描述Cauchy应力张量$\sigma$,那么应力可以写作
\begin{equation*}{}
	\sigma^{ij} e_i \otimes e_j.
\end{equation*}
从而根据\eqref{eq:Lie},我们可以得到
\begin{equation}
	\frac{\mathcal{D} \sigma}{\mathcal{D}t} = \partial_t \sigma + v^s \frac{\partial \sigma^{ij}}{\partial x^s} - \sigma^{lj} \frac{\partial v_i}{\partial x_l} -  \sigma^{il} \frac{\partial v_j}{\partial x_l}. 
\end{equation}
该导数即为Oldroyd论文中定义的右不变导数,又称作上对流Maxwell导数,记做$\stackrel{\nabla} \sigma$。如果采用反变坐标描述,则应力张量为$\sigma_{ij} e^i \otimes e^j$,其李导数为
\begin{equation}
	\frac{\mathcal{D} \sigma}{\mathcal{D}t} = \partial_t \sigma + v^s \frac{\partial \sigma_{ij}}{\partial x^s} + \sigma_{kj} \frac{\partial v_k}{\partial x_i} +  \sigma_{ik} \frac{\partial v_k}{\partial x_j}. 
\end{equation}
我们称这一导数为下对流Maxwell导数(或称作Cotter-Rivlin导数),记做$\stackrel{\Delta} \sigma$\cite{oldroyd1950formulation,dimitrienko2010nonlinear}。当然也可以采用混合基底来描述应力张量$\sigma$,从而可以定义混合导数。在连续介质力学中还存在其他客观导数,如Jaumann导数等。由于大部分粘弹性流体模型采用上对流Maxwell导数,我们在本章中仅考虑含上对流导数的模型,对于含其他导数的模型可以采用类似的方法得到。

另外,J. G. Oldroyd定义的张量变换关系考虑了空间压缩性的影响,此时张量的变换关系变为
\begin{equation*}
	(\mathcal{F}_t T)^{i_1,\cdots,i_p}_{j_1, \cdots,j_q} = {\det}^\gamma({\frac{\partial X}{\partial x}}) T^{l_1,\cdots,l_p}_{k_1, \cdots,k_q} \frac{\partial x^{k_1}}{\partial x_0^{j_1}} \cdots \frac{\partial x^{k_q}}{\partial x_0^{j_q}} \frac{\partial x_0^{i_1}}{\partial x^{l_1}} \cdots \frac{\partial x_0^{i_p}}{\partial x^{l_p}},  
\end{equation*}
% 定义形变张量
% \begin{equation*}
% 	F = \frac{\partial x}{\partial X}.
% \end{equation*}
从而我们可以得到$T$的全导数(考虑压缩性的导数记为$\frac{\mathcal{d}}{\mathcal{d} t}$,以上对流Maxwell导数为例)为
\begin{equation*}
	\frac{\mathcal{d} T}{\mathcal{d} t} =  T_t +v \cdot \nabla T + \gamma T \nabla \cdot v - (\nabla v) T - T(\nabla v)^T.   
\end{equation*}

由于坐标系的形变对张量的演化产生影响,所以张量的客观导数一般不是守恒形式的。这也意味着守恒-耗散理论对方程形式的假设不适用于张量。对于第二章考虑的线性粘弹性流体力学模型实际上不满足客观性原理,从而应用受到了很大的局限。通过直接将张量的导数$\partial_t \sigma  + v \cdot \nabla \sigma$写成客观导数的形式,可以推广线性Maxwell模型,例如下面的不可压粘弹性流体的上对流Maxwell模型就是通过推广线性的Maxwell模型得出的\cite{oldroyd1950formulation}:
\begin{subequations} \label{eq:UCM}
	\begin{align}
		\nabla \cdot v &= 0, \\
		v_t + v \cdot \nabla v + \nabla p  &= \nabla \cdot \sigma, \\
		\stackrel{\nabla} \sigma - \frac{2\eta_p}{\lambda} D &= - \frac{\sigma}{\lambda}
	\end{align}
\end{subequations}
其中$\eta_p$为高分子引起的粘性系数。
另外通过微观的方法推导出的粘弹性模型也经常包含客观导数,例如Giesekus模型、Phan-Thien-Tanner模型和FENE-P模型等。这些模型均具有下面的形式
\begin{equation*}
	\lambda \stackrel{\nabla} \tau  -2 \eta_p D = - \mathcal{G}(\tau,D),
\end{equation*}
其中$\mathcal{G}$的表达式为\cite{le2009multiscale}:
\begin{itemize}
\item Giesekus model:$\mathcal{G} = \tau_p + \alpha \frac{\lambda}{\eta_p} \tau_p\tau_p$。
\item PTT model:$\mathcal{G} = e^{\phi \lambda \frac{\mbox{Tr}(\tau_p)}{\eta_p}}\tau_p + \frac{\xi}{2} \lambda (D \tau_p + \tau_p D)$。
\item FENE-P model:$\mathcal{G}  = Z(\mbox{Tr}(\tau_p))\tau_p  -\lambda (\tau_p + \frac{\eta_p}{\lambda} I)\left( (\partial_t + v \cdot \nabla) \ln \mbox{Tr}(\tau_p) \right), Z(\mbox{Tr}(\tau_p)) = 1 + \frac{d}{b} (1+ \lambda \frac{\mbox{Tr}{(\tau_p)}}{d \eta_p})$。
\end{itemize}
% 另外,对于高分子熔融液,需要考虑高分子纠缠效应,例如Doi-Edwards模型的一个简单形式(Larson近似)
% \begin{equation}
% 	\stackrel{\nabla} \sigma + \frac{2}{3G} D:\sigma \sigma + \frac{1}{\lambda}(\sigma - G I) = 0
% \end{equation}
% 也包含上对流Maxwell导数\cite{larson2015modeling}。

上面提到的模型均为不可压模型,且没有考虑温度的影响。而由于张量客观导数的非守恒性,应力张量方程的形式不是守恒的,从而不满足守恒-耗散理论耗散变量演化方程为守恒形式的假设。为了克服这些困难,我们将推广守恒-耗散理论以纳入含客观导数的非线性粘弹性流体模型。

%see http://imechanica.org/files/Cauchy_Sress_Tensor.pdf

\section{推广的守恒-耗散理论}
我们仍然认为热力学过程可以采用守恒过程和耗散过程来描述。与经典守恒-耗散理论不同的是,我们在这里不再假设耗散变量$U_d$满足守恒形式的方程。我们仍然假设守恒变量$U_c$满足
\begin{equation*}
	\partial_t U_c + \sum_{j=1}^n \partial_{x_j} f_j(U) = 0.
\end{equation*}
耗散变量$U_d$的方程假设有形式(其中$\frac{\mathcal{D} }{\mathcal{D} t}$为客观导数)
\begin{eqnarray*}
	\frac{\mathcal{D} U_d}{\mathcal{D} t} + \sum_{j=1}^n B_{j}(U)U_{x_j} = \mathcal{q}(U).
\end{eqnarray*}
为使热力学第二定律成立,我们仍然假设熵函数的存在性。然而由于耗散变量$U_d$方程的非守恒性质,我们不能假设$\eta_{UU}F_{jU} = 0$,而假设熵函数满足类似经典守恒-耗散理论中熵函数的方程。对于源项,我们采用同样的方式描述。推广的守恒-耗散理论的基本假设如下:
\begin{enumerate}
		\item 存在严格上凸函数$\eta = \eta (U)$,称$\eta$为系统的熵函数,使得存在$J=J(U),\Delta = \Delta(U)$,成立
		\begin{equation}
			\eta_t + \nabla \cdot J = \Delta.
		\end{equation}
		\item 存在正定矩阵$M = M(U)$,称为耗散矩阵,使得$\mathcal{q}(U) = M \eta_{U_d}$。
	\end{enumerate}
	由这两条假设我们得到
	\begin{equation*}
		\Delta = \eta_{U_d}^T M \eta_{U_d} \ge 0,
	\end{equation*}
从而熵增大于0,即热力学第二定律成立。推广的守恒-耗散理论仅仅通过改变耗散变量满足的方程的形式来推广守恒-耗散理论,而守恒-耗散理论的核心假设仍然不变。守恒变量的守恒方程保证了物理守恒律,例如质量、动量、能量守恒的成立,熵的存在性保证了热力学第二定律的成立。另外,与经典守恒-耗散理论相同,我们一般假设$M$对称。

% 然而与经典守恒-耗散理论不同的是,熵函数的存在无法保证得到的方程是对称双曲组,从而其适定性需要单独考虑。另外形式上$\mathcal{Q}(U)=0$可以得到平衡态时耗散变量的值$U_d = U_{de}(U_c)$。此时可以得到对应的平衡态方程
% \begin{equation*}
% 	\partial_t U_c + \sum_{j=1}^d \partial_{x_j} f_j(U_c,U_{de}(U_c))=0.
% \end{equation*}
% 如何说明这一模型在$\mathcal{Q}$很小时对原模型的近似性也是需要单独考虑的一个问题。

% 我们期待类似第二章定理1的结论成立。needproof

% 另外,第二章提到的一样,我们可以令$\mathcal{q}=0$而得到下面的方程
% \begin{eqnarray*}
% 	\partial_t U_c + \sum_{j=1}^d f_j(U) = 0, \\
% 	\mathcal{D}_t U_d + \sum_{j=1}^d B_j(U) U_{x_j} = 0.	
% \end{eqnarray*}
% 这一模型亦可视为另一个平衡态体系的方程,对应的熵函数$\eta$满足$\eta_t + \nabla J = 0$。即我们可以认为推广的守恒-耗散理论所描述的体系是介于两个平衡态之间的。由一个平衡态到另一个平衡态的演化过程是非平衡态过程\cite{lieb2000fresh,lieb2013entropy,lieb2014entropy}。

下面我们将利用推广的守恒-耗散理论建立非线性粘弹性模型。

\section{推广的守恒-耗散理论在非线性粘弹性流体建模中的应用}
与上一章相同,我们取$U_c = \{ \rho,\rho v,\rho e\}$,满足方程\eqref{eq:fluid}。耗散变量仍取作$U_d =\{\rho w, \rho c\}$,假设熵函数$\eta =\eta(\rho,\rho v,\rho e,\rho w,\rho c)$可以表示成
\begin{equation*}
	\eta = \rho s(\nu,u,w,c).
\end{equation*}
其中$s$为比熵,$\nu = \frac{1}{\rho}$,$u$为系统的内能。根据Gibbs关系
\begin{equation*}
		\theta^{-1} = s_u, \quad \theta^{-1} p = s_{\nu}.
\end{equation*}
下面计算熵的产生率
\begin{eqnarray*}
		&&\eta_t + \nabla \cdot (\eta v) \\
		&=& \rho (s_t + v \cdot \nabla s) \\
		&=& -\nabla \cdot (\theta^{-1} q) + s_w \cdot [\rho (w_t + v \cdot \nabla w) + \nabla \theta^{-1}] \\
		&&+ (s_c:[\rho (c_t + v \cdot \nabla c - (\nabla v) c - c (\nabla v)^T) + \rho (\nabla v c + c (\nabla v)^T)] - \theta^{-1} \tau : D) \\
		&=& -\nabla \cdot (\theta^{-1} q) + s_w \cdot [\rho (w_t + v \cdot \nabla w) + \nabla \theta^{-1}] + [ s_c:\rho \stackrel{\nabla}{c} + (\rho s_c c - \frac{1}{2} \theta^{-1} \tau): 2 D]   \\
		&=& -\nabla \cdot J + \Delta.
	\end{eqnarray*}	
	这里我们假设了$q=s_w$,$c$为对称的。$w$的方程同经典的守恒-耗散理论相同。为了得到$c$的方程,有两个选择,一是令
	\begin{equation*}
		\theta^{-1} \tau = 2 \rho s_c c,
	\end{equation*}
	得到
	\begin{equation} \label{eq:ECDFgeneral1}
		\left( \begin{array}{c} 
			(\rho w)_t +  \nabla \cdot (\rho w \otimes v)  + \nabla \theta^{-1} \\
			(\rho c)_t +  \nabla \cdot (\rho c \otimes v) - (\nabla v) \rho c - (\rho c) (\nabla v)^T 
		\end{array} \right) = M \cdot
		\left( \begin{array}{c} 
			q \\ s_c
		\end{array}\right).
	\end{equation}
	我们称这样的选择得到的模型为第一类模型。
	另一个选择为令$\theta^{-1} \tau = 2 s_c + 2 \rho s_c c$,此时本构方程可以写为
	\begin{equation}
	\left( \begin{array}{c} 
			(\rho w)_t +  \nabla \cdot (\rho w \otimes v)  + \nabla \theta^{-1} \\
			(\rho c)_t +  \nabla \cdot (\rho c \otimes v) - (\nabla v) \rho c - (\rho c) (\nabla v)^T - 2 D 
		\end{array} \right) = M \cdot
		\left( \begin{array}{c} 
			q \\ s_c
		\end{array}\right).	
	\end{equation}
	我们称这样的选择得到的模型为第二类模型。

	\subsection{第一类模型}
	% 许多经典的模型都可以归为此类。
	这里我们将利用推广的守恒-耗散理论推广经典的不可压上对流Maxwell模型和FENE-P模型。
	\subsubsection{上对流Maxwell模型}
	首先我们推导上对流导数的Maxwell模型。取熵函数的形式为
	\begin{equation*}
			s = s_0(\nu,u)  - \frac{1}{2  \alpha_0} w^2 - \frac{1}{2  \alpha_1} (\mbox{Tr}(c) - \ln \det c).
	\end{equation*}
	其中$\alpha_0,\alpha_1$大于$0$,且$s_0(\nu,u)$为严格上凸函数,显然熵函数$\eta = \rho s$此时为上凸的。
	

	取$q=s_w=-\frac{ w}{\alpha_0},\tau = \frac{1}{\alpha_1} \rho \theta (I-c)$,$M$为
	\begin{equation*}
		M = \left( \begin{array}{ccc} 
			\frac{1}{\theta^2 \lambda} & 0 \\
			0 &  \frac{2 \rho^2 \theta c \otimes I}{\xi}   
		\end{array} \right).
	\end{equation*}
	我们得到
	\begin{eqnarray*}
		\alpha_0 [\partial_t q +  \nabla \cdot (q \otimes v)] - \nabla \theta^{-1} = -\frac{q}{\theta^2 \lambda}, \\
		\alpha_1[\partial_t (\theta^{-1} {\tau}) + \nabla \cdot (\theta^{-1} {\tau} \otimes v) - \nabla v \theta^{-1} \tau - \theta^{-1}\tau (\nabla v)^T] + 2 \rho D = -\frac{{\rho \tau}}{\xi}.
	\end{eqnarray*}
	
	形式上当$\alpha_0, \alpha_1$趋于0时,可以得到
	\begin{equation*}
		q = -\lambda \nabla \theta, \quad \tau = - 2 \xi {D} .
	\end{equation*}
	从而可知形式上这一模型与Fourier定律和牛顿定律一致。
	
	下面我们来推导压力$p$的表达式。设$\pi= s_{0\nu}$,由$\theta^{-1} p = s_\nu$,我们可以得到
	\begin{equation*}
		p = \theta s_\nu = \theta \pi . 
	\end{equation*}
	我们得到的最终方程为 
% \newpage
	\begin{subequations} \label{eq:genUCM}
		\begin{align}
			\rho_t + \nabla \cdot (\rho v) = 0 ,\\
			(\rho v)_t + \nabla \cdot (\rho v \otimes v) + \nabla (\theta \pi)  + \nabla \cdot ( \frac{\rho \theta(I-c)}{\alpha_1}) =0 ,\\
			(\rho e)_t + \nabla \cdot q + \nabla \cdot (P \cdot v) = 0, \\
			(\rho w)_t + \nabla \cdot (\rho w \otimes v) + \nabla \theta^{-1} = -\frac{1}{\lambda \theta^2 \alpha_0} w, \\
			(\rho c)_t +  \nabla \cdot (\rho c \otimes v) - (\nabla v) \rho c - (\rho c) (\nabla v)^T  = -\frac{2 \rho^2 \theta (c-I) }{\xi \alpha_1}.
		\end{align}
	\end{subequations}
	其中$P = pI + \tau,q =-\frac{w}{\alpha_0}$。

	在不可压缩($\rho=1$)和忽略温度的情形($\theta=1$),$\nabla \cdot v = 0$,应力$\sigma = -\tau$的演化方程为
	\begin{equation*}
		\partial_t \sigma + v \cdot \nabla \sigma - \nabla v \sigma - \sigma (\nabla v)^T - 2 D = -\frac{\sigma}{\xi}. 
	\end{equation*}
	此即上对流Maxwell粘弹性流体力学模型\eqref{eq:UCM}.	

	考虑温度的不可压Maxwell模型为
	\begin{eqnarray*}% \label{eq:Tmaxwell}
		v_t + v \cdot \nabla v + \nabla \cdot (-\theta \frac{c-I}{\alpha_1}) = 0, \\
		c_t + v \cdot \nabla c - (\nabla v) c - c (\nabla v)^T - 2 D = -\frac{1}{\xi} c.
	\end{eqnarray*}
	取$\alpha_1 = \frac{2}{\eta_p k}, T = \theta,\xi = \frac{\zeta}{2 kT \beta}$,即可得到由微观理论推出的含温度上对流Maxwell模型\eqref{eq:MicroUCM}。	实际上,上对流Maxwell模型微观上可以通过稀疏高分子溶液的Hooke弹簧模型得到\cite{larson1999structure,le2009multiscale}。弹簧受到的力$F_s$一般写作
	\begin{equation*}
		F_s = H R = 2 k_B T \beta^2 R.
	\end{equation*}
	其中$H$为弹性常数,$R$为位移。实际上可以推出$\langle RR \rangle$满足的方程同$c$一样(其中$\langle \cdot \rangle$表示对$R$的积分)\cite{larson1999structure},这样$c$实际上代表了微观量的统计量$c = \langle RR \rangle$。

	
	% 向第二章一样,如果假设熵函数的形式为
	% \begin{equation*}
	% 		s = s_0(\nu,u)  - \frac{1}{2 \nu \alpha_0} w^2 - \frac{1}{2 \nu \alpha_1} (\mbox{Tr}(c) - \ln \det c).
	% \end{equation*}
	% 其中$\alpha_0,\alpha_1$为常数。这样我们可以得到
	% \begin{equation*}
	% 	q = s_w = -\frac{\rho w}{\alpha_0} ,\quad \theta^{-1} \tau = \rho s_c c = \frac{1}{\alpha_1} \rho^2  (I-c).
	% \end{equation*}
	% 由压力$p$的定义,可以得到
	% \begin{equation*}
	% 	\theta^{-1} p = s_\nu = \pi +\frac{\rho^2}{2} w^2 + \frac{\rho}{2} c:c.
	% \end{equation*}
	% 取\begin{equation*}
	% 	M = \left( \begin{array}{ccc} 
	% 		\frac{1}{\theta^2 \lambda} & 0 \\
	% 		0 &  \frac{2 \rho^2 \theta c \otimes I}{\xi}   
	% 	\end{array} \right),
	% \end{equation*}	
	% 可以得到体系的方程为
	% \begin{subequations}\label{eq:generalizedUCM}
	% 	\begin{align}
	% 		\rho_t + \nabla \cdot (\rho v) = 0 ,\\
	% 		(\rho v)_t + \nabla \cdot (\rho v \otimes v) + \nabla (\theta \pi + \frac{\rho^2}{2} w^2 + \frac{\rho}{2} c:c)  + \nabla \cdot ( \frac{\rho^2 \theta(I-c)}{\alpha_1}) =0 ,\\
	% 		(\rho e)_t + \nabla \cdot q + \nabla \cdot (P \cdot v) = 0, \\
	% 		(\rho w)_t + \nabla \cdot (\rho w \otimes v) + \nabla \theta^{-1} = -\frac{1}{\lambda \theta^2} w, \\
	% 		(\rho c)_t +  \nabla \cdot (\rho c \otimes v) - (\nabla v) \rho c - (\rho c) (\nabla v)^T  = -\frac{\rho (c-I) }{\xi}.
	% 	\end{align}
	% \end{subequations}

	\subsubsection{FENE-P模型}
	由于粘弹性流体的弹簧模型中的Hooke弹簧是可以无限延展的,对于大分子这一假设并不成立,所以人们发展了FENE(Finitely Extensible, Nonlinear Elastic)模型。FENE模型假设弹簧受力满足
	\begin{equation*}
		F_s = \frac{2 \beta^2 k_B T}{1-(|R|/L)^2} R = H(R^2)R.
	\end{equation*}
	由于FENE模型无法写成宏观方程的形式\cite{le2009multiscale},人们将$H$对$R^2$的依赖关系改为对$R^2$的空间平均的依赖,从而得到FENE-P模型(这里$P$代表Peterlin矩封闭\cite{larson1999structure,le2009multiscale,masmoudi2011global})。FENE-P模型的应力张量的表达式如下\cite{larson1999structure}
	\begin{equation} \label{eq:FENEP}
		\sigma = \nu k_B T [ \beta^2 (1- \frac{\text{\footnotesize $\mbox{Tr}$}(S)}{2L^2})^{-1}S - I],  \quad \stackrel{\nabla} S + \frac{1}{\lambda} (\frac{S}{1-\frac{\text{\footnotesize $\mbox{Tr}$}(S)}{2L^2}} - \beta^{-2}I) = 0. 
	\end{equation}

	下面我们利用推广的守恒-耗散理论推广这一模型到可压缩、包含温度的情形。根据\cite{masmoudi2011global,hu2007new},我们取熵函数有形式
	\begin{equation*}
		s = s_0(\nu,u)  - \frac{1}{2  \alpha_0} w^2 + \frac{1}{2 \alpha_1} (a \ln \det c + b \ln(1-\frac{\text{\footnotesize $\mbox{Tr}$} (c)}{b})).
	\end{equation*}
	从而$q=s_w=-\frac{ w}{\alpha_0},\tau = \frac{1}{\alpha_1} \rho \theta (a I-\frac{c}{1-{\text{\footnotesize $\mbox{Tr}$} (c)}/{b}})$。取$M$为
	\begin{equation*}
		M = \left( \begin{array}{ccc} 
			\frac{1}{\theta^2 \lambda} & 0 \\
			0 &  \frac{2 \rho  c \otimes I}{\xi}   
		\end{array} \right).
	\end{equation*}
	我们得到
	\begin{subequations}\label{eq:generalizedFENEP}
		\begin{align}
			(\rho w)_t + \nabla \cdot (\rho w \otimes v) + \nabla \theta^{-1} = -\frac{1}{\lambda \theta^2 \alpha_0} w, \\
			(\rho c)_t +  \nabla \cdot (\rho c \otimes v) - (\nabla v) \rho c - (\rho c) (\nabla v)^T  = -\frac{\rho (\frac{c}{1-	{\text{\footnotesize $\mbox{Tr}$}(c)}/{b}}-a I) }{\xi}.
		\end{align}
	\end{subequations}
	取$\alpha_1 = \frac{\beta^2}{\nu k_B},a=\beta^{-2},b=2L^2,\xi = \lambda$,假设流体不可压($\rho=1$),那么我们可以由$c=S,\tau = -\sigma$得到\eqref{eq:FENEP},即我们得到了FENE-P模型。

	这一节我们利用了守恒-耗散理论推广了经典的不可压上对流Maxwell模型和FENE-P模型,通过本节的分析可以看出,推广的守恒-耗散理论可以很方便地对经典模型进行推广。

	\subsection{第二类模型:非等温可压上对流Maxwell模型}
	\subsubsection{模型的推导}
	下面我们将采用第二种方法推导非等温可压上对流Maxwell模型。
% 	首先选取熵函数为
% 	\begin{equation*}
% 			s = s_0(\nu,u)  - \frac{1}{2  \alpha_0} w^2 - \frac{1}{2  \alpha_1} c:c.
% 	\end{equation*}
% 	从而$q=s_w=-\frac{ w}{\alpha_0},\tau = - \frac{2}{\alpha_1}  \theta ( c + \rho c \cdot c)$。取$M$为
% 	\begin{equation*}
% 		M = \left( \begin{array}{ccc} 
% 			\frac{1}{\theta^2 \lambda} & 0 \\
% 			0 &  \frac{2 \rho^2 \theta c \otimes I}{\xi}   
% 		\end{array} \right).
% 	\end{equation*}
% 	我们得到
% \begin{eqnarray*}
% 			(\rho w)_t +  \nabla \cdot (\rho w \otimes v)  + \nabla \theta^{-1} = -\frac{1}{\lambda \theta^2}  w \\
% 			(\rho c)_t +  \nabla \cdot (\rho c \otimes v) - (\nabla v) \rho c - (\rho c) (\nabla v)^T - 2 D = - \frac{c}{\xi} .
% \end{eqnarray*}
% 从而流体的方程可以写为\begin{subequations}
% 		\begin{align*}
% 			\rho_t + \nabla \cdot (\rho v) = 0 ,\\
% 			(\rho v)_t + \nabla \cdot (\rho v \otimes v) + \nabla (\theta \pi)  - \nabla \cdot ( \frac{1}{\alpha_1}  \theta ( c + \rho c \cdot c)) =0 ,\\
% 			(\rho e)_t + \nabla \cdot q + \nabla \cdot (P \cdot v) = 0, \\
% 			(\rho w)_t + \nabla \cdot (\rho w \otimes v) + \nabla \theta^{-1} = -\frac{1}{\lambda \theta^2} w, \\
% 			(\rho c)_t +  \nabla \cdot (\rho c \otimes v) - (\nabla v) \rho c - (\rho c) (\nabla v)^T - 2 D = - \frac{c}{\xi} .
% 		\end{align*}
% 	\end{subequations}

	首先我们假设熵函数有形式(为了方便取$\alpha_0=\alpha_1=1$)
	\begin{equation*}
			s = s_0(\nu,u)  - \frac{1}{2 \nu } w^2 - \frac{1}{2  \nu } c:c.
	\end{equation*}
	从而$q=s_w=-{\rho w},\tau = - \theta (2 \rho c + 2 \rho c \cdot \rho c)$。取$M$为
	\begin{equation*}
		M = \left( \begin{array}{ccc} 
			\frac{1}{\theta^2 \lambda} & 0 \\
			0 &  (\frac{1}{\kappa}\dot{\mathcal{T}}  +  \frac{1}{\xi}\mathring{\mathcal{T}}  )
		\end{array} \right).
	\end{equation*}
	我们得到
\begin{eqnarray*}
			(\rho w)_t +  \nabla \cdot (\rho w \otimes v)  + \nabla \theta^{-1} = -\frac{1}{\lambda \theta^2}  \rho w, \\
			(\rho c)_t +  \nabla \cdot (\rho c \otimes v) - (\nabla v) \rho c - (\rho c) (\nabla v)^T - D = - \frac{\rho \dot{c}I}{\kappa} -  \frac{\rho \mathring{c}}{\xi} .
\end{eqnarray*}
假设$\pi=\pi(\nu,u) = s_{0\nu}(\nu,u)$,可以得到静压力$p$的表达式为
\begin{equation*}
			\theta^{-1} p = s_\nu = \pi +\frac{\rho^2}{2} w^2 + \frac{1}{2} \rho c: \rho c.
\end{equation*}
从而流体的方程可以写为
\begin{subequations} \label{eq:ECDFsecond}
		\begin{align}
			\rho_t + \nabla \cdot (\rho v) = 0 ,\\
			(\rho v)_t + \nabla \cdot (\rho v \otimes v) + \nabla (\theta\pi +\frac{1}{2} (\rho w)^2 + \frac{1}{2} \rho c: \rho c) \nonumber \\
			 - \nabla \cdot (\theta (2 \rho c + 2 \rho c \cdot \rho c)) =0 ,\\
			(\rho e)_t + \nabla \cdot q + \nabla \cdot (P \cdot v) = 0, \\
			(\rho w)_t + \nabla \cdot (\rho w \otimes v) + \nabla \theta^{-1} = -\frac{1}{\lambda \theta^2} w, \\
			(\rho c)_t +  \nabla \cdot (\rho c \otimes v) - (\nabla v) \rho c - (\rho c) (\nabla v)^T - 2 D = - \frac{\rho \dot{c}I}{\kappa} -  \frac{\rho \mathring{c}}{\xi}  .
		\end{align}
	\end{subequations}
	我们称这一模型为非等温可压上对流Maxwell模型。
	%本文中我们将这一模型称为粘弹性流体第二模型。我们将在后文讨论这一模型的数学性质。

	\subsubsection{与\"Ottinger模型的对比}
	上面的模型是受GENERIC启发提出的。文献\cite{ottinger2005beyond}第五章讨论了下对流导数的一个模型,我们在这里称为其\"Ottinger模型。下面我们将其中的Possion矩阵的下对流导数换为上对流导数,并选取简单的熵函数来采用GENERIC理论推导类似\eqref{eq:ECDFsecond}的模型。首先我们根据GENERIC理论选取描述系统的状态变量为$X=(\rho,\rho v, s ,w ,c)$,其中$\rho,v,s$分别代表密度、速度和熵。$w,c$与守恒-耗散理论中的选取相同,分别用来描述热传导和粘弹性流动的不可逆性。假设能量函数有下面的形式
	\begin{equation*}
		E(X) =	\int \frac{1}{2} \rho(x) v(x)^2 + \epsilon(x) dx,
	\end{equation*}
	其中$\epsilon$是内能密度。根据文献\cite{ottinger2005beyond}中5.1.1节的假设
	\begin{equation*}
		\epsilon = \epsilon (\rho,s,\mbox{Tr}(c^2),w^2) .
	\end{equation*}
	假设$\epsilon$为$\mbox{Tr}(c^2),w^2$的线性函数,即
	\begin{equation*}
		\epsilon = \epsilon_0(\rho,s) + \frac{1}{2} \phi_0 w^2  + \frac{1}{4} \phi_1 \mbox{Tr}(c^2) .
	\end{equation*}
	从而我们有
	\begin{equation*}
		\mu = \frac{\epsilon}{\rho},\quad T = \frac{\partial \epsilon}{\partial s}, \phi_0 = 2 \frac{\partial \epsilon}{\partial w}, \quad \phi_1 = 4 \frac{\partial \epsilon}{\partial \mbox{Tr}(c^2)}.
	\end{equation*}
	于是我们可以得到
	\begin{equation*}
		\frac{\delta E(X)}{ \delta X} = \left( \begin{array}{c} \mu - \frac{1}{2} v^2 \\ v \\ T \\ \phi_0 w \\ \frac{1}{2} \phi_1 c \end{array} \right).
	\end{equation*}
	Possion矩阵选为
	\begin{equation*}
		L(X) = -\left( \begin{array}{ccccc}
		0 & \frac{\partial}{\partial {x_{i'}}} \rho & 0 & 0 & 0 \\
		\rho \frac{\partial}{\partial {x_i}} & \frac{\partial}{\partial {x_{i'}}} ({\rho v_i}) + \rho v_{i'} \frac{\partial}{\partial {x_i}} & s \frac{\partial}{\partial {x_i}} & -[\frac{\partial w_m'}{\partial x_i}] + (\frac{\partial}{\partial {x_m'}} w_i) & L_{25} \\
		0 & \frac{\partial}{\partial x_{i'}} s & 0 & \frac{\partial }{\partial x_{m'}} & 0 \\
		0 & [\frac{\partial w_m}{\partial x_{i'}}] + w_{i'} \frac{\partial}{\partial x_{m}} & \frac{\partial}{\partial {x_m}} & 0 & 0  
		\end{array}\right)
	\end{equation*}
	其中
	\begin{eqnarray*}
		(L_{52})_{kl,i'} = [\frac{\partial {c_{kl}}}{\partial x_{i'}}] - (c_{ki'} + \delta_{ki'}) \frac{\partial}{\partial x_l} - ( c_{li'} + \delta_{li'}) \frac{\partial}{\partial x_{k}}, \\
		(L_{52})_{i,k'l'} = -[\frac{\partial {c_{k'l'}}}{\partial x_{i}}] - \frac{\partial}{\partial x_{k'}} (c_{il'} + \delta_{il'})  - \frac{\partial}{\partial x_{l'}} ( c_{ik'} + \delta_{ik'}).
	\end{eqnarray*}
	注意这里$[\cdot]$表示该项作为一个量,$(\cdot)$表示作为一个算子,
	可以验证$L(X)$满足Possion矩阵的性质。

	% 选取熵函数$s$为二次的。
	% \begin{equation*}
	% 	s = s_0(\rho, \epsilon) - \frac{1}{2} ( w^2 +  \mbox{Tr} (c^2) ).
	% \end{equation*}
	熵
	\begin{equation}
		S(X) = \int s(x) dx
	\end{equation}
	的导数为
	\begin{equation*}
		\frac{\delta S(X)}{ \delta X} = \left( \begin{array}{c} 0 \\ 0 \\ 1 \\  0 \\ 0 \end{array} \right).
	\end{equation*}
	这里我们假设了平衡态熵满足
	\begin{equation}
		\frac{\delta \int s_0(x) dx}{ \delta X} = 0. 
	\end{equation}
	假设耗散矩阵取值$M$如下
	\begin{equation*}
		M = \left( \begin{array}{ccccc} 
		0 & & & & \\
		& 0 & & &  \\
		& & \frac{1}{T \tau_1} \phi_0 w^2 + \frac{\phi_1 \dot{c}^2}{2 T \tau_0} + \frac{\phi_1 \mathring{c}:\mathring{c}}{2 T \tau_2} & -\frac{w}{\tau_1} & -\frac{1}{\tau_0} \dot{c} I - \frac{1}{\tau_2} \mathring{c} \\  
		& &  -\frac{w}{\tau_1} & \frac{T}{\tau_1} \phi_0^{-1} & 0  \\
	    & &  -\frac{1}{\tau_0} \dot{c} I - \frac{1}{\tau_2} \mathring{c} & 0 & \frac{2T}{\tau_0} \dot{\mathcal{T}}   + \frac{2T}{\tau_2}  \mathring{\mathcal{T}}   
		\end{array} \right).
	\end{equation*}
	我们假设参数$\tau_0,\tau_1,\tau_2,\phi_0,\phi_1$均为正数,从而$M$为半正定的。
	根据GENERIC理论,系统的方程可以写为
	\begin{equation*}
		\frac{dX}{dt} = L(X) \cdot \frac{\delta E(X)}{\delta X} + M(X) \cdot \frac{\delta S(X)}{\delta X} .
	\end{equation*}
	具体形式为
	
	\begin{subequations}\label{eq:Ottinger}
		\begin{align} 
			\partial_t \rho + \nabla \cdot (\rho v) = 0, \\
			\partial_t (\rho v) + \nabla \cdot (\rho v v ) + \nabla \cdot P + s \nabla T = 0, \\
			\partial_t s + \nabla \cdot (s v) + \phi_0  \nabla w = \frac{\phi_0 w^2}{T \tau_1} + \frac{1}{2} \frac{\phi_1 \dot{c}^2}{T \tau_0} + \frac{1}{2} \frac{\phi_1 \mathring{c}:\mathring{c}}{T \tau_2}, \\
			\partial_t w + v \cdot \nabla w + (\nabla v)^T \cdot w + \nabla T  = -\frac{w}{\tau_1}, \\
			\partial_t c + v \cdot \nabla c - (\nabla v) c - c (\nabla v)^T - (\nabla v + (\nabla v)^T) = -\frac{\dot{c}I}{\tau_0} - \frac{\mathring{c}}{\tau_1}.
		\end{align}
	\end{subequations}
	其中
	\begin{equation*}
		P = (\rho \mu + sT - \epsilon) I  - \phi_1(c^2 + c) + \phi_0 w^2 I.
	\end{equation*}
	采用\"Ottinger的理论得到的模型\eqref{eq:Ottinger}与模型\eqref{eq:ECDFsecond}的相同之处在于$w$和$c$的方程类似,且应力的表达式均出现了$c^2+c$和$w^2$。不同之处在于\eqref{eq:Ottinger}引入了熵函数作为系统的状态变量,在$w$的方程中包含了$(\nabla v)^T \cdot w$一项,并且$c$的演化方程并不包含密度$\rho$。
	%而实际上,$w$作为一个向量,其客观导数就是其Lagrange导数。$(\nabla v)^T \cdot w$一项的引入似乎没有意义。另一方面,$c$的方程中采用了上对流Maxwell导数,且在应力张量中含有$c$的项也不含有密度,
	\"Ottinger提出的模型抓住了系统的守恒律并采用耗散矩阵对不可逆过程进行了描述。但是我们认为这一模型无法很好地描述压缩性对应力张量的影响,也不符合J. G. Oldroyd提出的应力张量转换原理\cite{oldroyd1950formulation}。在耗散矩阵$M$的选取中,右下角$2\times 2$的子矩阵对于方程没有影响,但是对$M$的正定性影响很大。我们采用推广的守恒-耗散理论避免了GENERIC中对Possion矩阵、耗散矩阵选取的复杂性,并且仅仅假设耗散矩阵的正定性,抓住了不可逆过程建模最关键的两个原理:热力学第一和第二定律,并且这里得到的模型\eqref{eq:ECDFsecond}也具有类似GENERIC的耗散Hamilton结构。%下面我们将会看出,采用推广的守恒-耗散理论导出的粘弹性流体第二模型具有很好的数学结构。

%	\subsection{推广的守恒-耗散理论应用于热传导}
%	由于Cattaneo定律和热质理论均采用向量来描述热流,采用经典的守恒-耗散理论就可以很好地处理。对于推广的Guyer-Krumhansl理论\eqref{eq:EGK},我们采用了张量$Q$来描述热传导。从而我们也需要$Q$的导数满足客观性原理。

\section{一维等温可压上对流Maxwell模型数学分析}
由于推广的守恒-耗散理论破坏了方程的守恒结构,因而熵函数的存在无法保证方程的对称双曲性质。对于一般情况下由推广的守恒-耗散理论得到的模型的数学分析目前尚缺乏一般的结果。我们在本节将考虑采用第二种方法得到的非等温可压上对流Maxwell模型\eqref{eq:ECDFsecond}在等温时的情形。此情形下($\theta=1$),$s_0$只依赖于$\nu$,从而$\pi = s_{0\nu}$只依赖于密度$\rho$,此时我们得到
	\begin{subequations} \label{eq:ECDFsecondisothermal}
		\begin{align}
			\rho_t + \nabla \cdot (\rho v) = 0 ,\\
			(\rho v)_t + \nabla \cdot (\rho v \otimes v) + \nabla (\pi + \frac{1}{2} \rho c: \rho c)  - \nabla \cdot ( (2 \rho c + 2 \rho c \cdot \rho c)) =0 ,\\
			(\rho c)_t +  \nabla \cdot (\rho c \otimes v) - (\nabla v) \rho c - (\rho c) (\nabla v)^T - 2 D = - \frac{\rho \dot{c}I}{\kappa} -  \frac{\rho \mathring{c}}{\xi}  .
		\end{align}
	\end{subequations}
	在本文中,我们称这一模型为等温可压上对流Maxwell模型。下面我们将会验证这一模型的熵函数的Hessian矩阵无法对称化方程组\eqref{eq:ECDFsecondisothermal},但是对于一维的情况
% 由于推广的守恒-耗散理论破坏了方程的守恒结构,因而熵函数的存在无法保证方程的对称双曲性质。对于一般情况下由推广的守恒-耗散理论得到的模型的数学分析目前尚缺乏一般的结果。我们在本节将考虑采用第二种方法得到的等温可压上对流Maxwell模型\eqref{eq:ECDFsecondisothermal}一维情形时的数学性质。在一维时,对应的方程为
\begin{subequations} \label{eq:ECDFsecond1D}
		\begin{align}
			\rho_t + \partial_x (\rho v) = 0 ,\\
			(\rho v)_t + \partial_x (\rho v^2) + \partial_x (\pi)   -  (2+ 3 \rho c) \partial_x (  \rho c) =0 ,\\
			(\rho c)_t +  \partial_x (\rho c  v) - 2 \rho c \partial_x  v  - 2 \partial_x v = - \frac{\rho {c}}{\kappa}  .
		\end{align}
\end{subequations}
对称子可以找到,从而可以采用双曲方程的相关理论进行分析。本节我们将利用Yong的双曲平衡率方程组小解整体存在性理论和双曲方程松弛极限理论证明方程组\eqref{eq:ECDFsecond1D}平衡态附近的整体存在性和与一维可压Navier-Stokes方程组的兼容性。

下面我们首先给出其对称子,然后利用Yong的双曲平衡率方程组小解整体存在性理论\cite{yong2004entropy}证明其平衡态附近解的整体存在性,最后我们利用Yong、Yang提出的Chapman-Enskog展开的数学理论\cite{yang2015validity}证明当$\kappa$趋于$0$时方程组\eqref{eq:ECDFsecond1D}与由Maxwell迭代得到的一维Navier-Stokes方程组的兼容性。由于方程的非守恒形式,这里的分析无法直接利用\cite{yong2004entropy,yang2015validity}的结果,但其中的对方程的估计方法仍然适用。

% draft of the paper, built on Sep.29 by Huo Xiaokai, Email: hxk12@mails.tsinghua.edu.cn
% Chang on Dec. 3
% Revised on Dec. 7
% this is a combination of Oldroydnew1 and Oldroyddecay
\documentclass{article}
\usepackage{amsfonts}
\usepackage{amsmath}
\usepackage{amsthm}
\usepackage{cite}
%\usepackage{showkeys}
\theoremstyle{plain}
\begin{document}
\newtheorem{thm}{Theorem}
\newtheorem{lemma}{Lemma}
\theoremstyle{definition}
\theoremstyle{Remark}
\newtheorem{rem}{Remark}
\newtheorem{defn}{Definition}

\title{Structural stability of a 1D compressible viscoelastic fluid model}

\author{Xiaokai Huo \footnote{Zhou Pei-Yuan Center for Appl. Math., Tsinghua Univ., Beijing 100084, China; Email:
hxk12@mails.tsinghua.edu.cn},
~Wen-An Yong\footnote{Zhou Pei-Yuan Center for Appl. Math., Tsinghua Univ., Beijing 100084, China; Email:
Email: wayong@tsinghua.edu.cn}}

\date{}

\numberwithin{equation}{section}

\maketitle

\begin{abstract}

This paper is concerned with a compressible viscoelastic fluid model proposed by \"Ottinger.
%It consists of conservation laws (of mass and momentum) and a constitutive equation for a configuration tensor.
Although the model has a convex entropy, the Hessian matrix of the entropy does not symmetrize the system of first-order partial differential equations due to the non-conservative terms in the constitutive equation. We show that the corresponding 1D model is symmetrizable hyperbolic and dissipative and satisfies the Kawashima condition. Based on these, we prove the global existence of smooth solutions near equilibrium and justify the compatibility of the model with the Navier-Stokes equations.

\end{abstract}

\section{Introduction}

Complex fluids are ubiquitous in our daily life and in modern industry.
Examples include shampoo, toothpaste, silly putty, blood, polymeric solutions, emulsions, suspensions, liquid crystals, and glass-forming liquids.
When undergoing deformation, many of such materials exhibit both viscous and elastic characteristics and therefore
are often called viscoelastic fluids. The motion of such fluids is governed by conservation laws (of mass, momentum and energy) together with constitutive equations.
The constitutive equations, such as the Oldroyd-B and Upper-Convected Maxwell (UCM) models, link the stress in the conservation laws to the velocity field or strain and describe the viscoelastic characteristics. Most of the existing models are for dilute solutions or low-frequency mechanical responses, where the viscoelastic fluids can be well assumed to be incompressible. For incompressible models, many important mathematical results have been achieved. The interested reader is referred to \cite{larson1999structure,squires2009fluid,lin2012some,renardy2008mathematical} for what stated above.

%Since in viscoelastic fluids the viscous solvent is almost always incompressible, and the elastic components are also incompressible under low frequencies \cite{larson1999structure,squires2009fluid}, it is reasonable that the classical models take the incompressible assumptions on the velocity field.

However, recent studies show that the compressibility of viscoelastic fluids is non-negligible in many situations, such as high-frequency vibrations of nanoparticles immersed in viscoelastic fluids like water-glycerol mixtures \cite{yu2015compressible,galstyan2015note,chakraborty2015constitutive,pelton2009damping}, injecting/compression processes in plastic engineering \cite{kim1999numerical},
%polymeric melts,
and bubble collapses in compressible fluids \cite{lind2013bubble}.
%and on the hydrodynamics of polymeric melts \cite{housiadas2011perturbation} have attracted many interests in developing compressible models for the viscoelastic fluids. the vibrations of nanoscale mechanical devices in simple liquids.
%In these situations, the high-frequency vibration of nanoparticles and the high concentration of elastic components make the classical incompressible models invalid in describing the hydrodynamics of these fluids. In microrheology, the understanding of the compressibility effect of the elastic components on the flowing behavior is important \cite{squires2009fluid}. The compressibility effect is also vital in the other applications like understanding bubble dynamics \cite{lind2013bubble} and modeling injecting/compression process in plastic engineering \cite{kim1999numerical}.
Motivated  by these important applications, a number of compressible viscoelastic models have been developed over the past years. Most of them were derived by modifying the classical incompressible Oldroyd-B or UCM models \cite{beris2013thermodynamics,ottinger2005beyond,edwards1990remarks,belblidia2006stabilised,bollada2012mathematical,chakraborty2015constitutive,sureshkumar2004stability}. For example, that in \cite{edwards1990remarks} differs from the UCM model only by a term of the divergence of velocity field multiplying the stress. Those models often contain Oldroyd or other co-rotational derivatives to meet the principle of material indifference, which guarantees that the properties of the fluids are indifferent with the coordinate systems.
%Actually, according to \cite{bollada2012mathematical}, a widely accepted compressible model is not available.
%In \cite{bollada2012mathematical}, the models proposed in \cite{edwards1990remarks,beris2013thermodynamics,wilmanski2008continuum,belblidia2006stabilised} are discussed and a isothermal UCM like constitutive equations are derived.
%Among these models,  \cite{wilmanski2008continuum} adds a term with the contraction of stress tensor and velocity gradient tensor, and \cite{belblidia2006stabilised} decompose the stress into non-Newtonian polymeric parts and Newtonian solvent parts. In the model derived in \cite{bollada2012mathematical}, the dependence of viscosity on density is considered and a nonzero divergence of the velocity field is added in the constitutive equation.
In \cite{chakraborty2015constitutive}, the authors analyzed various compressible constitutive equations in the linear regime. They found that, except the model in \cite{yong2014newtonian}, all others do not capture the correct response in the high-frequency limit.
%One can actually apply a similar analysis on the models discussed in \cite{bollada2012mathematical} and find that they do not capture the high-frequency limit.
%with the Newtonian limit and high-frequency limit.
%Most of the models \cite{edwards1990remarks,sureshkumar2004stability,chakraborty2015constitutive} except the model proposed in \cite{yong2014newtonian} do not capture the proper response in the high-frequency limit.
But the model in \cite{yong2014newtonian} does not obey the material indifference principle.
In addition, a compressible model, using the deformation tensor instead of the stress tensor, was proposed in \cite{qian2010global,hu2011global} by generalizing the incompressible model \cite{lin2005hydrodynamics} through withdrawing the incompressibility assumption therein.
%evolution equations %in nonlinear continuum mechanics.
%intensively studied
Most of the compressible models mentioned above are isothermal and lack a thermodynamic basis.
%Unlike the thermodynamic considerations is largely avoided with the incompressibility assumption in the modeling of viscoelastic fluids, it is important in developing compressible viscoelastic fluid models. Most of the models derived from modifying the classical models \cite{edwards1990remarks,wilmanski2008continuum,belblidia2006stabilised,sureshkumar2004stability} or from the mathematical considerations \cite{qian2010global,hu2011global,yong2014newtonian}, do not take the temperature into consideration. And their consistencies with thermodynamic laws are not clear.
%It is expected that the constitutive equations should be derived from thermodynamic theories. Due to the non-conservative form of objective derivatives, the Posssion bracket formalism was not valid and a extended bracket formalism has been developed \cite{grmela1984bracket,morrison1984bracket,kaufman1984dissipative}.

On the other hand, with the GENERIC formalism for non-equilibrium thermodynamics \"Ottinger proposed a non-isothermal compressible model for complex fluids \cite{ottinger2005beyond}.
%applied the method in complex fluids \cite{grmela1997dynamicsI,ottinger1997dynamicsII,ottinger2005beyond}. It extends the Hamiltonian formalism for classical mechanics to general thermodynamic systems based on the second law of thermodynamics. Compared with the other models,
Different from the others, \"Ottinger's model uses an evolution equation for a configuration tensor and takes the stress as a function of the latter.
%The evolution equation contains and , a new configuration tensor is introduced with The Oldroyd derivative. And
It obeys both the principle of material indifference and the second law of thermodynamics.
%In a simple case without consideration of temperature,
In the isothermal case, the model reads as
\begin{subequations}\label{11}
  \begin{align}
  \partial_t \rho + \nabla \cdot (\rho \mathbf{v}) = 0, \\
  \partial_t (\rho \mathbf{v}) + \nabla \cdot (\rho \mathbf{v}\mathbf{v} + p) + \nabla \cdot \pi =0, \\
  \partial_t \mathbf{c} + \mathbf{v} \cdot \nabla \mathbf{c} + (\nabla \mathbf{v})  \mathbf{c }+ \mathbf{c}  (\nabla \mathbf{v})^T -(\nabla \mathbf{v} +(\nabla \mathbf{v})^T ) = -\frac{\mathbf{c}}{\tau}, \label{11-3}
\end{align}
\end{subequations}
where the first two equations are the standard conservation laws (of mass and momentum) and the third one is the constitutive equation for the configuration tensor $\mathbf{c}$.
In \eqref{11}, $\rho=\rho(x, t)$ is the fluid density, $\mathbf{v}$ is the velocity, $p=p(\rho)$ is the pressure, $\pi=2\mathbf{c} \cdot (\mathbf{c}-\mathbf{I})-\frac{1}{2}tr(\mathbf{c}^2)\mathbf{I}$ is the stress tensor with $\mathbf{I}$ the identity tensor, $tr(\mathbf{c}^2) = \sum_{i,j}c_{ij}c_{ji}$, $\nabla \mathbf{v}=[ \partial_{x_j} v_i]$, and $\tau$ is a relaxation time related to the viscosity of fluid. Notice that this model consists of only first-order partial differential equations and contains non-conservative terms in \eqref{11-3} due to the Oldroyd derivative.

The main goal of this paper is to study the structural stability of system \eqref{11} in one dimension:
\begin{subequations}\label{12}
  \begin{align}
  \partial_t \rho + \partial_x (\rho v) = 0 , \\
\partial_t (\rho v) + \partial_x (\rho v^2) + \partial_x p + (3c-2) \partial_x c = 0, \\
\partial_t c + v \partial_x c + (2c-2) \partial_x v =-\frac{c}{\tau}.
\end{align}
\end{subequations}
We show that this one-dimensional system is symmetrizable hyperbolic \cite{friedrichs1954symmetric} and dissipative in the sense of \cite{yong1992singular,yong1999singular} and satisfies the Kawashima condition \cite{kawashima1985systems}. With these nice properties,
we establish the global existence of small smooth solutions to the one-dimensional system and justify its compatibility with the Navier-Stokes equations.
%The global existence result is established with showing that this the compatibility with the Navier-Stokes equations is also rigorously justified.
We also show that, although the model \eqref{11} has a convex entropy, the Hessian of the entropy function does not symmetrize the system of
first-order partial differential equations due to the non-conservative terms. This is in contrast to the classical theory for hyperbolic conservation laws \cite{friedrichs1971systems} and explains why only the one-dimensional case is studied here.

Let us mention that there has been a substantial interest in the mathematical analysis of compressible viscoelastic fluids in recent years \cite{lei2006global,fang2014incompressible,hu2011global,qian2010global,hu2013global,hu2010local,qian2011initial, yong2014newtonian,guillope2010regular}. The non-conservative co-rotational derivatives
%and the inclusion of density equations
are the main difficulty. Most works are about the local and global existence of smooth solutions near equilibrium and the incompressible limit. In \cite{qian2010global,hu2011global, qian2011initial}, the authors established the local and global existence of smooth solutions in Besov spaces to the compressible model generalizing the incompressible one \cite{lin2005hydrodynamics}.
%The initial-boundary problem is investigated in \cite{qian2011initial}.
For a weakly compressible Oldroyd-B model, the local and global existence of smooth solutions were proved in \cite{guillope2010regular}.
On the other hand, the incompressible limit was justified in \cite{lei2006global,fang2014incompressible} for initial value problems of the Oldroyd-B model.
%Most of these work rewrites the equations in a hyperbolic-parabolic system with the non-conservative terms treated separately.
In addition, it was justified in \cite{yong2014newtonian} that a therein revised Maxwell model is compatible with the Navier-Stokes equations.
% by exploiting the conservation-dissipation property of the model.
%It is based on the fact the the system is a hyperbolic conservative balance system with entropy.

The paper is organized as follows. In the next section an entropy function for model (1.1) is discussed. We then show that the one-dimensional system (1.2) is symmetrizable hyperbolic and dissipative, and satisfies the Kawashima condition. The global existence result is established in Section 4. The last section is devoted to a proof of the compatibility with the Navier-Stokes equations.

%Complex fluids are prevalent in our daily life. Viscoelastic fluids cover a large content of complex fluids, including polymers, liquid crystals, electrorheological suspensions, glass-forming liquids, etc \cite{larson1999structure,lin2012some}.
%
%%The hydrodynamical and rheological properties interact in the flowing of the fluids, leading to various complex phenomena in the fluids.
%
%The motion of such fluids is governed by conservation laws of (mass, momentum and energy) and constitutive equations together \cite{oldroyd1950formulation}.
%
%Because of ?, most complex fluids are usually assumed to be incompressible. progress have been achieved in the past, both mathematically and . The interested reader is referred to for a comprehensive survey of the mathematical results.
%
%On the other hand,
%bubble dynamics
%in plastic engineering. A number of models have been developed over the past years. Most of them
%%Recently the study on the interactions between viscoelastic fluids-nanostructure interactions and the development of microrheology has attracted many interests in developing theoretical compressible models for the viscoelastic fluids \cite{yu2015compressible,galstyan2015note,chakraborty2015constitutive,pelton2009damping,squires2009fluid}. The compressibility of the fluids has a significant effects on the breathing modes of nanoparticles immersed in viscoelastic fluids like water-glycerol mixtures. In microrheology, the understanding of the compressibility effect of the elastic components on the flowing behavior is important \cite{yu2015compressible,galstyan2015note,chakraborty2015constitutive,pelton2009damping}. The compressibility effect is also vital in the other applications like understanding bubble dynamics \cite{lind2013bubble} and modeling injecting/compression process in plastic engineering \cite{kim1999numerical}. It is a challenge to develop reasonable mathematical models to explain the experimental results and to help understanding more complex phenomena.
%%Many theoretical models have been developed over the past years. In most macroscopic models, the hydrodynamic properties are described by the compressible classical Navier-Stokes equations or Euler equations with the stress determined by a constitutive equation, which is a relationship between the stress and velocity fields or strain measures and is responsible for the rheological properties \cite{larson1999structure}.
% %Most of the compressible models developed in the literatures
% are obtained by modifying the classical incompressible Oldroyd-B or Upper-Convected-Maxwell (UCM) model \cite{beris2013thermodynamics,ottinger2005beyond,edwards1990remarks,belblidia2006stabilised,bollada2012mathematical,chakraborty2015constitutive}. These models consist of Oldroyd-B derivatives or other objective derivates, which are widely used to meet the principle of material indifference to guarantee the coordinate independence of the models. However, to generalize the incompressible models to the compressible case is not trial. Actually, according to \cite{bollada2012mathematical}, a widely accepted compressible model is not available. In \cite{bollada2012mathematical}, the models proposed in \cite{edwards1990remarks,beris2013thermodynamics,wilmanski2008continuum,belblidia2006stabilised} are discussed and a isothermal UCM like constitutive equations are derived.
%
% %Among these models, \cite{edwards1990remarks} adds an additional term with divergence of velocity multiplying by the stress in the UCM model, \cite{wilmanski2008continuum} adds a term with the contraction of stress tensor and velocity gradient tensor, and \cite{belblidia2006stabilised} decompose the stress into non-Newtonian polymeric parts and Newtonian solvent parts. In the model derived in \cite{bollada2012mathematical}, the dependence of viscosity on density is considered.
%
% In \cite{chakraborty2015constitutive}, the compressible constitutive equations \cite{edwards1990remarks,sureshkumar2004stability,yong2014newtonian} are studied in the linear case with the Newtonian limit and high-frequency limit. Most of the models \cite{edwards1990remarks,sureshkumar2004stability} except the model proposed in \cite{yong2014newtonian} do not capture the proper response in the high-frequency limit. The compressional relaxation spectrum is considered in \cite{yong2014newtonian} while the others, mostly obtained by modifying the UCM or Oldroyd-B models, ignore it. However, the model in \cite{yong2014newtonian} is linear and violates the material indifference principle. In \cite{qian2010global,hu2011global}, a compressible models is proposed by generalizing the model \cite{lin2005hydrodynamics} using deformation tensor evolution equations in nonlinear continuum mechanics.
%
%Unlike the thermodynamics considerations is largely avoided with the assumption of incompressibility in the modeling of viscoelastic fluids, it is important in developing compressible viscoelastic fluid models. Most of the models derived from modifying the classical models \cite{edwards1990remarks,wilmanski2008continuum,belblidia2006stabilised,sureshkumar2004stability} or from the mathematical considerations \cite{qian2010global,hu2011global,yong2014newtonian}, do not take the temperature into consideration and their consistency with thermodynamic laws are not clear. It is expected that the constitutive equations should be derived from thermodynamic theories. Due to the non-conservative form of objective derivatives, the Posssion bracket formalism was not valid and a extended bracket formalism has been developed \cite{grmela1984bracket,morrison1984bracket,kaufman1984dissipative}. \"Ottinger and Grmela generalized it to develop the GENERIC formalism and applied the method in complex fluids \cite{grmela1997dynamicsI,ottinger1997dynamicsII,ottinger2005beyond}. It extends the Hamiltonian formalism for classical mechanics to general thermodynamic systems based on the second law of thermodynamics. Compared with the other models, a new configuration tensor is introduced with the stress as a function of it. The Oldroyd type derivatives are used in the evolution equations of the configuration tensor. And the second law of thermodynamics is satisfied with a explicit convex entropy formula. In a simple case without consideration of temperature, the model reads as
%\begin{subequations}\label{11}
%  \begin{align}
%  \partial_t \rho + \nabla \cdot (\rho \mathbf{v}) = 0, \\
%  \partial_t (\rho \mathbf{v}) + \nabla \cdot (\rho \mathbf{v}\mathbf{v} + p) + \nabla \cdot \pi =0, \\
%  \partial_t \mathbf{c} + \mathbf{v} \cdot \nabla \mathbf{c} + (\nabla \mathbf{v})  \mathbf{c }+ \mathbf{c}  (\nabla \mathbf{v})^T -(\nabla \mathbf{v} +(\nabla \mathbf{v})^T ) = -\frac{\mathbf{c}}{\tau}. \label{113}
%\end{align}
%\end{subequations}
%Here $\pi=2\mathbf{c} \cdot (\mathbf{c}-\mathbf{I})-\frac{1}{2}tr(\mathbf{c}^2)\mathbf{I}$, $\mathbf{I}$ is the identity tensor and $tr(\mathbf{c}^2) = \sum_{i,j}c_{ij}c_{ji}$, $\nabla \mathbf{v}=[ \partial_{x_j} v_i]$. And $\tau$ is a relaxation time related to the viscosity of the fluid. $\mathbf{c}$ is called configuration tensor. At equilibrium $\mathbf{c} =0$. $p$ is the pressure which depends only on the density. Notice that the first two equations are the same with Navier-Stokes equations and Euler equations in hydrodynamics. The third equation is the constitutive equation for the configuration tensor. The stress $\pi$ is a function of the configuration tensor $\mathbf{c}$.
%
%In recent years, there has been a substantial interests in the mathematical analysis of compressible viscoelatic fluids \cite{lei2006global,fang2014incompressible,hu2011global,qian2010global,hu2013global,hu2010local,qian2011initial}. The non-conservative Oldroyd type derivatives and the inclusion of density equations are the main difficulty. Most work is concerned with the local existence and global existence near equilibrium of smooth solutions. In the whole space, the local and global existence near equilibrium of smooth solutions to the generalized compressible model of the incompressible case \cite{lin2005hydrodynamics} are proved in the Besov spaces. The initial-boundary problem is investigated in \cite{qian2011initial}. The local and global existence of smooth solutions to a weakly compressible Oldroyd-B model is proved in \cite{guillope2010regular}. The incompressible limit in the whole space is discussed in \cite{lei2006global}. And the incompressible limit of Oldroyd-B fluids is studied in \cite{fang2014incompressible}. Most of these work rewrites the equations in a hyperbolic-parabolic system and deal with the non-conservative terms treated separately. In \cite{yong2014newtonian}, Yong proposed a linear Maxwell like model and studied its compatibility with Navier-Stokes equations. It is based on the fact the the system is a conservative balance system with entropy.
%
%The main goal of this paper is to establish the global existence of smooth solutions for the system \label{11} in one-dimensional case and justify the compatibility with Navier-Stokes equations. The one-dimensional version is
%\begin{subequations}\label{12}
%  \begin{align}
%  \partial_t \rho + \partial_x (\rho v) = 0 , \\
%\partial_t (\rho v) + \partial_x (\rho v^2) + \partial_x p + (3c-2) \partial_x c = 0, \\
%\partial_t c + v \partial_x c + (2c-2) \partial_x v =-\frac{c}{\tau}.
%\end{align}
%\end{subequations}
%The global existence result is establishing with showing that this one-dimensional system is symmetrizable hyperbolic \cite{majda1984compressible} and dissipative in the sense of \cite{yong1992singular,yong1999singular} and satisfies the Kawashima condition \cite{kawashima1985systems}. With these good properties, the compatibility with the Navier-Stokes equations is also rigorously justified. The theorems we prove in this paper can not be generalized to the multi-dimensional case since in higher dimensions these good properties fail except the existence of entropy.
%
%
%
%%The constitutive equations describe the rheological properties of the fluids and are responsible for the elastic effects. They are usually a relation between the stress tensor and the strain measure. One of the most important principles for the constitutive equations to obey is the principle of material indifference \cite{oldroyd1950formulation}. This principle ensures that the property of the material is indifferent with the coordinate system. Following this principle, many incompressible models have been proposed. But in many situations the fluids may become compressible \cite{edwards1990remarks,bollada2012mathematical}. According to a recent survey \cite{bollada2012mathematical}, a widely accepted compressible model is not available yet.
%
%In recent years, compressible complex fluids attract more and more attention. the study on the interactions between viscoelastic
%fluids-nanostructure interactions and the development of microrheology has attracted many interests
%in developing theoretical compressible models for the viscoelastic
%fluids [1{5]. The compressibility of the
%fluids has a significant effects on the breathing modes
%of nanoparticles immersed in viscoelastic
%fluids like water-glycerol mixtures. In
%microrheology, the understanding of the compressibility effect of the elastic components
%on the
%flowing behavior is important [1{4]. The compressibility effect is
%also vital in other applications like understanding bubble dynamics [6] and
%modeling injecting/compression process in plastic engineering [7].
%%to study the fluid-structure interaction of nanoscale mechanical devices vibrating in simple liquids.
%to study the vibrations of nanoscale mechanical devices in simple liquids.
%bubble dynamics
%in plastic engineering
%
%Jaumann, (upper (lower) convected) Oldroyd
%
%"objective derivative".
%
%In \cite{}, the authors showed that a recently revised Maxwell model is the only correct one up-to-now in the linear regime.
%
%In \cite{ottinger2005beyond}, \"Ottinger proposed the following compressible model
%\begin{subequations}\label{11}
%  \begin{align}
%  \partial_t \rho + \nabla \cdot (\rho \mathbf{v}) = 0, \\
%  \partial_t (\rho \mathbf{v}) + \nabla \cdot (\rho \mathbf{v}\mathbf{v} + p) + \nabla \cdot \pi =0, \\
%  \partial_t \mathbf{c} + \mathbf{v} \cdot \nabla \mathbf{c} + (\nabla \mathbf{v})  \mathbf{c }+ \mathbf{c}  (\nabla \mathbf{v})^T -(\nabla \mathbf{v} +(\nabla \mathbf{v})^T ) = -\frac{\mathbf{c}}{\tau}. \label{113}
%\end{align}
%\end{subequations}
%Here $\pi=2\mathbf{c} \cdot (\mathbf{c}-\mathbf{I})-\frac{1}{2}tr(\mathbf{c}^2)\mathbf{I}$, $\mathbf{I}$ is the identity tensor and $tr(\mathbf{c}^2) = \sum_{i,j}c_{ij}c_{ji}$. $\nabla \mathbf{v}=[\partial_{x_j} v_i]$.
%And $\tau$ is a relaxation time related to the viscosity of the fluid.
%$\mathbf{c}$ is called configuration tensor. At equilibrium $\mathbf{c} =0$.
%proposed by \"Ottinger \cite{ottinger2005beyond} seems to be a promising model. It is based on a theory of irreversible thermodynamics and satisfies the laws of thermodynamics, which leads to an entropy structure of the model. In addition, the material indifference principle is satisfied with the Oldroyd type derivatives used in the equation of state.
%In this paper, we focus on the following model proposed in \cite{ottinger2005beyond}:
%
% In recent years, there has been a considerable interest in the mathematical analysis of viscoelastic fluids. For a survey see \cite{lin2012some,bollada2012mathematical}. For incompressible models, Lions and Masmoudi proved the global existence of weak solutions for a Maxwell model with Janumann derivatives \cite{lions1998incompressible}. It is based on an $L^2$ estimate on the constitutive equation. The nonconservative term due to the Janumann derivative cancels when taking contraction with the stress tensor. This result could not be extended to the Oldroyd case where the nonconservative terms can not be canceled. Lin and his collaborators studied a model without damping by observing the deformation tensor has divergence zero and rewrite the equations in the conservative form with a nonlinear term in the equation of momentum \cite{lin2005hydrodynamics}. For the compressible case, Yong proposed a Maxwell type model without convected derivatives and justified the Newtonian limit of the model \cite{yong2014newtonian}. The model is conservative and includes a damping source term. However, the model is not material indifference. The introduce of material indifference will make the equations of state not conservative. The method used in \cite{yong2014newtonian}  then could not be applied here.
%
%For viscoelastic fluid models with Oldroyd type derivatives, the main difficulty for mathematical analysis comes from the nonconservative terms. The model analyzed in this paper include the Oldroyd derivatives. The lack of basic energy estimate for the Oldroyd type model is overcome by expressing the stress tensor as a function of configuration tensor. The evolution equation for the configuration tensor is of Maxwell type similiar with the Lower Convected Mawell (LCM) model for complex fluids. The basic energy estimate could thus be obtained with the help of the entropy structure of the model. However, we want to prove the global existence of smooth solutions. And it requires a $H^s, s \ge 2$ estimate. Thanks to the symmetrizable property in the one dimensional case, the global existence results can be proved by a modification of the method in \cite{kawashima2009decay}. After rewritting the one dimensional equations into a symmetric dissipative hyperbolic system, we verify the Kawashima condition of the equations, leading to the global existence theorem.
%
%In developing constitutive equations for viscoelastic fluids, compatibility should always be taken into account. This means that, when the relaxation time is short, solutions to the viscoelastic model is close to the corresponding ones of the Navier-Stokes equations. The UCM model and Oldroyd-B model are both compatible with the Navier-Stokes equations formally. However a rigorous justification is difficult due to the nonconservative convected derivative. In \cite{yong2014newtonian}, the compatibility was justified rigorously by ignoring the nonlinear terms in the convected derivative. The model we considered here has convected derivatives. The first-order approximation using the Chapman-Enskog expansion ( or the Maxwell iteration) gives a Navier-Stokes system. It is formal in 3D. But in the one dimensional case, we could treat it rigorously. The validity of the expansion is shown by modifying the method developed in \cite{yang2015validity}.
%
%
%The first two equations are the continuity equation and momentum equations. The third one is the constitutive equation for the configuration tensor $\mathbf{c}$. It describes the rheological properties. Notice that in Ottinger's model the stress tensor $\pi$ is a function of the configuration tensor $\mathbf{c}$. The nonlinear terms $ \nabla \mathbf{v} \mathbf{c}$ and $\mathbf{c} (\nabla \mathbf{v})^T $ come from the material indifference of the Oldroyd-type derivative \cite{oldroyd1950formulation}, from which the materal indifference principle holds. The model differs from the Lower Convected Maxwell model in the expression of the stress tensor. It is widely known that an energy estimate of the Lower Convected Maxwell model is not available due to the Oldroyd type derivatives \cite{saut1990existence}. Due to its entropy structure, Ottinger's model has a natural energy estimate.
%
%In this paper, we will mostly deal with the one-dimensional version of \eqref{11}:
%\begin{subequations}\label{12}
%  \begin{align}
%  \partial_t \rho + \partial_x (\rho v) = 0 , \\
%\partial_t (\rho v) + \partial_x (\rho v^2) + \partial_x p + (3c-2) \partial_x c = 0, \\
%\partial_t c + v \partial_x c + (2c-2) \partial_x v =-\frac{c}{\tau}.
%\end{align}
%\end{subequations}
%We show that this one-dimensional system is symmetrizable hyperbolic \cite{majda1984compressible} and dissipative in the sense of \cite{yong1992singular,yong1999singular} and satisfies the Kawashima condition \cite{kawashima1985systems}. With these nice properties, we prove the global-in-time existence of classical solutions and justify the compatibility with the Navier-Stokes equations.
%
%The paper is organized as follows. In the next section an entropy function for model \eqref{11} is discussed. We then show that the one-dimensional system \eqref{12} is symmetrizable hyperbolic and dissipative and satisfies the Kawashima condition. The global existence result is established in Section 4. The last section is devoted to a proof of the compatibility with the Navier-Stokes equations.

\section{Convex Entropy}
\setcounter{equation}{0}

In this section, we discuss an entropy function for the non-conservative system \eqref{11} and show that the Hessian matrix of the entropy does not symmetrize the system. Write the configuration tensor $\mathbf{c}=[c_{ij}]$ as a column vector (see \eqref{25} below for the 2D case). Then the equations \eqref{11} and \eqref{12} can be rewritten as a system of partial differential equations
\begin{eqnarray}\label{21}
  U_t + \sum_{j=1}^d A_j(U) U_{x_j} = Q(U).
\end{eqnarray}
Here $U=U(x, t)$ is the unknown function from $\mathbf{R}^d \times [0,+\infty)$ to $\mathbf{R}^n$, each $A_j(U)$ is an $n\times n$-matrix-valued function and $Q(U)$ an $n$-vector-valued function for $U\in G$ (an open set to be determined below).



\subsection{Entropy}
Notice that the model \eqref{11} is not of conservative form. We firstly generalize the entropy definition for conservation laws \cite{friedrichs1971systems} to general first-order systems of form \eqref{21}.
%And then the entropy structure of \"Ottinger's model could be derived. Based on the entropy structure, an entropy estimate could be obtained for \"Ottinger's model.

\begin{defn}
A convex function $S=S(U)$ is called an entropy for the system \eqref{21}, if there exists a $d$-vector-valued function $J=J(U)$ such that all the smooth solutions of the system \eqref{21} satisfy
$$
S_t(U) = -\nabla \cdot J(U) + \sigma^S(U)
$$
with $\sigma^S(U) \le 0$.
\end{defn}

For model \eqref{11}, we can show the following function from \cite{ottinger2005beyond,yong2014newtonian}:
\begin{eqnarray}\label{22}
  S = S(\rho,\rho \mathbf{v} ,\mathbf{c}) = \phi(\rho) + \frac{(\rho \mathbf{v})^2}{2\rho} + \frac{1}{2} tr(\mathbf{c}^2),
  \end{eqnarray}
with $\phi(\rho) = \rho \int_{\rho_0}^\rho \frac{p(z)}{z^2} dz$, is an entropy. Indeed, it is easy to check that $S(U)$ is strictly convex in $U$ provided that $\phi_{\rho \rho} = p_{\rho} / \rho \ge 0$. The latter is physically reasonable for the pressure gets high as the density increases.

With formula \eqref{22} and the equations in \eqref{11} we can compute
\begin{eqnarray}\label{23}
  S_t &=& S_{\rho} \rho_t + S_{\rho \mathbf{v}} \cdot \mathbf{v}_t + S_\mathbf{c} : \mathbf{c}_t \nonumber\\
%  =-S_{\rho} {\rho}_t \nabla \cdot (\rho \mathbf{v}) - S_{\rho \mathbf{v}} \cdot ( \nabla \cdot (\rho \mathbf{v} \mathbf{v}) + \nabla p + \nabla \cdot \pi) - S_{\mathbf{c}} : (\mathbf{v} \cdot \nabla \mathbf{c} + (\nabla \mathbf{v})^T \cdot \mathbf{c} + \mathbf{c} \cdot \nabla \mathbf{v} - (\nabla \mathbf{v} +(\nabla \mathbf{v})^T)) -\frac{\mathbf{c}:\mathbf{c}}{\tau} \\
 % =-\nabla \cdot (S\mathbf{v}+p \mathbf{v}+ \mathbf{\pi} \cdot \mathbf{v}) + (\rho S_\rho + \rho \mathbf{v}^2-p-\phi(\rho)-\frac{1}{2}\rho \mathbf{v}^2) \nabla \cdot \mathbf{v} + (\mathbf{c}\cdot 2\mathbf{c}-2\mathbf{c}-\frac{1}{2} \mathbf{c}:\mathbf{c}\mathbf{I} - \mathbf{\pi}):\nabla \mathbf{v} - \frac{1}{\tau}\mathbf{c}:\mathbf{c} \\
  &=& -\nabla \cdot (S\mathbf{v}+p \mathbf{v}+ \mathbf{\pi} \cdot \mathbf{v}) - \frac{1}{\tau}\mathbf{c}:\mathbf{c} \\
  &\equiv& -\nabla \cdot J(U) + \sigma^S(U) .\nonumber
\end{eqnarray}
%with $\mathbf{J} = S \mathbf{v} + p\mathbf{v} + \pi \cdot \mathbf{v}$ and $\sigma^S = -\frac{\mathbf{c}:\mathbf{c}}{\tau}$.
This equation indicates that the change of local entropy is caused by the entropy flux $J(U)$ and the local entropy production rate $\sigma^S(U)$. The non-positiveness of $\sigma^S(U)$ guarantees the second law of thermodynamics.
Based on the equation above, we have the following identity
\begin{multline*}
\int_\Omega \left[\phi(\rho(x,t))+\frac{1}{2}\rho(x,t)|{\mathbf{v}}(x,t)|^2+\frac{1}{2} tr(\mathbf{c}(x,t)^2)\right] dx +\int_0^t \int_\Omega \frac{tr(\mathbf{c}(x,t)^2)}{\tau} dx dt \\
= \int_\Omega \left[\phi(\rho(x,0))+\frac{1}{2}\rho(x,0)|{\mathbf{v}}(x,0)|^2+\frac{1}{2} tr(\mathbf{c}(x,0)^2)\right] dx .
\end{multline*}

%\emph{Remark 1}. The basic energy estimate does not hold in the incompressible Lower Convected Maxwell model
%\begin{eqnarray*}
%  \partial_t \mathbf{v} + \mathbf{v} \cdot  \nabla \mathbf{v} + \nabla  p= \nabla \cdot \pi, \\
%  \partial_t \pi + \mathbf{v} \cdot \nabla \pi + (\nabla \mathbf{v})\pi + \pi(\nabla \mathbf{v})^T -(\nabla \mathbf{v} + (\nabla \mathbf{v})^T) = -\frac{\pi}{\tau}.
%\end{eqnarray*}
%Taking contraction with $\pi$ in both sides of the third equation, the additional term $2\mathbf{\pi}\cdot{\pi} : \nabla v$ could not be canceled with the second equation multipling $\mathbf{v}$. In the model \eqref{11}, the nonconservative term could be canceled due to the formula of $\pi$.
%
%\emph{Remark 2}. When $\tau$ is small, $\mathbf{c} \approx \tau (\nabla \mathbf{v}+(\nabla \mathbf{v})^T)$ and $\pi \approx -2\tau(\nabla \mathbf{v}+(\nabla \mathbf{v})^T)$, the approximate equations will be
%\begin{eqnarray*}
%  \partial_t \rho + \nabla \cdot (\rho \mathbf{v}) = 0, \\
%  \partial_t (\rho \mathbf{v}) + \nabla \cdot (\rho \mathbf{v}\mathbf{v} + p) = 4\tau \Delta \mathbf{v}.
%\end{eqnarray*}
%which is the Navier-Stokes system for the Newtonian fluid. So the model \eqref{11} is consistency with the Newtonian fluid model. We will treat it rigorously for the one dimensional case later in the paper.
%%Remark. $\pi$ have the same evolution of the maxwell model?


\subsection{Non-symmetrizable hyperbolicity}
For systems of conservation laws with a convex entropy, it is well-known that the Hessian matrix of the entropy symmetrizes the system in the sense of Friedrichs \cite{friedrichs1971systems}. Here we show that the Hessian matrix is not a symmetrizer for the nonconservative system \eqref{11} in the 1D and 2D cases.

For the one-dimensional case ($d=1$), we have
\begin{eqnarray}\label{24}
U=(\rho, \rho v, c)^T, \qquad Q(U) = \frac{1}{\tau}(0, 0, -c)^T, \nonumber\\
A(U) \equiv A_1(U) = \left( \begin{array}{ccc}
	0 & 1 & 0 \\[2mm]
	p_\rho-v^2 & 2v & 3c-2 \\[2mm]
	-\frac{(2c-2)v}{\rho} & \frac{2c-2}{\rho} & v
	\end{array} \right).
\end{eqnarray}
Compute
\begin{eqnarray*}
  S_{UU}(U) = & \frac{1}{\rho} \left( \begin{array}{ccc} p_{\rho} + v^2 & -v & 0\\[2mm]
    -v & 1 & 0 \\[2mm]
      0 & 0 & \rho \end{array} \right) , \\[4mm]
    S_{UU} A(U) = & \frac{1}{\rho} \left( \begin{array}{ccc} -(p_\rho-v^2)v & p_\rho-v^2 & -(3c-2)v \\[2mm]
      p_\rho-v^2 & v  & 3c-2 \\[2mm]
      -(2c-2)v & 2c-2 & \rho v \end{array} \right).
\end{eqnarray*}
Obviously, the product $S_{UU} A(U)$ is not symmetric.
But in this case, we can find a symmetrizer by modifying the Hessian matrix $S_{UU}(U)$ in the next section.


In the two-dimensional case, the model can be written as
\begin{eqnarray}\label{25}
  U_t + A_1(U) U_{x_1} + A_2(U) U_{x_2} =Q(U)
\end{eqnarray}
with $U = (\rho, v_1, v_2, c_1, c_2, c_3)^T$.
Notice that $\mathbf{c}=[c_{ij}]_{2\times 2}$ is assumed to be symmetric and we take $(c_1, c_2, c_3)= (c_{11}, c_{12}, c_{22})$. The coefficient matrices $A_1(U)$ and $A_2(U)$ have expressions
\begin{eqnarray*}
  A_1 (U) = \left( \begin{array}{cccccc} 0 & 1 & 0 & 0 & 0 & 0 \\[2mm]
    p_\rho-v_1^2 & 2 v_1 & 0 & 3c_1-2 & 2c_2 & -c_3 \\[2mm]
    -v_1 v_2 &  v_2 & v_1 & 2c_2 & 2(c_1 +c_3 -1) & 2c_2 \\[2mm]
    -\frac{2c_1-2}{\rho} v_1 & \frac{2c_1-2}{\rho} & 0 & v_1 & 0 & 0 \\[2mm]
    -\frac{c_2 v_1 +(c_1-1)v_2}{\rho} & \frac{c_2}{\rho} & \frac{c_1-1}{\rho} & 0 & v_1 & 0 \\[2mm]
    -\frac{2c_2 v_2}{\rho} & 0 & \frac{2c_2}{\rho} & 0 & 0 & v_1
  \end{array} \right), \\[4mm]
  A_2(U) = \left( \begin{array}{cccccc} 0 & 0 & 1 & 0 & 0 & 0 \\[2mm]
    -v_1 v_2 & v_2 & v_1 & 2c_2 & 2(c_1 +c_3-1) & 2c_2 \\[2mm]
    p_\rho-v_2^2 & 0 & 2v_2 & -c_1 & 2c_2 & 3c_3-2 \\[2mm]
    -\frac{2c_2 v_1}{\rho} & \frac{2 c_2}{\rho} & 0 & v_2 & 0 & 0 \\[2mm]
    -\frac{c_2 v_2 +(c_3-1)v_1}{\rho} & \frac{c_3-1}{\rho} & \frac{c_2}{\rho} & 0 & v_2 & 0 \\[2mm]
   -\frac{2c_3-2}{\rho} v_2 & 0 & \frac{2c_3-2}{\rho} & 0 & 0 & v_2 \end{array} \right).
\end{eqnarray*}

Compute
\begin{eqnarray*}
  S_{UU}(U) =\frac{1}{\rho} \left( \begin{array}{cccccc} p_\rho + v_1^2+v_2^2 & -v_1 & -v_2 &0 &0 &0 \\
    -v_1 & 1 & 0  & 0 & 0 & 0  \\
    -v_2 & 0 & 1 & 0 & 0 & 0  \\
    0 & 0 & 0 & \rho & 0 & 0 \\
    0 & 0 & 0 & 0 & 2\rho & 0 \\
    0 & 0 & 0 & 0 & 0 & \rho \end{array} \right).
\end{eqnarray*}
We see that the entry at the second row and sixth column of $S_{UU}(U)A_1(U)$ is $-c_3/\rho$, but that at the second column and sixth row is 0.
Meanwhile, the entry at the second row and sixth column of $S_{UU}(U) A_2(U)$ is $2c_2/\rho$, but that at the second column and sixth row is 0.
%{\small  \arraycolsep=1.4pt
%\begin{multline*}
%  S_{UU} A_1(U) =
%  \frac{1}{\rho} \cdot \\ \left( \begin{array}{cccccc} -(p_\rho -v_x^2-v_y^2)v_x & p_\rho-v_x^2 & -v_x v_y & (2-3c_1)v_x-2c_2 v_y & -2c_2 v_x -2(c_1+c_3-1)v_y & c_3 v_x-2c_2 v_y \\
%  p_\rho-v_x^2 & vx & 0 & 3c_1-2 & 2c_2 & -c_3 \\
%  -v_x v_y & 0 & v_x & 2c_2 & 2(c_1+c_3-1) & 2c_2 \\
%  (2-2c_1)v_x & 2c_1-2 & 0 & \rho v_x & 0 & 0 \\
%  -2c_2 v_x +(2-2c_1)v_y & 2c_2 & 2c_1-2 & 0 & 2\rho v_x & 0 \\
%  -2c_2 v_y & 0 & 2c_2 & 0 & 0 & \rho v_x
%  \end{array} \right),
%\end{multline*}
%and
%\begin{multline*}
%  S_{UU} A_2(U) =
%  \frac{1}{\rho} \cdot \\ \left( \begin{array}{cccccc} -(p_\rho-v_x^2-v_y^2)v_y & -v_x v_y & p_\rho-v_y^2 & -2c_2 v_x + c_1 v_y & 2(1-c_1-c_3)v_x-2c_2 v_y & -2c_2 v_x + (2-3c_3)v_y \\
%    -v_x v_y & v_y & 0 & 2c_2 & 2(c_1 + c_3 -1)  & 2c_2 \\
%    p_\rho-v_y^2 & 0 & v_y & -c_1 & 2c_2 & 3c_3-2 \\
%    -2 c_2 v_x & 2c_2 & 0 & \rho v_y & 0 & 0 \\
%    (2-2c_3)v_x - 2c_2 v_y & 2c_3-2 & 2c_2 & 0 & 2 \rho v_y & 0 \\
%    (2-2c_3)v_y & 0 & 2c_3-2 & 0 & 0 & \rho v_y \end{array} \right).
%\end{multline*}
%}
Therefore, the products $S_{UU}(U)A_1(U)$ and $S_{UU}(U)A_2(U)$ are not symmetric.

%Next we will show that the system \eqref{12} is symmetrizable hyperbolic in the sense of Friedrichs \cite{majda2012compressible}. To do this, we need to find a positive-definite  symmetric matrix $A_0(U)$ symmetrizing the coefficient matrix $A(U)$.
%In case \eqref{12} can be casted into a conservative form, the Hessian $S_{UU}$ of the convex entropy $S=S(U)$ is a natural symmetrizer. Now \eqref{12} can not be casted into a conservative form and the Hessian
%\begin{eqnarray}
%S_{UU} = \left( \begin{array}{ccc}
%  \frac{p_\rho}{\rho} + \frac{v^2}{\rho} & -\frac{v}{\rho} & 0 \\[2mm]
%  -\frac{v}{\rho} & \frac{1}{\rho} & 0 \\[2mm]
%	0 & 0 & 1  \end{array} \right)
%\end{eqnarray}
%does not symmetrize $A(U)$.

\section{Structural properties of the 1D model}
\setcounter{equation}{0}

In this section, we show that the one-dimensional system satisfies the second structural stability condition proposed in \cite{yong1992singular,yong2001basic} for general hyperbolic relaxation systems and the Kawashima condition \cite{kawashima1985systems}. We start with the structural stability condition.

\subsection{Structural stability}

For the one-dimensional model \eqref{12}, we define
\begin{eqnarray}\label{31}
A_0(U) = \frac{1}{\rho} \left( \begin{array}{ccc}
	 p_\rho  +v^2 & -v & 0 \\ [2mm]
	-v & 1 & 0 \\[2mm]
	0 & 0 & \frac{3c-2}{2c-2}\rho  \end{array} \right).
\end{eqnarray}
Obviously, both this $A_0(U)$ and the product
\begin{eqnarray*}
A_0(U) A(U) = \frac{1}{\rho} \left( \begin{array}{ccc}
		-(p_{\rho}-v^2)v & {p_\rho -v^2} & -{(3c-2)v} \\[2mm]
		{p_\rho-v^2} & {v} & {3c-2} \\[2mm]
		-{(3c-2)v} & {3c-2} & \frac{3c-2}{2c-2}\rho v
		\end{array} \right)
\end{eqnarray*}
are symmetric (see \eqref{24} for $A(U)$). Moreover, $A_0(U)$ is positive definite if
\begin{eqnarray*}
p_\rho > 0 \quad \mbox{and} \quad c < \frac{2}{3} \quad \mbox{or} \quad  c > 1 .
\end{eqnarray*}
For this reason, we take
\begin{equation}\label{37}
G := \{(\rho, v, c): \rho>0, \quad c< 2/3\}
% \ \mbox{or} \ > 1
\end{equation}
as the domain for the one-dimensional model.

%Since we only consider the global existence near equilibrium where $c=0$ in this paper, the smallness of $c$ guarantees the positive definite property of $A_0(U)$ near equilibrium.
%\subsection{Dissipativity}

Furthermore, with $A_0(U)$ constructed above it is clear that
\begin{equation}\label{32}
A_0(U_e)Q_U(U_e) + Q_U^T(U_e)A_0(U_e) =-\frac{2}{\tau}\mbox{diag}(0, 0, 1)
\end{equation}
for any equilibrium state $U_e = (\rho_e, 0, 0)^T$. Thus, the system \eqref{12} is dissipative in the sense of \cite{yong1992singular,yong1999singular}.

\subsection{Kawashima condition}

The model \eqref{12} is a nonconservative symmetrizable hyperbolic system with a convex entropy. For hyperbolic balance laws with a convex entropy, a general global existence result can be found in \cite{yong2004entropy,hanouzet2003global}. The proof was based on the Kawashima condition \cite{kawashima1985systems}.
%Actually, the Kawashima was proposed for the general symmetrizable hyperbolic system. It describes the dissipative property of the system. The condition is for the linear hyperbolic system.
Here we show that the system \eqref{12} satisfies the Kawashima condition by explicitly giving a compensating matrix $K$ \cite{kawashima1985systems}.

Set
\begin{eqnarray}\label{33}
K=\left( \begin{array}{ccc}
	0 & 1 & 0 \\
	-1 & 0 & -\rho_e \\
	0 & \rho_e & 0
	\end{array} \right)
\end{eqnarray}
(a skew-symmetric matrix) and $\bar{L} = \eta\mbox{diag}(0, 0, 1)$ (the diagonal matrix in \eqref{32}). Then we compute
\begin{eqnarray*}
K A(U_e) + (K A(U_e))^T + \bar{L} =
	\left( \begin{array}{ccc}
	2p_\rho(\rho_e) & 0 & \rho_e p_\rho(\rho_e) -2  \\
	0 & 2 & 0 \\
	\rho_e p_\rho(\rho_e) - 2 & 0 & \eta -4\rho_e 	
	\end{array} \right).
\end{eqnarray*}
This matrix is positive definite if
\begin{eqnarray*}
p_\rho > 0, \ \eta >4\rho, \ 2p_\rho(\eta - 4\rho)-(\rho p_{\rho}-2)^2 = 2 p_\rho\eta  - (2 + \rho p_\rho)^2 >0.
\end{eqnarray*}
The first inequality is a physical requirement for $p=p(\rho)$. The last two inequalities are true provided that
$$
\eta \geq 2\rho_e +\frac{(2 + \rho_e p_\rho(\rho_e))^2}{2 p_\rho(\rho_e)}.
$$
Thus, for sufficiently large $\eta$, we have
\begin{eqnarray}\label{35}
K A(U_e) + (K A(U_e))^T + \bar{L} \ge C_s I
\end{eqnarray}
with $I$ the $3 \times 3$ unit matrix and $C_s$ a positive constant depending only on $\rho_e$.

\section{Global existence}
\setcounter{equation}{0}

In this section, we prove that the one-dimensional model \eqref{12} allows global smooth solutions near equilibrium states, although it contains non-conservative equations. Here we fix the relaxation time $\tau=1$. The proof is similar to that in \cite{yong2004entropy,kawashima2009decay} for hyperbolic balance laws, but we have to deal with the non-conservative equations. We use the standard notation from \cite{yong2004entropy,kawashima2009decay} and the main result reads as
%The main difficulty here is the lack of conservation laws, which is crucial in deriving $L^2$ estimate of $U$ in \cite{yong2004entropy}. Here we also use the entropy to derive $L^2$ estimate following the concrete analysis of the equation. The high order derivative estimate follows the reference \cite{yong2004entropy}. The Kawashima condition is used to obtain the $L^2_t(H^s_X)$ estimate of $U$ which is important to obtain the global existence result \cite{yong2004entropy}.

\begin{thm} \label{theoremglobal}
  Let $s \ge 2$ be an integer. Assume $U_0=U_0(x) \in H^s(\mathbb{R})$ and $\|U_0 -U_e\|_{H^s}$ is sufficiently small. Then there is a unique global solution $U=U(x,t)$ to the system \eqref{12} with initial data $U_0$ satisfying
$$
U-U_e \in C([0,+\infty),H^s(\mathbb{R}))
$$
and
\begin{eqnarray}\label{41}
\|U(T)-U_e\|^2_{H^s} + \int_0^T \left[ \|c(t)\|^2_{H^s} + \|\partial_x U(t)\|^2_{H^{s-1}} \right] dt \le C \|U_0 -U_e\|^2_{H^s}
\end{eqnarray}
for all $T>0$, where $C$ is a positive constant independent of $T$.
\end{thm}

\begin{proof}
Since the system  \eqref{12} is symmetrizable hyperbolic, by the standard theory
%the system \eqref{12}, defined in the open convex set $G$
\cite{majda1984compressible, kato1975cauchy} it has a unique local-in-time $H^s$ solution with initial data $U_0$. Then the global existence follows from a standard continuation argument (see, e.g., \cite{yong2004entropy}) based on the {\it a priori} estimate \eqref{41}. Thus, it is sufficient to get the {\it a priori} estimate, which will be done with the following three steps.

%The proof is divided into three steps. We first derive the basic $L^2$ estimate. And then the higher order derivative estimate is obtained by differentiate the system. Then the Kawashima condition is utilized to get the $L_t(H^s)$ estimate for $U$. The global existence result then follows from these estimates.

\emph{Step 1.} Define
\begin{eqnarray*}
E=E(U,U_e) = S(U)-S(U_e)-S_U(U_e)(U-U_e).
\end{eqnarray*}
Since $S=S(U)$ is strictly convex, there exist two positive constants $c_1$ and $C_1$ such that
\begin{eqnarray}\label{42}
c_1 |U-U_e|^2 \le E(U,U_e) \le C_1 |U-U|^2
\end{eqnarray}
for all $U$ close to $U_e$. On the other hand, from \eqref{23} and \eqref{21} we compute
\begin{eqnarray}\label{43}
E_t = S_t - S_U(U_e)U_t = -\nabla \cdot J(U) - \frac{c^2}{\tau} - S_U(U_e)(Q(U) - A(U)U_x).
\end{eqnarray}
Notice that
\begin{eqnarray*}
S_U(U_e) A(U) U_x &=& (\phi_\rho(\rho_e), 0, 0)
\left( \begin{array}{ccc}
0 & 1 & 0 \\
p_\rho-v^2 & 2v & 3c-2 \\
-\frac{2c-2}{\rho} & -\frac{2c-2}{\rho} & v
\end{array} \right)
\left( \begin{array}{ccc}
\rho \\ \rho v \\ c \end{array} \right)_x \\
&=& \phi_\rho(\rho_e) (\rho v )_x,
\end{eqnarray*}
we integrate the equation \eqref{43} over $(x, t)\in(-\infty, + \infty)\times[0,T]$ to obtain
\begin{eqnarray*}
\int_\mathbb{R}E(U(x, T), U_e)dx - \int_\mathbb{R} E(U_0, U_e)dx = - \int_0^T\int_\mathbb{R} \frac{c^2(x, t)}{\tau} dx dt.
\end{eqnarray*}
Combining this with \eqref{42}, we arrive at the following basic $L^2$ estimate
\begin{eqnarray}\label{44}
\|U(\cdot, T)-U_0\|^2_{L^2} + \int_0^T \|c(\cdot, t)\|^2_{L^2}dt \le C\|U_0 - U_e\|^2_{L^2},
\end{eqnarray}
where $C$ is a generic constant.

\emph{Step 2.} Next we estimate the higher-order derivatives. Let $l\le s$ be a positive integer. We take the $l$-th order derivative with $x$ in both sides of the system \eqref{21} to obtain
\begin{eqnarray*}
\partial^l_x U_t + A(U) \partial^l_x U_x = \partial^l_x Q(U) + [A(U),\partial^l_x]U_x.
\end{eqnarray*}
where $[X,Y]=XY-YX$ is the commutator. Since $A_0(U)$ and $A_0(U)A(U)$ are symmetric, we take the $L^2$-inner product with $A_0(U)\partial^l_x U$ to get
\begin{multline}\label{45}
  (A_0(U)\partial^l_x U,\partial^l_x U)_t + \int (\partial^l_x U^T A_0(U)A(U)\partial^l_x U)_x dx \\
  = ((\partial_t A_0(U)+\partial_x(A_0(U)A(U)))\partial^l_x U,\partial^l_x U) \\
  + 2(A_0(U)[A(U),\partial^l_x]U_x,\partial^l_x U) + 2(A_0(U) \partial^l_x Q(U), \partial^l_x U).
\end{multline}
Since $U(\cdot, t)\in H^s(\mathbb{R})$ for each $t$, the second term on the left-hand side vanishes.
%It follows from the postive definite property of $A_0(U)$ that
%\begin{eqnarray}
%  (A_0(U)\partial^l_x U, \partial^l_x U) \ge C \|\partial^l_x U\|^2_{L^2}
%\end{eqnarray}
The three terms on the right-hand side can be estimated as follows.

For the first term we have
\begin{eqnarray}\label{HsEST:1}
  && ((\partial_t A_0(U) + \partial_x (A_0(U)A(U)))\partial^l_x U, \partial^l_x U) \nonumber\\
  &\le&|\partial_t A_0(U) + \partial_x (A_0(U)A(U))|_{L^\infty}\|\partial^l_x U\|_{L^2}^2 \nonumber\\
  &\le& C (|\partial_t U|_{L^\infty}+|\partial_x U|_{L^{\infty}})\|\partial^l_x U\|_{L^2}^2  \\
  &\le& C (|\partial_x U|_{L^\infty}+|Q(U)|_{L^\infty})\|\partial^l_x U\|_{L^2}^2\nonumber\\
  &\le& C (|\partial_x U|_{L^\infty} + |c|_{L^\infty}) \|\partial^l_x U\|_{L^2}^2\nonumber \\
  &\le& C \|U-U_e\|_{H^s} \|\partial^l_x U\|_{L^2}^2. \nonumber
\end{eqnarray}
The second term can be estimated as
\begin{eqnarray*}
  2(A_0(U) [A,\partial^l_x] U_x, \partial^l_x U) \le C |A_0(U)|_{L^\infty} \|[A(U),\partial^l_x] U_x\|_{L^2} \|\partial^l_x U\|_{L^2}.
\end{eqnarray*}
For the commutator we use the calculus inequalities in Sobolev spaces \cite{majda1984compressible} to obtain
\begin{eqnarray*}
  \|[A(U),\partial^l_x] U_x\|_{L^2} &\le& C (\|\partial_x^s A(U)\|_{L^2}|U_x|_{L^\infty} + |\partial_x A(U)|_{L^{\infty}} \|\partial_x^{s-1} U_x\|_{L^2})  \\
  &\le& C|U_x|_{L^\infty} \|\partial_x^s U\|_{L^2}.
\end{eqnarray*}
Thus, we have
\begin{eqnarray}\label{HsEST:2}
  2(A_0(U) [A,\partial_x^l] U_x,\partial^l_x U) \le C \|U-U_e\|_{H^s}\|\partial_x U\|_{H^{s-1}}^2.
\end{eqnarray}
For the third term we recall the expressions of $Q(U)$ and $A_0(U)$ in \eqref{24} and \eqref{31} to get
\begin{eqnarray}\label{HsEST:3}
&& 2(A_0(U)\partial^l_x Q(U),\partial^l_x U) \nonumber \\
&=& 2( A_0(U_e)\partial^l_x Q(U),\partial^l_x U) + 2((A_0(U)-A_0(U_e)) \partial^l_x Q(U),\partial^l_x U) \nonumber \\
&\le&  -2 \|\partial^l_x c\|_{L^2}^2 + C|U-U_e|_{L^\infty} \|\partial^l_x U\|_{L^2}^2 \nonumber \\
&\le& -2 \|\partial^l_x c\|_{L^2}^2 + C\|U-U_e\|_{H^s} \|\partial^l_x U\|_{L^2}^2
\end{eqnarray}

Having the inequalities \eqref{HsEST:1},\eqref{HsEST:2} and \eqref{HsEST:3}, we integrate \eqref{45} over $[0,T]$ to obtain
\begin{eqnarray}\label{49}
  \|\partial^l_x U(T)\|_{L^2}^2&  + & \int_0^T \|\partial^l_x c(t)\|^2_{L^2}dt \le C\|\partial^l_x U(0)\|_{L^2}^2  \nonumber\\[3mm]
 & +& C\sup_{t \in [0,T]}\|U(t)-U_e\|_{H^s} \int_0^T \|\partial_x U(t)\|^2_{H^{s-1}}dt .
\end{eqnarray}
Here we have used the following fact that
\begin{eqnarray*}
   C^{-1} \|\partial^l_x U\|_{L^2}^2 \le  (A_0(U) \partial^l_x U,\partial^l_x U) \le C \|\partial^l_x U\|_{L^2}^2.
\end{eqnarray*}
Adding \eqref{44} and \eqref{49} with $1 \le l \le s$, we arrive at the following estimate
  \begin{eqnarray}\label{210}
    && \|U(T)-U_e\|^2_{H^{s}}  +  \int_0^T \|c(t)\|^2_{H^s} dt \nonumber \\
    & \le & C \|U_0-U_e\|_{H^{s}}^2 + C \sup_{t \in [0,T]} \|U(t) - U_e\|_{H^s} \int_0^T \|\partial_x U\|_{H^{s-1}}^2dt.
%  \\  \le C \|\partial_x U_0\|^2_{H^{s-1}} + C M(t)D_0(t)^2
\end{eqnarray}

\emph{Step 3.} Next we use the Kawashima condition to control the second term in the last inequality. For this purpose,  we rewrite \eqref{21} as
\begin{eqnarray*}
  U_t + A(U_e) U_x  = (A(U_e) -A(U))U_x + Q(U).
\end{eqnarray*}
From this equation, we take the $l$-th derivative with $x$ and multiply by the compensating matrix $K$ to obtain
\begin{eqnarray*}
  K \partial^l_x U_t + K A(U_e) \partial^l_x U_x  = K \partial^l_x ((A(U_e)-A(U))U_x) + K\partial^l_x Q(U).
\end{eqnarray*}
Then taking the inner product with $\partial^l_x U_x$ we get
\begin{eqnarray}\label{211}
  (K \partial^l_x U_t, \partial^l_x U_x) + (KA(U_e) \partial^l_x U_x, \partial^l_x U_x) \nonumber \\ = (K \partial^l_x((A(U_e)-A(U))U_x),\partial^l_x U_x) + (\partial^l_x( K Q(U) ),\partial^l_x U_x).
\end{eqnarray}
Recall that the matrix $K$ is skew-symmetric. The first term on the left-hand side of the last equation can be treated as
\begin{eqnarray}\label{212}
  (K\partial^l_x U_t, \partial^l_x U_x) &=& \frac{1}{2} \int_\mathbb{R} \left[(\partial^l_x U^T_x K \partial^l_x U )_t -  (\partial^l_x U_t^T K\partial^l_x U  )_x \right]dx  \nonumber \\
  &=& \frac{1}{2}(K \partial^l_x U,\partial^l_x U_x)_t .
\end{eqnarray}
For the second term, we use \eqref{35} to get
\begin{eqnarray}\label{213}
  2(KA(U_e) \partial^l_x U_x,\partial^l_x U_x) &=& ( (KA(U_e)+ (K A(U_e))^T +\bar{L}) \partial^l_x U_x, \partial^l_x U_x) - (\bar{L}\partial^l_x U_x, \partial^l_x U_x) \nonumber\\
  &\ge& C_s \|\partial^l_x U_x \|^2_{L^2} -\eta \|\partial^l_x c\|^2_{L^2},
\end{eqnarray}

The right-hand side of \eqref{211} can be simply estimated as
\begin{eqnarray}\label{214}
  (K \partial^l_x Q(U),\partial^l_x U_x) \le \epsilon \|\partial^l_x U_x\|_{L^2}^2 + \frac{C}{\epsilon} \|\partial^l_x c\|_{L^2}^2
\end{eqnarray}
and
\begin{eqnarray}\label{215}
  && (K\partial^l_x ((A(U_e)-A(U))U_x),\partial^l_x U_x) \le \epsilon\|\partial^l_x U_x\|_{L^2}^2 + \frac{C}{\epsilon} \|\partial^l_x( (A(U_e)-A(U))U_x)\|_{L^2}^2 \nonumber \\
  &\le& \epsilon \|\partial^l_x U_x \|_{L^2}^2 + C(\epsilon)( |A(U_e)-A(U )|_{L^\infty}^2 \|\partial^l_x U_x\|^2_{L^2}+\|\partial^l_x A(U)\|_{L^2}^2|U_x|_{L^\infty}^2) \nonumber \\
  &\le& \epsilon \|\partial^l_x U_x\|_{L^2}^2 + C(\epsilon) (|U-U_e|_{L^\infty}^2 \|\partial^l_x U_x\|_{L^2}^2 + |U_x|_{L^\infty}^2 \|\partial^l_x U\|_{L^2}^2).
\end{eqnarray}
Here we have used the calculus inequalities in Sobolev spaces.

Using the above inequalities \eqref{212}--\eqref{215}, we deduce from \eqref{211} that
\begin{multline*}
  C_s \|\partial^l_x U_x\|^2_{L^2} \le (\eta+\frac{C}{\epsilon}) \|\partial^l_x c\|_{L^2}^2-\frac{1}{2} (K \partial^l_x U,\partial^l_x U_x)_t  + 2\epsilon \|\partial^l_x U_x\|^2_{L^2} \\
  +C(\epsilon) |U-U_e|_{L^\infty}^2 \|\partial^l_x U_x\|_{L^2}^2 + C(\epsilon) |U_x|_{L^\infty}^2 \|\partial^l_x U\|_{L^2}^2.
\end{multline*}
Choose $\epsilon$ sufficiently small (for example $\epsilon = C_s/4$). By integrating the last inequality over $[0,T]$, we arrive at
\begin{multline*}
  \int_0^T \|\partial^l_x U_x(t)\|^2_{L^2} dt \le C \int_0^T \|\partial^l_x c(t)\|^2_{L^2} dt
  +C\|\partial^l_x U(T)\|_{H^1}^2 + C\|\partial^l_x U_0\|_{H^1}^2 \\
  + C \sup_{t\in [0,T]} (|U(t)-U_e|_{L^\infty}^2 + |U_x(t)|_{L^\infty}^2) \int_0^T \|\partial^l_x U(t)\|_{L^2}^2 dt
\end{multline*}
for $1\le l \le s-1$. Adding this inequality from $1 \le l \le s-1$ gives
\begin{multline}\label{216}
  \int_0^T \|\partial_x U(t)\|^2_{H^{s-1}} dt \le C \int_0^T \|c(t)\|^2_{H^s} dt + C\|U_0-U_e\|^2_{H^s} + C\|U(T)-U_e\|^2_{H^s} \\+C \sup_{0 \le t \le T}(|U(t)-U_e|_{L^\infty}^2 + |U_x(t)|_{L^\infty}^2) \int_0^T \|\partial_x U(t)\|_{H^{s-1}}^2 dt \\
  \le C \int_0^T \|c(t)\|_{H^s}^2 dt + C \|U_0-U_e\|_{H^s}^2 + C\|U(T) -U_e\|_{H^s}^2  \\ + C \sup_{t \in [0,T] }\|U(t)-U_e\|_{H^s}^2 \int_0^T \|\partial_x U(t)\|_{H^{s-1}}^2 dt.
\end{multline}

Let $\alpha$ be a positive number. We multiply \eqref{216} with $\alpha$ and then add the resultant inequality to \eqref{210} to get
\begin{eqnarray*}
  \|U(T)-U_e\|_{H^s}^2 + \int_0^T \|c(t)\|_{H^s}^2 dt + \alpha \int_0^T \|\partial_x U(t)\|_{H^{s-1}}^2 dt \\
  \le C \alpha \int_0^T \|c(t)\|_{H^s}^2 dt +C(1+\alpha)\|U_0 -U_e\|_{H^s}^2 + C \alpha \|U(T)-U_e\|_{H^s}^2  \\ + C(\sup_{t \in [0,T]} \|U(t)-U_e\|_{H^s} + \alpha \sup_{t \in [0,T]} \|U(t)-U_e\|_{H^s}^2)  \int_0^T \|\partial_x U\|_{H^{s-1}}^2 dt.
\end{eqnarray*}
With $\alpha$ sufficiently small, the last inequality leads to
\begin{eqnarray*}
  \|U(T)-U_e\|_{H^s}^2 + \int_0^T \|c(t)\|_{H^s}^2 dt + \int_0^T \|\partial_x U(t)\|_{H^{s-1}}^2 dt \\
  \le C \|U_0-U_e\|_{H^s}^2 + C \sup_{t \in [0,T]} \|U(t)-U_e\|_{H^s} \int_0^T \|\partial_x U(t)\|_{H^{s-1}}^2 dt \\
  +C\alpha  \sup_{t \in [0,T]} \|U(t)-U_e\|_{H^s}^2 \int_0^T \|\partial_x U(t)\|_{H^{s-1}}^2 dt.
\end{eqnarray*}
From this the {\it a prior} estimate in Theorem 4.1 follows if $\|U_0 -U_e\|_{H^s} $ is sufficiently small. This completes the proof.

\end{proof}
\section{Compatibility
%with the Navier-Stokes equations
}
\setcounter{equation}{0}

In this section, we show that the 1D model \eqref{12} is compatible with the Navier-Stokes equations as the relaxation time $\tau$ tends to zero. Formally, this can be simply seen with the Maxwell iteration. Indeed, we rewrite the third equation in \eqref{12} as
\begin{eqnarray*}
  c= -\tau(\partial_t c + v \partial_x c + (2c-2) \partial_x v)
\end{eqnarray*}
and iterate this relation once to obtain
\begin{eqnarray*}
  c = 2 \tau \partial_x v + O(\tau^2).
\end{eqnarray*}
Substituting the truncation into the second equation in \eqref{12},
we arrive at the 1D isentropic Navier-Stokes equations
  \begin{align}\label{51}
  \partial_t \rho + \partial_x (\rho v ) = 0, \nonumber \\
  \partial_t (\rho v) + \partial_x( \rho v^2 + p) = 4 \tau \partial^2_x v
\end{align}
with $4\tau$ as the viscosity coefficient. The rest of this section is to justify the above formal derivation of \eqref{51} from \eqref{12}.

Our justification is similar to that in \cite{yang2015validity} but we have to deal with the non-conservative terms in \eqref{12}. The main result reads as

\begin{thm}\label{theoremCE}
Let $s \ge 2$ be an integer and ${\bar u} =({\bar \rho}(x),\bar{\rho}(x){\bar v}(x))$ satisfy
  \begin{eqnarray*}
    \bar{u}\in H^{s+2} \quad  \mbox{and} \quad \inf_{x} \bar{\rho}(x)>0.
 \end{eqnarray*}
Then there exists $T_*>0$, independent of $\tau$, such that the 1D system \eqref{12} with initial data $(\bar{u},0)$ and the Navier-Stokes system \eqref{51} with initial data ${\bar u}$ have unique solutions $(u^\tau=(\rho^\tau,\rho^\tau v^\tau), c^\tau)(x,t)$ and $u^\tau_p=(\rho^\tau_p,\rho^\tau_p v^\tau_p)(x,t)$ in $C([0,T_*], H^s)$ satisfying
  \begin{equation}\label{52}
    \sup_{t \in [0, T_*]} \|(u^\tau-u^\tau_p)(\cdot,t)\|_{H^s} \le C(T_*) \tau^2
  \end{equation}
  for $\tau$ sufficiently small. Here $C(T_*)>0$ is independent of $\tau$.
\end{thm}

\begin{rem}
For the sake of simplicity, Theorem 2 is stated only for the special initial data. But it holds with \eqref{52} replaced by
\begin{equation*}\label{eq:Result}
    \sup_{t \in [-\tau\ln\tau, T_*]} \|(u^\tau-
    %u'_\tau-
    u^\tau_p)(\cdot,t)\|_{H^s} \le C(T_*) \tau^2,
  \end{equation*}
due to initial layers, for more general initial data
${\tilde u}_p =({\tilde \rho}_p(x, \tau),\tilde{\rho}_p(x,\tau){\tilde v}_p(x, \tau))$ for \eqref{51} and $ ({\tilde u} =({\tilde \rho}(x, \tau),\tilde{\rho}(x,\tau){\tilde v}(x, \tau)), \tilde {c}(x, \tau))$ for \eqref{12} satisfying
  \begin{eqnarray*}
    \tilde{u}_p(\cdot,\tau),\tilde{u}(\cdot,\tau),\tilde{c}(\cdot,\tau) \in H^{s+2}, \quad  \inf_{x, \tau} \tilde{\rho}_p(x,\tau), \inf_{x, \tau} \tilde{\rho}(x,\tau)>0, \quad \sup_{x, \tau} \tilde{c}(x,\tau)< 2/3, \nonumber  \\
    \|\tilde{\rho}(\cdot,\tau)-\tilde{\rho}_p(\cdot,\tau)\|_{H^s}, \| \tilde{\rho} \tilde{v}(\cdot,\tau)-\tilde{\rho}_p\tilde{v}_p(\cdot,\tau)- \frac{3}{2} \tau \bar{c}_{0x} \bar{c}_0 +2 \tau \bar{c}_{0x}\|_{H^s}  = O(\tau^2)
  \end{eqnarray*}
  with $\bar{c}_0=\tilde{c}(x,0)$. This general statement can be proved by slightly modifying the proof below.
\end{rem}

To prove the above theorem, we first  recall that the 1D system \eqref{12} satisfies the second structural stability condition \cite{yong1992singular,yong1999singular}. Therefore, the singular perturbation theory developed therein applies. Namely, the solution $U^\tau = (\rho^\tau, \rho^\tau v^\tau, c^\tau)^T$ to \eqref{12} with initial data $(\bar{u},0)$ exists in the time interval $[0,T_*]$ where an approximate solution $U_\tau^1=(\rho_\tau^1,\rho_\tau^1 v_\tau^1, c^1_\tau)^T$
was constructed and satisfies
\begin{eqnarray}\label{53}
  \sup_{t \in [0, T_*]} \|U^\tau(\cdot, t) - U_\tau^1(\cdot, t)\|_{H^s} \le K\tau^2
\end{eqnarray}
with $K$ a positive constant independent of $\tau$. Thus, the existence of $U^\tau$ on the time interval $[0,T_*]$ is out of question.

To show the other conclusions of the theorem, we refer to \cite{yong1992singular,yong1999singular} for the construction of the approximate solution $U_\tau^1$, which has the form
\begin{eqnarray*}
U^1_\tau  = U_0(x,t) + \tau U_1(x,t) + U'_0(x,t') + \tau U'_1(x,t')
\end{eqnarray*}
with $t'=t/\tau$.
Here the coefficients $U_0=(u_0, c_0)$ and $U_1=(u_1, c_1)$ of the outer solution solve
\begin{eqnarray}
  c_0 = 0, \quad
  \partial_t u_0+ \partial_x f(u_0,0)=0 , \nonumber \\
  c_1 = 2\partial_x v_0, \quad
  \partial_t u_1 + \partial_x (f_u(u_0,0) u_1)  -4 \partial^2_x v_0 = 0 \label{54}
\end{eqnarray}
with
\begin{eqnarray*}
  f(\rho,\rho v,c) = \left( \begin{array}{cc} \rho v \\ \rho v^2 + p + \frac{3}{2} c^2 - 2c \end{array} \right);
\end{eqnarray*}
while $U_0'=(u_0', c_0')$ and $U_1'=(u_1', c_1')$ satisfy
\begin{eqnarray*}
 \partial_{t'} u'_0 = 0,  \quad \partial_{t'} c'_0 = -c'_0 , \\
  \partial_{t'} u'_1 = \partial_x(f(\bar u_0,0) - f(\bar u_0,c'_0)), \\
  \partial_{t'} c'_1 = - c'_1 - \bar v_0 \partial_x c_0' -2c_0'\partial_x \bar v_0.
\end{eqnarray*}
The required initial data for the above equations were obtained with the matching conditions
\begin{eqnarray*}
  \lim_{t' \to \infty} U'_0(x,t') = 0 , \quad \lim_{t' \to \infty} U'_1(x,t') = 0 .
\end{eqnarray*}
These and the requirement that the approximate solution should fulfil the prescribed initial condition lead to
\begin{eqnarray*}
 u'_0 =0, \quad u_0(x,0) = \bar{u}(x), \quad  c'_0 =c'_0(x,0)e^{-t/\tau}=0 , \\[4mm]
 u'_1(x,0) = -\partial_x\int_0^{+\infty} (f(\bar u_0,0) - f(\bar u_0,c'_0)) dt'=0\\
% = \partial_x\int_0^{+\infty} \left( \begin{array}{cc} 0 \\  \frac{3}{2} (c_0')^2 - 2c'_0 \end{array} \right) dt' =
 %\left( \begin{array}{cc} 0 \\  \frac{3}{2} \bar c_{0x}\bar c_0 - 2\bar c_{0x} \end{array} \right), \\[4mm]
 u_1(x,0) =  - u'_1(x,0)=0, \quad c'_1(x,0) = - 2\partial_x\bar v_0(x).
\end{eqnarray*}
These together with \eqref{53} show that
\begin{eqnarray*}
  \|u^\tau(\cdot, t) - u_0 (\cdot, t) - \tau u_1(\cdot, t) \|_{H^s} \le K \tau^2
\end{eqnarray*}
for $t \in [0,T_*]$. Thus, it reduces to establish

\addtocounter{thm}{-1}
\renewcommand{\thethm}{\arabic{thm}$'$}
\begin{thm}
  Under the conditions of Theorem 2, %hold and $\tilde{u}^1(\cdot,\tau) \in H^{s+2}$,$\tilde{u}_p(\cdot,\tau)\in H^s$ satisfy
the system \eqref{51} with initial data $\bar{u} $ has a unique solution $u_p^\tau \in C([0,T_*],H^s)$ satisfying
\begin{eqnarray}\label{eq:wcediff}
  \sup_{t \in [0,T_*]} \| u^\tau_p(\cdot,t) - u_0 (\cdot, t) - \tau u_1(\cdot, t)\|_{H^s} \le C \tau^2
\end{eqnarray}
for $\tau$ sufficiently small. Here $C=C(T_*)$ is independent of $\tau$.
\end{thm}

\begin{proof}
From the local existence theory of the Navier-Stokes equations \cite{kawashima1984phd}, the system \eqref{51} has a unique solution $u^\tau_p=u^\tau_p(x,t)$ satisfying $u^\tau_p \in C([0,T],H^s)$. For $G_1\subset\subset G= \{(\rho, \rho v, c): \rho>0, \quad c< 2/3\}$, we define its maximum existence time as
\begin{eqnarray*}
  T^\tau :=\sup \{T>0 : u_p^\tau  \in C([0,T],H^s), u^\tau_p(x,t) \in G_1 \}.
\end{eqnarray*}
It can be shown as in \cite{yong2001basic} that
\begin{eqnarray*}
T^\tau > T_*,
\end{eqnarray*}
provided that, as $\tau$ tends to zero, the following error estimates
\begin{eqnarray*}
\sup_{x,t} |u_p^\tau(x,t) - u_0 (x, t) - \tau u_1(x,t)|=o(1), \\
\sup_t \| u_p^\tau(\cdot ,t) - u_0 (\cdot, t) - \tau u_1(\cdot, t) \|_{H^s} = O(1)
\end{eqnarray*}
hold for $t \in [0,\min\{T_*,T^\tau\})$. Thus it remains to prove the error estimate \eqref{eq:wcediff} with $T_*$ replaced by $\min\{T_*,T^\tau\}$.

For convenience, we will work with $w=(\rho,v)$, instead of $u=(\rho,\rho v)$, and  neglect the subscript $p$ to write $u^\tau_p$ as $u^\tau$. By the Sobolev calculus inequalities \cite{majda1984compressible}, the norms of $w$ are equivalent to those of $u$.

Set
\begin{eqnarray*}
  a(w) = \left( \begin{array}{cc} v & \rho \\ \frac{p_\rho}{\rho} & v \end{array} \right) \ \ \mbox{and} \ \
  a_0(w) = \left( \begin{array}{cc} \frac{p_\rho}{\rho^2} & 0 \\ 0 & 1 \end{array} \right).
\end{eqnarray*}
Note that $a_0(w)$ is symmetric positive definite for $\rho>0$ and $a_0(w)a(w)=a^T(w)a_0(w)$.
Then equation \eqref{51} can be rewritten as
\begin{eqnarray}\label{57}
  \partial_t w^\tau + a(w^\tau) \partial_x w^\tau = \left( \begin{array}{cc} 0 \\ \frac{4 \tau}{\rho^\tau} \partial^2_x v^\tau \end{array} \right) .
\end{eqnarray}

On the other hand, we deduce from \eqref{54} that $u_\tau \equiv u_0 +\tau u_1$ satisfies
\begin{eqnarray*}
  &&\partial_t u_\tau + f(u_\tau,0)_x \nonumber \\
  &=&  (f(u_\tau,0) - \tau f_u(u_0,0)u_1 - f(u_0,0))_x + \left( \begin{array}{c} 0 \\ 4 \tau \partial^2_x v_0 \end{array} \right) \nonumber \\
	&=& \left( \begin{array}{c} 0 \\ 4 \tau \partial^2_x v_\tau \end{array} \right)   + R
\end{eqnarray*}
with
$$
R=(f(u_\tau,0) - \tau f_u(u_0,0)u_1 - f(u_0,0))_x -\left( \begin{array}{c} 0 \\ 4 \tau^2 \partial^2_x v_1 \end{array} \right).
$$
Then it is not difficult to see that $w_\tau$ satisfies
\begin{eqnarray}\label{58}
  \partial_t w_\tau + a(w_\tau) \partial_x w_\tau = \left( \begin{array}{cc} 0 \\ \frac{4 \tau}{\rho_\tau} \partial^2_x v_\tau \end{array} \right) + \hat R
\end{eqnarray}
with
\begin{eqnarray*}
\hat{R} = \left( \begin{array}{cc} 1 & 0 \\ 0 & 1/\rho_\tau \end{array} \right)R .
\end{eqnarray*}
Obviously, we have $\|\hat{R}\|_{H^s} =O(\|R\|_{H^s})= O(\tau^2)$, for
\begin{eqnarray*}
  \|\partial^2_x v_1 \|_{H^s} &\le&  \|v_1\|_{H^{s+2}}
\end{eqnarray*}
and
\begin{eqnarray*}
  \|(f(u_\tau,0) &-& \tau f_u(u_0,0)u_1 - f(u_0,0))_x\|_{H^s} \\
  &=& \tau^2 \| u_1^Tf_{uu}(u_0+\tau \theta_2 \theta_1  u_1,0) \theta_1 u_1\|_{H^{s+1}} \\
  &\le&  C \tau^2 \|u_1\|_{H^{s+1}}^2 \|u_0+\tau \theta_2 \theta_1  u_1\|_{H^{s+1}},
\end{eqnarray*}
where $\theta_1,\theta_2 \in (0,1)$. Here we have used the Sobolev calculus inequalities \cite{majda1984compressible} and the fact that $u_0, u_1 \in H^{s+2}$.

Set $E = w^\tau - w_\tau$. It follows from \eqref{57} and \eqref{57} that
\begin{eqnarray*}
  \partial_t E + a(w^\tau) \partial_x E = (a(w_\tau) - a(w^\tau)) \partial_x w_\tau + \left( \begin{array}{cc} 0 \\ \frac{4\tau}{\rho^\tau} \partial^2_x v^\tau - \frac{4\tau}{\rho_\tau} \partial^2_x v_\tau \end{array} \right) - \hat{R}.
\end{eqnarray*}
This can be rewritten as
\begin{eqnarray*}
   \partial_t E + a(w^\tau) \partial_x E = (a(w_\tau) - a(w^\tau)) \partial_x w_\tau + \left( \begin{array}{cc} 0 \\ \frac{4\tau}{\rho^\tau} \partial^2_x (v^\tau - v_\tau) \end{array} \right) \\
   + \left( \begin{array}{cc} 0 \\ 4\tau(\frac{1}{\rho^\tau} -\frac{1}{ \rho_\tau})\partial^2_x v_\tau \end{array} \right)- \hat{R}.
\end{eqnarray*}
Taking the $l$-th order ($0 \le l \le s$) spatial derivatives leads to
\begin{eqnarray*}
  \partial_t \partial^l_x E + a(w^\tau) \partial^l_x E_x  = [a(w^\tau),\partial^l_x] E_x  + \left( \begin{array}{cc} 0 \\4\tau \partial^l_x(\frac{1}{\rho^\tau} \partial^2_x (v^\tau-v_\tau)) \end{array} \right) \\  + \left( \begin{array}{cc} 0 \\ 4 \tau \partial^l_x ((\frac{1}{\rho^\tau} - \frac{1}{\rho_\tau}) \partial^2_x v_\tau) \end{array} \right) -\partial^l_x \hat{R} + \partial^l_x ((a(w_\tau)-a(w^\tau))\partial_x w_\tau).
\end{eqnarray*}
We multiply the last equation with $\partial^l_x E^Ta_0(w^\tau)$ and integrate the resultant equation over $x$ to obtain
\begin{eqnarray}\label{58}
&& (a_0(w^\tau) \partial^l_x E,\partial^l_x E)_t + \int ( \partial^l_x E^T a_0(w^\tau) a(w^\tau) \partial^l_x E )_x dx \nonumber \\
&=&  ((a_0(w^\tau)_t + (a_0(w^\tau)a(w^\tau))_x) \partial^l_x E,\partial^l_x E)
+ 2(a_0(w^\tau)[a(w^\tau),\partial^l_x] E_x,\partial^l_x E)\nonumber \\
&& + 2 (4\tau \partial^l_x (\frac{1}{\rho^\tau} \partial^2_x (v^\tau-v_\tau)),\partial^l_x (v^\tau-v_\tau)) + 2(4\tau \partial^l_x ( (\frac{1}{\rho^\tau}-\frac{1}{\rho_\tau})\partial^2 v_\tau),\partial^l_x (v^\tau-v_\tau)) \nonumber\\
&& - 2(a_0(w^\tau)\partial^l_x\hat{R},\partial^l_x E)+ 2(a_0(w^\tau)\partial^l_x((a(w_\tau)-a(w^\tau))\partial_x w_\tau),\partial^l_x E) \nonumber \\
& \equiv& I_1 +I_2 +I_3 +I_4+I_5 + I_6.
\end{eqnarray}

Next we estimate the terms in the last equation. To do this, we notice that the second term on the left-hand side vanishes. Moreover, we recall that $w^\tau$ takes values in the compact set $G_1$ and $\|w_\tau(\cdot, t)\|_{H^{s+2}}$ is bounded uniformly with respect to $\tau$. Now, with the Sobolev inequalities \cite{majda1984compressible}, the terms $I_2, I_4, I_5$ and $I_6$ can be simply treated as follows
\begin{eqnarray*}
  I_2 &\le& 2 |a_0(w^\tau)|_{L^\infty} \|[a(w^\tau),\partial^l_x]E_x\|_{L^2} \|\partial^l_x E\|_{L^2} \nonumber \\
  &\le& C \|\partial^l_x E\|_{L^2} (|\partial_x a(w^\tau)|_{L^\infty}\|\partial^{s-1}_x E_x\|_{L^2} + |E_x|_{L^\infty} \|\partial^s_x a(w^\tau)\|_{L^2})  \nonumber \\
  &\le& C \|w^\tau\|_{H^s} \|E\|_{H^s}^2 \le C (1+\|E\|_{H^s}) \|E\|_{H^s}^2, \nonumber \\
  I_4 &\le&  C \tau \|\partial^l_x (v^\tau-v_\tau)\|_{L^2} \|\frac{1}{\rho^\tau}-\frac{1}{\rho_\tau}\|_{H^s} \|\partial^2_x v_\tau\|_{H^s}
  \le C \tau \|E\|_{H^s}^2, \nonumber\\
  I_5 &\le& C \|\partial^l_x \hat{R}\|_{L^2} \|\partial^l_x E\|_{L^2} \le C\tau^2\|\partial^l_x E\|_{L^2} \le C \tau^4 + C \|\partial^l_x E\|_{L^2}^2, \nonumber \\
  I_6 &\le& C\|\partial^l_x ((a(w_\tau)-a(w^\tau))\partial_x w_\tau)\|_{L^2} \|\partial^l_x E\|_{L^2}\nonumber\\
      &\le&  C \|w_\tau-w^\tau\|_{H^s} \|w_\tau\|_{H^{s+1}} \|\partial^l_x E\|_{L^2} \nonumber\\
      &\le&  C \|E\|_{H^s}^2. \label{515}
\end{eqnarray*}

For the first term, we deduce from equations in \eqref{51} that
\begin{eqnarray*}
  I_1 &\le&  |a_0(w^\tau)_t + (a_0(w^\tau) a(w^\tau))_x|_{L^\infty} \|\partial^l_x E\|_{L^2}^2 \nonumber\\
      &\le& C (|w^\tau_t|_{L^\infty} + |w^\tau_x|_{L^\infty}) \|\partial^l_x E\|_{L^2}^2 \nonumber\\
      &\le& C (|w^\tau_x|_{L^\infty} + \tau |\partial^2_x v^\tau|_{L^\infty}) \|\partial^l_x E\|_{L^2}^2 \nonumber\\
      &\le& C (\|w^\tau\|_{H^s} + \tau \|\partial_x(v^\tau-v_\tau)\|_{H^s} + \tau \|\partial_x v_\tau\|_{H^s} )\|\partial^l_x E\|_{L^2}^2 \nonumber\\
      &\le& C (\|w^\tau-w_\tau\|_{H^s}+\|w_\tau\|_{H^s}+\tau\|v_\tau\|_{H^{s+1}} + \tau \|\partial_x(v^\tau-v_\tau)\|_{H^s}) \|\partial^l_x E\|_{L^2}^2 \nonumber\\
      &\le& C (1+\|E\|_{H^s} + \tau \|\partial_x (v^\tau-v_\tau)\|_{H^s}) \|E\|_{H^s}^2. \label{516}
\end{eqnarray*}
The treatment of the third term $I_3$ needs integration by parts:
\begin{eqnarray*}
  I_3 &=& 8 \tau ([\partial^l_x,\frac{1}{\rho^\tau}] \partial^2_x(v^\tau-v_\tau) + \frac{1}{\rho^\tau} \partial^{l+2}_x(v^\tau-v_\tau),\partial^l_x (v^\tau-v_\tau) )\\
  &=& 8 \tau ([\partial^l_x,\frac{1}{\rho^\tau}] \partial^2_x(v^\tau-v_\tau),\partial^l_x (v^\tau-v_\tau))  -8 \tau(\partial_x (\frac{1}{\rho^\tau}) \partial^l_x(v^\tau-v_\tau),\partial^{l+1}_x (v^\tau-v_\tau)) \\ && -8\tau ( \frac{1}{\rho^\tau} \partial^{l+1}_x(v^\tau-v_\tau), \partial^{l+1}_x (v^\tau-v_\tau)) \\
  &\le& 8\tau \|[\partial^l_x,\frac{1}{\rho^\tau}]\partial^2_x(v^\tau-v_\tau)\|_{L^2} \|\partial^l_x (v^\tau-v_\tau)\|_{L^2} \\ & &+8\tau |\partial_x \frac{1}{\rho^\tau}|_{L^\infty} \|\partial^l_x (v^\tau-v_\tau)\|_{L^2} \|\partial^{l+1}_x(v^\tau-v_\tau)\|_{L^2}  -C_0 \tau \|\partial^{l+1}_x (v^\tau-v_\tau)\|_{L^2}^2\\
  &\le& C\tau \|\partial^l_x (v^\tau-v_\tau)\|_{L^2} (|\partial_x \frac{1}{\rho^\tau}|_{L^\infty} \|\partial_x^{s-1} \partial^2_x(v^\tau-v_\tau)\|_{L^2} + |\partial^2_x (v^\tau-v_\tau)|_{L^\infty} \|\partial^s_x \frac{1}{\rho^\tau}\|_{L^2})  \\ & & + C\tau \|\rho^\tau\|_{H^s} \|v^\tau-v_\tau\|_{H^s}\|\partial_x (v^\tau-v_\tau)\|_{H^s} -C_0 \tau\|\partial^{l+1}_x (v^\tau-v_\tau)\|_{L^2}^2\\
  &\le& C\tau (1+ \|E\|_{H^s})\|E\|_{H^s} \|\partial_x (v^\tau-v_\tau)\|_{H^s}-C_0 \tau \|\partial^{l+1}_x (v^\tau-v_\tau)\|_{L^2}^2 ,
\end{eqnarray*}
where $C_0 =\frac{ 8}{{\max_{x,t,\tau} \{\rho^\tau(x,t) \}}}$.

Combining \eqref{58} with the above estimates for the $I_k$'s and adding them together, we arrive at
\begin{eqnarray*}
 && \sum_{l=0}^s  (a_0(w^\tau) \partial^l_x E,\partial^l_x E)_t + C_0 \tau \|\partial_x (v^\tau-v_\tau)\|_{H^s}^2 \\
 &\le& C (1+\|E\|_{H^s})\|E\|_{H^s}^2 + C \tau(1+\|E\|_{H^s})\|E\|_{H^s} \|\partial_x(v^\tau-v_\tau)\|_{H^s} + C \tau^4  \\
 &\le& C (1+\|E\|_{H^s})\|E\|_{H^s}^2 + \frac{C_0\tau}{2} \|\partial_x(v^\tau-v_\tau)\|_{H^s}^2 + C \tau (1+\|E\|_{H^s})^2 \|E\|_{H^s}^2 + C \tau^4.
\end{eqnarray*}
Thus we have
\begin{eqnarray*}
  \sum_{l=0}^s  (a_0(w^\tau) \partial^l_x E,\partial^l_x E)_t + \frac{C_0}{2} \tau \|\partial_x (v^\tau-v_\tau)\|_{H^s}^2 \le   C (1+\|E\|_{H^s}^2)\|E\|_{H^s}^2 + C \tau^4.
\end{eqnarray*}
Recall that $a_0(w^\tau)$ is positive definite.
We integrate this inequality over $t \in [0,T]$ with $T \le \min \{T_*,T^\tau\}$ to obtain
\begin{eqnarray} \label{59}
 && \|E(T)\|_{H^s}^2 + \tau \int_0^T \|\partial_x(v^\tau(t)-v_\tau(t))\|_{H^s}^2 dt \nonumber \\
& \le &  C\int_0^T (1+\|E(t)\|_{H^s}^2)\|E(t)\|_{H^s}^2 dt +CT_* \tau^4.
\end{eqnarray}
Here we have used that $E(x, 0)=0$.

Set
\begin{eqnarray*}
  \phi(T) =  \int_0^T (1+\|E(t)\|_{H^s}^2)\|E(t)\|_{H^s}^2 dt + T_* \tau^4.
\end{eqnarray*}
It follows from \eqref{59} that
\begin{eqnarray*}
  \phi' \le C  \phi(1+C\phi)
\end{eqnarray*}
or
\begin{eqnarray*}
  -(\frac{1}{\phi})_t \le -C (-\frac{1}{\phi})+C^2.
\end{eqnarray*}
For this we use the Gronwall inequality to get
\begin{eqnarray*}
  \phi(T) \le \frac{e^{CT} \phi_0}{1-C e^{CT}  \phi_0}\le\frac{e^{CT_*} \phi_0}{1-C e^{CT_*} \phi_0} .
\end{eqnarray*}
Choosing $\tau$ sufficiently small such that $C e^{CT_*} \phi_0 \le \frac{1}{2}$ we arrive at
\begin{eqnarray*}
  \phi(T) \le 2T_*e^{CT_*}\tau^4.
\end{eqnarray*}
Hence we have
\begin{eqnarray*}
  \|E(T)\|_{H^s} \le C(T_*) \tau^2
\end{eqnarray*}
for $T \in [0,\min\{T_*,T^\tau\})$ and the proof is complete.
\end{proof}

\bibliography{Oldroydnew1}
\bibliographystyle{unsrt}

\end{document}







\section{本章小结}
本章通过推广守恒-耗散理论将含有客观导数的非线性粘弹性流体模型纳入了守恒-耗散理论的框架中。利用推广的守恒-耗散理论我们推广了上对流Maxwell模型和FENE-P模型,包含了温度和压缩性的影响。另外,我们还导出了非等温可压上对流Maxwell模型。最后利用Yong的双曲平衡率方程组小解的整体存在性理论和双曲方程松弛极限理论证明了一维等温可压上对流Maxwell模型光滑解在平衡态附近的整体存在性(定理\ref{theoremglobal})以及当松弛参数$\kappa$趋于$0$时与Navier-Stokes方程组的一致性(定理\ref{theoremCE})。
%应力张量的方程由于满足客观性原理不可以写成守恒形式,然而物理上的守恒律和熵增原理仍然成立。通过假设熵函数的存在性而放松对方程守恒形式的要求,推广的守恒-耗散理论很好地纳入了客观导数。本章利用推广的守恒-耗散理论推广了上对流Maxwell模型\eqref{eq:generalizedUCM}和FENE-P模型\eqref{eq:generalizedFENEP},这些模型包含了温度和压缩性的影响。另外,我们还利用守恒-耗散理论导出了类似\"Ottinger的一个模型——粘弹性流体力学第二模型\eqref{eq:ECDFsecond}。这一模型有很好的Hamilton结构。另外我们证明了其在一维时的光滑解在平衡态附近的整体存在性以及严格分析了在右端松弛参数趋于$0$时与Navier-Stokes方程组的一致性。


 % \bibliography{ref}
 % \bibliographystyle{unsrt}

% \end{document}