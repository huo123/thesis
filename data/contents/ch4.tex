%\chapter{有限形变理论在粘弹性流体建模中的应用}

\documentclass{article}
\usepackage{ctex}
\usepackage{amsmath}
\usepackage{cite}
\usepackage{amssymb}
\usepackage{amsthm}
\newtheorem{theorem}{定理}
\newtheorem{remark}{注释}
\newtheorem{lemma}{引理}

\begin{document}
本章我们将考虑有限形变理论在粘弹性流体力学中的应用。在第一章绪论中,我们已经看到积分形式的记忆衰减理论无法很好地用于粘弹性流体的建模。在本章中我们将利用守恒-耗散理论和有限形变理论对粘弹性流体力学进行建模。而有限形变理论采用形变张量$F$对流体的变形进行度量,由于形变张量满足力学上的适应性条件,可以将其方程写成守恒形式。从而我们可以将形变张量加入守恒变量中并可以利用守恒-耗散理论对粘弹性流通进行建模。在第一节中我们将给出连续介质运动方程的一般描述。第二节给出有限形变守恒-耗散理论的主要假设和一般框架。第三节讨论其在粘弹性流体中应用的几个例子。最后一节我们将讨论其中一个例子对应方程组的平衡态附近解的整体存在性。

\section{连续介质的一般方程}
在第二章第一节中,我们已经讨论了对连续介质变形的数学描述。连续介质的形变可以用函数$x=\mathcal{F}_t (X,t)$来描述。定义形变张量
\begin{equation*}\label{eq:Fdef}
	F_{ij} = \frac{\partial x_i}{\partial X_j}.
\end{equation*}
$F$的运动方程为
\begin{eqnarray}\label{eq:Feq}
F_t + v \cdot \nabla F = \nabla v F.
\end{eqnarray}
由$F$的定义\eqref{eq:Fdef},$F$满足下面的适应性条件
% \begin{eqnarray}
\begin{subequations}
\begin{align}
\nabla \cdot (\rho F^T) = 0,\label{eq:compatibility1} \\
 F_{lj} \partial_{x_l} F_{ik} = F_{lk} \partial_{x_k} F_{il} , \label{eq:compatibility2} \\
 \rho \det F = 1. \label{eq:compatibilty3} 
% \end{eqnarray}
\end{align}
\end{subequations}
这三个适应性条件中不是独立的。由\eqref{eq:compatibility2}和\eqref{eq:compatibilty3}可以推出\eqref{eq:compatibility1}。实际上,由
Jacobi公式
$$
\partial_{x_i}\det A = \det A \mbox{Tr}(A^{-1} \partial_{x_i} A),
$$
成立
\begin{eqnarray*}
\nabla \cdot (\rho F^T)  & = &\partial_{x_i} ( \rho F_{ij}) = \partial_{x_i} ( \frac{1}{\det F} F_{ij})  \\
&=& \frac{1}{\det F} \partial_{x_i} F_{ij}  - \frac{1}{(\det F)^2} F_{ij} \det F \mbox{Tr}(F^{-1} \partial_{x_i} F) \\
&=&\frac{1}{\det F} (\partial_{x_i} F_{ij} - F_{ij} (F^{-1})_{mn} \partial_{x_i} F_{nm}) \\
&=& \frac{1}{\det F} (\partial_{x_i} F_{ij} - (F^{-1})_{mn} F_{ij} \partial_{x_i} F_{nm}) \\
&=&  \frac{1}{\det F} (\partial_{x_i} F_{ij} - (F^{-1})_{mn} F_{im} \partial_{x_i} F_{nj}) \\
&=&  \frac{1}{\det F} (\partial_{x_i} F_{ij} - \delta_{in} \partial_{x_i} F_{nj}) \\
&=&  \frac{1}{\det F} (\partial_{x_i} F_{ij} - \partial_{x_i} F_{ij}) =0 ,
\end{eqnarray*}
其中倒数第五个等式用到了条件\eqref{eq:compatibilty2}
由适应性条件\eqref{eq:compatibility1},$\rho F$的演化方程可以写成守恒形式
\begin{eqnarray*}
(\rho F)_t + \nabla \cdot (\rho \otimes v) - \nabla \cdot (\rho v \otimes F^T) = 0 .
\end{eqnarray*} 

将$F$的方程加入描述连续介质运动的一般方程\eqref{eq:fluid}中,我们得到对连续介质的运动描述如下。
\begin{subequations}\label{eq:continuum}
\begin{align}
\rho_t + \nabla \cdot (\rho v )=0, \\
(\rho v)_t + \nabla \cdot (\rho v \otimes v) + \nabla \cdot P = 0, \\
(\rho e)_t + \nabla \cdot (\rho e v) + \nabla \cdot q + \nabla \cdot (P \cdot v) = 0 ,\\
(\rho F)_t + \nabla \cdot (\rho F \otimes v) - \nabla \cdot (\rho v \otimes F^T) = 0 .
\end{align}
\end{subequations}

在有限形变理论中应力$P$为形变张量$F$的函数。如何得到它们之间的本构关系是热力学理论的主要目的。而在粘弹性流体力学中,由于粘弹性的存在,应变能是耗散的,如何描述这一不可逆过程是非平衡态热力学理论的一个重要目的和粘弹性流体建模的关键。

\section{有限形变守恒-耗散理论}
为了描述粘弹性流体流动的不可逆性,我们仍然借助非平衡态热力学的守恒-耗散理论。由于$\rho F$的方程是守恒形式,我们可以将其加入守恒量中。选取守恒量
$$U_c = (\rho, \rho v, \rho e, \rho F).$$
我们仍引入耗散变量$U_d = (\rho w,\rho c)$来描述不可逆过程。$U$满足方程\eqref{eq:FCDF},即
\begin{equation}\label{eq:FCDF}
		\partial_t U + \sum_{j=1}^3 \partial_{x_j} F_j(U) = \mathcal{Q} (U) .
\end{equation}
假设系统存在熵函数函数$\eta$如下
\begin{eqnarray*}
\eta(U) = \eta(\rho, \rho v, \rho e,\rho F,\rho w, \rho c) = \rho s(\nu,u,F,w,C)
\end{eqnarray*}
其中$s$为比熵。

由Gibbs关系我们得到
$$\theta^{-1}:=s_{u} (\nu,u,F, w,C), \quad \theta^{-1} p := s_\nu (\nu,u,w,C).$$
计算熵的变化率得到
\begin{eqnarray*}
		\eta_t + \nabla \cdot (\eta v) &=& \rho (s_t + v \cdot \nabla s), \\
		&=& -\nabla \cdot (\theta^{-1} q) + s_w \cdot [\rho (w_t + v \cdot \nabla w) + \nabla \theta^{-1}] + \rho s_F : \nabla v F  \\
		&& + (s_c:[\rho (c_t + v \cdot \nabla c)] - \theta^{-1} \tau : D) \\
		&=& -\nabla \cdot (\theta^{-1} q) + s_w \cdot [\rho (w_t + v \cdot \nabla w)+\nabla \theta^{-1}] + \rho s_F F^T : D  \\
		&& + (s_c:[\rho (c_t + v \cdot \nabla c)] - \theta^{-1} \tau : D + \rho s_F F^T : D)  \\
		&& = -\nabla \cdot J + \Delta.
\end{eqnarray*}
这里我们假设了$q=s_w$。并且假设
\begin{equation}
	s_F F^T = F s_F^T.	
\end{equation}
这个假设是角动量守恒的要求得到的\cite{dimitrienko2010nonlinear,dafermos2010hyperbolic},下面可以看出其保证了应力张量的对称性。
对于$\tau$,像在第三章第二节中的那样,我们有两种选择。一是令
\begin{eqnarray*}
\theta^{-1} \tau = \rho s_F F^T 
\end{eqnarray*}
得到本构关系
\begin{eqnarray}\label{eq:finite1}
\left( \begin{array}{c} \partial_t (\rho w) + \nabla \cdot (\rho w \otimes v) + \nabla \theta^{-1}) \\
	\partial_t (\rho c) + \nabla \cdot (\rho c \otimes v) \end{array} \right)
=
M \cdot \left( \begin{array}{c} q \\ s_c \end{array} \right)
\end{eqnarray}
我们称这样的选择得到是第一类模型。另外我们可以选取
\begin{eqnarray*}
	\theta^{-1} \tau = \rho s_F F^T + s_c.
\end{eqnarray*}
得到
\begin{eqnarray}\label{eq:finite2}
\left( \begin{array}{c} \partial_t (\rho w) + \nabla \cdot (\rho w \otimes v) + \nabla \theta^{-1}) \\
	\partial_t (\rho c) + \nabla \cdot (\rho c \otimes v) - D \end{array} \right)
=
M \cdot \left( \begin{array}{c} q \\ s_c  \end{array} \right)
\end{eqnarray}
我们称这样的选择的得到的模型是第二类模型。

由于我们的选取使得熵函数$\eta$的演化均可以写作以下形式
\begin{equation}\label{eq:entropypro}
	\eta_t + \nabla \cdot J = \eta_U^T M \eta_U.
\end{equation}
为了保证熵增大于等于$0$,我们仍然假设守恒-耗散理论的第二个假设成立,即耗散矩阵$M$是正定的。

然而由于在方程组的守恒性需要适应性条件\eqref{eq:compatibility1}来保证。所以守恒-耗散理论第一条假设中$\eta_{UU} F_{jU}$对称的条件不一定满足。但是实际上方程组\eqref{eq:FCDF}满足带对合条件(involutions)的守恒形式方程组\cite{dafermos1986quasilinear,dafermos2013non,benzoni2007multi}。即存在$n$个矩阵$R_j,j=1,2,\cdots,n$,使得
\begin{equation*}
	M_i F_{j} + M_j F_i = 0, \ i,j = 1, \cdots,n .
\end{equation*}
其中$M_j$为($U= (\rho,\rho v_i,\rho e, \rho F_{kl},\rho w_m, \rho c_{rs})$)
\begin{equation*}
	M_{j} = \left( \begin{array}{ccccccc}
		0 \\
		& 0 \\
		& & 0 \\
		& & & \delta_{jk'}\delta_{ll'} \\
		& & & & 0 \\
		& & & & & 0
	\end{array} \right).
\end{equation*}
从而
\begin{equation}\label{eq:involutions}
	M_j \partial_j U = 0.
\end{equation}
该条件称为对合条件。该对合条件的具体形式为
\begin{equation*}
	\partial_{j} F_{jl} = 0
\end{equation*}
这就是第一个适应性条件\eqref{eq:compatibility1}。对于带对合的双曲守恒律方程组,Serre等发展了附随熵(contingent entropy entropy)的概念。如果存在函数$\eta = \eta(U)$和函数$J_j=J_j(U),j=1, \cdots n$使得
\begin{equation*}
	J_{jU}  = \eta_U F_{jU} + \Xi(U)^T M_j,\quad j=1,2, \cdots n,
\end{equation*}
那么成立
\begin{equation*}
	\partial_t U + \sum_{j=1}^n \partial_{x_j} J_{j}(U) = \eta_U^T \mathcal{Q}(U).
\end{equation*}
所以守恒-耗散理论的第一条假设中$\eta_{UU} F_{jU}$应改为存在函数$\Xi(U)$,使得
\begin{equation}\label{eq:invsym}
	\eta_{UU} F_{jU} + \Xi_U^T M_j = F_{jU}^T \eta_{UU} + M_j \Xi_U(U).
\end{equation}
另外对于本章考虑的两种选择得到的方程组\eqref{eq:FCDF}(其中$w,c$满足\eqref{eq:finite1}或\eqref{eq:finite2})与适应性条件\eqref{eq:compatibility1},如果存在函数$\eta = \eta(U)$满足\eqref{eq:entropypro},则存在$\Xi$,使得\eqref{eq:invsym}成立。实际上在我们的计算中采用的是下面的方程
\begin{equation*}
	\rho F_t + v \nabla \cdot F = \nabla v F.
\end{equation*}
或者写作
\begin{equation*}
(\rho F)_t + \nabla \cdot (\rho F \otimes v) - \nabla \cdot (\rho v \otimes F^T) + v \otimes \nabla \cdot (\rho F^T) = 0 .
\end{equation*}
所以$U$的方程为
\begin{equation*}
	\partial_t U + \sum_{j=1}^3 (\partial_{x_j} F_j(U) + v\otimes M_j \partial_{x_j} U)= \mathcal{Q} (U) .
\end{equation*}
由\eqref{eq:entropypro},和两种方法的计算过程可知
\begin{equation*}
	\eta_U F_{jU} + \eta_{U} (v \otimes M_{j}) = J_{jU}.
\end{equation*}
从而实际上取$\Xi = \eta_U \otimes v^T$即可得到$\eta$满足\eqref{eq:invsym}。

在守恒-耗散理论中,我们还假设熵函数$\eta=\eta(U)$为$U$的上凸函数。然而对于守恒律方程组\eqref{eq:continuum},如果我们不考虑$w,c$而假设系统处于平衡态,那么根据连续介质的相关理论\cite{ball1976convexity,dimitrienko2010nonlinear,dafermos2013non},比熵函数$s$不一定是$F$的上凸函数。实际上,根据动量矩守恒和客观性原理,$s$需要满足
\begin{equation}\label{eq:sObjective}
	s_F F^T = F s_F F^T, \ s(OF) = O s(F).
\end{equation}
其中$O$为旋转矩阵\cite{dafermos2013non}。而根据\cite{ball1976convexity},满足这两个条件的凸函数必须是二次的。在弹性力学中,人们考虑的弹性能函数不仅仅是二次的,$s$只需要是$F$的秩一上凸函数(Rank-One concave),即存在$\mu>0$使得
\begin{equation*}
	\frac{\partial^2 s(F)}{\partial F_{i\alpha} \partial F_{j\beta}} \xi_i \xi_j \nu_\alpha \nu_\beta \le -\mu |\xi|^2 |\nu|^2, \xi \in \mathbf{R}^n, \ \nu \in \mathbf{R}^n. 
\end{equation*}
恒成立。然而对于我们考虑的系统$s$还依赖于非平衡态变量$c,w$,我们假设$s$是$c,w$的凸函数,且为$F$的秩一上凸函数,且对$s=s(W),\ W= (\nu,u,F,c,w)$成立
\begin{eqnarray}
	\Omega^T s_{WW} \omega \le -\mu |\Omega|^2, \label{eq:FCDFconvex}\\ 
	\Omega = \left( \begin{array}{c}
		\omega_1 \\ \omega_2 \\ \omega_3 \\ \xi \otimes \nu \\ \omega_5 \\ \omega_6
	\end{array}\right). \nonumber
\end{eqnarray}
其中$\omega_1,\omega_2,\omega_3,\omega_5,\omega_6$分别为任意量,大小分别与$\rho, v,e, F,w,c$相同。

综上所述,我们得到有限形变守恒-耗散理论的两条基本假设如下。
\begin{enumerate}
		\item 存在函数$\eta = \eta (U) = \rho s(\nu,u,F,w,c)$,满足\eqref{eq:FCDFconvex}与\eqref{eq:sObjective},且存在$\Xi=\Xi(U)$,使得$\eta_{UU} F_{jU} + \Xi_U^T M_j $。我们称$\eta$为系统的熵函数($s$为比熵函数)。
		\item 存在正定矩阵$M = M(U)$,使得$\mathcal{q}(U) = M \eta_{U_d}$。我们称$M$为耗散矩阵。
\end{enumerate}

\subsection{第一类模型}
由于第一类模型中熵函数不依赖于耗散变量时,应力张量$\tau$不依赖于耗散变量$c$,此时我们无法描述粘弹性流体的不可逆过程。所以在这里我们假设熵函数包含$F$和$c$的耦合项。由于我们要求\eqref{eq:sObjective}成立,所以我们假设$s$有下面的形式 
 $$s = s_0(\nu,u) - \frac{1}{2 \nu \alpha_0} |w|^2 - \Phi (F) + \mbox{Tr} (FcF^T) - \Psi(c) .$$
由有限形变守恒耗散理论的第一条假设,我们要求$s$满足\eqref{eq:FCDFconvex},从而有
\begin{equation*}
	s_0(\nu,u) - \frac{1}{2\nu \alpha_0}|w|^2
\end{equation*}
为$\nu,u$的上凸函数。假设$u = c_v \ln \theta$且$p$仅依赖于密度$\rho$,$s_0(\nu,u) = c_v \ln u - \int_{\rho_0}^\rho \frac{p(z)}{z^2}dz$,那么$s_0$下凸给出
\begin{equation*}
	s_{0\nu u,\nu u} = \left( \begin{array}{ccc}
		- \rho^2 \frac{\partial p}{\partial \rho} - \frac{\rho^3}{\alpha_0}|w|^2 & 0 & 0 \\
		0 & -\frac{c_v}{u^2} & 0 \\
		0 & 0 & -\frac{\rho}{2\alpha_0}
	\end{array}\right)
\end{equation*}
负定。从而我们要求$\rho>0,\alpha_0>0$与$p_\rho>0$。这与物理上的要求相同。对于$\Phi(F)$和$\Psi(c)$,我们要求
\begin{equation*}
	\left( \begin{array}{cc}
		\xi_k \zeta_l & \omega_{rs} 
	\end{array}\right)
	\left( \begin{array}{cc}
		\Phi_{F_{kl}F_{k'l'}} - 2\delta_{kk'} c_{ll'}& -2 F_{kr'} \delta_{ls'} \\
		-2 F_{k'r} \delta_{l's} & \Psi_{c_{rs}c_{r's'}}
	\end{array}\right) 
	\left( \begin{array}{c}
		\xi_{k'}  \zeta_{l'} \\ \omega_{s'r'} 
	\end{array}\right) \\
	\ge \mu |\xi|^2|\zeta|^2 + \mu \omega:\omega
\end{equation*}

\subsection{第二类模型}
在第二类模型中$\tau$可以看做由弹性应力$\tau_p = \theta \rho s_F F^T$和粘弹性应力$\tau_v=\theta s_c$
两部分构成的。假设熵函数有形式
\begin{equation*}
 	s = s_0(\nu,u) - \frac{1}{2 \nu \alpha_0} |w|^2 - \Phi (F)  - \frac{1}{2\nu \alpha_1}c:c .
\end{equation*}
其中为了满足\eqref{eq:FCDFconvex},我们要求$\Phi(F)$为秩一下凸的。选取
\begin{equation*}
	M = \left( \begin{array}{ccc} 
			\frac{1}{\theta^2 \lambda} & 0 \\
			0 &  \theta(\frac{1}{\kappa} \dot{\mathcal{T}} + \frac{1}{\xi} \mathring{\mathcal{T}})  ,
		\end{array} \right)
\end{equation*}
从而我们根据\eqref{eq:finite2}得到
	\begin{subequations}
		\begin{align*}
			\alpha_0 [\partial_t q +  \nabla \cdot (q \otimes v)] - \nabla \theta^{-1} = -\frac{q}{\theta^2 \lambda}, \\
			\alpha_1[\partial_t (\theta_v^{-1} \dot{\tau}) + \nabla \cdot (\theta_v^{-1} \dot{\tau_v} \otimes v)] + \dot{D} = -\frac{\dot{\tau_v}}{\kappa}, \\
			\alpha_1[\partial_t (\theta_v^{-1} \mathring{\tau}) + \nabla \cdot (\theta_v^{-1} \mathring{\tau_v} \otimes v)] + \mathring{D} = -\frac{\dot{\tau_v}}{\xi}. 
		\end{align*}
	\end{subequations}
这与第二章得到的线性粘弹性模型\eqref{eq:CDFMaxwell}相同。但是我们这里应力张量应取作
\begin{equation*}
	\tau = -\theta \rho \Phi_F F^T - \theta \frac{\rho c}{\alpha_1}.
\end{equation*}
取$\Phi(F)$为最简单的二次函数,$\Phi(F) = F:F$,那么
\begin{equation*}
	\tau = -\theta \rho F F^T - \theta \frac{\rho c}{\alpha_1}.
\end{equation*}

不考虑温度($\theta=1$)的等温方程为
\begin{subequations}\label{eq:compressibleRelax}
  \begin{align}
  \rho_t + \nabla \cdot (\rho \mathbf{v}) = 0, \\
  (\rho \mathbf{v})_t + \nabla \cdot ( \rho \mathbf{v} \otimes \mathbf{v}) + \nabla p = \nabla \cdot (\rho F F^T) + \nabla \cdot ( \rho c), \\
  (\rho F)_t + \nabla \cdot (F \otimes \rho \mathbf{v}) = (\nabla \mathbf{v}) \rho F, \\
  (\rho c)_t + \nabla \cdot (\rho c) - D  = -\frac{\rho \dot{c}}{\kappa} - -\frac{\rho \mathring{c}}{\kappa}
\end{align}
\end{subequations}
%\documentclass{article}
%\usepackage{ctex}
% \usepackage{amssymb}
% \usepackage{amsmath}
% \usepackage{amsthm}
% \newtheorem{theorem}{定理}
% \newtheorem{remark}{注释}
% \newtheorem{lemma}{引理}
% \begin{document}
\section{有限形变粘弹性流体模型的数学分析}
本小节将讨论方程\eqref{eq:compressibleRelax}解的存在性。
%这里我们用$W=(\rho,\rho v,\rho F,\rho c)$表示上面的$U$。而利用$U$表示$U = (\rho, v, F, 、\rho c)$,
首先由熵$\eta(U) = \rho s$的形式为
\begin{equation*}
      s = s_0(\nu) - \frac{v^2}{2} - \frac{1}{2 \nu \alpha_0} |w|^2 - \Phi (F)  - \frac{1}{2\nu \alpha_1}c:c .
\end{equation*}
可知$s$是下凸的。这里$s_0$定义为
\begin{equation*}
    s_0(\nu) = -\int_{\rho_0}^{1/\nu} \frac{\pi(z)}{z^2} dz.
\end{equation*}
由有限形变的守恒-耗散理论我们无法得到$\eta_{UU}$可以将$U$的方程对称化。然而由于熵函数是二次的,$\eta_{UU}$是负定的。可以验证$A_0(U) = -\eta_{UU}$可以将$U$的方程对称化。首先将$U$的方程写成下面的形式。 
\begin{eqnarray*}
U_t + \sum_{j=1}^n A_j(U) U_{x_j} = \mathcal{Q}(U),\\
  U = \left( \begin{array}{c} \rho \\ \rho v_i \\ \rho  F_{kl} \\ \rho c_{rs} \end{array} \right), \quad \mathcal{Q}(U) = 
\left( \begin{array}{c} 0 \\ 0  \\ -\frac{1}{\kappa} \dot{\mathcal{T}}I - \frac{1}{\xi} \mathring{\mathcal{T}} \end{array} \right)U, \quad   A_j(U)= \\
 \tiny \left( \begin{array}{cccc} 
     0 & \delta_{i'j} & 0 & 0 \\
     p_\rho - v_i v_j - F_{jl'} F_{il'} & v_i \delta_{i'j} v_j \delta_{i'i} & -\rho F_{jl'} \delta_{ik'} & -\frac{1}{2}(\delta_{jr'} \delta_{is'} +\delta_{js'}\delta_{ir'}) \\
    0 & -F_{kl} v_j + F_{jl} v_k  & F_{kl} \delta_{i'j} - F_{jl} \delta_{i'k} & v_j \delta_{kk'} \delta_{ll'} & 0  \\
    -c_{rs} v_j - \frac{1}{2\rho} (\delta_{jr}\delta_{i's} + \delta_{js} \delta_{i'r}) &  c-\frac{1}{2}(\delta_{jr} \delta_{i's} +\delta_{js}\delta_{i'r}) & 0 & v_j \delta_{rr'} \delta_{ss'} \end{array} \right), 
\end{eqnarray*}
计算$s$的Hessian矩阵为
\begin{equation*}
    s_{UU} = \left( \begin{array}{cccc}
        p_{\nu} - \frac{c:c}{\nu^3} & 0 & 0 & \frac{1}{2\nu^2} c_{r's'} \\
        0 & - \delta_{ii'} & 0 & 0 \\
        0 & 0 & -\delta_{kk'} \delta_{ll'} & 0 \\
        \frac{1}{2\nu^2} c_{rs} & 0 & 0 & -\frac{1}{2\nu} \delta_{rr'} \delta_{ss'}
    \end{array}\right)
\end{equation*}
从而$A_0 = -s_{UU}$与$A_j$的乘积为
\begin{equation*}
    A_0(U) A_j(U) = 
\end{equation*}


Maxwell迭代一阶近似方程组的整体存在性。首先由\eqref{eq:compressibleRelax}中$c$的方程得到
\begin{eqnarray*}
    (\rho \dot{c})_t + \nabla \cdot (\rho \dot{c}) -  \nabla \cdot v  = - \frac{1}{\kappa} (\rho  \dot{c}), \\
    (\rho \mathring{c})_t + \nabla \cdot (\rho \mathring{c}) -  \frac{1}{2} (\nabla v + (\nabla v)^T  - \frac{2}{3} \nabla \cdot v I)  = - \frac{1}{\xi} (\rho  \mathring{c}).
\end{eqnarray*}
可以写为
\begin{eqnarray*}
    \rho \dot{c} = - \kappa\left( (\rho \dot{c})_t + \nabla \cdot (\rho \dot{c}) -  \nabla \cdot v\right), \\
    \rho \mathring{c} = - \xi \left( (\rho \mathring{c})_t + \nabla \cdot (\rho \mathring{c}) -  \frac{1}{2} (\nabla v + (\nabla v)^T  - \frac{2}{3} \nabla \cdot v I) \right).
\end{eqnarray*}
假设$\kappa,\xi$很小,迭代一次得到
\begin{equation*}
    \rho \dot{c} = \kappa \nabla \cdot v + O(\kappa^2), \ \rho \mathring{c} =  \frac{\xi}{2} (\nabla v + (\nabla v)^T  - \frac{2}{3} \nabla \cdot v I) + O(\xi^2).
\end{equation*}
代入方程\eqref{eq:compressibleRelax}中可以得到
\begin{subequations}\label{eq:compressible}
  \begin{align}
  \rho_t + \nabla \cdot (\rho  v ) = 0, \\
  (\rho  v )_t + \nabla \cdot ( \rho  v  \otimes  v ) + \nabla p = \nabla \cdot (\rho F F^T) + \mu \Delta  v  + \mu' \nabla \nabla \cdot  v , \\
  (\rho F)_t + \nabla \cdot (F \otimes \rho  v ) = (\nabla  v ) \rho F
\end{align}
\end{subequations}
其中$\mu = \kappa,\ \mu'=\xi + \frac{\kappa}{3}$均大于$0$。

下面我们将考虑该方程组在平衡态附近整体解的存在性。首先我们将说明该方程组可以看作对称双曲-抛物系统。从而其整体存在性可由Kawashima等人的理论\cite{}进行研究。然而通过验证,Kawashima条件并不成立。幸运的是我们可以通过力学上的适应性条件来弥补这一缺陷,从而可以采用Kawashima的一般框架得到存在性定理的证明。

为了方便本小节假设$U = (\rho, v ,F)$,而采用$W$表示上文中的$U$,即$W=(\rho, \rho v ,\rho F)$。平衡态取$U_e$定义为$\rho=\rho_e>0,  v =0 $和$F=I_{n^2}$ ($I_k$表示$k$阶单位矩阵)。本节的主要结果如下。
\begin{theorem}\label{theoremcom}
令正整数$s > \frac{n}{2}+1$。假设$U_0-U_e\in H^s$且$\|U_0-U_e\|_{H^s}$足够。并且$U_0 = U_0(x)$满足适应性条件\eqref{eq:compatibility1}和\eqref{eq:compatibility2}。那么方程组\eqref{eq:compressible}以$U_0$为初值的Cauchy问题存在唯一整体解$U=U(x,t)$,满足
    \begin{eqnarray}\label{eq:thmincom}
       U - U_e\in C([0,+\infty), H^s) \cap L^2([0,+\infty), H^{s}), \nonumber \quad
         v \in L^2([0,+\infty), H^{s+1}),\\[2mm]
      \|U(T)-U_e\|_{H^s}^2 +  \int_0^T \left[\|\nabla  v (t) \|_{H^s}^2 + \|\nabla U(t)\|_{H^{s-1}}^2\right] dt
      \le C \|U_0-U_e\|_{H^s}^2 .
    \end{eqnarray}
\end{theorem}

实际上,本定理已经在\cite{qian2010global,hu2011global}中得到证明。对于不可压缩的情形,文献\cite{lin2005hydrodynamics,lei2008global,chen2006global}也证明了类似的定理。在这里我们将给出一个新的证明。这一证明利用了双曲-抛物方程的Kawashima理论,给出了统一的证明方法,阐释了整体存在性成立的原因。并且这里提出的证明方法对于不可压方程仍成立。另外,我们还分析了不同适应性条件,解释了文献\cite{lin2005hydrodynamics,lei2008global,chen2006global,qian2010global,hu2011global}中不同适应性条件下整体存在性成立的原因。

我们的证明将方程\eqref{eq:compressible}的结构特点和细致的估计。首先由前面的推导可以得到系统的熵函数存在。然后我们将给出方程\eqref{eq:compressible}的对称子。注意到由有限形变的守恒-耗散理论我们并得不到该方程的对称性质。从而该方程组可以写为对称双曲-抛物方程组。接下来我们将验证其不满足Kawashima等人对双曲-抛物方程组提出的Kawashima条件。为了得到定理\ref{theoremcom},我们利用适应性条件\eqref{eq:compatibility1}和\eqref{eq:compatibility2}来得到整体存在性所需的估计。最后我们将采用\cite{kawashima1984systems,yong2004entropy}中的方法给出证明的过程。

\subsection{熵函数与对称性}
由前面的分析,我们可以得到方程组\eqref{eq:compressible}的熵函数为
\begin{eqnarray}\label{22}
  \eta(\rho,\rho  v ,\rho F) = \rho s(\nu,u,F) =  \rho \int_{\rho_0}^\rho \frac{p(\zeta)}{\zeta^2} d\zeta + \frac{1}{2} \rho | v |^2 + \frac{1}{2} \rho |F|^2.
\end{eqnarray}
$\eta = \eta(W)$为$W$的上凸函数。注意我们这里的$\eta$与前文的熵函数相差一个符号,这是为了之后分析的简便。

由计算可以得到熵函数的时间演化满足
\begin{eqnarray}\label{eq:entropy}
  \eta_t  &=& S_\rho \rho_t + S_{\rho  v } \cdot (\rho  v )_t  + S_{\rho F} : (\rho F)_t \nonumber \\
  &=& -S_{\rho} \nabla \cdot (\rho  v )  - S_{\rho  v  } \cdot [\nabla \cdot (\rho  v \otimes v ) + \nabla p] + S_{\rho  v } \cdot \nabla \cdot \sigma \nonumber \\
    &&- S_{\rho F} : \nabla \cdot (\rho F\otimes  v  ) + S_{\rho F}:(\nabla   v ) \rho F \nonumber \\
    &=& - \nabla \cdot (S  v +p  v  +\sigma^T \cdot v ) + (p+S - \rho S_\rho - \rho  v  \cdot S_{\rho  v } - \rho F : S_{\rho F}) \nabla \cdot  v  \nonumber \\
    &&- (\sigma - S_{\rho F} \rho F^T) : \nabla  v \nonumber \\
    &=& - \nabla \cdot (S  v +p  v  +\sigma^T \cdot v ) - \mu|\nabla  v |^2 - \mu'|\nabla\cdot v |^2.
\end{eqnarray}
从而$\eta$满足
\begin{eqnarray}\label{eq:entropyvol}
  \eta_t(W) = -\nabla\cdot J(W) + \sigma^S.
\end{eqnarray}
其中熵流$J(W)\equiv S  v +p  v  +\sigma^T \cdot v $,熵的产生率为$\sigma^S \equiv - \mu|\nabla  v |^2 - \mu'|\nabla\cdot v |^2 \le 0$。从而满足热力学第二定律。

利用\eqref{eq:compatibility1},方程\eqref{eq:compressible}可以写成如下的坐标形式
\begin{eqnarray*}
  \left( \begin{array}{c} \rho \\ v_i \\ F_{kl} \end{array} \right)_t
  +
  \left( \begin{array}{ccc} v_j & \rho \delta_{i'j} & 0 \\
    \frac{p_\rho}{\rho} \delta_{ij} & v_j \delta_{ii'} & -F_{jl'} \delta_{ik'} \\
    0 & - F_{jl}\delta_{ki'}  & v_j \delta_{kk'} \delta_{ll'} \end{array} \right)
  \left( \begin{array}{c} \rho \\ v_{i'} \\ F_{k'l'} \end{array} \right)_{x_j}
=
\left( \begin{array}{c} 0 \\ \frac{\mu}{\rho} \Delta {v_i} + \frac{\mu'}{\rho} \partial_{x_i} \partial_{x_j} v_j  \\ 0 \end{array} \right).
\end{eqnarray*}
即
\begin{eqnarray}\label{eq:symmetrichyperbolic}
  U_t + \sum_{j=1}^d A_j(U) U_{x_j} = \mathcal{Q}[U].
\end{eqnarray}
其中
\begin{eqnarray*}
  Q[U] = \frac{1}{\rho}\left( \begin{array}{c} 0\\ \mu\Delta  v + \mu'\nabla \nabla \cdot  v  \\ 0_{d^2} \end{array} \right).
\end{eqnarray*}
取正定对称矩阵
\begin{eqnarray}
  A_0(U) = \left( \begin{array}{ccc} \frac{p_\rho}{\rho^2} & 0 & 0 \\
                                                   0 & I_d & 0 \\
						   0 & 0 & I_{d^2}\end{array} \right),
\end{eqnarray}
从而可以验证$A_0(U) A_j(U)$对称,即方程组\eqref{eq:symmetrichyperbolic}存在对称子$A_0$。

\subsubsection{Kawashima条件}
下面我们将说明对于方程组\eqref{eq:compressible},Kawashima条件在平衡点$U_e$不成立。但是Kawashima等人论文\cite{}中的类似估计可以利用适应性条件\eqref{eq:compatibility1}与\eqref{eq:compatibility2}得到。

首先我们检验Kawashima条件。首先我们考虑平衡点$U_e$附近双曲-抛物组\eqref{eq:symmetrichyperbolic}的线性方程。
\begin{eqnarray}\label{eq:symmtrickawashima}
  U_t + \sum_{j=1}^n A_j(U_e) U_{x_j} -\sum_{j=1}^n \sum_{k=1}^n D_{jk}(U_e) U_{x_j x_k}=  0.
\end{eqnarray}
其中
\begin{eqnarray*}
  D_{jk}(U) = \left( \begin{array}{ccc} 0 & 0 & 0 \\ 0 & \frac{\mu}{\rho} \delta_{jk} + \frac{\mu'}{\rho} \delta_{ij}\delta_{ki'}& 0 \\ 0 & 0 & 0 \end{array} \right).
\end{eqnarray*}
令$\xi=(\xi_1, \xi_2, \cdots, \xi_d)\in S^{n-1}$($S$表示单位球)。这里我们取Kawashima条件的一个形式为
\begin{itemize}
    \item 矩阵$ \sum_{j=1}^n \xi_j A_j(U_e)$的特征向量不在矩阵$\sum_{j=1}^n \sum_{k=1}^n \xi_j \xi_k D_{jk}(U_e)$的零空间中。
\end{itemize}

假设$\hat{U} = (\hat{\rho}, \hat{ v },\hat{F})$位于矩阵$\sum_{j=1}^n \sum_{k=1}^n  \xi_j \xi_k D_{jk}(U_e)$的零空间中,那么有
\begin{equation*}
    \frac{\mu}{\rho_e} |\xi|^2 \hat{v}_i + \frac{\mu'}{\rho_e} \xi_i \xi_{i'} \hat{v}_{i'} = 0.
\end{equation*}
假设$v_{i} \neq = 0$,$v$与$\xi$方向相同,从而存在数$m \neq 0$,使得$v_i = m \xi_i$,代入上式得到
\begin{equation*}
    \mu |\xi|^2 \hat{v}_i + \mu' |\xi|^2 \hat{v}_i = 0. 
\end{equation*}
从而$v=0$。假设$\hat{U}$也是矩阵$ \sum_{j=1}^n \xi_j A_j(U_e)$的特征向量。从而存在数$\lambda$,使得
\begin{eqnarray*}
    \sum_{j=1}^n \xi_j A_j(U_e)\hat{U} = \lambda \hat{U}, \\
     \left( \begin{array}{ccc} 0 & \rho_e \xi_{i'} & 0 \\
    \frac{p_\rho(\rho_e)}{\rho_e} \xi_i & 0 & -\xi_{l'} \delta_{ik'} \\
    0 & - \xi_{l}\delta_{ki'}  & 0 \end{array} \right)
    \left(\begin{array}{c}
        \hat{\rho} \\ \hat{ v}_{i'} \\ \hat{F}_{k'l'}
    \end{array} \right) = \lambda
    \left(\begin{array}{c}
        \hat{\rho} \\ \hat{ v }_i \\ \hat{F}_{kl}
    \end{array} \right).
\end{eqnarray*}
我们得到
\begin{eqnarray*}
    \rho_e \xi \cdot \hat{v}  = \lambda \hat{\rho} , \\
    \frac{p_\rho(\rho_e)}{\rho_e} \hat{\rho} \xi_i - \hat{F}_{il} \xi_l = \lambda \hat{v}_i, \\
    -\xi_l \hat{v}_k = \lambda \hat{F}_{kl}.
\end{eqnarray*}
将$\hat{v}=0$代入可以得到$\lambda = 0$(由$\xi \neq 0$与第三个式子)。从而任意特征向量$U$满足
\begin{eqnarray} \label{eq:linear}
    \hat{ v }=0, \quad \xi_i \frac{p_\rho(\rho_e)}{\rho_e} \hat{\rho} - \xi_{l'} \hat{F}_{il'} = 0.
\end{eqnarray}
例如$\hat{U} = (1,0,\frac{p_\rho(\rho_e)}{\rho_e} I_{n^2})$即满足既是矩阵$ \sum_{j=1}^n \xi_j A_j(U_e)$的特征向量又在矩阵$\sum_{j=1}^n \sum_{k=1}^n \xi_j \xi_k D_{jk}(U_e)$的零空间中。从而Kawashima条件对于\eqref{eq:compressible}不成立。

然而我们可以利用适应性条件\eqref{eq:compatibility1}和\eqref{eq:compatibility2}来弥补Kawashima条件的缺失。在定义下面的线性算子
\begin{eqnarray}\label{eq:cformula}
{\mathcal C}_1(U) = & -\nabla\rho - \rho_e \nabla\cdot F^T, \nonumber \\
{[{\mathcal C}_2(U)]}_{kmj} = & \partial_{x_m} F_{kj} - \partial_{x_j} F_{km}
\end{eqnarray}
分别为适应性条件\eqref{eq:compatibility1}和\eqref{eq:compatibility2}在平衡点附近的线性部分。假设上面的$\hat{U}$也在算子${\mathcal C}_1(U)$和${\mathcal C}_2(U)$的符号矩阵的零空间中,我们有
\begin{eqnarray}
\xi_i \hat{\rho} + \rho_e \xi_{j} \hat{F}_{ji} = 0 , \label{eq:linear1}\\
\xi_{m} \hat{F}_{kj} - \xi_{j} \hat{F}_{km} =0. \label{eq:linear2}
\end{eqnarray}
将\eqref{eq:linear}乘以$\xi_i$,并对$i$求和。利用\eqref{eq:linear1},我们可以推出
$$
|\xi|^2 \frac{p_\rho(\rho_e)}{\rho_e} \hat{\rho}  =  \xi_i \xi_{l'} \hat{F}_{il'}  = - \xi_{l'} \frac{\xi_{l'}  }{\rho_e} \hat{\rho} = -\frac{|\xi|^2}{\rho_e} \hat{\rho},
$$
从而$\hat{\rho}=0.$将此结果代入\eqref{eq:linear}给出
 $\xi_{l'} \hat{F}_{il'} = 0 .$
乘以$\hat{F}_{im}$并对$i$求和,利用\eqref{eq:linear2}可以导出
$$0=\xi_{l'} \hat{F} _{il'} \hat{F}_{im} = \xi_m \hat{F} _{il'}\hat{F} _{il'}.$$
从上面的等式由$\xi \neq 0$可以知道$\hat{F}=0$。这样对应零特征值的非零特征向量$U$不在线性算子${\mathcal C}_1(U)$ and ${\mathcal C}_2(U)$的符号矩阵的零空间中。基于这样的结果,我们期望Kawashima条件在适应性条件的约束下成立。

在文献\cite{shizuta1985systems}中,Kawashima条件等价于存在反对称矩阵$K=K(U_e, \xi)$,使得
\begin{eqnarray*}
  K\sum_{j}\xi_j A_j(U_e) - \sum_{j}\xi_j A_j^T(U_e)K + \sum_{j, k}\xi_j\xi_kD_{jk}(U_e)
\end{eqnarray*}
对任意$\xi\in S^{d-1}$是正定的。我们上面的分析表明这样的$K$不存在,但是在适应性条件\eqref{eq:compatibility1}与\eqref{eq:compatibility2}的帮助下,成立下面的引理。
\begin{lemma}\label{lemmaK}
取反对称矩阵
$$
K_j = \mbox{diag}\left(\frac{p'(\rho_e)}{\rho_e^2}, -I_d, I_{d^2}\right)A_j(U_e),
$$
那么存在正常数$\eta$和$C_S$,使得对任意的光滑函数$U =U(x)$,下面的不等式成立。
\begin{eqnarray}\label{eq:prop}
  &&\sum_{j,m=1}^d [( \eta K_m A_j(U_e) U_{x_j},U_{x_m}) + (D_{mj}(U_e) U_{x_j},U_{x_m})]\nonumber \\
  &\ge& C_S \|\nabla U \|_{L^2}^2 +\eta\frac{2p'(\rho_e)}{\rho_e^2}({\mathcal C}_1(U - U_e), \nabla \rho) + \eta({\mathcal C}_2(U - U_e), \nabla F).
\end{eqnarray}

\end{lemma}

\begin{proof}
由\eqref{eq:compressible},我们有
$$
\sum_j A_j(U_e) U_{x_j}=\left( \begin{array}{cc} \rho_e\nabla\cdot\mathbf v\\ \frac{p'(\rho_e)}{\rho_e}\nabla\rho - \nabla\cdot F\\ -\nabla \mathbf v\end{array} \right).
$$
于是成立
$$
-\sum_j K_jU_{x_j}=-\mbox{diag}\left(\frac{p'(\rho_e)}{\rho_e^2}, -I_d, I_{d^2}\right)\sum_j A_j(U_e) U_{x_j}=\left( \begin{array}{cc} -\frac{p'(\rho_e)}{\rho_e}\nabla\cdot\mathbf v\\ \frac{p'(\rho_e)}{\rho_e}\nabla \rho - \nabla\cdot F\\ \nabla \mathbf v\end{array} \right).
$$
下面我们计算 (省略下标$e$)
\begin{eqnarray*}
   && \sum_{m,j=1}^d (K_m  A_j(U_e) U_{x_j},U_{x_m}) = -(\sum_j A_j(U_e) U_{x_j}, \sum_m K_m U_{x_m})\\
  &=& - p' \|\nabla \cdot  v \|_{L^2}^2 - \|\nabla v \|_{L^2}^2+ \|\frac{p'}{\rho} \nabla \rho \|_{L^2}^2 - 2\frac{p'}{\rho} (\nabla\rho, \nabla\cdot F) +  \|\nabla\cdot F\|_{L^2}^2  \\
  &=& - p' \|\nabla \cdot  v \|_{L^2}^2 - \|\nabla v \|_{L^2}^2+ \|\frac{p'}{\rho} \nabla \rho \|_{L^2}^2 -  \frac{2p'}{\rho}( \partial_{x_j} \rho,\partial_{x_m} F_{jm}) + ( \partial_{x_j} F_{kj},\partial_{x_m} F_{km})\\
     &=& - p' \|\nabla \cdot  v \|_{L^2}^2 - \|\nabla v \|_{L^2}^2+ \|\frac{p'}{\rho} \nabla \rho \|_{L^2}^2-  \frac{2p'}{\rho} (\partial_{x_m} \rho,\partial_{x_j} F_{jm}) +  ( \partial_{x_m} F_{kj},\partial_{x_j} F_{km})\\
     &=&  - p' \|\nabla \cdot  v \|_{L^2}^2 - \|\nabla v \|_{L^2}^2+ \|\frac{p'}{\rho} \nabla \rho \|_{L^2}^2-  \frac{2p'}{\rho}(\nabla \rho,\nabla\cdot F^T)\\
     && + \|\nabla F\|_{L^2}^2
     + ( \partial_{x_m} F_{kj} - \partial_{x_j} F_{km},\partial_{x_j} F_{km}) \\
     &=&  - p' \|\nabla \cdot  v \|_{L^2}^2 - \|\nabla v \|_{L^2}^2+ \frac{p'^2 + 2p'}{\rho^2} \|\nabla \rho \|_{L^2}^2+ \|\nabla F\|_{L^2}^2 \\
     &&+ \frac{2p'}{\rho^2}(\nabla \rho, {\mathcal C}_1(U - U_e)) + ({\mathcal C}_2(U - U_e), \nabla F).
\end{eqnarray*}
这样下面的式子成立
\begin{eqnarray*}
 && \sum_{j,m=1}^d[ ( \eta K_m A_j(U_e) U_{x_j},U_{x_m}) + (D_{mj}(U_e) U_{x_j},U_{x_m})] \\
  &=& \eta\frac{p'^2 + 2 p'}{\rho^2} \|\nabla \rho \|_{L^2}^2 + (\frac{\mu}{\rho} - \eta   ) \|\nabla v \|_{L^2}^2+ (\frac{\mu'}{\rho} - \eta  p' ) \|\nabla \cdot  v \|_{L^2}^2+ \eta \|\nabla F\|_{L^2}^2\\
  && + \eta\frac{2p'}{\rho^2}(\nabla \rho, {\mathcal C}_1(U - U_e)) + \eta({\mathcal C}_2(U - U_e), \nabla F).
\end{eqnarray*}
选取$\eta=\min\{ \frac{\mu}{2\rho}, \frac{\mu'}{2\rho p'}\}$可得\eqref{eq:prop}成立。这样我们证明了引理\ref{lemmaK}。
\end{proof}{}

\begin{remark}
引理\ref{lemmaK}中利用了适应性条件\eqref{eq:compatibility1}和\eqref{eq:compatibility2}。实际上我们亦可以采用适应性条件\eqref{eq:compatibility3}得到类似的结果。下面我们分析这三个适应性条件对Kawashima条件的影响。首先适应性条件\eqref{eq:compatibility1}可由\eqref{eq:compatibility2}和\eqref{eq:compatibility3}导出。实际上,由
Jacobi公式
$$
\partial_{x_i}\det A = \det A \mbox{Tr}(A^{-1} \partial_{x_i} A),
$$
成立
\begin{eqnarray*}
\nabla \cdot (\rho F^T)  & = &\partial_{x_i} ( \rho F_{ij}) = \partial_{x_i} ( \frac{1}{\det F} F_{ij})  \\
&=& \frac{1}{\det F} \partial_{x_i} F_{ij}  - \frac{1}{(\det F)^2} F_{ij} \det F \mbox{Tr}(F^{-1} \partial_{x_i} F) \\
&=&\frac{1}{\det F} (\partial_{x_i} F_{ij} - F_{ij} (F^{-1})_{mn} \partial_{x_i} F_{nm}) \\
&=& \frac{1}{\det F} (\partial_{x_i} F_{ij} - (F^{-1})_{mn} F_{ij} \partial_{x_i} F_{nm}) \\
&=&  \frac{1}{\det F} (\partial_{x_i} F_{ij} - (F^{-1})_{mn} F_{im} \partial_{x_i} F_{nj}) \\
&=&  \frac{1}{\det F} (\partial_{x_i} F_{ij} - \delta_{in} \partial_{x_i} F_{nj}) \\
&=&  \frac{1}{\det F} (\partial_{x_i} F_{ij} - \partial_{x_i} F_{ij}) =0 ,
\end{eqnarray*}
从而我们可知对适应性条件\eqref{eq:compatibility2}和\eqref{eq:compatibility3},引理\ref{lemmaK}成立。

另外,在二维的情况我们可以证明引理\ref{lemmaK}对适应性条件\eqref{eq:compatibility1}和\eqref{eq:compatibility3}也成立。实际上,对条件\eqref{eq:compatibility3}求导得到
$$
0=\nabla (\rho \det F) = \nabla \rho \det F + \rho \det F \mbox{Tr}(F^{-1} \nabla F).
$$
其线性化算子为
\begin{equation*} \label{3}
\mathcal{C}_3(U):=-\nabla \rho - \rho_e \nabla \mbox{Tr}(F).
\end{equation*}
从而二维$n=2$时$\mathcal{C}_2(U)$的唯一两个独立量可以用$\mathcal{C}_1$和$\mathcal{C}_3$表示如下。
\begin{eqnarray*}
\mathcal{C}_2(U)_{112}&=&\partial_{x_1}F_{12} - \partial_{x_2}F_{11}= {\mathcal{C}_3(U)}_2 - {\mathcal{C}_1(U)}_2, \\ \mathcal{C}_2(U)_{221}&=&\partial_{x_2}F_{21}-\partial_{x_1}F_{22} = {\mathcal{C}_3(U)}_1 - {\mathcal{C}_1(U)}_1.
\end{eqnarray*}
这表明算子${\mathcal C}_1(U)$同${\mathcal C}_3(U)$亦可以排除不满足Kawashima条件的非零特征向量$\hat{U}$。

然而在维数$n \ge 3$的情况,适应性条件\eqref{eq:compatibility1}和\eqref{34}并不能保证不满足Kawashima条件的非零特征向量$\hat{U}$不存在,例如$n=3$时,对$\xi = (1,0,0)$,我们可以验证$(0,0, G)$,
 $$
 G = \left( \begin{array}{ccc}
 0 & 0 & 0 \\
 0 & 1 & 0 \\
 0 & 0 & -1 \end{array} \right)
 $$
是矩阵$ \sum_{j=1}^n \xi_j A_j(U_e)$的零特征值对应的一个特征向量。从而引理\ref{lemmaK}对适应性条件\eqref{eq:compatibility1}和\eqref{eq:compatibility3}在大于二维时不成立。这也给出了文献\cite{}和\cite{}中二维和三维假设不同适应性条件的一个解释。
\end{remark}

\subsubsection{定理\ref{theoremcom}的证明}
因为方程组\eqref{eq:compressible}可以对称化,其解的局部存在性成立\cite{kawashima1984systems}。另外方程组\eqref{eq:compressible}的局部存在性定理的证明也可以在文献\cite{lin2005hydrodynamics,lei2008global,hu2013global}中找到。下面我们将推导定理\ref{theoremcom}中的\emph{先验估计}\eqref{eq:thmincom}。

证明分三步,第一步利用熵的演化方程得到$U-U_e$的$L^2$估计,然后求导得到高阶导数的估计。最后利用引理\ref{lemmaK}完成证明。

\emph{第一步:}令
\begin{eqnarray*}
  G(W) = S(W) - S(W_e) - S_W(W_e) (W-W_e).
\end{eqnarray*}
其中$W_e = (\rho_e, 0_d, \rho_e I_{d^2})$。由熵函数的上凸性可知存在常数$C_1$ and $C_2$,使得
\begin{eqnarray*}
  C_1 |W-W_e|^2 \le G(W) \le C_2 |W-W_e|^2, \quad C_1 |U-U_e|^2 \le |W-W_e|^2 \le C_2 |U-U_e|^2
\end{eqnarray*}
对于靠近$U_e$的$U$成立。于是根据\eqref{eq:entropy}与\eqref{eq:rhoF},我们得到
\begin{eqnarray*}
  G(W)_t + \nabla \cdot (J(W) + \phi(\rho_e) \rho  v  + \rho (\mbox{Tr}(F) - F) v ) = -\mu |\nabla  v |^2 - \mu' |\nabla \cdot  v |^2.
\end{eqnarray*}
其中$J(W)$同\eqref{eq:entropyvol}中的熵流,$\phi(\rho) = \frac{p(\rho)}{\rho} + \int_{\rho_0}^{\rho}\frac{p(\zeta)}{\zeta^2}d\zeta - \frac{d}{2}$。对此等式在$x\in\mathbf{R}^n$和$t\in[0, T]$上积分给出下面的估计。
\begin{eqnarray}\label{eq:EnergyESTcom}
  \|U(T) - U_e\|_{L^2}^2 + c \int_0^T[ \mu \|\nabla v (t)\|_{L^2}^2 +  \mu' \|\nabla \cdot  v (t)\|_{L^2}^2] dt
  \le C\|U_0-U_e\|_{L^2}^2 .
\end{eqnarray}

\emph{第二步:}将$\partial_x^\alpha$($\alpha$为多重指标且满足$|\alpha|\le s$)作用到方程\eqref{eq:symmetrichyperbolic}两边得到
\begin{eqnarray}\label{eq:higher}
    \partial_x^\alpha U_t + \sum_{j=1}^d A_j(U) \partial_x^\alpha U_{x_j} = \partial_x^\alpha Q[U] + \sum_{j=1}^d [A_j(U),\partial_x^\alpha] U_{x_j}.
\end{eqnarray}
与$A_0(U) \partial_x^\alpha U$取$L^2$内积得到
\begin{eqnarray}\label{eq:HighOrderInner}
  (A_0(U) \partial_x^\alpha U, \partial_x^\alpha U)_t
  %+ \sum_{j=1}^d \int (\partial_x^\alpha U^T A_0(U) A_j(U) \partial_x^\alpha U)_{x_j} dx
 & = & 2(A_0(U) \partial_x^\alpha Q[U],\partial_x^\alpha U) + 2 \sum_{j=1}^d (A_0(U) [A_j(U),\partial_x^\alpha]U_{x_j},\partial_x^\alpha U)  \nonumber\\
  &&+ \left( \big( \partial_t A_0(U) + \sum_{j=1}^d \partial_{x_j} (A_0(U) A_j(U))\big) \partial_x^\alpha U,\partial_x^\alpha U \right)  \\
  &\equiv& I_1 + I_2 + I_3 .  \nonumber
\end{eqnarray}
右端项可以采用Sobolev演算不等式\cite{majda2012compressible}估计如下。
\begin{eqnarray*}
 && I_1 = 2(\partial_x^\alpha(\frac{\mu}{\rho} \Delta  v  + \frac{\mu'}{\rho} \nabla \nabla \cdot  v ), \partial_x^\alpha  v ) \\
  &=& 2(\partial_x^\alpha (\frac{\mu}{\rho_e} \Delta  v  + \frac{\mu'}{\rho_e} \nabla \nabla \cdot  v ), \partial_x^\alpha  v ) + 2( \partial_x^\alpha ( \mu (\frac{1}{\rho}-\frac{1}{\rho_e}) \Delta  v  + \mu'(\frac{1}{\rho} - \frac{1}{\rho_e}) \nabla \nabla \cdot  v ), \partial_x^\alpha v ) \\
  &=& 2(\partial_x^\alpha (\frac{\mu}{\rho_e} \Delta  v  + \frac{\mu'}{\rho_e} \nabla \nabla \cdot  v ), \partial_x^\alpha  v )
  - 2( \partial_x^{\alpha-1} ( \mu (\frac{1}{\rho}-\frac{1}{\rho_e}) \Delta  v  + \mu'(\frac{1}{\rho} - \frac{1}{\rho_e}) \nabla \nabla \cdot  v ),\partial_x^{\alpha +1}  v ) \\
  &\le& -\frac{2\mu}{\rho_e} \|\partial_x^\alpha \nabla  v \|_{L^2}  -\frac{2\mu'}{\rho_e} \|\partial_x^\alpha \nabla \cdot  v \|_{L^2} + C(\mu+\mu')\| \rho -\rho_e \|_{H^s} \|\nabla  v \|_{H^s}^2 ,
\end{eqnarray*}
\begin{eqnarray*}
  I_2 &\le& 2|A_0(U)|_{L^\infty} \|\sum_{j=1}^d [A_j(U),\partial_x^\alpha]U_{x_j} \|_{L^2} \|\partial_x^\alpha U\|_{L^2} \\
  &\le& C \|\partial_x^\alpha U\|_{L^2} \sum_{j=1}^d (|\nabla A_j(U)|_{L^\infty} \|D^{s-1}_x U_{x_j}\|_{L^2}+ |U_{x_j}|_{L^\infty} \|D^s_x A_j(U)\|_{L^2}) \\
  &\le& C \|\partial_x^\alpha U\|_{L^2} \|\nabla U\|_{H^{s-1}}^2 \le C\|\nabla U\|_{H^{s-1}}^3,
\end{eqnarray*}
及
\begin{eqnarray*}
  I_3 &\le& C (|\rho_t|_{L^\infty} + |\nabla U|_{L^\infty}) (\partial_x^\alpha U,\partial_x^\alpha U) \\
      &\le& C |\nabla U|_{L^\infty} \|\partial_x^\alpha U\|_{L^2}^2 \\
      &\le& C \|\nabla U\|_{H^{s-1}} \|\nabla U\|_{H^{s-1}}^2.
\end{eqnarray*}
这里对$I_3$的估计利用了\eqref{eq:compressible}中密度的方程以及$A_0(U)$仅仅依赖于$\rho$这个事实。

将$I_1, I_2$和$I_3$的估计代入\eqref{eq:HighOrderInner},我们得到
\begin{eqnarray*}
  & (A_0(U)\partial_x^\alpha U,\partial_x^\alpha U)_t + \frac{2\mu}{\rho_e} \|\partial_x^\alpha \nabla  v \|_{L^2}  + \frac{2\mu'}{\rho_e} \|\partial_x^\alpha \nabla \cdot  v \|_{L^2}   \\
\le &C \| \nabla U\|_{H^{s-1}}^3 + C(\mu+\mu')\|\rho -\rho_e\|_{H^s} \|\nabla  v \|_{H^s}^2.\nonumber
\end{eqnarray*}
在$ t\in [0,T]$上对上述不等式积分并对$1 \le |\alpha| \le s$相加,结合\eqref{eq:EnergyESTcom},我们最终得到
\begin{eqnarray}\label{eq:Step2Resultcom}
  && \| U(T) - U_e\|_{H^s}^2 + c\int_0^T \|\nabla  v  (t)\|_{H^s}^2 dt  \\
  &\le& C \|U_0 - U_e\|_{H^s}^2+ C (\mu + \mu') \sup_{t \in [0,T]} (\|\rho(t) -\rho_e\|_{H^s}) \int_0^T \|\nabla  v (t)\|_{H^s}^2 dt \nonumber\\
  &&+ C \sup_{t \in [0,T]} (\|U(t) - U_e\|_{H^s}) \int_0^T \|\nabla U(t)\|_{H^{s-1}}^2 dt . \nonumber
\end{eqnarray}

\emph{第三步:}为了控制\eqref{eq:Step2Resultcom}的最后一项,我们利用引理\ref{lemmaK}。首先将\eqref{eq:higher}写为下面的形式。
\begin{eqnarray*}
  \partial_x^\alpha  U_t + \sum_{j=1}^d A_j(U_e) \partial_x^\alpha  U_{x_j}
 = \sum_{j,k=1}^d  \partial_x^\alpha ( D_{jk} (U)U_{x_i x_j} ) + \sum_{j=1}^d \partial_x^\alpha ((A_j(U_e) - A_j(U))U_{x_j}).
\end{eqnarray*}
其中$\alpha$满足$0\le|\alpha|\le s-1$。回忆引理\ref{lemmaK}中的反对称矩阵$K$。我们将上述等式与$-K_m\partial_x^\alpha U_{x_m}$做内积并将得到的结果对$m \in [1,d]$求和得到
\begin{eqnarray} \label{eq:KsDU}
  && \quad \sum_{m=1}^d  (K_m \partial_x^\alpha U_t,\partial_x^\alpha U_{x_m}) + \sum_{m,j=1}^d (K_m A_j(U_e) \partial_x^\alpha U_{x_j}, \partial_x^\alpha U_{x_m}) \\
  &=&  \sum_{m,j=1}^d(K_m\partial_x^\alpha ( (A_j(U_e) - A_j(U)) U_{x_j}), \partial_x^\alpha U_{x_m}) \nonumber \\
 && +
  \sum_{m,j,k=1}^d (K_m \partial_x^\alpha (D_{jk} (U) U_{x_i x_j} ),\partial_x^\alpha U_{x_m}) \nonumber \\
 & \le&
  C\sum_j \|\partial_x^\alpha ((A_j(U_e)-A_j(U))U_{x_j})\|_{L^2} \|\partial_x^\alpha \nabla U\|_{L^2} \nonumber \\
  && + \epsilon \|\partial_x^\alpha \nabla U\|_{L^2}^2  + \frac{C}{\epsilon} \sum_{j,k} \| \partial_x^\alpha ([D_{jk} (U_e) + D_{jk} (U) - D_{jk} (U_e)]U_{x_i x_j}) \|_{L^2}^2  \nonumber \\
 &\le& C \|U-U_e\|_{H^s} \|\nabla U\|_{H^{s-1}}^2 + \epsilon \|\partial_x^\alpha \nabla U\|_{L^2}^2 + \frac{C}{\epsilon}  \|\partial_x^{\alpha+1} \nabla  v \|_{L^2}^2  \nonumber \\
&& + \frac{C}{\epsilon}\| U-U_e\|_{H^{s-1}}^2  \|\nabla  v \|_{H^s}^2 . \nonumber
\end{eqnarray}
对等式\eqref{eq:KsDU}左端第一项,我们采用分部积分得到
\begin{eqnarray} \label{eq:L1}
  &&(K_m \partial_x^\alpha U_t,\partial_x^\alpha U_{x_m})  \\
  &=& \frac{1}{2} \int (\partial_x^\alpha U_{x_m}^T K_m \partial_x^\alpha U)_t dx - \frac{1}{2} \int (\partial_x^\alpha U^T K_m \partial_x^\alpha U_t)_{x_m} dx \nonumber \\
  &=& \frac{1}{2} (K_m \partial_x^\alpha U,\partial_x^\alpha U_{x_m})_t. \nonumber
\end{eqnarray}

对第二项,由$U$满足适应性条件\eqref{eq:compatibility1}和\eqref{eq:compatibility2},我们得到
\begin{eqnarray} \label{c}
{\mathcal C}_1(U - U_e) &=& \nabla\cdot[(\rho - \rho_e)(F^T - I_{d^2})], \nonumber\\[2mm]
{[{\mathcal C}_2(U - U_e)]}_{kmj} &=& (F_{lj} - \delta_{lj})\partial_{x_l}F_{km} - (F_{lm} - \delta_{lm})\partial_{x_l}F_{kj}.
\end{eqnarray}
其中${\mathcal C}_1$和${\mathcal C}_2$的定义见\eqref{eq:cformula}。这样由引理\ref{lemmaK}和\eqref{c},成立
\begin{eqnarray}\label{eq:L2com}
 && \sum_{j,m} (K_m A_j(U_e) \partial_x^\alpha U_{x_j}, \partial_x^\alpha U_{x_m}) = \\
  &&\sum_{j,m}[ (K_m A_j(U_e) \partial_x^\alpha U_{x_j}, \partial_x^\alpha U_{x_m}) + (D_{mj}(U_e) \partial_x^\alpha U_{x_j}, \partial_x^\alpha U_{x_m}) ] \nonumber \\
  && - \sum_{m,j} (D_{mj}(U_e) \partial_x^\alpha  U_{x_j}, \partial_x^\alpha U_{x_m}) \nonumber \\
   &\ge& C_S \|\partial_x^\alpha \nabla U\|_{L^2}^2 - \frac{\mu}{\rho_e} \|\partial_x^\alpha \nabla  v \|_{L^2}^2 -\frac{\mu'}{\rho_e} \|\partial_x^\alpha \nabla \cdot  v \|_{L^2}^2\nonumber \\
  &&+ \eta\frac{2p'(\rho_e)}{\rho_e^2}(\partial_x^\alpha{\mathcal C}_1(U - U_e), \nabla \partial_x^\alpha\rho) + \eta(\partial_x^\alpha{\mathcal C}_2(U - U_e), \nabla \partial_x^\alpha F), \nonumber\\
  &\ge& C_S \|\partial_x^\alpha \nabla U\|_{L^2}^2 - \frac{\mu}{\rho_e} \|\partial_x^\alpha \nabla  v \|_{L^2}^2 -\frac{\mu'}{\rho_e} \|\partial_x^\alpha \nabla \cdot  v \|_{L^2}^2\nonumber \\
  &&- C\|(\partial_x^\alpha{\mathcal C}_1(U - U_e)\|_{L^2}\|\nabla \partial_x^\alpha\rho\|_{L^2} - C\|\partial_x^\alpha{\mathcal C}_2(U - U_e)\|_{L^2}\|\nabla \partial_x^\alpha F\|_{L^2}\nonumber\\
  &\ge& C_S \|\partial_x^\alpha \nabla U\|_{L^2}^2  - \frac{\mu}{\rho_e} \|\partial_x^\alpha \nabla  v \|_{L^2}^2 -\frac{\mu'}{\rho_e} \|\partial_x^\alpha \nabla \cdot  v \|_{L^2}^2\nonumber \\
  &&- C\|\nabla U\|_{H^{s-1}}^2(\|\nabla \partial_x^\alpha\rho\|_{L^2} + \|\nabla \partial_x^\alpha F\|_{L^2}). \nonumber
\end{eqnarray}

将\eqref{eq:L1}和\eqref{eq:L2com}代入\eqref{eq:KsDU}中,取$\epsilon = \frac{C_S}{2}$并对$0 \le |\alpha| \le s-1$求和得到
\begin{eqnarray*}\label{eq:nablaUcom}
  C_S \|\nabla U\|_{H^{s-1}}^2 &\le& - \sum_{m} (K^m D^l_x U,D^l_x U_{x_m})_t  + C\|U-U_e\|_{H^s} \|\nabla U\|_{H^{s-1}}^2 \nonumber \\
  &&+ C \|\nabla  v \|_{H^s}^2+ C \|U-U_e\|_{H^s}^2 \|\nabla  v \|_{H^s}^2 .
\end{eqnarray*}
在$t \in [0,T]$对这个不等式积分得到
\begin{eqnarray*}
  \int_0^T \| \nabla U(t)\|_{H^{s-1}}^2 \le C \|U(T)-U_e \|_{H^s}^2 + C \|U_0 -U_e\|_{H^s}^2+ C\int_0^T \|\nabla  v \|_{H^s}^2 dt  \nonumber \\
  + C \sup_{t \in [0,T]} \|U(t)-U_e\|_{H^s} \int_0^T \|\nabla U(t)\|_{H^{s-1}}^2dt
  + C \sup_{t \in [0,T]} \|U(t)-U_e\|_{H^s}^2 \int_0^T\|\nabla  v (t)\|_{H^s}^2 dt. \nonumber
\end{eqnarray*}
结合\eqref{eq:Step2Resultcom}给出定理\ref{theoremcom}中的先验估计。从而定理得证。



%    Text of article.

%    Bibliographies can be prepared with BibTeX using amsplain,
%    amsalpha, or (for "historical" overviews) natbib style.
%\bibliographystyle{amsplain}
%    Insert the bibliography data here.
%\bibliography{ref}





%          \section{Introduction}\label{intro}
%
%          Put a general  introduction to your paper here. Separate text
%          sections with other sections.
%\newpage
%          \section{Put title of the next section here}\label{an apprpriate label}
%\newpage
%          %If you have subsections use:
%          \subsection{Subsection title.}\label{another label}
%          Don't forget to give each section, subsection, equation, theorem,
%          corollary, etc. a unique label, and when you refer to the results
%          later in the text use \ref{labelname} instead of explicitly writing
%          the number of the environment in question.
%%\begin{align}\label{labelname}
%%   &  \\
%%   & 
%%\end{align}
%          This use of \label and \ref is REQUIRED for  papers.
%
%          Similarly, always use \cite{taubes1}(biblabelname) to refer to bibliographic
%          references, which would then be entered in the bibliography via
%          %\bibitem[visible label]{<biblabelname>}.
%\newpage
%          %
%          % For figures use
%
%          %\begin{figure}
%
%          %The use of .eps files is encouraged, in which case you should
%          %un-comment the \uspackage{graphics} command above, and use the
%          %command
%          %\include{figure.eps}
%          % to insert the figure file.
%
%          %\end{figure}
%
%
%          % BibTeX users please use
%
%          % \bibliographystyle{}
%
%          % \bibliography{}
%
%          %
%
%          % Non-BibTeX users please use
%
%          \begin{thebibliography}{}
%
%          %
%
%          % and use \bibitem to create references.
%
%          %
%
%          \bibitem{mnemonic}Author, {\em title of paper}, Journal Name
%         Volume, page numbers, year.
%
%          % Format for Journal Reference. For example
%
%          \bibitem{taubes1} C. Taubes, {\em The Seiberg-Witten invariants
%          and
%          symplectic forms}, Math. Res. Letters, 1, 809--822, 1994.
%          \end{thebibliography}


%          \end{document}





\bibliographystyle{amsplain}
%    Insert the bibliography data here.
\bibliography{ref}
\end{document}