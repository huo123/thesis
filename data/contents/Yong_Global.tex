%\documentclass{article}
%\usepackage{ctex}
% \usepackage{amssymb}
% \usepackage{amsmath}
% \usepackage{amsthm}
% \newtheorem{theorem}{定理}
% \newtheorem{remark}{注释}
% \newtheorem{lemma}{引理}
% \begin{document}
%\section{有限形变粘弹性流体模型的数学分析}
% 本小节将讨论等温粘弹性流体模型\eqref{eq:compressibleRelax}解的存在性。首先我们将说明这一方程是可对称对称双曲组,然后利用Kawashima理论证明其解在平衡态附近的全局存在性。最后我们将证明$\kappa,\xi$同时趋于$0$时对应的方程组平衡态附近解的整体存在性。

% \subsection{粘弹性流体模型\eqref{eq:compressibleRelax}解的存在性}
% 我们将讨论方程\eqref{eq:compressibleRelax}解的局部存在性以及平衡态附近解的整体存在性。
% \subsubsection{局部存在性}
% %这里我们用$W=(\rho,\rho v,\rho F,\rho c)$表示上面的$U$。而利用$U$表示$U = (\rho, v, F, 、\rho c)$,
% 首先由有限形变的守恒-耗散理论我们无法得到$\eta_{UU}$可以将$U$的方程对称化。然而由于熵函数是二次的,$\eta_{UU}$是负定的。可以验证$A_0(U) = -\eta_{UU}$可以将$U$的方程对称化。从而方程局部解的存在性可由对称双曲方程的一般理论得到。

% 下面我们将验证$A_0(U) = -\eta_{UU}$可以将$U$的方程对称化。首先将$U$的方程写成下面的形式。 
% \begin{eqnarray*}
% U_t + \sum_{j=1}^n A_j(U) U_{x_j} = \mathcal{Q}(U),\\
%   U = \left( \begin{array}{c} \rho \\ \rho v_i \\ \rho  F_{kl} \\ \rho c_{rs} \end{array} \right), \quad \mathcal{Q}(U) = 
% \left( \begin{array}{c} 0 \\ 0  \\ -\frac{1}{\kappa} \dot{\mathcal{T}}I - \frac{1}{\xi} \mathring{\mathcal{T}} \end{array} \right)U, \quad   A_j(U)= \\
%  % \tiny \left( \begin{array}{cccc} 
%  %     0 & \delta_{i'j} & 0 & 0 \\
%  %     p_\rho - v_i v_j - F_{jl'} F_{il'} & v_i \delta_{i'j} v_j \delta_{i'i} & -\rho F_{jl'} \delta_{ik'} & -\frac{1}{2}(\delta_{jr'} \delta_{is'} +\delta_{js'}\delta_{ir'}) \\
%  %    0 & -F_{kl} v_j + F_{jl} v_k  & F_{kl} \delta_{i'j} - F_{jl} \delta_{i'k} & v_j \delta_{kk'} \delta_{ll'} & 0  \\
%  %    -c_{rs} v_j - \frac{1}{2\rho} (\delta_{jr}\delta_{i's} + \delta_{js} \delta_{i'r}) &  c-\frac{1}{2}(\delta_{jr} \delta_{i's} +\delta_{js}\delta_{i'r}) & 0 & v_j \delta_{rr'} \delta_{ss'} \end{array} \right), 
%  \left(
%  \begin{smallmatrix}
%      0 & \delta_{i'j} & 0 & 0 \\
%      p_\rho \delta_{ij} - v_i v_j + F_{jl'} F_{il'} & v_i \delta_{i'j} + v_j \delta_{i'i} & - F_{jl'} \delta_{ik'} & \rho c_{r's'}-\frac{1}{2}(\delta_{jr'} \delta_{is'} +\delta_{js'}\delta_{ir'}) \\
%     -F_{kl} v_j + F_{jl} v_k  & F_{kl} \delta_{i'j} - F_{jl} \delta_{i'k} & v_j \delta_{kk'} \delta_{ll'} & 0  \\
%     -c_{rs} v_j + \frac{1}{2\rho} (\delta_{jr}v_s+ \delta_{js} v_r) &  c_{rs}-\frac{1}{2\rho }(\delta_{jr} \delta_{i's} +\delta_{js}\delta_{i'r}) & 0 & v_j \delta_{rr'} \delta_{ss'} 
%     \end{smallmatrix}\right).
% \end{eqnarray*}
% 计算$-\eta$的Hessian矩阵为
% \begin{equation*}
%     -\eta_{UU} = \left( \begin{array}{cccc}
%         \frac{p_{\rho} + v^2 + F:F}{\rho} & -\frac{v_i}{\rho} & -\frac{F_{kl}}{\rho} & 0 \\
%         -\frac{v_{i''}}{\rho} &  \frac{1}{\rho}\delta_{ii''} & 0 & 0 \\
%         -\frac{F_{k''l''}}{\rho} & 0 & \frac{1}{\rho} \delta_{kk''} \delta_{ll''} & 0 \\
%         0 & 0 & 0 & \delta_{rr''} \delta_{ss''}
%     \end{array}\right).
% \end{equation*}
% 从而$A_0 = -s_{UU}$与$A_j$的乘积为
% \begin{eqnarray*}
%    && A_0(U) A_j(U) = \\
%    && \left(
%     \begin{smallmatrix}
%         * & \frac{1}{\rho} (p_\rho \delta_{i'j}- v_{i'} v_j + F_{jl} F_{{i'}l})  &  \frac{1}{\rho}(v_{k'} F_{jl'} -  v_j F_{k'l'}) & (14) \\
%         %(14) - c_{r's'}v_i + \frac{1}{2\rho} (v_{s'}\delta_{jr'} + v_{r'} \delta_{js'}) \\
%         \frac{1}{\rho} (p_\rho \delta_{i''j}- v_{i''} v_j + F_{jl'} F_{{i''}l'}) & * & -\frac{1}{\rho} F_{jl'} \delta_{i''k'} & (24) \\
%         %c_{r''s''}-\frac{1}{2\rho}(\delta_{i''s'}\delta_{jr'} + \delta_{i''r'} \delta_{js'})  \\
%         \frac{1}{\rho} (-F_{k''l''} v_j + F_{jl''} v_{k''}) & -\frac{F_{jl}\delta_{ik'}}{\rho} & 0 & 0 \\
%         (41) & (42) & 0 & * 
%         %c_{rs}-\frac{1}{2\rho }(\delta_{jr''} \delta_{i's''} +\delta_{js''}\delta_{i'r''}) & 0 & * 
%         % (41)-c_{r''s''} v_j + \frac{1}{2\rho} (\delta_{jr''} v_{s''} + \delta_{js''} v_{r''}) &  c_{rs}-\frac{1}{2\rho }(\delta_{jr''} \delta_{i's''} +\delta_{js''}\delta_{i'r''}) & 0 & * 
%     \end{smallmatrix}
%      \right)
% \end{eqnarray*}
% 其中$(14) = - c_{r's'}v_i + \frac{1}{2\rho} (v_{s'}\delta_{jr'} + v_{r'} \delta_{js'}),(41) = -c_{r''s''} v_j + \frac{1}{2\rho} (\delta_{jr''} v_{s''} + \delta_{js''} v_{r''})$,
% $(24) = c_{r''s''}-\frac{1}{2\rho}(\delta_{i''s'}\delta_{jr'} + \delta_{i''r'} \delta_{js'}), (42)=c_{rs}-\frac{1}{2\rho }(\delta_{jr''} \delta_{i's''} +\delta_{js''}\delta_{i'r''})$。从而可以看出$A_0A_j$对称。

% 于是由对称双曲方程组的相关理论\cite{majda2012compressible},其光滑解局部存在。

% \subsubsection{整体存在性}
% 我们在平衡态$U_e = (\rho_e=0,\rho_e v_e = 0, \rho_e F_e = \rho_e I, \rho_e c_e = 0)$附近考虑方程\eqref{eq:compressibleRelax}的整体存在性。由于其为可对称双曲的,可以采用验证Kawashima条件得到。我们这里采用的Kawashima条件的形式为
% \begin{itemize}
%   \item $\sum_j \omega_j A_j(U_e)$的特征向量不在矩阵$\mathcal{Q}_U(U_e)$的零空间中。
% \end{itemize}

% 首先计算$\mathcal{Q}_U(U_e)$如下。
% \begin{equation*}
%     \mathcal{Q}_U(U_e) = \left( \begin{matrix}
%         0 & & \\
%         & 0 & \\
%         & & -\frac{1}{\kappa} \dot{\mathcal{T}}I - \frac{1}{\xi} \mathring{\mathcal{T}}
%     \end{matrix} \right)
% \end{equation*}
% 其中$\omega \in S^{d-1}$,$S$为单位球。我们假设$W = (W_1,W_{2i},W_{3kl},W_{4rs})$在$\mathcal{Q}_U(U_e)$的零空间中,那么成立
% \begin{equation}\label{eq:Wzero}
%     W_{4rs} = 0
% \end{equation}
% 对任意$r,s$成立。下面我们假设其还是$\sum_j \omega_j A_j(U_e)$的特征向量。即存在数$\mu$,使得
% \begin{equation*}
%     \sum_j \omega_j A_j(U_e) W = \mu W.
% \end{equation*}
% 表示成坐标形式并利用\eqref{eq:Wzero},得到
% \begin{eqnarray}
%     \omega \cdot W_2  = \mu W_1, \label{eq:WCondition1} \\
%     (p_\rho(\rho_e) +1 ) W_1 \omega_i  - \omega_j W_{3ij} = \mu W_{2i},  \label{eq:WCondition2}\\
%      \omega \cdot W_2 \delta_{kl}  - \omega_l W_{2k} = \mu W_{3kl}, \label{eq:WCondition3}\\
%      -\frac{1}{2\rho_e} (\omega_r W_{2s} + \omega_s W_{2r}) = \mu W_{4rs} = 0 \label{eq:WCondition4}.
% \end{eqnarray}
% 首先假设$\mu \neq 0$。
% 将式子\eqref{eq:WCondition3}带入\eqref{eq:WCondition2}中得到
% \begin{equation*}
%     (p_\rho(\rho_e) +1 ) W_1 \omega_i  - \frac{1}{\mu} \omega_i (\omega \cdot W_2) + \frac{1}{\mu}|\omega|^2  W_{2i} = \mu W_{2i}.
% \end{equation*}
% 将$W_1 = \frac{1}{\mu} \omega \cdot W_2$代入上式得到
% \begin{equation*}
%     (p_\rho(\rho_e) +1 )\frac{1}{\mu} \omega \cdot W_2 \omega_i - \frac{1}{\mu} \omega_i (\omega \cdot W_2)+ \frac{1}{\mu}|\omega|^2  W_{2i} = \mu W_{2i}.
% \end{equation*}
% 于是
% \begin{equation*}
%       p_\rho(\rho_e)\frac{1}{\mu} \omega \cdot W_2 \omega_i  =(\mu - \frac{1}{\mu}|\omega|^2) W_{2i}.
% \end{equation*}
% 若$\mu - \frac{1}{\mu}|\omega|^2 = 0$,那么$\omega \cdot W_2 = 0$,代入\eqref{eq:WCondition3}可知$W_3=0$,代入\eqref{eq:WCondition2}可知$W_1=0$,从而$W=0$。若$\mu - \frac{1}{\mu}|\omega|^2 \neq 0$,$W_2$与$\omega$共线,假设$W_2 = \beta \omega$,代入\eqref{eq:WCondition4}得到
% \begin{equation*}
%     \frac{2}{\beta} W_{2r} W_{2s} = 0.
% \end{equation*}
% 于是$W_2=0$。从而有上面的分析可以得到$W=0$。从而非零特征值不存在。
% 下面假设$\mu=0$,那么由\eqref{eq:WCondition1}得到
% \begin{equation*}
%     \omega \cdot W_2 = 0.
% \end{equation*}
% 将\eqref{eq:WCondition4}两边同乘以$\omega_r$并对$r$相加,利用上式得到
% \begin{equation*}
%     |\omega|^2 W_{2rs} = 0.
% \end{equation*}
% 从而$W_2 = 0$。
% % 由此代入\eqref{eq:WCondition3}可知$W_3=0$,代入\eqref{eq:WCondition2}可知$W_1=0$,从而$W=0$。即零特征值对应的特征向量不存在。下面
% 我们得到对应的特征向量需要满足
% \begin{equation*}
%     (p_\rho(\rho_e) + 1)\xi_i W_1  - \xi_j W_{3ij} = 0.
% \end{equation*}
% 例如可取$W = (1,0,(p_\rho(\rho_e) + 1),0)$满足条件。
% 从而存在$\sum_j \omega_j A_j(U_e)$的特征向量在矩阵$\mathcal{Q}_U(U_e)$的零空间中。于是Kawashima条件不成立。

% 虽然Kawashima条件对方程组\eqref{eq:compressibleRelax}不成立,但是由于力学适应性条件的存在,我们仍然可以证明其平衡态附近解的整体存在性定理。实际上,我们可以利用适应性条件\eqref{eq:compatibility1}和\eqref{eq:compatibility2}来弥补Kawashima条件的缺失。在定义下面的线性算子
% \begin{eqnarray}\label{eq:cformula}
% {\mathcal C}_1(U) = & -\nabla\rho - \rho_e \nabla\cdot F^T, \nonumber \\
% {[{\mathcal C}_2(U)]}_{kmj} = & \partial_{x_m} F_{kj} - \partial_{x_j} F_{km}
% \end{eqnarray}
% 分别为适应性条件\eqref{eq:compatibility1}和\eqref{eq:compatibility2}在平衡点附近的线性部分。
% 假设上面的$W$也在算子${\mathcal C}_1(U)$的符号矩阵的零空间中,我们有
% \begin{equation*}
%     \omega_i W_{3ij} = 0.
% \end{equation*}
% 将\eqref{eq:WCondition2}乘以$\omega_i$,我们得到
% \begin{equation*}
%         (p_\rho(\rho_e) +1 ) W_1 |\omega|^2  - \omega_j \omega_i W_{3ij} = (p_\rho(\rho_e) +1 ) W_1 |\omega|^2 =0.
% \end{equation*}
% 于是得到$W_1=0$,以及
% \begin{equation}\label{eq:W3ij}
%     \omega_j W_{3ij} =0.
% \end{equation}
% 假设上面的$W$也在算子${\mathcal C}_2(U)$的符号矩阵的零空间中,我们有
% \begin{equation*}
%     \omega_k W_{3ij} = \omega_j W_{3ik}
% \end{equation*}
% 对任意$i,j,k$成立。将等式\eqref{eq:W3ij}乘以$F_{im}$并利用上式得到
% \begin{equation*}
%     \omega_j W_{3im} W_{3ij} = \omega_m W_{3ij}W_{3ij} =0.
% \end{equation*}
% 于是$W_3=0$,从而我们得到$W=0$,即不存在$\sum_j \omega_j A_j(U_e)$的特征向量既在矩阵$\mathcal{Q}_U(U_e)$的零空间中也在$\mathcal{C}_1,\mathcal{C}_2$的符号矩阵的零空间中。




% \section{松弛参数趋于$0$时方程组\eqref{eq:compressibleRelax}极限方程的解的存在性}
下面我们指出,这一模型在$\kappa,\xi$趋于$0$时形式上可以近似Lin等人的模型。为此,我们利用Maxwell迭代,
% 下面考虑\eqref{eq:compressibleRelax}的Maxwell迭代一阶近似方程组的整体存在性。
首先由\eqref{eq:compressibleRelax}中$c$的方程得到
\begin{eqnarray*}
    (\rho \dot{c})_t + \nabla \cdot (\rho \dot{c}) -  \frac{1}{n} \nabla \cdot v  = - \frac{1}{\kappa} (\rho  \dot{c}), \\
    (\rho \mathring{c})_t + \nabla \cdot (\rho \mathring{c}) -  \frac{1}{2} (\nabla v + (\nabla v)^T  - \frac{2}{n} \nabla \cdot v I)  = - \frac{1}{\xi} (\rho  \mathring{c}).
\end{eqnarray*}
可以写为
\begin{eqnarray*}
    \rho \dot{c} = - \kappa\left( (\rho \dot{c})_t + \nabla \cdot (\rho \dot{c}) - \frac{1}{n} \nabla \cdot v\right), \\
    \rho \mathring{c} = - \xi \left( (\rho \mathring{c})_t + \nabla \cdot (\rho \mathring{c}) -  \frac{1}{2} (\nabla v + (\nabla v)^T  - \frac{2}{n} \nabla \cdot v I) \right).
\end{eqnarray*}
假设$\kappa,\xi$很小,迭代一次得到
\begin{equation*}
    \rho \dot{c} = \frac{\kappa}{n} \nabla \cdot v + O(\kappa^2), \ \rho \mathring{c} =  \frac{\xi}{2} (\nabla v + (\nabla v)^T  - \frac{2}{n} \nabla \cdot v I) + O(\xi^2).
\end{equation*}
代入方程\eqref{eq:compressibleRelax}中,并忽略高阶项,可以得到
\begin{subequations}\label{eq:compressible}
  \begin{align}
  \rho_t + \nabla \cdot (\rho  v ) = 0, \\
  (\rho  v )_t + \nabla \cdot ( \rho  v  \otimes  v ) + \nabla p = \nabla \cdot (\rho F F^T) + \mu \Delta  v  + \mu' \nabla \nabla \cdot  v , \\
  (\rho F)_t + \nabla \cdot (F \otimes \rho  v ) = (\nabla  v ) \rho F
\end{align}
\end{subequations}
其中$\mu = \xi/2,\ \mu'=\xi/2 + (\kappa - \xi)/n$均大于$0$($n\ge 2$时显然成立,$n=1$时$\mu + \mu' = \kappa>0$),此即Lin等人的模型\eqref{eq:lincompressible}\cite{lin2005hydrodynamics,lei2008global}。

\section{Lin等人的模型平衡态附近解的整体存在性}

下面我们将考虑Lin等人的模型\eqref{eq:compressible}在平衡态附近整体解的存在性。首先我们将说明该方程组可以看作对称双曲-抛物系统,从而其整体存在性可由Kawashima等人的理论\cite{kawashima1984systems,kawashima1985systems,yong2004entropy}进行研究。然而通过验证,Kawashima条件并不成立。幸运的是我们可以通过力学上的适应性条件来弥补这一缺陷,从而可以采用Kawashima的一般框架得到存在性定理的证明。

为了方便,本节假设$U = (\rho, v ,F)$,而采用$W$表示上文中的$U$,即$W=(\rho, \rho v ,\rho F)$。平衡态$U_e$为$\rho=\rho_e>0,  v =0 $和$F=I_{n^2}$ ($I_k$表示$k$阶单位矩阵)。本节的主要结果如下:
\begin{theorem}\label{theoremcom}
令正整数$s > \frac{n}{2}+1$。假设$U_0-U_e\in H^s$且$\|U_0-U_e\|_{H^s}$足够小。并且$U_0 = U_0(x)$满足适应性条件\eqref{eq:compatibility1}和\eqref{eq:compatibility2}。那么方程组\eqref{eq:compressible}以$U_0$为初值的Cauchy问题存在整体唯一解$U=U(x,t)$,满足
    \begin{eqnarray}\label{eq:thmincom}
       U - U_e\in C([0,+\infty), H^s) \cap L^2([0,+\infty), H^{s}), \nonumber \quad
         v \in L^2([0,+\infty), H^{s+1}),\\[2mm]
      \|U(T)-U_e\|_{H^s}^2 +  \int_0^T \left[\|\nabla  v (t) \|_{H^s}^2 + \|\nabla U(t)\|_{H^{s-1}}^2\right] dt
      \le C \|U_0-U_e\|_{H^s}^2 .
    \end{eqnarray}
\end{theorem}

实际上,本定理已经在\cite{qian2010global,hu2011global}中得到证明,对于不可压缩的情形,文献\cite{lin2005hydrodynamics,lei2008global,chen2006global}也证明了类似的定理。在这里我们将给出一个新的证明,这一证明利用了双曲-抛物方程的Kawashima理论,给出了统一的证明方法,并且这里提出的证明方法对于不可压方程仍成立。另外,我们还分析了不同的适应性条件,解释了文献\cite{lin2005hydrodynamics,lei2008global,chen2006global,qian2010global,hu2011global}中不同适应性条件下整体存在性成立的原因。

我们的证明依赖于方程组\eqref{eq:compressible}的结构特点和细致的估计。首先由前面的推导可以得到系统的熵函数存在。然后我们将给出方程\eqref{eq:compressible}的对称子,从而该方程组可以写为对称双曲-抛物方程组。接下来我们将验证其不满足Kawashima等人对双曲-抛物方程组提出的Kawashima条件,为了得到定理\ref{theoremcom},我们利用适应性条件\eqref{eq:compatibility1}和\eqref{eq:compatibility2}来得到整体存在性所需的估计。最后我们将采用\cite{kawashima1984systems,yong2004entropy}中的方法给出证明的过程。

\subsection{熵函数与对称性}
由前面的分析,我们可以得到方程组\eqref{eq:compressible}的熵函数为
\begin{eqnarray}\label{22}
  \eta(\rho,\rho  v ,\rho F)  =  \rho \int_{\rho_0}^\rho \frac{p(\zeta)}{\zeta^2} d\zeta + \frac{1}{2} \rho | v |^2 + \frac{1}{2} \rho |F|^2.
\end{eqnarray}
$\eta = \eta(W)$为$W$的下凸函数。注意我们这里的$\eta$与前文的熵函数相差一个符号,这是为了之后分析的简便。

由计算可以得到熵函数的时间演化满足
\begin{eqnarray}\label{eq:entropy}
  \eta_t  &=& S_\rho \rho_t + S_{\rho  v } \cdot (\rho  v )_t  + S_{\rho F} : (\rho F)_t \nonumber \\
  &=& -S_{\rho} \nabla \cdot (\rho  v )  - S_{\rho  v  } \cdot [\nabla \cdot (\rho  v \otimes v ) + \nabla p] + S_{\rho  v } \cdot \nabla \cdot \sigma \nonumber \\
    &&- S_{\rho F} : \nabla \cdot (\rho F\otimes  v  ) + S_{\rho F}:(\nabla   v ) \rho F \nonumber \\
    &=& - \nabla \cdot (S  v +p  v  +\sigma^T \cdot v ) + (p+S - \rho S_\rho - \rho  v  \cdot S_{\rho  v } - \rho F : S_{\rho F}) \nabla \cdot  v  \nonumber \\
    &&- (\sigma - S_{\rho F} \rho F^T) : \nabla  v \nonumber \\
    &=& - \nabla \cdot (S  v +p  v  +\sigma^T \cdot v ) - \mu|\nabla  v |^2 - \mu'|\nabla\cdot v |^2.
\end{eqnarray}
从而$\eta$满足
\begin{eqnarray}\label{eq:entropyvol}
  \eta_t(W) = -\nabla\cdot J(W) + \sigma^S.
\end{eqnarray}
其中熵流$J(W)\equiv S  v +p  v  +\sigma^T \cdot v $,熵的产生率为$\sigma^S \equiv - \mu|\nabla  v |^2 - \mu'|\nabla\cdot v |^2 \le 0$,从而满足热力学第二定律。

利用\eqref{eq:compatibility1},方程组\eqref{eq:compressible}可以写成如下的坐标形式
\begin{eqnarray*}
  \left( \begin{array}{c} \rho \\ v_i \\ F_{kl} \end{array} \right)_t
  +
  \left( \begin{array}{ccc} v_j & \rho \delta_{i'j} & 0 \\
    \frac{p_\rho}{\rho} \delta_{ij} & v_j \delta_{ii'} & -F_{jl'} \delta_{ik'} \\
    0 & - F_{jl}\delta_{ki'}  & v_j \delta_{kk'} \delta_{ll'} \end{array} \right)
  \left( \begin{array}{c} \rho \\ v_{i'} \\ F_{k'l'} \end{array} \right)_{x_j}
=
\left( \begin{array}{c} 0 \\ \frac{\mu}{\rho} \Delta {v_i} + \frac{\mu'}{\rho} \partial_{x_i} \partial_{x_j} v_j  \\ 0 \end{array} \right).
\end{eqnarray*}
即
\begin{eqnarray}\label{eq:symmetrichyperbolic}
  U_t + \sum_{j=1}^n A_j(U) U_{x_j} = \mathcal{Q}[U].
\end{eqnarray}
其中
\begin{eqnarray*}
  Q[U] = \frac{1}{\rho}\left( \begin{array}{c} 0\\ \mu\Delta  v + \mu'\nabla \nabla \cdot  v  \\ 0 \end{array} \right).
\end{eqnarray*}
取正定对称矩阵
\begin{eqnarray}
  A_0(U) = \left( \begin{array}{ccc} \frac{p_\rho}{\rho^2} & 0 & 0 \\
                                                   0 & I_n & 0 \\
						   0 & 0 & I_{n^2}\end{array} \right),
\end{eqnarray}
从而可以验证$A_0(U) A_j(U)$对称,即方程组\eqref{eq:symmetrichyperbolic}存在对称子$A_0$。

\subsection{Kawashima条件}
下面我们将说明对于方程组\eqref{eq:compressible},Kawashima条件在平衡点$U_e$不成立,但是Kawashima等人论文\cite{shizuta1985systems,kawashima1984systems}中的类似估计可以利用适应性条件\eqref{eq:compatibility1}与\eqref{eq:compatibility2}得到。

首先我们检验Kawashima条件。考虑平衡点$U_e$附近双曲-抛物组\eqref{eq:symmetrichyperbolic}的线性方程
\begin{eqnarray}\label{eq:symmtrickawashima}
  U_t + \sum_{j=1}^n A_j(U_e) U_{x_j} -\sum_{j=1}^n \sum_{k=1}^n D_{jk}(U_e) U_{x_j x_k}=  0.
\end{eqnarray}
其中
\begin{eqnarray*}
  D_{jk}(U) = \left( \begin{array}{ccc} 0 & 0 & 0 \\ 0 & \frac{\mu}{\rho} \delta_{jk} \delta_{ii'} + \frac{\mu'}{\rho} \delta_{ij}\delta_{ki'}& 0 \\ 0 & 0 & 0 \end{array} \right).
\end{eqnarray*}
令$\xi=(\xi_1, \xi_2, \cdots, \xi_d)\in \mathbf{S}^{n-1}$($\mathbf{S}^{n-1}$表示$\mathbf{R}^{n}$中的单位球)。这里我们取Kawashima条件的一个形式为
\begin{itemize}
    \item 矩阵$ \sum_{j=1}^n \xi_j A_j(U_e)$的特征向量不在矩阵$\sum_{j=1}^n \sum_{k=1}^n \xi_j \xi_k D_{jk}(U_e)$的零空间中。
\end{itemize}

假设$\hat{U} = (\hat{\rho}, \hat{ v },\hat{F})$位于矩阵$\sum_{j=1}^n \sum_{k=1}^n  \xi_j \xi_k D_{jk}(U_e)$的零空间中,那么有
\begin{equation*}
    \frac{\mu}{\rho_e} |\xi|^2 \hat{v}_i + \frac{\mu'}{\rho_e} \xi_i \xi_{i'} \hat{v}_{i'} = 0.
\end{equation*}
假设$\hat{v}_{i} \neq 0$,$\hat{v}$与$\xi$方向相同,从而存在数$m \neq 0$,使得$\hat{v}_i = m \xi_i$,代入上式得到
\begin{equation*}
    \mu |\xi|^2 \hat{v}_i + \mu' |\xi|^2 \hat{v}_i = 0. 
\end{equation*}
从而$\hat{v}=0$。假设$\hat{U}$也是矩阵$ \sum_{j=1}^n \xi_j A_j(U_e)$的特征向量,从而存在数$\lambda$,使得
\begin{eqnarray*}
    \sum_{j=1}^n \xi_j A_j(U_e)\hat{U} = \lambda \hat{U}, \\
     \left( \begin{array}{ccc} 0 & \rho_e \xi_{i'} & 0 \\
    \frac{p_\rho(\rho_e)}{\rho_e} \xi_i & 0 & -\xi_{l'} \delta_{ik'} \\
    0 & - \xi_{l}\delta_{ki'}  & 0 \end{array} \right)
    \left(\begin{array}{c}
        \hat{\rho} \\ \hat{ v}_{i'} \\ \hat{F}_{k'l'}
    \end{array} \right) = \lambda
    \left(\begin{array}{c}
        \hat{\rho} \\ \hat{ v }_i \\ \hat{F}_{kl}
    \end{array} \right).
\end{eqnarray*}
我们得到
\begin{eqnarray*}
    \rho_e \xi \cdot \hat{v}  = \lambda \hat{\rho} , \\
    \frac{p_\rho(\rho_e)}{\rho_e} \hat{\rho} \xi_i - \hat{F}_{il} \xi_l = \lambda \hat{v}_i, \\
    -\xi_l \hat{v}_k = \lambda \hat{F}_{kl}.
\end{eqnarray*}
将$\hat{v}=0$代入可以得到$\lambda = 0$(由$\xi \neq 0$与第三个式子)。从而任意特征向量$\hat{U}$满足
\begin{eqnarray} \label{eq:linear}
    \hat{ v }=0, \quad \xi_i \frac{p_\rho(\rho_e)}{\rho_e} \hat{\rho} - \xi_{l'} \hat{F}_{il'} = 0.
\end{eqnarray}
例如$\hat{U} = (1,0,\frac{p_\rho(\rho_e)}{\rho_e} I_{n^2})$既是矩阵$ \sum_{j=1}^n \xi_j A_j(U_e)$的特征向量又在矩阵$\sum_{j=1}^n \sum_{k=1}^n \xi_j \xi_k D_{jk}(U_e)$的零空间中,从而Kawashima条件对于\eqref{eq:compressible}不成立。

然而我们可以利用适应性条件\eqref{eq:compatibility1}和\eqref{eq:compatibility2}来弥补Kawashima条件的缺失。
定义下面的线性算子
 \begin{eqnarray}\label{eq:cformula}
 {\mathcal C}_1(U) = & -\nabla\rho - \rho_e \nabla\cdot F^T, \nonumber \\
 {[{\mathcal C}_2(U)]}_{kmj} = & \partial_{x_m} F_{kj} - \partial_{x_j} F_{km}
 \end{eqnarray}
分别为适应性条件\eqref{eq:compatibility1}和\eqref{eq:compatibility2}在平衡点附近的线性部分。
假设上面的$\hat{U}$也在算子${\mathcal C}_1(U)$和${\mathcal C}_2(U)$(定义见\eqref{eq:cformula})的符号矩阵的零空间中,我们有
\begin{eqnarray}
\xi_i \hat{\rho} + \rho_e \xi_{j} \hat{F}_{ji} = 0 , \label{eq:linear1}\\
\xi_{m} \hat{F}_{kj} - \xi_{j} \hat{F}_{km} =0. \label{eq:linear2}
\end{eqnarray}
将\eqref{eq:linear}乘以$\xi_i$,并对$i$求和,利用\eqref{eq:linear1},我们可以推出
$$
|\xi|^2 \frac{p_\rho(\rho_e)}{\rho_e} \hat{\rho}  =  \xi_i \xi_{l'} \hat{F}_{il'}  = - \xi_{l'} \frac{\xi_{l'}  }{\rho_e} \hat{\rho} = -\frac{|\xi|^2}{\rho_e} \hat{\rho},
$$
从而$\hat{\rho}=0$。将此结果代入\eqref{eq:linear}给出
 $\xi_{l'} \hat{F}_{il'} = 0 $,
乘以$\hat{F}_{im}$并对$i$求和,利用\eqref{eq:linear2}可以导出
$$0=\xi_{l'} \hat{F} _{il'} \hat{F}_{im} = \xi_m \hat{F} _{il'}\hat{F} _{il'}.$$
从上面的等式由$\xi \neq 0$可以知道$\hat{F}=0$。这样对应零特征值的非零特征向量$\hat{U}$不在线性算子${\mathcal C}_1(U)$ 和 ${\mathcal C}_2(U)$的符号矩阵的零空间中。基于这样的结果,我们期望Kawashima条件在适应性条件的约束下成立。

在文献\cite{shizuta1985systems}中,Kawashima条件等价于存在反对称矩阵$K=K(U_e, \xi)$,使得
\begin{eqnarray*}
  K\sum_{j}\xi_j A_j(U_e) - \sum_{j}\xi_j A_j^T(U_e)K + \sum_{j, k}\xi_j\xi_kD_{jk}(U_e)
\end{eqnarray*}
对任意$\xi\in S^{d-1}$是正定的。我们上面的分析表明这样的$K$不存在,但是在适应性条件\eqref{eq:compatibility1}与\eqref{eq:compatibility2}的帮助下,成立下面的引理:
\begin{lemma}\label{lemmaK}
取反对称矩阵
$$
K_j = \mbox{diag}\left(\frac{p'(\rho_e)}{\rho_e^2}, -I_n, I_{n^2}\right)A_j(U_e),
$$
那么存在正常数$\eta$和$C_S$,使得对任意的光滑函数$U =U(x)$,下面的不等式成立。
\begin{eqnarray}\label{eq:prop}
  &&\sum_{j,m=1}^n [( \eta K_m A_j(U_e) U_{x_j},U_{x_m}) + (D_{mj}(U_e) U_{x_j},U_{x_m})]\nonumber \\
  &\ge& C_S \|\nabla U \|_{L^2}^2 +\eta\frac{2p'(\rho_e)}{\rho_e^2}({\mathcal C}_1(U - U_e), \nabla \rho) + \eta({\mathcal C}_2(U - U_e), \nabla F).
\end{eqnarray}

\end{lemma}

\begin{proof}
由\eqref{eq:compressible},我们有
$$
\sum_j A_j(U_e) U_{x_j}=\left( \begin{array}{cc} \rho_e\nabla\cdot\mathbf v\\ \frac{p'(\rho_e)}{\rho_e}\nabla\rho - \nabla\cdot F\\ -\nabla \mathbf v\end{array} \right).
$$
于是成立
$$
-\sum_j K_jU_{x_j}=-\mbox{diag}\left(\frac{p'(\rho_e)}{\rho_e^2}, -I_d, I_{d^2}\right)\sum_j A_j(U_e) U_{x_j}=\left( \begin{array}{cc} -\frac{p'(\rho_e)}{\rho_e}\nabla\cdot\mathbf v\\ \frac{p'(\rho_e)}{\rho_e}\nabla \rho - \nabla\cdot F\\ \nabla \mathbf v\end{array} \right).
$$
下面我们计算 (省略下标$e$)
\begin{eqnarray*}
   && \sum_{m,j=1}^n (K_m  A_j(U_e) U_{x_j},U_{x_m}) = -(\sum_j A_j(U_e) U_{x_j}, \sum_m K_m U_{x_m})\\
  &=& - p' \|\nabla \cdot  v \|_{L^2}^2 - \|\nabla v \|_{L^2}^2+ \|\frac{p'}{\rho} \nabla \rho \|_{L^2}^2 - 2\frac{p'}{\rho} (\nabla\rho, \nabla\cdot F) +  \|\nabla\cdot F\|_{L^2}^2  \\
  &=& - p' \|\nabla \cdot  v \|_{L^2}^2 - \|\nabla v \|_{L^2}^2+ \|\frac{p'}{\rho} \nabla \rho \|_{L^2}^2 -  \frac{2p'}{\rho}( \partial_{x_j} \rho,\partial_{x_m} F_{jm}) + ( \partial_{x_j} F_{kj},\partial_{x_m} F_{km})\\
     &=& - p' \|\nabla \cdot  v \|_{L^2}^2 - \|\nabla v \|_{L^2}^2+ \|\frac{p'}{\rho} \nabla \rho \|_{L^2}^2-  \frac{2p'}{\rho} (\partial_{x_m} \rho,\partial_{x_j} F_{jm}) +  ( \partial_{x_m} F_{kj},\partial_{x_j} F_{km})\\
     &=&  - p' \|\nabla \cdot  v \|_{L^2}^2 - \|\nabla v \|_{L^2}^2+ \|\frac{p'}{\rho} \nabla \rho \|_{L^2}^2-  \frac{2p'}{\rho}(\nabla \rho,\nabla\cdot F^T)\\
     && + \|\nabla F\|_{L^2}^2
     + ( \partial_{x_m} F_{kj} - \partial_{x_j} F_{km},\partial_{x_j} F_{km}) \\
     &=&  - p' \|\nabla \cdot  v \|_{L^2}^2 - \|\nabla v \|_{L^2}^2+ \frac{p'^2 + 2p'}{\rho^2} \|\nabla \rho \|_{L^2}^2+ \|\nabla F\|_{L^2}^2 \\
     &&+ \frac{2p'}{\rho^2}(\nabla \rho, {\mathcal C}_1(U - U_e)) + ({\mathcal C}_2(U - U_e), \nabla F).
\end{eqnarray*}
这样下面的式子成立
\begin{eqnarray*}
 && \sum_{j,m=1}^n[ ( \eta K_m A_j(U_e) U_{x_j},U_{x_m}) + (D_{mj}(U_e) U_{x_j},U_{x_m})] \\
  &=& \eta\frac{p'^2 + 2 p'}{\rho^2} \|\nabla \rho \|_{L^2}^2 + (\frac{\mu}{\rho} - \eta   ) \|\nabla v \|_{L^2}^2+ (\frac{\mu'}{\rho} - \eta  p' ) \|\nabla \cdot  v \|_{L^2}^2+ \eta \|\nabla F\|_{L^2}^2\\
  && + \eta\frac{2p'}{\rho^2}(\nabla \rho, {\mathcal C}_1(U - U_e)) + \eta({\mathcal C}_2(U - U_e), \nabla F).
\end{eqnarray*}
选取$\eta=\min\{ \frac{\mu}{2\rho}, \frac{\mu'}{2\rho p'}\}$可得\eqref{eq:prop}成立,这样我们证明了引理\ref{lemmaK}。
\end{proof}{}

\begin{remark}
引理\ref{lemmaK}中利用了适应性条件\eqref{eq:compatibility1}和\eqref{eq:compatibility2},实际上我们亦可以采用适应性条件\eqref{eq:compatibility3}得到类似的结果。下面我们分析这三个适应性条件对Kawashima条件的影响。

首先根据第一节的分析,适应性条件\eqref{eq:compatibility1}可由\eqref{eq:compatibility2}和\eqref{eq:compatibility3}导出,
%实际上,由
% Jacobi公式
% $$
% \partial_{x_i}\det A = \det A \mbox{Tr}(A^{-1} \partial_{x_i} A),
% $$
% 成立
% \begin{eqnarray*}
% \nabla \cdot (\rho F^T)  & = &\partial_{x_i} ( \rho F_{ij}) = \partial_{x_i} ( \frac{1}{\det F} F_{ij})  \\
% &=& \frac{1}{\det F} \partial_{x_i} F_{ij}  - \frac{1}{(\det F)^2} F_{ij} \det F \mbox{Tr}(F^{-1} \partial_{x_i} F) \\
% &=&\frac{1}{\det F} (\partial_{x_i} F_{ij} - F_{ij} (F^{-1})_{mn} \partial_{x_i} F_{nm}) \\
% &=& \frac{1}{\det F} (\partial_{x_i} F_{ij} - (F^{-1})_{mn} F_{ij} \partial_{x_i} F_{nm}) \\
% &=&  \frac{1}{\det F} (\partial_{x_i} F_{ij} - (F^{-1})_{mn} F_{im} \partial_{x_i} F_{nj}) \\
% &=&  \frac{1}{\det F} (\partial_{x_i} F_{ij} - \delta_{in} \partial_{x_i} F_{nj}) \\
% &=&  \frac{1}{\det F} (\partial_{x_i} F_{ij} - \partial_{x_i} F_{ij}) =0 ,
% \end{eqnarray*}
从而我们可知对适应性条件\eqref{eq:compatibility2}和\eqref{eq:compatibility3},引理\ref{lemmaK}成立,从而整体存在性定理也成立。

另外,在二维的情况我们可以证明引理\ref{lemmaK}对适应性条件\eqref{eq:compatibility1}和\eqref{eq:compatibility3}也成立。实际上,对条件\eqref{eq:compatibility3}求导得到
$$
0=\nabla (\rho \det F) = \nabla \rho \det F + \rho \det F \mbox{Tr}(F^{-1} \nabla F).
$$
其线性化算子为
\begin{equation*} \label{3}
\mathcal{C}_3(U):=-\nabla \rho - \rho_e \nabla \mbox{Tr}(F).
\end{equation*}
从而二维$n=2$时$\mathcal{C}_2(U)$的唯一两个独立量可以用$\mathcal{C}_1$和$\mathcal{C}_3$表示如下
\begin{eqnarray*}
\mathcal{C}_2(U)_{112}&=&\partial_{x_1}F_{12} - \partial_{x_2}F_{11}= {\mathcal{C}_3(U)}_2 - {\mathcal{C}_1(U)}_2, \\ \mathcal{C}_2(U)_{221}&=&\partial_{x_2}F_{21}-\partial_{x_1}F_{22} = {\mathcal{C}_3(U)}_1 - {\mathcal{C}_1(U)}_1.
\end{eqnarray*}
这表明算子${\mathcal C}_1(U)$同${\mathcal C}_3(U)$亦可以排除不满足Kawashima条件的非零特征向量$\hat{U}$。

然而在维数$n \ge 3$的情况,适应性条件\eqref{eq:compatibility1}和\eqref{eq:compatibility3}并不能保证不满足Kawashima条件的非零特征向量$\hat{U}$不存在,例如$n=3$时,对$\xi = (1,0,0)$,我们可以验证$(0,0, G)$,
 $$
 G = \left( \begin{array}{ccc}
 0 & 0 & 0 \\
 0 & 1 & 0 \\
 0 & 0 & -1 \end{array} \right)
 $$
是矩阵$ \sum_{j=1}^n \xi_j A_j(U_e)$的零特征值对应的一个特征向量,从而引理\ref{lemmaK}对适应性条件\eqref{eq:compatibility1}和\eqref{eq:compatibility3}在大于二维时不成立。这也给出了文献\cite{lin2005hydrodynamics}和\cite{liu2008global}中二维和三维假设不同适应性条件的一个解释。
\end{remark}

\subsection{整体存在性定理(定理4.1)的证明}
因为方程组\eqref{eq:compressible}可以被对称化,其解的局部存在性成立\cite{kawashima1984systems},另外方程组\eqref{eq:compressible}的局部存在性定理的证明也可以在文献\cite{lin2005hydrodynamics,lei2008global,hu2013global}中找到。下面我们将推导定理\ref{theoremcom}中的\emph{先验估计}\eqref{eq:thmincom}。

证明分三步,第一步利用熵的演化方程得到$U-U_e$的$L^2$估计,然后求导得到高阶导数的估计,最后利用引理\ref{lemmaK}完成证明。

\emph{第一步:}令
\begin{eqnarray*}
  G(W) = S(W) - S(W_e) - S_W(W_e) (W-W_e),
\end{eqnarray*}
其中$W_e = (\rho_e, 0_n, \rho_e I_{n^2})$。由熵函数的上凸性可知存在常数$C_1$ 和 $C_2$,使得
\begin{eqnarray*}
  C_1 |W-W_e|^2 \le G(W) \le C_2 |W-W_e|^2, \quad C_1 |U-U_e|^2 \le |W-W_e|^2 \le C_2 |U-U_e|^2
\end{eqnarray*}
对于靠近$U_e$的$U$成立。于是根据\eqref{eq:entropy}与$\rho F$的方程,我们得到
\begin{eqnarray*}
  G(W)_t + \nabla \cdot (J(W) + \phi(\rho_e) \rho  v  + \rho (\mbox{Tr}(F) - F) v ) = -\mu |\nabla  v |^2 - \mu' |\nabla \cdot  v |^2.
\end{eqnarray*}
其中$J(W)$同\eqref{eq:entropyvol}中的熵流,$\phi(\rho) = \frac{p(\rho)}{\rho} + \int_{\rho_0}^{\rho}\frac{p(\zeta)}{\zeta^2}d\zeta - \frac{n}{2}$。对此等式在$x\in\mathbf{R}^n$和$t\in[0, T]$上积分给出下面的估计
\begin{eqnarray}\label{eq:EnergyESTcom}
  \|U(T) - U_e\|_{L^2}^2 + c \int_0^T[ \mu \|\nabla v (t)\|_{L^2}^2 +  \mu' \|\nabla \cdot  v (t)\|_{L^2}^2] dt
  \le C\|U_0-U_e\|_{L^2}^2 .
\end{eqnarray}

\emph{第二步:}将$\partial_x^\alpha$($\alpha$为多重指标且满足$|\alpha|\le s$)作用到方程\eqref{eq:symmetrichyperbolic}两边得到
\begin{eqnarray}\label{eq:higher}
    \partial_x^\alpha U_t + \sum_{j=1}^n A_j(U) \partial_x^\alpha U_{x_j} = \partial_x^\alpha Q[U] + \sum_{j=1}^n [A_j(U),\partial_x^\alpha] U_{x_j}.
\end{eqnarray}
与$A_0(U) \partial_x^\alpha U$取$L^2$内积得到
\begin{eqnarray}\label{eq:HighOrderInner}
  (A_0(U) \partial_x^\alpha U, \partial_x^\alpha U)_t
  %+ \sum_{j=1}^n \int (\partial_x^\alpha U^T A_0(U) A_j(U) \partial_x^\alpha U)_{x_j} dx
 & = & 2(A_0(U) \partial_x^\alpha Q[U],\partial_x^\alpha U) + 2 \sum_{j=1}^n (A_0(U) [A_j(U),\partial_x^\alpha]U_{x_j},\partial_x^\alpha U)  \nonumber\\
  &&+ \left( \big( \partial_t A_0(U) + \sum_{j=1}^n \partial_{x_j} (A_0(U) A_j(U))\big) \partial_x^\alpha U,\partial_x^\alpha U \right)  \\
  &\equiv& I_1 + I_2 + I_3 .  \nonumber
\end{eqnarray}
右端项可以采用Sobolev演算不等式\cite{majda2012compressible}估计如下
\begin{eqnarray*}
 && I_1 = 2(\partial_x^\alpha(\frac{\mu}{\rho} \Delta  v  + \frac{\mu'}{\rho} \nabla \nabla \cdot  v ), \partial_x^\alpha  v ) \\
  &=& 2(\partial_x^\alpha (\frac{\mu}{\rho_e} \Delta  v  + \frac{\mu'}{\rho_e} \nabla \nabla \cdot  v ), \partial_x^\alpha  v ) + 2( \partial_x^\alpha ( \mu (\frac{1}{\rho}-\frac{1}{\rho_e}) \Delta  v  + \mu'(\frac{1}{\rho} - \frac{1}{\rho_e}) \nabla \nabla \cdot  v ), \partial_x^\alpha v ) \\
  &=& 2(\partial_x^\alpha (\frac{\mu}{\rho_e} \Delta  v  + \frac{\mu'}{\rho_e} \nabla \nabla \cdot  v ), \partial_x^\alpha  v )
  - 2( \partial_x^{\alpha-1} ( \mu (\frac{1}{\rho}-\frac{1}{\rho_e}) \Delta  v  + \mu'(\frac{1}{\rho} - \frac{1}{\rho_e}) \nabla \nabla \cdot  v ),\partial_x^{\alpha +1}  v ) \\
  &\le& -\frac{2\mu}{\rho_e} \|\partial_x^\alpha \nabla  v \|_{L^2}  -\frac{2\mu'}{\rho_e} \|\partial_x^\alpha \nabla \cdot  v \|_{L^2} + C(\mu+\mu')\| \rho -\rho_e \|_{H^s} \|\nabla  v \|_{H^s}^2 ,
\end{eqnarray*}
\begin{eqnarray*}
  I_2 &\le& 2|A_0(U)|_{L^\infty} \|\sum_{j=1}^n [A_j(U),\partial_x^\alpha]U_{x_j} \|_{L^2} \|\partial_x^\alpha U\|_{L^2} \\
  &\le& C \|\partial_x^\alpha U\|_{L^2} \sum_{j=1}^n (|\nabla A_j(U)|_{L^\infty} \|D^{s-1}_x U_{x_j}\|_{L^2}+ |U_{x_j}|_{L^\infty} \|D^s_x A_j(U)\|_{L^2}) \\
  &\le& C \|\partial_x^\alpha U\|_{L^2} \|\nabla U\|_{H^{s-1}}^2 \le C\|\nabla U\|_{H^{s-1}}^3,
\end{eqnarray*}
及
\begin{eqnarray*}
  I_3 &\le& C (|\rho_t|_{L^\infty} + |\nabla U|_{L^\infty}) (\partial_x^\alpha U,\partial_x^\alpha U) \\
      &\le& C |\nabla U|_{L^\infty} \|\partial_x^\alpha U\|_{L^2}^2 \\
      &\le& C \|\nabla U\|_{H^{s-1}} \|\nabla U\|_{H^{s-1}}^2.
\end{eqnarray*}
这里对$I_3$的估计利用了\eqref{eq:compressible}中密度的方程以及$A_0(U)$仅仅依赖于$\rho$这个事实。

将$I_1, I_2$和$I_3$的估计代入\eqref{eq:HighOrderInner},我们得到
\begin{eqnarray*}
  & (A_0(U)\partial_x^\alpha U,\partial_x^\alpha U)_t + \frac{2\mu}{\rho_e} \|\partial_x^\alpha \nabla  v \|_{L^2}  + \frac{2\mu'}{\rho_e} \|\partial_x^\alpha \nabla \cdot  v \|_{L^2}   \\
\le &C \| \nabla U\|_{H^{s-1}}^3 + C(\mu+\mu')\|\rho -\rho_e\|_{H^s} \|\nabla  v \|_{H^s}^2.\nonumber
\end{eqnarray*}
在$ t\in [0,T]$上对上述不等式积分并对$1 \le |\alpha| \le s$相加,结合\eqref{eq:EnergyESTcom},我们最终得到
\begin{eqnarray}\label{eq:Step2Resultcom}
  && \| U(T) - U_e\|_{H^s}^2 + c\int_0^T \|\nabla  v  (t)\|_{H^s}^2 dt  \\
  &\le& C \|U_0 - U_e\|_{H^s}^2+ C (\mu + \mu') \sup_{t \in [0,T]} (\|\rho(t) -\rho_e\|_{H^s}) \int_0^T \|\nabla  v (t)\|_{H^s}^2 dt \nonumber\\
  &&+ C \sup_{t \in [0,T]} (\|U(t) - U_e\|_{H^s}) \int_0^T \|\nabla U(t)\|_{H^{s-1}}^2 dt . \nonumber
\end{eqnarray}

\emph{第三步:}为了控制\eqref{eq:Step2Resultcom}的最后一项,我们利用引理\ref{lemmaK}。首先将\eqref{eq:higher}写为下面的形式
\begin{eqnarray*}
  \partial_x^\alpha  U_t + \sum_{j=1}^n A_j(U_e) \partial_x^\alpha  U_{x_j}
 = \sum_{j,k=1}^n  \partial_x^\alpha ( D_{jk} (U)U_{x_i x_j} ) + \sum_{j=1}^n \partial_x^\alpha ((A_j(U_e) - A_j(U))U_{x_j}).
\end{eqnarray*}
其中$\alpha$满足$0\le|\alpha|\le s-1$。回忆引理\ref{lemmaK}中的反对称矩阵$K$。我们将上述等式与$-K_m\partial_x^\alpha U_{x_m}$做内积并将得到的结果对$m \in [1,n]$求和得到
\begin{eqnarray} \label{eq:KsDU}
  && \quad \sum_{m=1}^n  (K_m \partial_x^\alpha U_t,\partial_x^\alpha U_{x_m}) + \sum_{m,j=1}^n (K_m A_j(U_e) \partial_x^\alpha U_{x_j}, \partial_x^\alpha U_{x_m}) \\
  &=&  \sum_{m,j=1}^n(K_m\partial_x^\alpha ( (A_j(U_e) - A_j(U)) U_{x_j}), \partial_x^\alpha U_{x_m}) \nonumber \\
 && +
  \sum_{m,j,k=1}^n (K_m \partial_x^\alpha (D_{jk} (U) U_{x_i x_j} ),\partial_x^\alpha U_{x_m}) \nonumber \\
 & \le&
  C\sum_j \|\partial_x^\alpha ((A_j(U_e)-A_j(U))U_{x_j})\|_{L^2} \|\partial_x^\alpha \nabla U\|_{L^2} \nonumber \\
  && + \epsilon \|\partial_x^\alpha \nabla U\|_{L^2}^2  + \frac{C}{\epsilon} \sum_{j,k} \| \partial_x^\alpha ([D_{jk} (U_e) + D_{jk} (U) - D_{jk} (U_e)]U_{x_i x_j}) \|_{L^2}^2  \nonumber \\
 &\le& C \|U-U_e\|_{H^s} \|\nabla U\|_{H^{s-1}}^2 + \epsilon \|\partial_x^\alpha \nabla U\|_{L^2}^2 + \frac{C}{\epsilon}  \|\partial_x^{\alpha+1} \nabla  v \|_{L^2}^2  \nonumber \\
&& + \frac{C}{\epsilon}\| U-U_e\|_{H^{s-1}}^2  \|\nabla  v \|_{H^s}^2 . \nonumber
\end{eqnarray}
对等式\eqref{eq:KsDU}左端第一项,我们采用分部积分得到
\begin{eqnarray} \label{eq:L1}
  &&(K_m \partial_x^\alpha U_t,\partial_x^\alpha U_{x_m})  \\
  &=& \frac{1}{2} \int (\partial_x^\alpha U_{x_m}^T K_m \partial_x^\alpha U)_t dx - \frac{1}{2} \int (\partial_x^\alpha U^T K_m \partial_x^\alpha U_t)_{x_m} dx \nonumber \\
  &=& \frac{1}{2} (K_m \partial_x^\alpha U,\partial_x^\alpha U_{x_m})_t. \nonumber
\end{eqnarray}

对第二项,由$U$满足适应性条件\eqref{eq:compatibility1}和\eqref{eq:compatibility2},我们得到
\begin{eqnarray} \label{c}
{\mathcal C}_1(U - U_e) &=& \nabla\cdot[(\rho - \rho_e)(F^T - I_{d^2})], \nonumber\\[2mm]
{[{\mathcal C}_2(U - U_e)]}_{kmj} &=& (F_{lj} - \delta_{lj})\partial_{x_l}F_{km} - (F_{lm} - \delta_{lm})\partial_{x_l}F_{kj}.
\end{eqnarray}
其中${\mathcal C}_1$和${\mathcal C}_2$的定义见\eqref{eq:cformula}。这样由引理\ref{lemmaK}和\eqref{c},成立
\begin{eqnarray}\label{eq:L2com}
 && \sum_{j,m} (K_m A_j(U_e) \partial_x^\alpha U_{x_j}, \partial_x^\alpha U_{x_m}) = \\
  &&\sum_{j,m}[ (K_m A_j(U_e) \partial_x^\alpha U_{x_j}, \partial_x^\alpha U_{x_m}) + (D_{mj}(U_e) \partial_x^\alpha U_{x_j}, \partial_x^\alpha U_{x_m}) ] \nonumber \\
  && - \sum_{m,j} (D_{mj}(U_e) \partial_x^\alpha  U_{x_j}, \partial_x^\alpha U_{x_m}) \nonumber \\
   &\ge& C_S \|\partial_x^\alpha \nabla U\|_{L^2}^2 - \frac{\mu}{\rho_e} \|\partial_x^\alpha \nabla  v \|_{L^2}^2 -\frac{\mu'}{\rho_e} \|\partial_x^\alpha \nabla \cdot  v \|_{L^2}^2\nonumber \\
  &&+ \eta\frac{2p'(\rho_e)}{\rho_e^2}(\partial_x^\alpha{\mathcal C}_1(U - U_e), \nabla \partial_x^\alpha\rho) + \eta(\partial_x^\alpha{\mathcal C}_2(U - U_e), \nabla \partial_x^\alpha F), \nonumber\\
  &\ge& C_S \|\partial_x^\alpha \nabla U\|_{L^2}^2 - \frac{\mu}{\rho_e} \|\partial_x^\alpha \nabla  v \|_{L^2}^2 -\frac{\mu'}{\rho_e} \|\partial_x^\alpha \nabla \cdot  v \|_{L^2}^2\nonumber \\
  &&- C\|(\partial_x^\alpha{\mathcal C}_1(U - U_e)\|_{L^2}\|\nabla \partial_x^\alpha\rho\|_{L^2} - C\|\partial_x^\alpha{\mathcal C}_2(U - U_e)\|_{L^2}\|\nabla \partial_x^\alpha F\|_{L^2}\nonumber\\
  &\ge& C_S \|\partial_x^\alpha \nabla U\|_{L^2}^2  - \frac{\mu}{\rho_e} \|\partial_x^\alpha \nabla  v \|_{L^2}^2 -\frac{\mu'}{\rho_e} \|\partial_x^\alpha \nabla \cdot  v \|_{L^2}^2\nonumber \\
  &&- C\|\nabla U\|_{H^{s-1}}^2(\|\nabla \partial_x^\alpha\rho\|_{L^2} + \|\nabla \partial_x^\alpha F\|_{L^2}). \nonumber
\end{eqnarray}

将\eqref{eq:L1}和\eqref{eq:L2com}代入\eqref{eq:KsDU}中,取$\epsilon = \frac{C_S}{2}$并对$0 \le |\alpha| \le s-1$求和得到
\begin{eqnarray*}\label{eq:nablaUcom}
  C_S \|\nabla U\|_{H^{s-1}}^2 &\le& - \sum_{m} (K^m D^l_x U,D^l_x U_{x_m})_t  + C\|U-U_e\|_{H^s} \|\nabla U\|_{H^{s-1}}^2 \nonumber \\
  &&+ C \|\nabla  v \|_{H^s}^2+ C \|U-U_e\|_{H^s}^2 \|\nabla  v \|_{H^s}^2 .
\end{eqnarray*}
在$t \in [0,T]$对这个不等式积分得到
\begin{eqnarray*}
  \int_0^T \| \nabla U(t)\|_{H^{s-1}}^2 \le C \|U(T)-U_e \|_{H^s}^2 + C \|U_0 -U_e\|_{H^s}^2+ C\int_0^T \|\nabla  v \|_{H^s}^2 dt  \nonumber \\
  + C \sup_{t \in [0,T]} \|U(t)-U_e\|_{H^s} \int_0^T \|\nabla U(t)\|_{H^{s-1}}^2dt
  + C \sup_{t \in [0,T]} \|U(t)-U_e\|_{H^s}^2 \int_0^T\|\nabla  v (t)\|_{H^s}^2 dt. \nonumber
\end{eqnarray*}
结合\eqref{eq:Step2Resultcom}给出定理\ref{theoremcom}中的先验估计\eqref{eq:thmincom},从而定理得证。



%    Text of article.

%    Bibliographies can be prepared with BibTeX using amsplain,
%    amsalpha, or (for "historical" overviews) natbib style.
%\bibliographystyle{amsplain}
%    Insert the bibliography data here.
%\bibliography{ref}


% \bibliographystyle{amsplain}
% %    Insert the bibliography data here.
% \bibliography{ref}


%          \section{Introduction}\label{intro}
%
%          Put a general  introduction to your paper here. Separate text
%          sections with other sections.
%\newpage
%          \section{Put title of the next section here}\label{an apprpriate label}
%\newpage
%          %If you have subsections use:
%          \subsection{Subsection title.}\label{another label}
%          Don't forget to give each section, subsection, equation, theorem,
%          corollary, etc. a unique label, and when you refer to the results
%          later in the text use \ref{labelname} instead of explicitly writing
%          the number of the environment in question.
%%\begin{align}\label{labelname}
%%   &  \\
%%   & 
%%\end{align}
%          This use of \label and \ref is REQUIRED for  papers.
%
%          Similarly, always use \cite{taubes1}(biblabelname) to refer to bibliographic
%          references, which would then be entered in the bibliography via
%          %\bibitem[visible label]{<biblabelname>}.
%\newpage
%          %
%          % For figures use
%
%          %\begin{figure}
%
%          %The use of .eps files is encouraged, in which case you should
%          %un-comment the \uspackage{graphics} command above, and use the
%          %command
%          %\include{figure.eps}
%          % to insert the figure file.
%
%          %\end{figure}
%
%
%          % BibTeX users please use
%
%          % \bibliographystyle{}
%
%          % \bibliography{}
%
%          %
%
%          % Non-BibTeX users please use
%
%          \begin{thebibliography}{}
%
%          %
%
%          % and use \bibitem to create references.
%
%          %
%
%          \bibitem{mnemonic}Author, {\em title of paper}, Journal Name
%         Volume, page numbers, year.
%
%          % Format for Journal Reference. For example
%
%          \bibitem{taubes1} C. Taubes, {\em The Seiberg-Witten invariants
%          and
%          symplectic forms}, Math. Res. Letters, 1, 809--822, 1994.
%          \end{thebibliography}


% %          \end{document}
% \bibliographystyle{amsplain}
% %    Insert the bibliography data here.
% \bibliography{ref}
