% \chapter{守恒-耗散理论与线性粘弹性流体力学模型}
 \documentclass{article}
 \usepackage{ctex}
 \usepackage{amsmath}
 \usepackage{amsthm}
\newtheorem{theorem}{定理}

 \begin{document}
 	非平衡态热力学的核心是热力学第二定律。有热力学第二定律可以知道不可逆过程必然伴随熵增,而如何描述这一定律变得十分重要。无论是CIT、EIT,还是RT、GENERIC,其出发点均为热力学第二定律。通过引入非平衡状态变量,拓展经典平衡态热力学的Gibbs关系,以及采用合适的方式描述热力学熵的增长是这些理论的共有特性。守恒-耗散理论就是基于热力学第二定律的直接表述得到的。与之前的理论不同的是,守恒-耗散理论是建立在严格的数学结构上的。数学上对方程适定性等性质的要求同物理上的热力学第一和第二定律相联系,从而采用合适的方式对不可逆过程进行建模\cite{}。
	
	\section{守恒-耗散理论}
	首先,守恒-耗散理论认为,热力学过程可以由守恒过程和耗散过程来描述。守恒率可以采用守恒变量$U_c$来描述。例如$U_c$可以是质量、动量、能量等守恒量。耗散过程采用耗散变量$U_d$来描述。类似于其他非平衡态热力学理论的非平衡态状态变量,然而守恒-耗散理论并不假定其形式。守恒-耗散理论假设不可逆过程可以用下面的偏微分方程来描述
	\begin{equation}\label{eq:CDF}
		\partial_t U + \sum_{j=1^3} \partial_{x_j} F_j(U) = Q(U) .
	\end{equation}
	其中
	\begin{equation*}
		U = \left( \begin{array}{c}
			U_c \\ U_d 
			\end{array} \right) , \quad
			 \left( \begin{array}{c}
			f_j(U) \\ g_j(U)
			\end{array} \right), \quad 
			Q(U) = \left( \begin{array}{c}
			0 \\ q(U) 
			\end{array} \right).
	\end{equation*}
	根据文献\cite{},如果假设$Q(U)$前有参数$\frac{1}{\epsilon}$。当$\epsilon$很小时我们期望方程和平衡态的方程相近。基于数学上的考虑\cite{},结合热力学第二定律,守恒-耗散理论的假设如下\cite{}。
	\begin{enumerate}
		\item 存在严格上凸函数$\eta = \eta (U)$,使得$\eta_{UU} F_{jU}$对称。我们称$\eta$为系统的熵函数。
		\item 存在正定矩阵$M = M(U)$,使得$q(U) = M \eta_{U_d}$。我们称$M$为耗散矩阵。
	\end{enumerate}
	由这两条假设可以得到,存在$J = J(U)$使得下面的式子成立
	\begin{equation*}
		\eta_t + \nabla \cdot J = \eta_{U_d}^T M \eta_{U_d} \ge 0.
	\end{equation*}
	可以看出熵的产生率大于等于0,而这是由第二个假设保证的($M$正定)。于是守恒-耗散理论通过这两个条件使得模型满足热力学第二定律。且不可逆过程是由耗散变量引起的\cite{}。
	
	另外,一般我们假定$M$的零空间不依赖于$U$。这样我们可以得到熵的产生率为$0$对应着$\eta_U (U_d)= 0$。我们可以将对应的$U_d=U_{de}(U_c)$称为平衡态时耗散变量的值。此时我们可以得到下面的模型。
	\begin{equation*}
		\partial_t U_c + \sum_{j=1}^3 \partial_{x_j} f_j(U,U_d) = 0 .
	\end{equation*}	
	我们可以得到下面的简单定理。
	\begin{theorem}
		存在$\eta' = \eta'(U_c)$和$J' = J'(U_c)$,使得下面的式子成立
		\begin{equation*}
			\eta' + \nabla \cdot J' = 0.
		\end{equation*}
	\end{theorem}
	\begin{proof}
		首先由$\eta_{UU} F_{jU}$对称可知$\eta_{U_c U_c} f_{jU_c}(U_c,U_d)$对称。从而$\eta_{U_c U_c} f_{jU_c}(Uc, U_{de})$对称。
		2%		May be uncorrect
%		\begin{equation*}
%			\eta_{U_c} (U_c,U_d) F_{j U_c} (U_c,U_d) + \eta_{U_d} (U_c,U_d) F_{jU_d} (U_c,U_d) = J_{jU_c} (U_c,U_d)
%		\end{equation*}
	\end{proof}
	$Q(U) = 0$时对应的体系可以看作平衡态的方程。
	另一方面,如果我们令$M=0$,从而系统的演化方程为
	\begin{equation*}
			\partial_t U + \sum_{j=1^3} \partial_{x_j} F_j(U) = 0 .
	\end{equation*}
	体系的熵函数$\eta$满足
	\begin{equation*}
		\eta_t + \nabla \cdot J = 0
	\end{equation*}
	从而我们可以将这个方程所描述的体系也视为平衡态。这样我们可以得到守恒-耗散理论所描述的非平衡态体系实际上是介于两个平衡态体系之间的。如果没有耗散,则为平衡态$M=0$,如果时间趋于无穷则方程会趋于另一平衡态体系。这于Elliott Lieb等所讨论的非平衡态的概念有些类似\cite{}。他们认为非平衡态为由一个平衡态到另一个平衡态的演化过程。
	
	守恒-耗散理论中耗散矩阵$M$一般假设为对称的。这可以看做Onsager倒易关系的一个推广。实际上如果熵函数是二次的,那么$\eta_{U_d} = \mu U_d$,而熵的产生率
	\begin{equation*}
		\Delta = \eta_{U_d}^T M \eta_{U_d} = \mu^2 U_d^2.
	\end{equation*}
	在Onager理论中,熵的产生率为热力学力$X$和热力学流$J_X$的乘积。Onsager倒易关系认为热力学流和力之间的关系是线性的。及存在矩阵$M'$使得
	\begin{eqnarray*}
		J_X = M X ,\\
		\Delta'= J_X \cdot X = X^T M X
	\end{eqnarray*}
	这也是EIT理论中大部分熵函数都是二次的一个原因。由于Onsager倒易关系的线性性,导致了无法考虑远离平衡态的情况。我们假设$X$不在为平衡态状态变量,而是熵对耗散变量的导数,从而可以包含更加复杂的热力学流和力之间的关系。另一方面,守恒-耗散理论并不假设热力学流和力的区分,仅仅假设耗散变量的存在性。实际上,热力学流和力之间的区分并非有非常明显的界限。

	守恒-耗散理论同GENERIC一样均采用了耗散矩阵来表示熵的耗散。然而其避免了GENERIC理论中Possion括号和耗散括号定义的复杂性,给耗散矩阵的选取带来了更大的自由度。然而,由于守恒-耗散理论假设了方程都可以写成守恒的形式,从而为应用带来了局限。

	守恒-耗散理论可以成功地用于线性粘弹性流体的建模。然而,由于客观导数的非守恒形式,在应用到非线性粘弹性模型时会有困难,如何推广并应用于非线性粘弹性流体模型将在下一章给出。这里,我们首先考察线性粘弹性流体力学模型。

	\section{守恒-耗散理论在线性粘弹性流体模型中的应用}
	本节的内容摘自文献\cite{}。

	首先考虑流体运动的守恒率方程\eqref{eq:fluid}。从而我们取
	\begin{equation*}
		U_c = \{\rho, \rho v, \rho e\}.
	\end{equation*}
	其中$e = \frac{1}{2} v^2 + u$为单位质量流体的总能量。由于我们希望考虑流体的粘弹性和热传导,从而我们希望采用与流体应力张量和热流所对应的非平衡态变量,我们取
	\begin{equation*}
		U_d = \{ w, c\}
	\end{equation*}
	作为耗散变量,分别用来描述应变能的衰减和热传导的能量损失。我们假设熵函数$\eta = \eta (\rho,\rho v ,\rho e, \rho w, \rho c)$。根据熵函数的可加性,比熵$s$(单位质量的熵$\eta = \rho s$)可以写为
	\begin{equation}
		s = s(\nu, u, w, c).
	\end{equation}
	其中$\nu = \frac{1}{\rho}$,$w,c$分别为向量和张量。根据Gibbs关系,我们可以定义系统的温度和静压力
	\begin{equation}
		\theta^{-1} = s_u, \quad \theta^{-1} p = s_{\nu}, \quad \theta^{-1} \tau = s_c.
	\end{equation}
	其中$\tau = P - pI$。前两个式子来源于平衡态热力学的Gibbs关系,最后一个式子是我们的假设(广义Gibbs关系)。
	这里我们假定广义的Gibbs关系成立
	\begin{equation*}
		ds = \theta^{-1} pd\nu + \theta^{-1} du + s_w \cdot dw + s_c :dc
	\end{equation*}
	这里$A:B = \sum_{i,j}A_{ij}B_{ij}$。计算熵的产生率。
	\begin{eqnarray*}
		\eta_t + \nabla \cdot (\eta v) &=& \rho s_t + v \cdot \nabla s, \\
		&=& -\nabla \cdot (\theta^{-1} q) + s_w \cdot [\rho (w_t + v \cdot \nabla w)] \\
		&&+ (s_c:[\rho (c_t + v \cdot \nabla v)] - \theta^{-1} \tau : D) \\
		&& = -\nabla \cdot J + \Delta.
	\end{eqnarray*}
	其中$D = \frac{1}{2} [\nabla v + (\nabla v)^T], J =\theta^{-1} q $,
	\begin{equation*}
		\Delta = s_w \cdot [\rho (w_t + v \cdot \nabla w)] + (s_c:[\rho (c_t + v \cdot \nabla v)] - \theta^{-1} \tau : \nabla v) .
	\end{equation*}
	这里我们用到了$\tau$的对称性,这是动量矩守恒的自然结果\cite{}。
	取$q =s_w,\tau = \theta s_c$,由守恒-耗散理论的假设,可以得到
	\begin{equation*}
		\left( \begin{array}{c} 
			(\rho w)_t +  \nabla \cdot (\rho w \otimes v)  + \nabla \theta^{-1} \\
			(\rho c)_t +  \nabla \cdot (\rho c \otimes v)  - D
		\end{array} \right) = M \cdot
		\left( \begin{array}{c} 
			q \\ \theta^{-1} \tau
		\end{array}\right).
	\end{equation*}
	其中$M=M(\nu,u,w,c)$是正定的,这里我们假设其也是对称的。

	考虑不可压缩流体并忽略温度的影响,则$c$的演化方程可以写为
	\begin{equation*}
		\partial_t c + v \cdot \nabla c - D = M \tau.
	\end{equation*}
	这与不可压线性粘弹性的Maxwell模型相类似。只是这里左端$c$代替了$\tau$。而$\tau$可以表达为$c$的复杂函数。当$\tau$和$c$之间存在线性关系时,例如$\tau = c$时,$\tau$的演化方程为
	\begin{equation*}
		\partial_t \tau+ v \cdot \nabla \tau - D = M \tau.
	\end{equation*}
	这就是我们第一章提到的Maxwell模型。

	为了推广Maxwell模型以考虑压缩性和温度的影响。我们假设熵函数是二次的。
	\begin{equation*}
		s = s_0(\nu,u)  - \frac{1}{2\nu \alpha_0} w^2 - \frac{1}{2\nu \alpha_1} \dot{c}:\dot{c}- \frac{1}{2\alpha_2 \nu} \mathring{c}:\mathring{c}.
	\end{equation*}
	其中$s_0(\nu,u)$为平衡态时的熵函数。$\dot{A} = \frac{1}{3} \mbox{Tr} (A)I, \mathring{A} = \frac{1}{2} (A + A^T) - \frac{1}{3} \mbox{Tr} (A) I$分别表示张量的迹部分和无迹对称部分。取耗散矩阵$M$为
	\begin{equation*}
		M = \left( \begin{array}{ccc} 
			\frac{1}{\theta^2 \lambda} & 0 \\
			0 &  \theta(\frac{1}{\kappa} \dot{\mathcal{T}} + \frac{1}{\xi} \mathring{\mathcal{T}})  
		\end{array} \right)
	\end{equation*}
	其中$\dot{\mathcal{T}}, \mathring{\mathcal{T}}$均为四阶张量,其坐标表示为$\dot{\mathcal{T}}_{kl,k'l'} = \frac{1}{3}\delta_{kl} \delta_{k'l'}, \mathring{\mathcal{T}}_{kl,k'l'} =\frac{1}{2}(\delta_{kk'}\delta_{ll'} + \delta_{kl'} \delta_{lk'} ) -\frac{1}{3}\delta_{kl} \delta_{k'l'} $。从而$\dot{\mathcal{T}} A = \dot{A},\mathring{\mathcal{T}} A = \mathring{A}$。$M$的正定性参看附录1。needapp
	于是我们可以得到系统的演化方程为\cite{}.
	\begin{subequations}
		\begin{align}
			\alpha_0 [\partial_t q +  \nabla \cdot (q \otimes v)] - \nabla \theta^{-1} = -\frac{q}{\theta^2 \lambda}, \\
			\alpha_1[\partial_t (\theta^{-1} \dot{\tau}) + \nabla \cdot (\theta^{-1} \dot{\tau} \otimes v)] + \dot{D} = -\frac{\dot{\tau}}{\kappa}, \\
			\alpha_2[\partial_t (\theta^{-1} \mathring{\tau}) + \nabla \cdot (\theta^{-1} \mathring{\tau} \otimes v)] + \mathring{D} = -\frac{\dot{\tau}}{\xi}. 
		\end{align}
	\end{subequations}
	当$\alpha_0, \alpha_1, \alpha_1$趋于0时,可以得到
	\begin{equation*}
		q = -\lambda \nabla \theta, \quad \tau = - \xi \mathring{D} - \kappa \dot{D}.
	\end{equation*}
	分别为经典的Fourier-Newton-Stokes本构关系\cite{}。

	在不可压缩和忽略温度的情形,$\nabla \cdot v = 0, \dot{\tau}=0$,$\tau = \mathring{\tau}$,则$\sigma = -\tau$的演化方程为
	\begin{equation*}
		\partial_t \sigma + v \cdot \nabla \sigma - D = -\frac{\sigma}{\xi}. 
	\end{equation*}
	此即Maxwell粘弹性流体力学模型\eqref{eq:maxwell}.	

	考虑温度的不可压Maxwell模型为
	\begin{eqnarray} \label{eq:Tmaxwell}
		v_t + v \cdot \nabla v + \nabla \cdot (-\theta \frac{c}{\alpha_2}) = 0, \\
		c_t + v \cdot \nabla c - D = -\frac{1}{\xi} c.
	\end{eqnarray}
	而由微观方法得到的高分子稀溶液的模型为 \cite{}
	\begin{eqnarray*}
		v_t + v \cdot \nabla v = \eta_p (H A - KT I), \\
		A_t + v \cdot \nabla A - \nabla v A - A(\nabla v)^T  = -\frac{4 H}{\zeta} A+ \frac{4kT}{\zeta} I.
	\end{eqnarray*}
	其中$H = \frac{3kT}{Nb^2}$.其中$k$为Boltzmann常数,$N$为高分子链弹簧模型的弹簧个数,$b$为弹簧长度。取$\beta = \frac{H}{kT} = \frac{3}{Nb^2}$,则应力张量$\sigma = k T(\beta A-I)$。令$c'=\frac{1}{2}(\beta A-I)$,可以得到$c'$的方程如下。
	\begin{equation*}
		c'_t + v \cdot \nabla c'- \nabla v c' - c'(\nabla v)^T -  D  = -\frac{2 kT \beta}{\zeta} c'.
	\end{equation*}
	其对应的线性模型为
	\begin{eqnarray*}
		v_t + v \cdot \nabla v = \frac{1}{2} \eta_p kT c', \\
		c'_t + v \cdot \nabla c' -  D  = -\frac{2 kT \beta}{\zeta} c'.
	\end{eqnarray*}
	在\eqref{eq:Tmaxwell}中令$\alpha_2 = \frac{2}{\eta_p k}, T = \theta,\xi = \frac{\zeta}{2 kT \beta}$即可以得到本模型。可见守恒-耗散理论可以很好地用来对线性粘弹性模型进行建模。

	对于一般的情况,我们亦可选取$M$为
	\begin{equation*}
		M_{ikl,i'k'l'} = \left( \begin{array}{ccc} 
			\mu_1 \delta_{ii' } & \frac{1}{3}\beta'_i \delta_{k'l'} +  [\frac{1}{2} (\beta''_{k'} \delta_{il'} + \beta''_{l'} \delta_{ik'}) - \frac{1}{3} \beta_{i} \delta_{k'l'}]\\
			\frac{1}{3}\beta'_{i'} \delta_{kl} +  [\frac{1}{2} (\beta''_{k} \delta_{i'l} + \beta''_{l'} \delta_{i'k}) - \frac{1}{3} \beta_{i'} \delta_{kl}] &  \frac{1}{\kappa} \dot{\mathcal{T}} + \frac{1}{\xi} \mathring{\mathcal{T}}  
		\end{array} \right)
	\end{equation*} 
	注意参数$\mu_1,\beta',\beta'',\kappa,\zeta$的选取需要保证$M$的正定型。
	从而得到一般的线性本构关系为
	\begin{subequations}
		\begin{align*}
			\alpha_0 [\partial_t q +  \nabla \cdot (q \otimes v)] - \nabla \theta^{-1} = - \mu_1 q -\beta' \theta^{-1} \dot{\tau} - \beta'' \theta^{-1} \mathring{\tau} , \\
			\alpha_1[\partial_t (\theta^{-1} \dot{\tau}) + \nabla \cdot (\theta^{-1} \dot{\tau} \otimes v)] + \dot{D} = -\frac{\dot{\tau}}{\kappa} - \frac{1}{3}(\beta' \cdot q) I, \\
			\alpha_2[\partial_t (\theta^{-1} \mathring{\tau}) + \nabla \cdot (\theta^{-1} \mathring{\tau} \otimes v)] + \mathring{D} = -\frac{\dot{\tau}}{\xi} - \frac{1}{2} (\beta''  \otimes q + q \otimes \beta'') + \frac{1}{3}(\beta'' \cdot q) I. 
		\end{align*}
	\end{subequations}
	这与第一章EIT得到的本构关系\eqref{eq:EITconstitutive}类似。这里我们采用耗散变量$w,c$替代了EIT中的$\nabla T, \nabla \dot{\tau}$和$\nabla \cdot{\mathring{\tau}}$。

	\section{粘弹性流体力学热传导模型的守恒-耗散理论}
	粘弹性流体中的热传导模型目前研究较少,目前大部分模型采用的仍然是Fouerier热传导定律或者Cattaneo定律。本小节我们将讨论推广的Cattaneo定律和热质理论,以及Guyer-Krumhans模型,首先我们采用守恒-耗散理论给出这些模型的理论推导,然后结合上一节粘弹性流体的建模,给出这些这些热传导模型在粘弹性流体力学模型中的应用。

首先我们仅考虑热传导而忽略流体的运动和压缩性(假设$\rho = 1$),那么体系的方程可以写为
\begin{equation*}
	 u_t  + \nabla \cdot q = 0 .	
\end{equation*}
那么$w$的方程为
\begin{equation*}
	 w_t + \nabla \theta^{-1} = M' q. 
\end{equation*}
其中$q = S_w$。

\subsection{Cattaneo定律}
上一节熵的选取为$s = s( u)  -\frac{1}{2 \alpha_0} w^2, M' = \frac{1}{\theta^2 \lambda} $。得到的热传导定律为
\begin{equation*}
	\alpha_0 \partial_t q  - \nabla \theta^{-1} = -\frac{q}{\theta^2 \lambda}.
\end{equation*}
若选取$\alpha_0 = \frac{\chi}{\lambda\theta^2}$,那么可以得到经典的Cattaneo热传导定律
\begin{equation*}
	 \chi  \partial_t q = - (q +\lambda \nabla \theta).
\end{equation*}
此时假设$u = c_v \theta, s_0 = c_v \ln \theta$,那么可以得到$s = s (\theta,w)$的Hessian矩阵为
\begin{equation*}
	\left( \begin{array}{cc} 
		-\frac{c_v}{\theta^2} - \frac{\lambda}{\chi} w^2 & \frac{2 \theta \lambda w }{\chi} \\
		\frac{2 \theta \lambda w }{\chi}  & -\frac{\lambda \theta^2}{\chi }
	\end{array} \right).
\end{equation*}
 这一矩阵在$c_v >0$时对一切$\theta, w$均为负定的。即熵函数为上凸的。

\subsection{推广的Cattaneo定律}
 然而注意到$\alpha_0$假设是常数,所以这里我们需要$\frac{\chi}{\lambda \theta^2}$为常数。
如果上面的熵函数中$\alpha_0$依赖于内能(温度)$a_0=\alpha(u)$,那么$q$的方程应写为
\begin{equation*}
({\alpha(u)}{q})_t-\nabla \theta^{-1}= M'{q}
\end{equation*}
我们称这个方程为推广的Cattaneo定律。其中$M'>0$。

如果我们采用平衡态温度$T$来替代$\theta$,由$T$定义
\begin{equation*}
	T^{-1}=\frac{\partial S_{eq}(u)}{\partial u},
\end{equation*}
可以得到
\begin{equation}
\theta^{-1}=T^{-1}-\frac{1}{2}(\frac{1}{\alpha(u)})'{c}^2
\end{equation}
这里$()'$为对$u$的导数。从而
\begin{equation*}
{w}_t+\nabla(T^{-1}-\frac{1}{2}(\frac{1}{\alpha(u)})' {w}^2)=M'{q}
\end{equation*}
或
\begin{equation*}
({\alpha(u)}{{q}})_t-\nabla(T^{-1}- \frac{\alpha'(u){q}^2}{2})-M{q}=0
\end{equation*}
令$M=\frac{1}{\lambda T^2}, \alpha(u)=\frac{\rho c_v}{2\gamma u^3}=\frac{\rho}{2\gamma c_v^2 T^3}$,其中$\rho$为流体的密度,$c_v$为比热。
最终我们得到
\begin{equation}
\tau_{TM} {q}_t-3c_v {L}T_t+3(\nabla {q})^T \cdot {L}+\lambda(1-6M_H^2)\nabla T+{q}=0
\end{equation}
其中
\begin{equation*}
\tau_{TM}=\frac{\lambda \rho}{2 \gamma c_v^2 T} \quad {L}=\frac{\lambda \rho}{2 \gamma c_v^3 T^2}{q} \quad M_H^2=\frac{\lambda \rho {q}^2}{2\gamma c_v^3 T^3}
\end{equation*}

由过增元等人发展的热质理论的模型为\cite{}
\begin{equation}
\tau_{TM} {q}_t-c_v {L}T_t+\nabla {q} \cdot {L}+\lambda(1-M_H^2)\nabla T+{q}=0.
\end{equation}
这里我们导出的模型和热质理论的模型仅系数由略微差异。

下面我们来考察$s=s(u,w)$的上凸性。$s$的Hessian矩阵为
\begin{eqnarray*}
\left( \begin{array}{ll} -\frac{1}{c_v T^2}-\frac{1}{2}(\frac{1}{\alpha(u)})'' {c}^2 & (\frac{1}{\alpha})'{c} \\ (\frac{1}{\alpha})'{c} & -\frac{1}{\alpha(u)} \end{array} \right).
\end{eqnarray*}
为保证这一矩阵的负定性,我们要求对角元是负的且矩阵为矩阵的行列式为负的。从而可以得到$c_v >0$与
\begin{eqnarray*}
{q}=\alpha(u){c} \le \frac{\rho c_v T}{\sqrt{6}} \sqrt{\frac{\lambda}{\rho c_v \tau_{TM}}}
\end{eqnarray*}
其中 $\sqrt{\frac{\lambda}{\rho c_v \tau_{TM}}}$ 为热波的最大速度\cite{Jou1996extended}。 所以熵函数仅在热流小于特定值时为负定的。这一特定值对应于热波允许的最大速度。

从这一例子可以看出,简单的熵函数的形式就可以导出复杂的热传导本构关系。推广的Cattaneo定律在特定的参数选择下可以得到与热质理论方程类似的模型。

\subsection{Guyer-Krumhansl模型}
物质中的热传导可以采用线性化的声子Boltzmann方程来导出。1966年Guyer和Krumhansl发展了固体结构热传导的Guyer-Krumhansl模型\cite{guyer1966solution}。这一模型考虑了固体结构对热传导的影响。而由于粘弹性流体中高分子也具有一定的结构,这一模型也可以用来描述液晶等粘弹性流体的热传导\cite{}。

由于晶格等结构的存在,在固体中热传导可能出现各向异性。同样的在粘弹性流体中也可能与高分子取向等结构引起的热传导各向异性。为了描述热传导的各项异性我们采用张量$Q$来描述。我们假设张量${Q}$对称。其可以分解为${Q}=\mathring{{Q}}+\dot{Q}{I}$。

假设熵具有形式
\begin{equation*}
s(u,{w},\mathring{{Q}},\dot{Q})=s_{eq}(u)-\frac{1}{2 \alpha_0}{w}^2-\frac{1}{2\tau_1} {\mathring{{Q}}}:{\mathring{{Q}}}-\frac{1}{2\tau_2}\dot{Q}^2
\end{equation*}
由计算可以得到(利用Gibbs关系)
\begin{eqnarray*}
s_t &=& \theta^{-1} u_t +s_{w} \cdot {w}_t + s_{\mathring{{Q}}}:\mathring{{Q}}_t+s_{\dot{Q}} \dot{Q}_t \\
    &=& -\nabla \cdot (\theta^{-1} {q}+\nabla (s_{\mathring{{Q}}}+s_{\dot{Q}}) \cdot {q})\\
    	&&+({w}_t+\nabla \cdot (s_{\mathring{{Q}}}+s_{\dot{Q}})+\nabla \theta^{-1}) \cdot {q} \\
&& +s_{\mathring{{Q}}}:(\mathring{{Q}}_t+(\mathring{\nabla {q}})^{sym})+s_{\dot{Q}}(\dot{Q}_t+\frac{1}{3}\nabla \cdot {q}).
\end{eqnarray*}
为了保证上的产生率大于0(守恒-耗散理论的第二条假设)的成立,我们就假设存在矩阵$M''$,使得
\begin{equation*}
\left( \begin{array}{ll} {w}_t+\nabla \cdot (S_{\mathring{{Q}}}+S_Q)+\nabla \theta^{-1} \\ \mathring{{Q}}_t+(\mathring{\nabla {q}})^{sym} \\ \dot{Q}_t+\frac{1}{3} \nabla \cdot {q} \end{array} \right) = M \left( \begin{array}{l} {q} \\ S_{\mathring{{Q}}} \\S_{\dot{Q}} \end{array} \right).
\end{equation*}
我们考虑最简单的情况,取
\begin{equation*}
	M=\left( \begin{array}{lll} M_0 & 0 & 0 \\0 & M_1 & 0 \\0 & 0 & M_2 \end{array} \right)
\end{equation*}
其中$M_0,M_1,M_2$均为正定的。

我们有
\begin{eqnarray*}
\mathring{{Q}}_t+(\mathring{\nabla {q}})^{sym}=-\frac{M_1}{\tau_1}\mathring{{Q}}, \\
\dot{Q}_t+\frac{1}{3} \nabla \cdot {q}=-\frac{M_2}{\tau_2}\dot{Q}.
\end{eqnarray*}
当 $\tau_1 \to 0, \tau_2 \to 0$时,上面的方程近似为
\begin{eqnarray*} 
\mathring{{Q}}=-\frac{\tau_1}{M_1}(\mathring{\nabla {q}})^{sym} ,\quad  \quad \mathring{S}_{Q}=\frac{1}{M_1}(\mathring{\nabla {q}})^{sym} \\
\dot{Q}=-\frac{\tau_2}{3M_2}\nabla \cdot {q} \quad , \quad \bar{S}_{\dot{Q}}=\frac{1}{3M_2} \nabla \cdot {q}
\end{eqnarray*}
取$M$为
\begin{equation}
M_0=\frac{1}{\lambda \theta^2}, M_1=\frac{\lambda \theta^2}{d\tau}, M_2=\frac{2\lambda \theta^2}{5d\tau}
\end{equation}
与$a_0=\frac{\tau_0}{\lambda \theta^2}$,
我们可以得到Guyer-Krumhansl模型(以$T$代替$\theta$)\cite{Jou1996extended}
\begin{equation}
{q}_t+\frac{{q}}{\tau_0}+\frac{\lambda}{\tau_0}\nabla T=\frac{1}{2}d(\nabla^2 {q}+2\nabla \nabla \cdot {q})
\end{equation}
注意这里我们假设了$a_0,\tau_1,\tau_2$均为常数。

我们可以得到了推广的Guyer-Krumhansl模型。
\begin{subequations}
	\begin{align}
(\alpha_0 {q})_t - \nabla \cdot ({\frac{1}{\tau_1} \mathring{{Q}}}+\frac{1}{\tau_2}\dot{Q})+\nabla \theta^{-1} = M_0 {q} \\
\mathring{{Q}}_t+(\mathring{\nabla {q}})^{sym}=-\frac{M_1}{\tau_1}\mathring{{Q}} \\
\dot{Q}_t+\frac{1}{3} \nabla \cdot {q}=-\frac{M_2}{\tau_2} \dot{Q}
	\end{align}
\end{subequations}

在$a_0,\tau_1,\tau_2$均为常数时,熵函数的Hessian矩阵为
\begin{eqnarray*}
\left(\begin{array}{llll}  -\frac{1}{c_v \theta^2} & 0 & 0 & 0 \\
                                           0  & -\frac{1}{\alpha_0} & 0 & 0 \\
										   0 & 0 & -\frac{1}{2\tau_1} & 0 \\
										   0 & 0 & 0 & -\frac{1}{2\tau_2}
										   \end{array} \right).
\end{eqnarray*}
熵的上凸性的等价于$c_v$和参数$\alpha_0,\tau_1,\tau_2$为正的。

利用推广的Guyer-Krumhansl模型可以得到热质理论和Guyer-Krumhansl模型的组合模型。即令$M_0=\frac{1}{\lambda T^2}, \alpha_0 = \alpha_0 (u) =\frac{\rho c_v}{2\gamma u^3}=\frac{\rho}{2\gamma c_v^2 T^3}$,于是可以得到
\begin{eqnarray*}
\tau_{TM} {q}_t-c_v {L}T_t+\nabla {q} \cdot {L}+\lambda(1-M_H^2)\nabla T+{q} - \lambda T^2 \nabla \cdot ({\frac{1}{\tau_1} \mathring{{Q}}}+\frac{1}{\tau_2}\dot{Q})=0, \\
\mathring{{Q}}_t+(\mathring{\nabla {q}})^{sym}=-\frac{M_1}{\tau_1}\mathring{{Q}}, \\
\dot{Q}_t+\frac{1}{3} \nabla \cdot {q}=-\frac{M_2}{\tau_2} \dot{Q}.
\end{eqnarray*}
令$\tau_1,\tau_2$趋于0,并取$M_1 = \frac{\lambda T^2}{d}, M_2 = \frac{2\lambda T^2}{5 d }$,可以得到下面的Guyer-Kramhansl-Thermomass模型。
\begin{eqnarray*}
	\tau_{TM} {q}_t-c_v {L}T_t+\nabla {q} \cdot {L}+\lambda(1-M_H^2)\nabla T+{q} = \frac{1}{2}d (\nabla^2 q +2 \nabla \nabla \cdot q). \\
\end{eqnarray*}


\subsection{考虑热传导的线性粘弹性流体模型}
上面对热传导模型的分析可以用于粘弹性流体力学的建模之中。我们仍然选取守恒变量$U_c$为密度、动量和能量。耗散变量我们选取$w,Q$和$c$。假设



	\section{守恒-耗散理论的数学结果}
	守恒-耗散理论一方面考虑了热力学第一定律和第二定律和Onsager倒易关系等物理原理,另一方面它建立在严格的数学理论之上。熵函数的存在性和方程的守恒形式共同保证方程可以被对陈化,从而根据对称双曲组的相关理论,方程的局部适定性可以保证。而对右端项的要求可以使得方程满足文献\cite{}中的条件,从而可以保证有端项含有松弛参数$\epsilon$时松弛极限$\epsilon$很小时可以与形式上得到的方程相近似,从而保证了离平衡态“很近”的体系可以很好地采用平衡态体系近似。

	守恒-耗散理论得到的方程均为可对称双曲方程组,所以可以采用对称双曲组的相关理论对方程的解进行分析。例如在平衡态附近方程解的全局适定性可以采用Kawashima理论进行分析\cite{}。可压缩模型的小马赫数极限可以采用类似可压Euler方程的小马赫数极限进行分析\cite{}。方程的弱解的存在性和间断解可以采用相关的理论进行分析\cite{}。

	本节主要考虑方程的强解,我们采用的解空间为Sobolev空间$H^s, s \in N$。

	\section{方程解的存在性定理}
	由守恒-耗散理论的第一条假设,存在正定对称矩阵$A_0(U)  = -\eta_{UU}$使得$A_0(U) A_j(U)$对称。$A_0$的正定性可由$\eta$的上凸性得到。利用对称双曲方程组的存在性理论,可以得到下面的定理\cite{}。
	\begin{theorem}
		假设$U_0 \in H^{s}$,其中$s>\frac{d}{2} +1$,其中$d$为空间的维数。
	\end{theorem}
	于是对于线性弹性流体力学模型
	\end{document}
