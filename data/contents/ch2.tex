% \chapter{守恒-耗散理论与线性粘弹性流体力学模型}
 \documentclass{article}
 \usepackage{ctex}
 \usepackage{amsmath}
 \usepackage{amsthm}
\newtheorem{theorem}{定理}

 \begin{document}
 	非平衡态热力学的核心是热力学第二定律。有热力学第二定律可以知道不可逆过程必然伴随熵增,而如何描述这一定律变得十分重要。无论是CIT、EIT,还是RT、GENERIC,其出发点均为热力学第二定律。通过引入非平衡状态变量,拓展经典平衡态热力学的Gibbs关系,以及采用合适的方式描述热力学熵的增长是这些理论的共有特性。守恒-耗散理论就是基于热力学第二定律的直接表述得到的。与之前的理论不同的是,守恒-耗散理论是建立在严格的数学结构上的。数学上对方程适定性等性质的要求同物理上的热力学第一和第二定律相联系,从而采用合适的方式对不可逆过程进行建模\cite{}。
	
	\section{守恒-耗散理论}
	首先,守恒-耗散理论认为,热力学过程可以由守恒过程和耗散过程来描述。守恒率可以采用守恒变量$U_c$来描述。例如$U_c$可以是质量、动量、能量等守恒量。耗散过程采用耗散变量$U_d$来描述。类似于其他非平衡态热力学理论的非平衡态状态变量,然而守恒-耗散理论并不假定其形式。守恒-耗散理论假设不可逆过程可以用下面的偏微分方程来描述
	\begin{equation}\label{eq:CDF}
		\partial_t U + \sum_{j=1^3} \partial_{x_j} F_j(U) = Q(U) .
	\end{equation}
	其中
	\begin{equation*}
		U = \left( \begin{array}{c}
			U_c \\ U_d 
			\end{array} \right) , \quad
			 \left( \begin{array}{c}
			f_j(U) \\ g_j(U)
			\end{array} \right), \quad 
			Q(U) = \left( \begin{array}{c}
			0 \\ q(U) 
			\end{array} \right).
	\end{equation*}
	根据文献\cite{},如果假设$Q(U)$前有参数$\frac{1}{\epsilon}$。当$\epsilon$很小时我们期望方程和平衡态的方程相近。基于数学上的考虑\cite{},结合热力学第二定律,守恒-耗散理论的假设如下\cite{}。
	\begin{enumerate}
		\item 存在严格上凸函数$\eta = \eta (U)$,使得$\eta_{UU} F_{jU}$对称。我们称$\eta$为系统的熵函数。
		\item 存在正定矩阵$M = M(U)$,使得$q(U) = M \eta_{U_d}$。我们称$M$为耗散矩阵。
	\end{enumerate}
	由这两条假设可以得到,存在$J = J(U)$使得下面的式子成立
	\begin{equation*}
		\eta_t + \nabla \cdot J = \eta_{U_d}^T M \eta_{U_d} \ge 0.
	\end{equation*}
	可以看出熵的产生率大于等于0,而这是由第二个假设保证的($M$正定)。于是守恒-耗散理论通过这两个条件使得模型满足热力学第二定律。且不可逆过程是由耗散变量引起的\cite{}。
	
	另外,一般我们假定$M$的零空间不依赖于$U$。这样我们可以得到熵的产生率为$0$对应着$\eta_U (U_d)= 0$。我们可以将对应的$U_d=U_{de}(U_c)$称为平衡态时耗散变量的值。此时我们可以得到下面的模型。
	\begin{equation*}
		\partial_t U_c + \sum_{j=1}^3 \partial_{x_j} f_j(U,U_d) = 0 .
	\end{equation*}	
	我们可以得到下面的简单定理。
	\begin{theorem}
		存在$\eta' = \eta'(U_c)$和$J' = J'(U_c)$,使得下面的式子成立
		\begin{equation*}
			\eta' + \nabla \cdot J' = 0.
		\end{equation*}
	\end{theorem}
	\begin{proof}
		首先由$\eta_{UU} F_{jU}$对称可知$\eta_{U_c U_c} f_{jU_c}(U_c,U_d)$对称。从而$\eta_{U_c U_c} f_{jU_c}(Uc, U_{de})$对称。
		2%		May be uncorrect
%		\begin{equation*}
%			\eta_{U_c} (U_c,U_d) F_{j U_c} (U_c,U_d) + \eta_{U_d} (U_c,U_d) F_{jU_d} (U_c,U_d) = J_{jU_c} (U_c,U_d)
%		\end{equation*}
	\end{proof}
	$Q(U) = 0$时对应的体系可以看作平衡态的方程。
	另一方面,如果我们令$M=0$,从而系统的演化方程为
	\begin{equation*}
			\partial_t U + \sum_{j=1^3} \partial_{x_j} F_j(U) = 0 .
	\end{equation*}
	体系的熵函数$\eta$满足
	\begin{equation*}
		\eta_t + \nabla \cdot J = 0
	\end{equation*}
	从而我们可以将这个方程所描述的体系也视为平衡态。这样我们可以得到守恒-耗散理论所描述的非平衡态体系实际上是介于两个平衡态体系之间的。如果没有耗散,则为平衡态$M=0$,如果时间趋于无穷则方程会趋于另一平衡态体系。这于Elliott Lieb等所讨论的非平衡态的概念有些类似\cite{}。他们认为非平衡态为由一个平衡态到另一个平衡态的演化过程。
	
	
\end{document}