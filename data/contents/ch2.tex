% \chapter{守恒-耗散理论与线性粘弹性流体力学模型}
 \documentclass{article}
 \usepackage{ctex}
 \usepackage{amsmath}
 \usepackage{}
 \usepackage{amsthm}
\newtheorem{theorem}{定理}

 \begin{document}
 	非平衡态热力学的核心是热力学第二定律。有热力学第二定律可以知道不可逆过程必然伴随熵增,而如何描述这一定律变得十分重要。无论是CIT、EIT,还是RT、GENERIC,其出发点均为热力学第二定律。通过引入非平衡状态变量,拓展经典平衡态热力学的Gibbs关系,以及采用合适的方式描述热力学熵的增长是这些理论的共有特性。守恒-耗散理论就是基于热力学第二定律的直接表述得到的。与之前的理论不同的是,守恒-耗散理论是建立在严格的数学结构上的。数学上对方程适定性等性质的要求同物理上的热力学第一和第二定律相联系,从而采用合适的方式对不可逆过程进行建模\cite{}。
	
	\section{守恒-耗散理论}
	首先,守恒-耗散理论认为,热力学过程可以由守恒过程和耗散过程来描述。守恒率可以采用守恒变量$U_c$来描述。例如$U_c$可以是质量、动量、能量等守恒量。耗散过程采用耗散变量$U_d$来描述。类似于其他非平衡态热力学理论的非平衡态状态变量,然而守恒-耗散理论并不假定其形式。守恒-耗散理论假设不可逆过程可以用下面的偏微分方程来描述
	\begin{equation}\label{eq:CDF}
		\partial_t U + \sum_{j=1}^3 \partial_{x_j} F_j(U) = \mathcal{Q} (U) .
	\end{equation}
	其中
	\begin{equation*}
		U = \left( \begin{array}{c}
			U_c \\ U_d 
			\end{array} \right) , \quad
			 \left( \begin{array}{c}
			f_j(U) \\ g_j(U)
			\end{array} \right), \quad 
			\mathcal{Q}(U) = \left( \begin{array}{c}
			0 \\ \mathcal{q} (U) 
			\end{array} \right).
	\end{equation*}
	根据文献\cite{},如果假设$Q(U)$前有参数$\frac{1}{\epsilon}$。当$\epsilon$很小时我们期望方程和平衡态的方程相近。基于数学上的考虑\cite{},结合热力学第二定律,守恒-耗散理论的假设如下\cite{}。
	\begin{enumerate}
		\item 存在严格上凸函数$\eta = \eta (U)$,使得$\eta_{UU} F_{jU}$对称。我们称$\eta$为系统的熵函数。
		\item 存在正定矩阵$M = M(U)$,使得$\mathcal{q}(U) = M \eta_{U_d}$。我们称$M$为耗散矩阵。
	\end{enumerate}
	由这两条假设可以得到,存在$J = J(U)$使得下面的式子成立
	\begin{equation*}
		\eta_t + \nabla \cdot J = \eta_{U_d}^T M \eta_{U_d} \ge 0.
	\end{equation*}
	可以看出熵的产生率大于等于0,而这是由第二个假设保证的($M$正定)。于是守恒-耗散理论通过这两个条件使得模型满足热力学第二定律。且不可逆过程是由耗散变量引起的\cite{}。
	
	另外,一般我们假定$M$的零空间不依赖于$U$。这样我们可以得到熵的产生率为$0$对应着$\eta_U (U_d)= 0$。我们可以将对应的$U_d=U_{de}(U_c)$称为平衡态时耗散变量的值。此时我们可以得到下面的模型。
	\begin{equation*}
		\partial_t U_c + \sum_{j=1}^3 \partial_{x_j} f_j(U,U_d) = 0 .
	\end{equation*}	
	我们可以得到下面的简单定理。
	\begin{theorem}
		存在$\eta' = \eta'(U_c)$和$J' = J'(U_c)$,使得下面的式子成立
		\begin{equation*}
			\eta' + \nabla \cdot J' = 0.
		\end{equation*}
	\end{theorem}
	\begin{proof}
		首先由$\eta_{UU} F_{jU}$对称可知$\eta_{U_c U_c} f_{jU_c}(U_c,U_d)$对称。从而$\eta_{U_c U_c} f_{jU_c}(Uc, U_{de})$对称。
		2%		May be uncorrect
%		\begin{equation*}
%			\eta_{U_c} (U_c,U_d) F_{j U_c} (U_c,U_d) + \eta_{U_d} (U_c,U_d) F_{jU_d} (U_c,U_d) = J_{jU_c} (U_c,U_d)
%		\end{equation*}
	\end{proof}
	$Q(U) = 0$时对应的体系可以看作平衡态的方程。
	另一方面,如果我们令$M=0$,从而系统的演化方程为
	\begin{equation*}
			\partial_t U + \sum_{j=1^3} \partial_{x_j} F_j(U) = 0 .
	\end{equation*}
	体系的熵函数$\eta$满足
	\begin{equation*}
		\eta_t + \nabla \cdot J = 0
	\end{equation*}
	从而我们可以将这个方程所描述的体系也视为平衡态。这样我们可以得到守恒-耗散理论所描述的非平衡态体系实际上是介于两个平衡态体系之间的。如果没有耗散,则为平衡态$M=0$,如果时间趋于无穷则方程会趋于另一平衡态体系。这于Elliott Lieb等所讨论的非平衡态的概念有些类似\cite{}。他们认为非平衡态为由一个平衡态到另一个平衡态的演化过程。
	
	守恒-耗散理论中耗散矩阵$M$一般假设为对称的。这可以看做Onsager倒易关系的一个推广。实际上如果熵函数是二次的,那么$\eta_{U_d} = \mu U_d$,而熵的产生率
	\begin{equation*}
		\Delta = \eta_{U_d}^T M \eta_{U_d} = \mu^2 U_d^2.
	\end{equation*}
	在Onager理论中,熵的产生率为热力学力$X$和热力学流$J_X$的乘积。Onsager倒易关系认为热力学流和力之间的关系是线性的。及存在矩阵$M'$使得
	\begin{eqnarray*}
		J_X = M X ,\\
		\Delta'= J_X \cdot X = X^T M X
	\end{eqnarray*}
	这也是EIT理论中大部分熵函数都是二次的一个原因。由于Onsager倒易关系的线性性,导致了无法考虑远离平衡态的情况。我们假设$X$不在为平衡态状态变量,而是熵对耗散变量的导数,从而可以包含更加复杂的热力学流和力之间的关系。另一方面,守恒-耗散理论并不假设热力学流和力的区分,仅仅假设耗散变量的存在性。实际上,热力学流和力之间的区分并非有非常明显的界限。

	守恒-耗散理论同GENERIC一样均采用了耗散矩阵来表示熵的耗散。然而其避免了GENERIC理论中Possion括号和耗散括号定义的复杂性,给耗散矩阵的选取带来了更大的自由度。然而,由于守恒-耗散理论假设了方程都可以写成守恒的形式,从而为应用带来了局限。

	守恒-耗散理论可以成功地用于线性粘弹性流体的建模。然而,由于客观导数的非守恒形式,在应用到非线性粘弹性模型时会有困难,如何推广并应用于非线性粘弹性流体模型将在下一章给出。这里,我们首先考察线性粘弹性流体力学模型。

	\section{守恒-耗散理论在线性粘弹性流体模型中的应用}
	本节的内容摘自文献\cite{}。

	首先考虑流体运动的守恒率方程\eqref{eq:fluid}。从而我们取
	\begin{equation*}
		U_c = \{\rho, \rho v, \rho e\}.
	\end{equation*}
	其中$e = \frac{1}{2} v^2 + u$为单位质量流体的总能量。由于我们希望考虑流体的粘弹性和热传导,从而我们希望采用与流体应力张量和热流所对应的非平衡态变量,我们取
	\begin{equation*}
		U_d = \{ w, c\}
	\end{equation*}
	作为耗散变量,分别用来描述应变能的衰减和热传导的能量损失。我们假设熵函数$\eta = \eta (\rho,\rho v ,\rho e, \rho w, \rho c)$。根据熵函数的可加性,比熵$s$(单位质量的熵$\eta = \rho s$)可以写为
	\begin{equation}
		s = s(\nu, u, w, c).
	\end{equation}
	其中$\nu = \frac{1}{\rho}$,$w,c$分别为向量和张量。根据Gibbs关系,我们可以定义系统的温度和静压力
	\begin{equation}
		\theta^{-1} = s_u, \quad \theta^{-1} p = s_{\nu}, \quad \theta^{-1} \tau = s_c.
	\end{equation}
	其中$\tau = P - pI$。前两个式子来源于平衡态热力学的Gibbs关系,最后一个式子是我们的假设(广义Gibbs关系)。
	这里我们假定广义的Gibbs关系成立
	\begin{equation*}
   	\end{equation*}
	这里$A:B = \sum_{i,j}A_{ij}B_{ij}$。计算熵的产生率。
	\begin{eqnarray*}
		\eta_t + \nabla \cdot (\eta v) &=& \rho (s_t + v \cdot \nabla s), \\
		&=& -\nabla \cdot (\theta^{-1} q) + s_w \cdot [\rho (w_t + v \cdot \nabla w)] \\
		&&+ (s_c:[\rho (c_t + v \cdot \nabla v)] - \theta^{-1} \tau : D) \\
		&& = -\nabla \cdot J + \Delta.
	\end{eqnarray*}
	其中$D = \frac{1}{2} [\nabla v + (\nabla v)^T], J =\theta^{-1} q $,
	\begin{equation*}
		\Delta = s_w \cdot [\rho (w_t + v \cdot \nabla w)] + (s_c:[\rho (c_t + v \cdot \nabla v)] - \theta^{-1} \tau : \nabla v) .
	\end{equation*}
	这里我们用到了$\tau$的对称性,这是动量矩守恒的自然结果\cite{}。
	取$q =s_w,\tau = \theta s_c$,由守恒-耗散理论的假设,可以得到
	\begin{equation} \label{eq:CDFgeneral  }
		\left( \begin{array}{c} 
			(\rho w)_t +  \nabla \cdot (\rho w \otimes v)  + \nabla \theta^{-1} \\
			(\rho c)_t +  \nabla \cdot (\rho c \otimes v)  - D
		\end{array} \right) = M \cdot
		\left( \begin{array}{c} 
			q \\ \theta^{-1} \tau
		\end{array}\right).
	\end{equation}
	其中$M=M(\nu,u,w,c)$是正定的,这里我们假设其也是对称的。

	考虑不可压缩流体并忽略温度的影响,则$c$的演化方程可以写为
	\begin{equation*}
		\partial_t c + v \cdot \nabla c - D = M \tau.
	\end{equation*}
	这与不可压线性粘弹性的Maxwell模型相类似。只是这里左端$c$代替了$\tau$。而$\tau$可以表达为$c$的复杂函数。当$\tau$和$c$之间存在线性关系时,例如$\tau = c$时,$\tau$的演化方程为
	\begin{equation*}
		\partial_t \tau+ v \cdot \nabla \tau - D = M \tau.
	\end{equation*}
	这就是我们第一章提到的Maxwell模型。

	为了推广Maxwell模型以考虑压缩性和温度的影响。我们假设熵函数是二次的。
	\begin{equation*}
		s = s_0(\nu,u)  - \frac{1}{2\nu \alpha_0} w^2 - \frac{1}{2\nu \alpha_1} \dot{c}^2- \frac{1}{2\alpha_2 \nu} \mathring{c}:\mathring{c}.
	\end{equation*}
	其中$s_0(\nu,u)$为平衡态时的熵函数。$\dot{A} = \frac{1}{3} \mbox{Tr} (A)I, \mathring{A} = \frac{1}{2} (A + A^T) - \frac{1}{3} \mbox{Tr} (A) $分别表示张量的迹部分和无迹对称部分。取耗散矩阵$M$为
	\begin{equation*}
		M = \left( \begin{array}{ccc} 
			\frac{1}{\theta^2 \lambda} & 0 \\
			0 &  \theta(\frac{1}{\kappa} \dot{\mathcal{T}} + \frac{1}{\xi} \mathring{\mathcal{T}})  
		\end{array} \right)
	\end{equation*}
	其中$\dot{\mathcal{T}}, \mathring{\mathcal{T}}$均为四阶张量,其坐标表示为$\dot{\mathcal{T}}_{kl,k'l'} = \frac{1}{3}\delta_{kl} \delta_{k'l'}, \mathring{\mathcal{T}}_{kl,k'l'} =\frac{1}{2}(\delta_{kk'}\delta_{ll'} + \delta_{kl'} \delta_{lk'} ) -\frac{1}{3}\delta_{kl} \delta_{k'l'} $。从而$\dot{\mathcal{T}} A = \dot{A},\mathring{\mathcal{T}} A = \mathring{A}$。$M$的正定性参看附录1。needapp
	于是我们可以得到系统的演化方程为\cite{}.
	\begin{subequations}\label{eq:CDFMaxwell}
		\begin{align}
			\alpha_0 [\partial_t q +  \nabla \cdot (q \otimes v)] - \nabla \theta^{-1} = -\frac{q}{\theta^2 \lambda}, \\
			\alpha_1[\partial_t (\theta^{-1} \dot{\tau}) + \nabla \cdot (\theta^{-1} \dot{\tau} \otimes v)] + \dot{D} = -\frac{\dot{\tau}}{\kappa}, \\
			\alpha_2[\partial_t (\theta^{-1} \mathring{\tau}) + \nabla \cdot (\theta^{-1} \mathring{\tau} \otimes v)] + \mathring{D} = -\frac{\dot{\tau}}{\xi}. 
		\end{align}
	\end{subequations}
	当$\alpha_0, \alpha_1, \alpha_1$趋于0时,可以得到
	\begin{equation*}
		q = -\lambda \nabla \theta, \quad \tau = - \xi \mathring{D} - \kappa \dot{D} I.
	\end{equation*}
	分别为经典的Fourier-Newton-Stokes本构关系\cite{}。

	在不可压缩和忽略温度的情形,$\nabla \cdot v = 0, \dot{\tau}=0$,$\tau = \mathring{\tau}$,则$\sigma = -\tau$的演化方程为
	\begin{equation*}
		\partial_t \sigma + v \cdot \nabla \sigma - D = -\frac{\sigma}{\xi}. 
	\end{equation*}
	此即Maxwell粘弹性流体力学模型\eqref{eq:maxwell}.	

	考虑温度的不可压Maxwell模型为
	\begin{eqnarray} \label{eq:Tmaxwell}
		v_t + v \cdot \nabla v + \nabla \cdot (-\theta \frac{c}{\alpha_2}) = 0, \\
		c_t + v \cdot \nabla c - D = -\frac{1}{\xi} c.
	\end{eqnarray}
	而由微观方法得到的高分子稀溶液的模型为 \cite{}
	\begin{eqnarray*}
		v_t + v \cdot \nabla v = \eta_p (H A - KT I), \\
		A_t + v \cdot \nabla A - \nabla v A - A(\nabla v)^T  = -\frac{4 H}{\zeta} A+ \frac{4kT}{\zeta} I.
	\end{eqnarray*}
	其中$H = \frac{3kT}{Nb^2}$.其中$k$为Boltzmann常数,$N$为高分子链弹簧模型的弹簧个数,$b$为弹簧长度。取$\beta = \frac{H}{kT} = \frac{3}{Nb^2}$,则应力张量$\sigma = k T(\beta A-I)$。令$c'=\frac{1}{2}(\beta A-I)$,可以得到$c'$的方程如下。
	\begin{equation*}
		c'_t + v \cdot \nabla c'- \nabla v c' - c'(\nabla v)^T -  D  = -\frac{2 kT \beta}{\zeta} c'.
	\end{equation*}
	其对应的线性模型为
	\begin{eqnarray*}
		v_t + v \cdot \nabla v = \frac{1}{2} \eta_p kT c', \\
		c'_t + v \cdot \nabla c' -  D  = -\frac{2 kT \beta}{\zeta} c'.
	\end{eqnarray*}
	在\eqref{eq:Tmaxwell}中令$\alpha_2 = \frac{2}{\eta_p k}, T = \theta,\xi = \frac{\zeta}{2 kT \beta}$即可以得到本模型。可见守恒-耗散理论可以很好地用来对线性粘弹性模型进行建模。

	对于一般的情况,我们亦可选取$M$为
	\begin{equation*}
		M_{ikl,i'k'l'} = \left( \begin{array}{ccc} 
			\mu_1 \delta_{ii' } & \frac{1}{3}\beta'_i \delta_{k'l'} +  [\frac{1}{2} (\beta''_{k'} \delta_{il'} + \beta''_{l'} \delta_{ik'}) - \frac{1}{3} \beta_{i} \delta_{k'l'}]\\
			\frac{1}{3}\beta'_{i'} \delta_{kl} +  [\frac{1}{2} (\beta''_{k} \delta_{i'l} + \beta''_{l'} \delta_{i'k}) - \frac{1}{3} \beta_{i'} \delta_{kl}] &  \frac{1}{\kappa} \dot{\mathcal{T}} + \frac{1}{\xi} \mathring{\mathcal{T}}  
		\end{array} \right)
	\end{equation*} 
	注意参数$\mu_1,\beta',\beta'',\kappa,\zeta$的选取需要保证$M$的正定型。
	从而得到一般的线性本构关系为
	\begin{subequations}
		\begin{align*}
			\alpha_0 [\partial_t q +  \nabla \cdot (q \otimes v)] - \nabla \theta^{-1} = - \mu_1 q -\beta' \theta^{-1} \dot{\tau} - \beta'' \theta^{-1} \mathring{\tau} , \\
			\alpha_1[\partial_t (\theta^{-1} \dot{\tau}) + \nabla \cdot (\theta^{-1} \dot{\tau} \otimes v)] + \dot{D} = -\frac{\dot{\tau}}{\kappa} - \frac{1}{3}(\beta' \cdot q) I, \\
			\alpha_2[\partial_t (\theta^{-1} \mathring{\tau}) + \nabla \cdot (\theta^{-1} \mathring{\tau} \otimes v)] + \mathring{D} = -\frac{\dot{\tau}}{\xi} - \frac{1}{2} (\beta''  \otimes q + q \otimes \beta'') + \frac{1}{3}(\beta'' \cdot q) I. 
		\end{align*}
	\end{subequations}
	这与第一章EIT得到的本构关系\eqref{eq:EITconstitutive}类似。这里我们采用耗散变量$w,c$替代了EIT中的$\nabla T, \nabla \dot{\tau}$和$\nabla \cdot{\mathring{\tau}}$。

	\section{粘弹性流体力学热传导模型的守恒-耗散理论}
	粘弹性流体中的热传导模型目前研究较少,目前大部分模型采用的仍然是Fouerier热传导定律或者Cattaneo定律。本小节我们将讨论推广的Cattaneo定律和热质理论,以及Guyer-Krumhans模型,首先我们采用守恒-耗散理论给出这些模型的理论推导,然后结合上一节粘弹性流体的建模,给出这些这些热传导模型在粘弹性流体力学模型中的应用。

首先我们仅考虑热传导而忽略流体的运动和压缩性(假设$\rho = 1$),那么体系的方程可以写为
\begin{equation*}
	 u_t  + \nabla \cdot q = 0 .	
\end{equation*}
那么$w$的方程为
\begin{equation*}
	 w_t + \nabla \theta^{-1} = M' q. 
\end{equation*}
其中$q = S_w$。

\subsection{Cattaneo定律}
上一节熵的选取为$s = s( u)  -\frac{1}{2 \alpha_0} w^2, M' = \frac{1}{\theta^2 \lambda} $。得到的热传导定律为
\begin{equation*}
	\alpha_0 \partial_t q  - \nabla \theta^{-1} = -\frac{q}{\theta^2 \lambda}.
\end{equation*}
若选取$\alpha_0 = \frac{\chi}{\lambda\theta^2}$,那么可以得到经典的Cattaneo热传导定律
\begin{equation*}
	 \chi  \partial_t q = - (q +\lambda \nabla \theta).
\end{equation*}
此时假设$u = c_v \theta, s_0 = c_v \ln \theta$,那么可以得到$s = s (\theta,w)$的Hessian矩阵为
\begin{equation*}
	\left( \begin{array}{cc} 
		-\frac{c_v}{\theta^2} - \frac{\lambda}{\chi} w^2 & \frac{2 \theta \lambda w }{\chi} \\
		\frac{2 \theta \lambda w }{\chi}  & -\frac{\lambda \theta^2}{\chi }
	\end{array} \right).
\end{equation*}
 这一矩阵在$c_v >0$时对一切$\theta, w$均为负定的。即熵函数为上凸的。

\subsection{推广的Cattaneo定律}
 然而注意到$\alpha_0$假设是常数,所以这里我们需要$\frac{\chi}{\lambda \theta^2}$为常数。
如果上面的熵函数中$\alpha_0$依赖于内能(温度)$a_0=\alpha(u)$,那么$q$的方程应写为
\begin{equation*}
({\alpha(u)}{q})_t-\nabla \theta^{-1}= M'{q}
\end{equation*}
我们称这个方程为推广的Cattaneo定律。其中$M'>0$。

如果我们采用平衡态温度$T$来替代$\theta$,由$T$定义
\begin{equation*}
	T^{-1}=\frac{\partial S_{eq}(u)}{\partial u},
\end{equation*}
可以得到
\begin{equation}
\theta^{-1}=T^{-1}-\frac{1}{2}(\frac{1}{\alpha(u)})'{c}^2
\end{equation}
这里$()'$为对$u$的导数。从而
\begin{equation*}
{w}_t+\nabla(T^{-1}-\frac{1}{2}(\frac{1}{\alpha(u)})' {w}^2)=M'{q}
\end{equation*}
或
\begin{equation*}
({\alpha(u)}{{q}})_t-\nabla(T^{-1}- \frac{\alpha'(u){q}^2}{2})-M{q}=0
\end{equation*}
令$M=\frac{1}{\lambda T^2}, \alpha(u)=\frac{\rho c_v}{2\gamma u^3}=\frac{\rho}{2\gamma c_v^2 T^3}$,其中$\rho$为流体的密度,$c_v$为比热。
最终我们得到
\begin{equation}
\tau_{TM} {q}_t-3c_v {L}T_t+3(\nabla {q})^T \cdot {L}+\lambda(1-6M_H^2)\nabla T+{q}=0
\end{equation}
其中
\begin{equation*}
\tau_{TM}=\frac{\lambda \rho}{2 \gamma c_v^2 T} \quad {L}=\frac{\lambda \rho}{2 \gamma c_v^3 T^2}{q} \quad M_H^2=\frac{\lambda \rho {q}^2}{2\gamma c_v^3 T^3}
\end{equation*}

由过增元等人发展的热质理论的模型为\cite{}
\begin{equation}
\tau_{TM} {q}_t-c_v {L}T_t+\nabla {q} \cdot {L}+\lambda(1-M_H^2)\nabla T+{q}=0.
\end{equation}
这里我们导出的模型和热质理论的模型仅系数由略微差异。

下面我们来考察$s=s(u,w)$的上凸性。$s$的Hessian矩阵为
\begin{eqnarray*}
\left( \begin{array}{ll} -\frac{1}{c_v T^2}-\frac{1}{2}(\frac{1}{\alpha(u)})'' {c}^2 & (\frac{1}{\alpha})'{c} \\ (\frac{1}{\alpha})'{c} & -\frac{1}{\alpha(u)} \end{array} \right).
\end{eqnarray*}
为保证这一矩阵的负定性,我们要求对角元是负的且矩阵为矩阵的行列式为负的。从而可以得到$c_v >0$与
\begin{eqnarray*}
{q}=\alpha(u){c} \le \frac{\rho c_v T}{\sqrt{6}} \sqrt{\frac{\lambda}{\rho c_v \tau_{TM}}}
\end{eqnarray*}
其中 $\sqrt{\frac{\lambda}{\rho c_v \tau_{TM}}}$ 为热波的最大速度\cite{Jou1996extended}。 所以熵函数仅在热流小于特定值时为负定的。这一特定值对应于热波允许的最大速度。

从这一例子可以看出,简单的熵函数的形式就可以导出复杂的热传导本构关系。推广的Cattaneo定律在特定的参数选择下可以得到与热质理论方程类似的模型。

\subsection{Guyer-Krumhansl模型}
物质中的热传导可以采用线性化的声子Boltzmann方程来导出。1966年Guyer和Krumhansl发展了固体结构热传导的Guyer-Krumhansl模型\cite{guyer1966solution}。这一模型考虑了固体结构对热传导的影响。而由于粘弹性流体中高分子也具有一定的结构,这一模型也可以用来描述液晶等粘弹性流体的热传导\cite{}。

由于晶格等结构的存在,在固体中热传导可能出现各向异性。同样的在粘弹性流体中也可能与高分子取向等结构引起的热传导各向异性。为了描述热传导的各项异性我们采用张量$Q$来描述。我们假设张量${Q}$对称。其可以分解为${Q}=\mathring{{Q}}+\dot{Q}{I}$。

假设熵具有形式
\begin{equation*}
s(u,{w},\mathring{{Q}},\dot{Q})=s_{eq}(u)-\frac{1}{2 \alpha_0}{w}^2-\frac{1}{2\tau_1} {\mathring{{Q}}}:{\mathring{{Q}}}-\frac{1}{2\tau_2}\dot{Q}^2
\end{equation*}
由计算可以得到(利用Gibbs关系)
\begin{eqnarray*}
s_t &=& \theta^{-1} u_t +s_{w} \cdot {w}_t + s_{\mathring{{Q}}}:\mathring{{Q}}_t+s_{\dot{Q}} \dot{Q}_t \\
    &=& -\nabla \cdot (\theta^{-1} {q}+ (s_{\mathring{{Q}}}+s_{\dot{Q}}) \cdot {q})\\
    	&&+({w}_t+\nabla \cdot (s_{\mathring{{Q}}}+s_{\dot{Q}})+\nabla \theta^{-1}) \cdot {q} \\
&& +s_{\mathring{{Q}}}:(\mathring{{Q}}_t+(\mathring{\nabla {q}})^{sym})+s_{\dot{Q}}(\dot{Q}_t+\frac{1}{3}\nabla \cdot {q}).
\end{eqnarray*}
为了保证上的产生率大于0(守恒-耗散理论的第二条假设)的成立,我们就假设存在矩阵$M''$,使得
\begin{equation*}
\left( \begin{array}{ll} {w}_t+\nabla \cdot (s_{\mathring{{Q}}}+s_Q)+\nabla \theta^{-1} \\ \mathring{{Q}}_t+(\mathring{\nabla {q}})^{sym} \\ \dot{Q}_t+\frac{1}{3} \nabla \cdot {q} \end{array} \right) = M \left( \begin{array}{l} {q} \\ s_{\mathring{{Q}}} \\s_{\dot{Q}} \end{array} \right).
\end{equation*}
我们考虑最简单的情况,取
\begin{equation*}
	M=\left( \begin{array}{lll} M_0 & 0 & 0 \\0 & M_1 & 0 \\0 & 0 & M_2 \end{array} \right)
\end{equation*}
其中$M_0,M_1,M_2$均为正定的。

我们有
\begin{eqnarray*}
\mathring{{Q}}_t+(\mathring{\nabla {q}})^{sym}=-\frac{M_1}{\tau_1}\mathring{{Q}}, \\
\dot{Q}_t+\frac{1}{3} \nabla \cdot {q}=-\frac{M_2}{\tau_2}\dot{Q}.
\end{eqnarray*}
当 $\tau_1 \to 0, \tau_2 \to 0$时,上面的方程近似为
\begin{eqnarray*} 
\mathring{{Q}}=-\frac{\tau_1}{M_1}(\mathring{\nabla {q}})^{sym} ,\quad  \quad s_{\mathring{Q}}=\frac{1}{M_1}(\mathring{\nabla {q}})^{sym} \\
\dot{Q}=-\frac{\tau_2}{3M_2}\nabla \cdot {q} \quad , \quad s_{\dot{Q}}=\frac{1}{3M_2} \nabla \cdot {q}
\end{eqnarray*}
取$M$为
\begin{equation}
M_0=\frac{1}{\lambda \theta^2}, M_1=\frac{\lambda \theta^2}{d\tau}, M_2=\frac{2\lambda \theta^2}{5d\tau}
\end{equation}
与$a_0=\frac{\tau_0}{\lambda \theta^2}$,
我们可以得到Guyer-Krumhansl模型(以$T$代替$\theta$)\cite{Jou1996extended}
\begin{equation}
{q}_t+\frac{{q}}{\tau_0}+\frac{\lambda}{\tau_0}\nabla T=\frac{1}{2}d(\nabla^2 {q}+2\nabla \nabla \cdot {q})
\end{equation}
注意这里我们假设了$a_0,\tau_1,\tau_2$均为常数。

我们可以得到了推广的Guyer-Krumhansl模型。
\begin{subequations}
	\begin{align}
(\alpha_0 {q})_t - \nabla \cdot ({\frac{1}{\tau_1} \mathring{{Q}}}+\frac{1}{\tau_2}\dot{Q})+\nabla \theta^{-1} = M_0 {q} \\
\mathring{{Q}}_t+(\mathring{\nabla {q}})^{sym}=-\frac{M_1}{\tau_1}\mathring{{Q}} \\
\dot{Q}_t+\frac{1}{3} \nabla \cdot {q}=-\frac{M_2}{\tau_2} \dot{Q}
	\end{align}
\end{subequations}

在$a_0,\tau_1,\tau_2$均为常数时,熵函数的Hessian矩阵为
\begin{eqnarray*}
\left(\begin{array}{llll}  -\frac{1}{c_v \theta^2} & 0 & 0 & 0 \\
                                           0  & -\frac{1}{\alpha_0} & 0 & 0 \\
										   0 & 0 & -\frac{1}{2\tau_1} & 0 \\
										   0 & 0 & 0 & -\frac{1}{2\tau_2}
										   \end{array} \right).
\end{eqnarray*}
熵的上凸性的等价于$c_v$和参数$\alpha_0,\tau_1,\tau_2$为正的。

利用推广的Guyer-Krumhansl模型可以得到热质理论和Guyer-Krumhansl模型的组合模型。即令$M_0=\frac{1}{\lambda T^2}, \alpha_0 = \alpha_0 (u) =\frac{\rho c_v}{2\gamma u^3}=\frac{\rho}{2\gamma c_v^2 T^3}$,于是可以得到
\begin{eqnarray*}
\tau_{TM} {q}_t-c_v {L}T_t+\nabla {q} \cdot {L}+\lambda(1-M_H^2)\nabla T+{q} - \lambda T^2 \nabla \cdot ({\frac{1}{\tau_1} \mathring{{Q}}}+\frac{1}{\tau_2}\dot{Q})=0, \\
\mathring{{Q}}_t+(\mathring{\nabla {q}})^{sym}=-\frac{M_1}{\tau_1}\mathring{{Q}}, \\
\dot{Q}_t+\frac{1}{3} \nabla \cdot {q}=-\frac{M_2}{\tau_2} \dot{Q}.
\end{eqnarray*}
令$\tau_1,\tau_2$趋于0,并取$M_1 = \frac{\lambda T^2}{d}, M_2 = \frac{2\lambda T^2}{5 d }$,可以得到下面的Guyer-Kramhansl-Thermomass模型。
\begin{eqnarray*}
	\tau_{TM} {q}_t-c_v {L}T_t+\nabla {q} \cdot {L}+\lambda(1-M_H^2)\nabla T+{q} = \frac{1}{2}d (\nabla^2 q +2 \nabla \nabla \cdot q). \\
\end{eqnarray*}


\subsection{考虑热传导的线性粘弹性流体模型}
上面对热传导模型的分析可以用于粘弹性流体力学的建模之中。我们仍然选取守恒变量$U_c$为密度、动量和能量。耗散变量我们选取$w,Q$和$c$。假设熵函数
\begin{equation*}
	\eta = \eta (\rho,\rho v,\rho e,\rho w,\rho Q,\rho c) = \rho s(\rho,v,e,w,Q,c).
\end{equation*}
是其变量的上凸函数。
根据Gibbs关系我们有
\begin{equation*}
			ds = \theta^{-1} pd\nu + \theta^{-1} du + s_w \cdot dw + s_Q : d Q+ s_c :dc.
\end{equation*}
仍然将张量$Q,c$进行分解$Q = \mathring{Q} +\dot{Q}I$。
从而可以得到熵的产生率
\begin{eqnarray*}
		\eta_t + \nabla \cdot (\eta v) &=& \rho (s_t + v \cdot \nabla s), \\
		&=& -\nabla \cdot (\theta^{-1} q + (s_{\mathring{{Q}}}+s_{\dot{Q}}) \cdot {q} )  \\
		&&+ q \cdot [\rho (w_t + v \cdot \nabla w) + \nabla \cdot (s_{\mathring{{Q}}}+s_{\dot{Q}})] \\
		&&+ (s_c:[\rho (c_t + v \cdot \nabla v)] - \theta^{-1} \tau : D) \\
		&&+s_{\mathring{{Q}}}:(\rho (\mathring{{Q}}_t + v \cdot \nabla \mathring{Q})+(\mathring{\nabla {q}})^{sym}) \\
		&&+s_{\dot{Q}}(\rho (\dot{Q}_t + v \cdot \nabla \dot{Q})+\frac{1}{3}\nabla \cdot {q}) \\
		&& = -\nabla \cdot J + \Delta. 
\end{eqnarray*}
这里我们用到了$q=s_w,\tau = \theta s_c$。为了满足守恒-耗散理论的第二条假设,我们假设存在正定对称矩阵$M$,使得下面的本构关系成立。
\begin{equation*}
   	\left( \begin{array}{c} 
			(\rho w)_t +  \nabla \cdot (\rho w \otimes v)  + \nabla \theta^{-1} + \nabla \cdot (s_{\mathring{{Q}}}+s_{\dot{Q}})\\
			(\rho c)_t +  \nabla \cdot (\rho c \otimes v)  - D \\
			(\rho \mathring{{Q}})_t + \nabla \cdot (\rho \mathring{Q} \otimes v)+(\mathring{\nabla {q}})^{sym} \\ (\rho \dot{Q})_t + \nabla \cdot (\rho \dot{Q} v)+\frac{1}{3} \nabla \cdot {q}
		\end{array} \right) = M \cdot
		\left( \begin{array}{c} 
			q \\ \theta^{-1} \tau \\s_{\mathring{{Q}}} \\s_{\dot{Q}}
		\end{array}\right).
\end{equation*} 
假设熵函数是二次的且具有下面的形式。
	\begin{equation*}
		s = s_0(\nu,u)  - \frac{1}{2\nu \alpha_0} w^2 - \frac{1}{2\nu \alpha_1} \dot{c}^2- \frac{1}{2\alpha_2 \nu} \mathring{c}:\mathring{c} - \frac{1}{2 \tau_1 \nu} \mathring{Q}: \mathring{Q} - \frac{1}{2  \tau_2 \nu} \dot{Q}^2.
 	\end{equation*}  
我们得到了可压缩线性粘弹性流体包含温度的一般方程。
\begin{equation}\label{eq:CNSTgeneral}
   	\left( \begin{array}{c} 
			(\alpha_0 q)_t +  \nabla \cdot (\alpha_0  q \otimes v)  + \nabla \theta^{-1} + \nabla \cdot (\frac{1}{\tau_1} \rho \mathring{{Q}}+\frac{1}{\tau_2} \rho \dot{Q})\\
			(\alpha_1 \theta^{-1}\tau)_t +  \nabla \cdot (\alpha_1 \theta^{-1} \tau \otimes v)  - D \\
			(\rho \mathring{{Q}})_t + \nabla \cdot (\rho \mathring{Q} \otimes v)+(\mathring{\nabla {q}})^{sym} \\ (\rho \dot{Q})_t + \nabla \cdot (\rho \dot{Q} v)+\frac{1}{3} \nabla \cdot {q}
		\end{array} \right) = M \cdot
		\left( \begin{array}{c} 
			q \\ \theta^{-1} \tau \\ -\frac{1}{\tau_1} \rho \mathring{{Q}} \\ -\frac{1}{\tau_2} \rho \dot{{Q}}
		\end{array}\right).
\end{equation}
可以取耗散矩阵$M$为对角的,即
\begin{equation*}
	M = \left(\begin{array}{cccc}
	M_0 & & & \\
	& M_1 & & \\
	& & M_2 & \\
	& & & M_3 \end{array} \right).
\end{equation*}
通过选取合适的参数,我们可以得到Navier-Stokes-GGK模型(这里GGK代表Generalized Guyer-Krumhansl模型)。

从上面的建模过程可以看出,守恒-耗散理论十分简洁。其应用的步骤可以总结如下:
\begin{enumerate}
	\item 选取合适的守恒变量$U_c$和耗散变量$U_d$。
	\item 假设熵函数$\eta = \eta(U_c,U_d)$为守恒变量和耗散变量的函数。假设熵函数的形式并保证其为下凸的(可以包含参数)。利用Gibbs关系将熵的产生率写成$\eta_t+ \nabla \cdot (\eta v) + \nabla \cdot J = \Delta$的形式。
	\item 选取合适的耗散矩阵$M$使得方程满足守恒-耗散理论的第二条假设。
	\item 选取合适的参数已得到本构关系。
\end{enumerate}
可以看出整个过程的关键在于选取合适的熵函数和耗散矩阵。而守恒-耗散理论明确给出了这两者选取的准则,即分别满足守恒-耗散理论的第一条和第二条假设。

与EIT相比,守恒-耗散理论采用的耗散变量具有更大的自由度。EIT经常采用温度梯度、热流等量作为耗散变量,而守恒-耗散理论同GENERIC等理论一样,并不事先假设耗散变量的具体形式。类似GENERIC的观点,我们认为系统的不可逆性是由其内部结构决定的,而描述这一耗散结构经常需要微观的建模。采用$w,c$而不是$q,\tau$可以为建模带来更大的自由度,有时甚至可以与微观模型的统计量产生联系。另外EIT采用的熵函数大部分为二次的。虽然我们上面讨论的模型中熵函数大部分也为二次的,但是守恒-耗散理论仅仅假设熵函数的上凸性。由于这两点,我们认为守恒-耗散理论可以为原理平衡态的体系提供建模的框架。另外守恒-耗散理论对耗散矩阵仅仅要求是正定的。虽然大部分情况下我们假定其也为对称的(如果类似Onsager的倒易关系成立)。这样我们避免了GENERIC等理论对于耗散矩阵的复杂要求。

另外守恒-耗散理论还具有很好的数学基础。下一节将讨论守恒-耗散理论的部分数学结果。

	\section{守恒-耗散理论的数学结果}
	守恒-耗散理论一方面考虑了热力学第一定律和第二定律和Onsager倒易关系等物理原理,另一方面它建立在严格的数学理论之上。熵函数的存在性和方程的守恒形式共同保证方程可以被对称化,从而根据对称双曲组的相关理论,方程的局部适定性可以保证。而对右端项的要求可以使得方程满足文献\cite{}中的条件,从而可以保证有端项含有松弛参数$\epsilon$时松弛极限$\epsilon$很小时可以与形式上得到的方程相近似,从而保证了离平衡态“很近”的体系可以很好地采用平衡态体系近似。

	守恒-耗散理论得到的方程均为可对称双曲方程组,所以可以采用对称双曲组的相关理论对方程的解进行分析。例如在平衡态附近方程解的全局适定性可以采用Kawashima理论进行分析\cite{}。可压缩模型的小马赫数极限可以采用类似可压Euler方程的小马赫数极限进行分析\cite{}。方程的弱解的存在性和间断解可以采用相关的理论进行分析\cite{}。

	本节主要考虑方程的强解(经典解),我们采用的解空间为Sobolev空间$H^s, s > \frac{n}{2}+1$,其中$n$为空间的维数。

	\subsection{方程解的局部存在性}
	由守恒-耗散理论的第一条假设,存在正定对称矩阵$A_0(U)  = -\eta_{UU}$使得$A_0(U) A_j(U)$对称。$A_0$的正定性可由$\eta$的上凸性得到。假设状态变量$U$定义域在$R^{n}$的开集$G$中,$G_1$紧包含在$G$中$\bar{G_1} \subset\subset G$。利用对称双曲方程组的存在性理论,可以得到下面的定理\cite{}。
	\begin{theorem}
		假设$U_0 =U_0(x) \in H^{s}$,其中$s>\frac{n}{2} +1$。那么存在时间区间$[0,T]$,其中$T>0$,使得方程\eqref{eq:CDF}对于初值问题$U=U_0(x)$,存在唯一解$U = U(x,t) \in G_2, \bar{G_2} \subset \subset G$,且
		\begin{equation*}
			u \in C([0,T],H^s) \cap C^1([0,T],H^{s-1}).
		\end{equation*}
		其中$T$依赖于$\|u_0\|_{H^s}$和$G_1$。
	\end{theorem}
	于是对于线性弹性流体力学模型\eqref{eq:CNSTgeneral},可以应用上面的定理,其中$U = (\rho,\rho v, \rho c,\rho w, \rho Q)$。这样我们就可以得到由\eqref{eq:fluid}与\eqref{eq:CNSTgeneral}构成的方程组的经典解局部存在且唯一。

	近些年来,Iftimie、Chae和徐江等人证明了Besov空间下(临界指数下)的双曲对称方程组解的存在唯一性定理\cite{}。文献\cite{}证明的定理如下。
	\begin{theorem}
		假设$U_0 =U_0(x) \in B_{2,1}^{s}(R^n)$,其中$s=\frac{n}{2} +1$。那么存在时间区间$[0,T]$,其中$T>0$,使得方程\eqref{eq:CDF}对于初值问题$U=U_0(x)$,存在唯一解$U = U(x,t) \in C^1([0,T],R^n)$满足
		\begin{equation*}
			u \in \tilde{C}_{T}([0,T],B_{2,1}^s(R^n)) \cap \tilde{C}^1_T(B_{2,1}^{s-1}).
		\end{equation*}
		其中$T$依赖于$\|u_0\|_{B^{s}_{2,1}}$。这里$\tilde{C}_T$为Chemin-Lerner空间,定义为
		\begin{eqnarray*}
			\tilde{C}_T(B_{2,1}^s):= \tilde{L}_T^\infty(B_{2,1}^s) \cap C([0,T],B_{2,1}^s), \\
			\tilde{L}_T^\infty(B_{2,1}^s) :=\{ u \in L^\infty: \|u\|_{\tilde{L}_T^\infty (B_{2,1}^s)} < \infty\}.
		\end{eqnarray*}
	\end{theorem}
	
	\subsection{平衡态附近解的整体存在性}
	我们定义$\tilde{Q}(U) = 0$对应的$U$称为平衡流形。我们考虑在常数平衡态$U=U_e$附近方程解的存在性。  
	守恒-耗散理论得到的方程\eqref{eq:CDF}带有部分耗散源项。对于这样的系统的平衡态附近整体存在性的研究可以参看\cite{}。Shuichi Kawashima和Yasushi Shizuta最早对带有部分扩散源项的系统进行了讨论并提出了Kawashima条件。Wen-An Yong和Shuichi Kawashima讨论了带有部分耗散源项的系统\cite{}。关于带有部分耗散原项的双曲守恒律方程组的研究可以参看文献\cite{}。

	这里我们考察方程在$\mathcal{q}(U)=0$对应状态$U=U_e$的行为。这样一来,熵增
	\begin{equation*}
		\Delta = \eta_{U_d}^T M \eta_{U_d} = 0
	\end{equation*}
	对应的$U$满足$\eta_{U_d}(U) = 0$。
	注意到守恒耗散理论的第一条假设和第二条假设以及$M$的零空间不依赖于$U$正好构成了文献\cite{}中对方程\eqref{eq:CDF}的三条假设。而由于$M$的正定性,其零空间为0,不依赖于$U$。从而根据\cite{}中的定理2.4,我们有
	\begin{theorem}
		假设方程\eqref{eq:CDF}满足守恒-耗散定理的两条基本假设,耗散矩阵$M$对称且零空间不依赖于$U$。那么下面的结论成立(文献\cite{}定理2.4)。
		\begin{enumerate}
			\item $$-\eta_{U_d}^T (U) \mathcal{q}(U) \le -\lambda^{-1}(U) |\mathcal{q}(U)|^2$$对任意的$U$成立。这里$\lambda(U)$为$M$的最大特征值。
			\item $\mathcal{Q} (U) = 0$等价于$\eta_{U_d} (U) = 0$。
			\item 若$U$满足$\mathcal{Q}(U)=0$,那么对应的$\mathcal{q}_{U_d}(U) \eta_{U_d U_d}^{-1}(U)$对称且正定。
			\item 若$U$满足$\mathcal{Q}(U)=0$,那么对应的$\mathcal{q}_{U_d}(U)$是可逆的。
		\end{enumerate}
	\end{theorem}
	下面我们证明这些结果。
	\begin{proof}
		文献\cite{}考虑的第一个条件即为守恒-耗散理论的第一条假设(只不过\cite{}假设熵函数为下凸的,即对应于我们的$-\eta$)。第二个条件为存在对称半正定矩阵$\mathcal{L} = \mathcal{L}(U)$使得$\mathcal{Q}(U) = \mathcal{L}(U) \eta_{U} (U)$。第三个条件为$\mathcal{L}(U)$的零空间不依赖于$U$。在守恒-耗散理论中我们假设的$\mathcal{L}$为
		\begin{equation*}
			\mathcal{L} (U)= \left( \begin{array}{cc}
			0 & 0 \\ 0 & M(U)
			\end{array} \right).
		\end{equation*}
		其中$M$的假设为负定的。即守恒-耗散理论的假设要强于文献\cite{}的假设。而由于$M$的正定性$\mathcal{L}$的零空间即$(U_c,U_d = 0)$(对应$\mathcal{L}(U) = 0$,注意这里为简单采用了$U$表示其零空间的向量,并不代表方程的解)。于是文献\cite{}中定理2.4的(1)、(2)、(4)、(5)分别对应这里的四个结论。下面我们给出这四个结论的直接证明。
		\begin{enumerate}
			\item 由于$M$的最大特征值为$\lambda$,下面的式子成立。
			\begin{equation*}
				\eta_{U_d}^T M \eta_{U_d} = (M \eta_{U_d})^T M^{-1} M \eta_{U_d}  \ge \lambda^{-1} | M\eta_{U_d}|^2. 
			\end{equation*}
			从而结论1成立。
			\item 由$\mathcal{Q}(U) = 0$,得到$ \mathcal{q}(U) = M\eta_{U_d} = 0$。由$M$的正定性可以得到$\eta_{U_d}(U) = 0$。反之亦然。
			\item $U$满足$\mathcal{Q}(U) = 0$,从而由第二条结论$\eta_{U_d}{Q}(U) = 0$。于是
			\begin{equation*}
				\mathcal{q}_{U_d} (U) = (M(U) \eta_{U_d}(U))_{U_d} = M_{U_{d}}(U)\eta_{U_d}(U) +M(U) \eta_{U_d U_d} = M(U) \eta_{U_d U_d}(U).  
			\end{equation*}
			所以$\mathcal{q}_{u_d}(U) \eta_{U_d U_d}^{-1} = M(U)$是对称正定的。
			\item 因$\mathcal{q}_{U_d} (U) = M(U) \eta_{U_d U_d}(U)$,其中$M$正定,$\eta_{U_d U_d}$负定,均非奇异,所以$\mathcal{q}_{U_d} (U)$可逆。
			\end{enumerate}
	\end{proof}

	由本定理的结论1,2,4可以得到文献\cite{}中的假设(i),(ii),(iii)成立。即$\mathcal{q}_{u_d}(U)$可逆,$\eta_{U U} F_{jU}$对称,且存在$c_G = \lambda^{-1}$,使得
	\begin{equation*}
		[\eta_U(U) - \eta_U (U_e)] \mathcal{Q}(U) \le -c_G |\mathcal{Q}(U)|^2.
	\end{equation*}
	由文献\cite{},可知如果方程\eqref{eq:CDF}还满足Kawashima条件
	\begin{itemize}
		\item 矩阵$\sum_j \omega_j F_{jU}(U_e), \omega = (\omega_1,\omega_2, \cdots,\omega_n) \in S^{n-1}$ ($\mathbf{R}^n$中的单位球)的特征向量不在$\mathcal{Q}$的Jacobi矩阵$\mathcal{Q}_{U}(U_e)$的零空间中。
	\end{itemize}
	根据文献\cite{},我们可以得到下面的平衡态$U=U_e$附近方程\eqref{eq:CDF}的整体存在性定理。
	\begin{theorem} \label{th:Kawashima}
		设整数$s \ge [\frac{n}{2}]+2$,且方程\eqref{eq:CDF}满足守恒-耗散理论的两个假设,其中耗散矩阵$M$为正定对称的。且满足Kawashima条件。那么存在常数$C_1,C_2$,若$U_0 = U_0(x) \in H^s(\mathbf{R}^n)$满足
		\begin{equation*}
			\|U_0 -U_e\|_{H^s} \le C_1.
		\end{equation*}
		则方程\eqref{eq:CDF}以$U_0$为初值的Cauchy问题存在唯一的整体解$U=U(x,t) \in C([0,\infty),H^s(\mathbf{R}^d)]$,且对任意$T>0$满足下面的估计。
		\begin{eqnarray*}
			&& \|U(\cdot,T) - U_e \|_{H^s}^2 + \int_0^T \| \mathcal{Q}(U)(\cdot,t)\|_{H^s}^2 dt + \int_0^T \|\nabla U (\cdot,t)\|_{H^{s-1}}^2 dt \\
			&& \le C_2 \| U_0 -U_e\|_{H^s}^2
		\end{eqnarray*}
	\end{theorem}
	定理的证明见\cite{}。

	下面我们考虑可压线性粘弹性流体模型的Kawashima条件。
    {}
	首先我们讨论温度不变的情况,即$\theta = 1$。选取$\alpha_1 = \alpha_2 =\rho$,
	\begin{equation*}
		M = \left( \begin{array}{cc} \frac{1}{\kappa} & 0 \\ 0 & \frac{1}{\xi} \end{array} \right)
	\end{equation*}
	此时方程\eqref{eq:fluid}、\eqref{eq:CDFMaxwell}可以写为
	\begin{subequations}\label{eq:CDFspecific}
		\begin{align}
			\rho_t + \nabla \cdot (\rho v) = 0, \\
			(\rho v)_t + \nabla \cdot (\rho v \otimes v) + \nabla p - \nabla \cdot ( \dot{c} I +  \mathring{c}) = 0, \\
			(\rho \dot{c})_t  + \nabla \cdot(\rho \dot{c} \otimes v) -  \nabla \cdot v = - \frac{\dot{c}}{\kappa}. \\
			(\rho \mathring{c})_t + \nabla \cdot (\rho c \otimes v) - \frac{1}{2} (\nabla v + (\nabla v)^T) = - \frac{ \mathring{c}}{\xi}.
		\end{align}
	\end{subequations}
	其中$p(\rho) = s_{\nu}(\nu) = -s_{\rho} \rho^2$。可以写成下面的坐标分量的形式。
	\begin{eqnarray*}
		U_t + \sum_j A_j(U) U_{x_j} = \mathcal{Q}(U), \\
		U =\left( \begin{array}{cccc} 
			\rho \\ \rho v_i \\ \rho \dot{c} \\ \rho \mathring{c}_{kl}
		\end{array}\right), \quad 
		\mathcal{Q}(U) = 
		\left( \begin{array}{cccc} 
			0 & & & \\
			& 0 & & \\
			& & -\frac{1}{\rho \kappa} &\\
			& & & -\frac{1}{\rho \xi}
 		\end{array}\right) U, \\
 		\tiny{ A^{j}(U) = \left( \begin{array}{cccc} 
 		v_j & \rho \delta_{i'j} & 0 & 0 \\
 		-v_i v_j + p_{\rho} \delta_{ij} + \frac{\dot{c}}{\rho} \delta_{ij}  + \frac{\mathring{c}_{ij}}{\rho} & v_i \delta_{i'j} + v_j \delta_{ii'} & -\frac{1}{\rho}\delta_{ij} & -\frac{1}{\rho}(\delta_{ik'} \delta_{jl'}+\delta_{jk'}\delta_{il'}) \\
 		- \dot{c} v_j + \frac{v_j}{\rho} & \frac{\dot{c}}{\rho} \delta_{i'j} - \frac{1}{\rho} \delta_{i'j} & v_{j} & 0 \\
 		- \mathring{c}_{kl} v_j + \frac{1}{2}(\frac{v_k}{\rho} \delta_{jl} + \frac{v_l}{\rho} \delta_{jk}) & \frac{\mathring{c}_{kl}}{\rho} \delta_{i'j} - \frac{1}{\rho}(\delta_{i'k} \delta_{jl} + \delta_{jk} \delta_{il}) & 0 & v_j \delta_{kk'} \delta_{ll'}  
 		\end{array}\right) }.
	\end{eqnarray*}
	在平衡态附近$\rho = \bar{\rho},v=0,\dot{c} = 0,\mathring{c}=0$,对应的$A_j(U_e)$为
	\begin{eqnarray*}
		A^{j}(U_e) = \left( \begin{array}{cccc} 
 		0 & \bar{\rho} \delta_{i'j} & 0 & 0 \\
 		p_{\rho}(\bar{\rho}) \delta_{ij} & 0 & -\frac{1}{\bar{\rho}}\delta_{ij} & -\frac{1}{\bar{\rho}}(\delta_{ik'} \delta_{jl'}+\delta_{jk'}\delta_{il'}) \\
 		0 & -\frac{1}{\bar{\rho}} \delta_{i'j} & 0 & 0 \\
 		0 & - \frac{1}{\bar{\rho}}(\delta_{i'k} \delta_{jl} + \delta_{jk} \delta_{il}) & 0 & 0 
 		\end{array}\right).
	\end{eqnarray*}
	假设$W = (W_1,W_2,W_3,W_4)^T$满足$\mathcal{Q}_U(U_e) W = 0$,由
	\begin{equation*}
		\mathcal{Q}_U(U) = \left( \begin{array}{cccc} 
			0 & & & \\
			& 0 & & \\
			\frac{\rho \dot{c}}{\kappa} & 0 & -\frac{1}{\rho \kappa} & 0 \\
			\frac{\rho \mathring{c}}{\kappa} & 0 & 0 & -\frac{1}{\rho \kappa}  
		\end{array}\right)
	\end{equation*}
	得到
	\begin{equation*}
		\left( \begin{array}{cccc} 
		0 & & & \\
		& 0 & & \\
		& & -\frac{1}{\bar{\rho}\kappa} & \\
		& & & \frac{1}{\bar{\rho}\xi} 
		\end{array} \right) 
		\left( \begin{array}{c} W_1 \\ W_2 \\W_3 \\W_4 \end{array} \right) = 0.
	\end{equation*}
	从而$W_3=0,W_4=0$。假设$W$为矩阵$\sum_j \omega_j A_j(U_e)$的特征向量,即存在$\mu$,使得$A_j(U_e) W = \mu W$。
	\begin{eqnarray*}
		&& \sum_j \omega_j \left( \begin{array}{cccc} 
 		0 & \bar{\rho} \delta_{i'j} & 0 & 0 \\
 		p_{\rho}(\bar{\rho}) \delta_{ij} & 0 & -\frac{1}{\bar{\rho}}\delta_{ij} & -\frac{1}{\bar{\rho}}(\delta_{ik'} \delta_{jl'}+\delta_{jk'}\delta_{il'}) \\
 		0 & -\frac{1}{\bar{\rho}} \delta_{i'j} & 0 & 0 \\
 		0 & - \frac{1}{\bar{\rho}}(\delta_{i'k} \delta_{jl} + \delta_{jk} \delta_{il}) & 0 & 0 
 		\end{array}\right) \left( \begin{array}{c} W_1 \\ W_2 \\W_3 \\W_4 \end{array} \right) \\
 		&& = \mu \left( \begin{array}{c} W_1 \\ W_2 \\W_3 \\W_4 \end{array} \right).
	\end{eqnarray*}
	得到
	\begin{eqnarray*}
		\bar{\rho} \omega \cdot W_2 = \mu W_1, \\ \quad [(\bar{\rho} p_{\rho}(\bar{\rho}) W_1 - W_3) I - W_4] \cdot \omega =\mu W_2, \\
		-\frac{1}{\bar{\rho}} W_2 \cdot \omega  = 0, \\
		-\frac{1}{\bar{\rho}} (\omega \otimes W_2 + W_2 \otimes \omega) = 0.
	\end{eqnarray*}
	假设$\mu \neq 0$,将第三个式子代入第一个得到$W_1 = 0$,再将$W_2=0,W_3=0,w_4=0$代入第二个式子得到$W_2 = 0$,从而$W = 0$。若$\mu = 0$,由第四个式子得到
	\begin{equation*} \label{eq:Womega}
		W_{2k} \omega_l + W_{2l} \omega_k = 0.
	\end{equation*}
	由第一或第三个式子得到
	\begin{equation}\label{eq:omega}
		\sum_{j} \omega_j W_{2j} = 0.
	\end{equation}
	在等式\eqref{eq:Womega}两边同乘以$\omega_k$并对$k$求和,利用\eqref{eq:omega}得到
	\begin{equation*}
		\sum_{k} W_{2l} \omega_k^2 = W_{wl} = 0
	\end{equation*}
	对任意的$l$成立。即$W_2=0$。从而$W=0$。这样我们说明了不存在非零向量$W$使得$A^j(U_e) W= \mu W $和$\mathcal{Q}_U(U_e) W=0$同时成立,即Kawashima条件成立。

	于是对方程组\eqref{eq:CDFspecific},定理\ref{th:Kawashima}成立。

	对于$\alpha_1,\alpha_2$为常数的情况,此时$p$不仅依赖于$\rho$,还依赖于$\dot{c},\mathring{c}$。
	\begin{equation*}
		p = s_\nu = s_{0\nu} +\frac{1}{2\nu^2 \alpha_1} \dot{c}^2 + \frac{1}{2\nu^2 \alpha_2} \mathring{c} : \mathring{c}. 
	\end{equation*}
	设$\pi= \pi(\rho) = s_{0\nu}$。取$\alpha_1=\alpha_2=1$。则此时的粘弹性流体模型的方程为   
	\begin{subequations}\label{eq:CDFalphaConst}
		\begin{align}
			\rho_t + \nabla \cdot (\rho v) = 0, \\
			(\rho v)_t + \nabla \cdot (\rho v \otimes v) + \nabla \pi(\rho) + \frac{1}{2} \nabla [ (\rho \dot{c})^2+ (\rho \mathring{c}) : (\rho \mathring{c})]- \nabla \cdot ( \rho \dot{c} I +  \rho \mathring{c}) = 0, \\
			(\rho \dot{c})_t  + \nabla \cdot(\rho \dot{c} \otimes v) -  \nabla \cdot v = - \frac{\rho \dot{c}}{\kappa}. \\
			(\rho \mathring{c})_t + \nabla \cdot (\rho c \otimes v) - \frac{1}{2} (\nabla v + (\nabla v)^T) = - \frac{\rho \mathring{c}}{\xi}.
		\end{align}
	\end{subequations}
	此时由于动量方程中出现的高阶项$\nabla [ (\rho \dot{c})^2+ (\rho \mathring{c}) : (\rho \mathring{c})]$在平衡态附近的线性化为0。所以方程\eqref{eq:CDFalphaConst}的线性化与方程\eqref{eq:CDFspecific}的线性化结果类似。从而亦满足Kawashima条件。于是对于方程组\eqref{eq:CDFalphaConst},定理\ref{th:Kawashima}亦成立。

	对于包含温度的情况,我们取$\alpha_0=\alpha_1=\alpha_2=\rho$,令$\pi(\rho) = s_\nu(\nu)$,得到下面的模型。
	\begin{subequations}\label{eq:CDFTemp}
		\begin{align}
			\rho_t + \nabla \cdot (\rho v) = 0, \\
			(\rho v)_t + \nabla \cdot (\rho v \otimes v) + \nabla (\theta \pi) - \nabla \cdot ( \theta \dot{c} I + \theta \mathring{c}) = 0, \\
			(\rho  e)_t + \nabla \cdot (\rho e v) + \nabla \cdot q + \nabla \cdot (P \cdot v) = 0\\
			(\rho w)_t + \nabla \cdot (\rho w \otimes v) + \nabla \theta^{-1} = - \frac{1}{\lambda \theta^2} w \\
			(\rho \dot{c})_t  + \nabla \cdot(\rho \dot{c} \otimes v) -  \nabla \cdot v = - \frac{\dot{c}}{\kappa}. \\
			(\rho \mathring{c})_t + \nabla \cdot (\rho c \otimes v) - \frac{1}{2} (\nabla v + (\nabla v)^T) = - \frac{ \mathring{c}}{\xi}.
		\end{align}
	\end{subequations}
其中$q= -w,P = (\pi - \theta \dot{c}) I - \theta \mathring{c}$。写成坐标的形式为
	\begin{eqnarray*} \tiny
		U_t + \sum_j A_j(U) U_{x_j} = \mathcal{Q}(U), \\
		U =\left( \begin{array}{cccc} 
			\rho \\ \rho v_i \\ \rho e \\ \rho w_m \\ \rho \dot{c} \\ \rho \mathring{c}_{kl}
		\end{array}\right), \quad 
		\mathcal{Q}(U) = 
		\left( \begin{array}{cccccc} 
			0 & & & & &\\
			& 0 & & & &\\
			& & 0 & & & \\
			& & &  -\frac{1}{\rho \lambda \theta^2} & &  \\
			& & & & -\frac{1}{\rho \kappa} & \\
			& & & & & -\frac{1}{\rho\xi}  \\
 		\end{array}\right) U, \\
	 		 A^{j}(U) = \\
 		 \tiny{ \left( \begin{array}{cccccc} 
 		v_j & \rho \delta_{i'j} & 0 & 0 & 0 & 0  \\
 		-v_i v_j + P_{ij} (-\frac{e}{\rho} + \frac{v^2}{\rho} ) + (\pi_\rho + \frac{\dot{c}}{\rho^2})\theta \delta_{ij} + \frac{\mathring{c}_{ij} \theta}{\rho}  & v_i \delta_{i'j} + v_j \delta_{ii'} - \frac{P_{ij}}{\rho} v_{i'}  & \frac{P_{ij}}{\rho} & 0 & -\frac{\theta}{\rho} \delta_{ij} & -\frac{\theta}{2\rho} (\delta_{ik'}\delta_{jl'} + \delta_{jk'}\delta_{il'})  \\
 		-e v_j - \frac{w_j}{\rho} -\frac{P_{ij} v_i}{\rho} + v_j (\pi_\rho + \frac{\dot{c}}{\rho}) + \frac{\mathring{c}_{ij} v_i}{\rho}   & e \delta_{i'j} + \frac{P_{ij}}{\rho} \delta_{ii'} & v_{j} & \frac{1}{\rho} \delta_{jm}  & -\frac{v_j}{\rho} & -\frac{v_i}{\rho} \delta_{ik'} \delta_{jl'} \\
 		-w_m v_j - \theta^{-2} (-\frac{e}{\rho} + \frac{v^2}{\rho}) \delta_{jm} & w_m \delta_{i'j} + \frac{\theta^{-2} v_{i'}}{\rho} \delta_{jm} & -\frac{\theta^{-2}}{\rho} \delta_{jm} & v_j \delta_{m m'} & 0 & 0 \\
 		-\dot{c} v_j + \frac{v_j}{\rho} & \dot{c} - \frac{1}{\rho} & 0 & 0 & v_j & 0 \\
 		-\mathring{c}_{kl} v_j + \frac{1}{2} (v_k \delta_{jl} + v_l \delta_{jk}) & - \frac{1}{2\rho} (\delta_{jk} \delta_{i'l} + \delta_{jl} \delta_{i'k}) + \mathring{c}_{kl} \delta_{i'j} & 0 & 0 & 0 & v_j 
 		\end{array}\right) }.
	\end{eqnarray*}
	在平衡态$\rho=\bar{\rho},v=0,e=\bar{e} = \bar{\theta},w= 0,\dot{c} = 0, \mathring{c}=0$,$\bar{P} = \bar{\theta} \pi(\bar{\rho}) I$,假设$W=(W_1,W_2,W_3,W_4,W_5,W_6)$在$\mathcal{Q}_U(U_e)$的零空间中。由
	\begin{equation*}
		\mathcal{Q}_U(U) = 
		\left( \begin{array}{cccccc} 
			0 & & & & &\\
			& 0 & & & &\\
			& & 0 & & & \\
			\frac{w}{\rho \lambda \theta^2} + \frac{2w}{\lambda \theta^3 \rho}(-e + v^2) & \frac{2 w \otimes v}{\rho \lambda \theta^3} & \frac{2w}{\rho \lambda \theta^3} &  -\frac{1}{\rho \lambda \theta^2} & &  \\
			\frac{\rho \dot{c}}{\kappa} & & & & -\frac{1}{\rho \kappa} & \\
			\frac{\rho \mathring{c}}{\xi}& & & & & -\frac{1}{\rho \xi} \end{array} \right).
	\end{equation*}
	于是$\mathcal{Q}_U(U_e) W=0$可以推出$W_3 = W_4 = W_5 = 0$。而
	平衡态点$A_j(U_e)$取值如下
	\begin{equation*}
		 \tiny{ \left( \begin{array}{cccccc} 
 		0 & \bar{\rho} \delta_{i'j} & 0 & 0 & 0 & 0  \\
 		\bar{\theta} \pi(\bar{\rho}) (-\frac{\bar{\theta}}{\bar{\rho}} ) \delta_{ij} + \pi_\rho(\bar{\rho})\bar{\theta} \delta_{ij}   &  0  & \frac{\bar{\theta} \pi(\bar{\rho})}{\bar{\rho}}\delta_{ij} & 0 & -\frac{\bar{\theta}}{\bar{\rho}} \delta_{ij} & -\frac{\bar{\theta}}{2\bar{\rho}} (\delta_{ik'}\delta_{jl'} + \delta_{jk'}\delta_{il'})  \\
 		0  & \bar{\theta} \delta_{i'j} + \frac{\bar{\theta} \pi(\bar{\rho})}{\rho} \delta_{ji'} & 0 & \frac{1}{\bar{\rho}} \delta_{jm}  & 0 & 0 \\
 		- \bar{\theta}^{-2} (-\frac{\bar{\theta}}{\rho} ) \delta_{jm} & 0 & -\frac{\bar{\theta}^{-2}}{\rho} \delta_{jm} & 0 & 0 & 0 \\
 		0 &  -\frac{1}{\rho} & 0 & 0 & 0 & 0 \\
 		0 & -\frac{1}{2\bar{\rho}} (\delta_{jk} \delta_{i'l} + \delta_{jl} \delta_{i'k}) & 0 & 0 & 0 & 0  
 		\end{array}\right) }.
	\end{equation*}
	假设$W$为矩阵$\sum_j \omega_j A_j(U_e)$的特征值,即存在$\mu$,使得$\sum_h \omega_j A_j(U_e)W = \mu W$。我们得到
	\begin{eqnarray*}
				\bar{\rho} \omega \cdot W_2 = \mu W_1, \\ 
				\quad [( (\bar{\theta}  \pi_{\rho}(\bar{\rho})-\frac{\pi(\bar{\rho})\bar{\theta}^2}{\bar{\rho}}) W_1 + \frac{\bar{\theta} \pi(\bar{\rho})}{\bar{\rho}}W_3 I - \frac{\bar{\theta}}{\bar{\rho}} W_5 )I- \frac{\bar{\theta} \pi(\bar{\rho})}{\bar{\rho}} W_6] \cdot \omega =\mu W_2, \\
				\left( (\bar{\theta} + \frac{\bar{\theta} \pi(\bar{\theta})}{\bar{\rho}})W_2 + \frac{1}{\bar{\rho}} W_3 I \right) \cdot \omega = \mu W_3 \\
				\frac{\bar{\theta}^{-1}}{\bar{\rho}} W_1 -\frac{\bar{\theta}^{-2}}{\bar{\rho}} W_3 = \mu W_4\\
		-\frac{1}{\bar{\rho}} W_2 \cdot \omega  = \mu W_5, \\
		-\frac{1}{\bar{\rho}} (\omega \otimes W_2 + W_2 \otimes \omega) = \mu W_6.
	\end{eqnarray*}
	首先根据前面的分析我们知道第五个和第六个式子给出$W_2=0$。$\mu \neq 0$时将其代入第一个式子得到$W_1=0$。代入第四个式子得到$W_3=0$。从而$W=0$。$\mu=0$时,将$W_2=0$代入第三个式子得到$W_3=0$。将$W_3=0$代入第四个式子可得到$W_1=0$,从而$W=0$。这样我们验证了Kawashima条件成立。即对于模型\eqref{eq:CDFTemp},定理\ref{th:Kawashima}也成立。

	\subsection{松弛极限}
	实际中经常需要考虑系统在平衡态附近的情况。这是方程的源项经常会包含小参数\cite{}。即方程\eqref{eq:CDF}具有下面的形式。
	\begin{equation}\label{eq:CDFSingular}
		\partial_t U + \sum_{j=1}^3 \partial_{x_j} F_j(U) = \frac{1}{\epsilon}\mathcal{Q} (U) .
	\end{equation}
	其中$\epsilon$很小。对$\epsilon \rightarrow 0$时的极限的研究可以参看文献\cite{}。雍稳安的博士论文\cite{}中提出了极限的奇异摄动展开收敛的充分条件。在文献\cite{}中讨论了$\epsilon \rightarrow 0$时采用Maxwell迭代(Chapman-Enskog展开)得到近似方程的有效性。下面我们将这些结果用于线性粘弹性流体力学模型。为了简单我们在本节考虑模型\eqref{eq:CDFspecific}。其他情况的计算复杂些但思想是相同的。我们的主要目的是要说明当$\epsilon$很小时可以采用Navier-Stokes方程组近似本章发展的粘弹性流体力学模型。

	\subsubsection{线性粘弹性模型的摄动展开}
	
	\subsubsection{线性粘弹性模型的Chapman-Enskog展开}

	\subsection{可压模型的不可压极限}
	经典可压Euler方程的小马赫数极限的数学分析依赖于Euler方程的双曲性\cite{}。由于守恒-耗散理论得到的方程是对称双曲的,前面导出的可压粘弹性流体力学方程的小马赫数极限问题可以采用类似Euler方程的分析方法进行处理。本小节我们将忽略热传导并假设压力$p$为密度的函数,满足$p(\rho) = \kappa \rho^\gamma, \gamma>1$。
	\subsection{本章小结}

	\end{document}
