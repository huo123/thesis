% \chapter{绪论}
 \documentclass{article}
 \usepackage{ctex}
 \usepackage{amsmath}
 \begin{document}

\section{问题背景}
粘弹性流体广泛存在于日常生活中。粥、冰淇凌、油漆、细胞液等都是粘弹性流体。由于这些流体中高分子的存在,它们表现出一定的弹性效应。且这一效应是随着流动而衰减的\cite{coleman1961foundations,larson1999structure}, 我们把这一特性称为粘弹性。材料粘弹性行为的数学模型已经有很长的研究历史。\cite{maxwell2013scientific,kelvin1887stability,oldroyd1950formulation,weissenberg1947continuum,zimm1956dynamics,ferry1980viscoelastic,larson1999structure}. 随着近些年来材料科学和生命科学的发展,越来越多的新材料被发现,例如液晶、凝胶、活性物质等\cite{de1975physics,prost2015active,marchetti2013hydrodynamics,ramaswamy2010mechanics,berthier2013non}。利用数学物理方法对这些流体的粘弹性特性进行建模,获得了快速发展\cite{lin2012some,larson1999structure,joseph2013fluid}。

由于实际中的粘弹性流体大多是水溶液,目前大部分模型考虑的都是不可压情形,即假设流体不可压缩。然而随着纳米科学等学科的发展,一些粘弹性流体的压缩性变得不可忽略\cite{yu2015compressible,galstyan2015note,chakraborty2015constitutive,pelton2009damping,kim1999numerical,lind2013bubble}。且大部分模型并没有考虑温度的影响,虽然实际上粘弹性流体的流动行为与温度密切相关。如何将流体的压缩性和温度纳入粘弹性流体的模型中,是一个具有挑战的问题\cite{grmela1997dynamicsI,ottinger1997dynamicsII,larson1999structure}。

另外,由于粘弹性效应,粘弹性流体在热力学上处于非平衡态。经典平衡态热力学无法很好地对其进行描述。近些年来随着非平衡态热力学的发展,许多理论被应用于粘弹性流体的建模中\cite{jou1996extended,beris2013thermodynamics,ottinger2005beyond,coleman1961foundations,,coleman1964thermodynamics,truesdell2012rational,truesdell2004non,eringen2012microcontinuum,eringen1965theory,zhu2014conservation}。然而,目前大多数理论是从物理原理出发,缺乏数学上对方程适定性的分析。如何建立一个既满足物理原理,又能够满足数学上的要求的理论,对于粘弹性流体的数学建模具有重要意义\cite{zhu2014conservation,larson1999structure}。

为了研究这些问题,本文考虑的粘弹性流体仅限于简单的高分子溶液。

\section{粘弹性流体建模的研究历程与现状}
对物质粘弹性的建模的早期研究出现在19世纪下半叶。James Clerk Maxwell从阻滞弹簧的模型出发建立了线性粘弹性理论\cite{maxwell2013scientific}. 经典的Maxwell模型方程如下
picneed
\begin{eqnarray*}
\frac {d\epsilon} {dt} = \frac {\sigma} {\eta} + \frac {1} {E} \frac {d\sigma} {dt}.
\end{eqnarray*}
其中$\sigma$和$\epsilon$分别为材料的应力张量和应变张量,$\eta$和$E$分别为材料的粘性系数和弹性常数(模量)。在流体中$\frac {d\epsilon} {dt}$一般采用速度梯度量$\nabla v$的对称部分$D=\frac{1}{2} (\nabla v + (\nabla v)^T)$。一个简化的形式为
\begin{equation} \label{eq:maxwell}
			\partial_t \sigma + v \cdot \nabla \sigma - D = -\frac{\sigma}{\xi}. 
\end{equation}
交换上面方程中的应力应变张量得到的模型称为Kelvin–Voigt模型,由物理学家Lord Kelvin和Woldemar Voigt在19世纪末提出\cite{kelvin1887stability}。

19世纪中叶Karl Weissenberg发现了Weissenberg效应。这一效应是指当转动的杆置于粘弹性流体中时,流体将会沿着杆“爬上”或者"爬下"\cite{weissenberg1947continuum}。为了解释这一效应,James G. Oldroyd提出了粘弹性建模的重要法则:物质坐标不变性原理。这一原理是说模型不应依赖于坐标系的选取。基于这一原理,他提出了客观导数的概念。他提出了经典的Oldroyd—B模型。这一模型如下
\begin{equation} \label{eq:Oldroyd}
	{\sigma} + \lambda_1 \stackrel{\nabla}{{\sigma}} = 2\eta_0 ({D} + \lambda_2 \stackrel{\nabla}{{D}}).
\end{equation}
其中$\sigma$为应力张量,$\lambda_1$和$\lambda_2$为松弛时间和迟滞时间,$\eta_0$为流体的粘性系数。$\stackrel{\sigma} = \partial_t \sigma + v \cdot \nabla \sigma - (\nabla v)\  \sigma + \sigma (\nabla v)^T$为$\sigma$的上对流Maxwell导数\cite{oldroyd1950formulation}。J. Oldroyd的这一论文对粘弹性流体的建模这一领域产生了深远的影响。基于这一原理,修改经典Maxwell模型和Kelvin-Voigt模型中的导数为客观导数可以得到许多模型\cite{lin2005hydrodynamics,larson1999structure}。

然而,直接推广线性粘弹性模型仅能考虑简单的情况,并且不能很好地纳入温度的影响以及考虑模型的热力学性质。为了解决这些问题,热力学家和力学家提出了许多理论。这些理论主要有两个方向。一是利用非平衡态热力学理论对粘弹性流体进行建模\cite{jou1996extended,ottinger2005beyond,zhu2014conservation}。二是基于热力学原理和有限形变理论的记忆衰减理论\cite{coleman1961foundations,truesdell2012rational}。

\subsection{非平衡态热力学在粘弹性流体建模中的应用},
热力学第二定律告诉我们热力学体系不可逆状态的存在性。为了研究热力学不可逆过程,科学家发展了非平衡态热力学。Lars Onsager在1931和1932年的两篇文章中提出了著名的Onsager倒易关系\cite{onsager1931reciprocal,onsager1931reciprocalII}。由Onsager与Prigogine等人发展的经典非平衡态热力学理论(Classical Irreversible Thermodynamics,简称CIT),自20世纪40年代以来获得了蓬勃发展,并且得到了广泛的承认\cite{jou1996extended,truesdell2012rational}。CIT理论可以将经典的牛顿流体和Fourier热传导定律纳入其中。然而无法将Maxwell粘弹性模型和Cattaneo热传导模型纳入其理论框架中来。为了推广经典的CIT理论,许多新的非平衡态理论被提出。其中包括有理热力学(Rational Thermodynamics,简称RT)、扩展不可逆热力学(Extended Irreversible Thermodynamics,简称EIT)、非平衡可逆-不可逆一般方程(Gerneral Equation for the Non-Equilibrium Reversible-Irreversible Coupling,简称GENERIC)、守恒-耗散理论(Convervation-Dissipation Formalism of irreversible thermodynamics,简称CDF)等\cite{truesdell2012rational,jou1996extended,ottinger2005beyond,zhu2014conservation}。

考虑一般的流体方程
\begin{subequations} \label{eq:fluid}
	\begin{align}
		\partial_t \rho + \nabla \cdot (\rho v) = 0 ,\\
		\partial_t (\rho v) + \nabla \cdot (P + \rho v \otimes v), \\
		\partial_t[\rho (u + \frac{1}{2} v^2] + \nabla \cdot [q + \rho (u+\frac{1}{2}v^2) v + P \cdot v] = 0.
	\end{align}
\end{subequations}
其中$\rho,v,u$分别表示流体的密度、速度和内能,$P$表示流体的应力张量。这三个方程分别表示质量守恒、动量守恒和能量守恒。CIT假设平衡态的热力学状态参量仍然可以用来描述非平衡态,且体系的熵依赖于这些状态变量。这里我们假设体系的比熵(单位质量的熵)为$s=s(\frac{1}{\rho},u)$。利用经典平衡态热力学的Gibbs关系,有
\begin{equation*}
	T ds = du + p d(\frac{1}{\rho}).
\end{equation*}
通过计算可以得到
\begin{equation*}
	\rho \dot{s} = - \frac{1}{T} \nabla \cdot q - \frac{1}{T} P: \nabla v.
\end{equation*}
其中$\dot{s} = \partial_t s + v \cdot s$表示Lagrange导数。根据热力学第二定律,为保证不可逆过程熵函数的增长,需要$\rho \dot{s} \ge 0 $。根据Onsager倒易关系,热力学流和热力学力之间为线性关系。另$P = p + \pi I + \mathring{P}$,其中$\mathring{P} = P - \frac{1}{3} \mbox{Tr}(P) I$为$P$的无迹部分。依据下表的热力学流和力的定义,根据Onsager倒易关系,我们可以得到下面的本构关系。
\begin{eqnarray*}
	q = -\lambda \nabla T, \\
	\pi =  - \zeta \nabla \cdot v, \\
	\mathring{P} = - 2 \eta \mathring{D}.
\end{eqnarray*}

tableneed

然而,由于局部平衡假设的存在,CIT仅能描述远离平衡态不远的系统。为了拓展CIT,大部分理论假设非平衡态不仅需要平衡态变量,而且需要非平衡态变量来描述。Coleman、Truesdell、Noll等人发展了RT。RT理论认为非平衡态不仅依赖于平衡态状态变量,还依赖于状态变量的空间梯度和历史等信息。RT理论不在假设局部平衡原理,而是认为系统存在“记忆性”,流体当前的状态取决于之前的形变历史。通过推广Clausius-Duhem不等式,利用Onsager倒易关系得到本构关系。Stoks-Fourier可以由RT理论导出\cite{jou1996extended}。

另外,David Jou、Georgy Lebon等人发展了EIT。EIT通过拓展非平衡态状态变量,将CIT中的热力学流引入非平衡态状态变量中,来对不可逆过程进行建模。EIT仍然假设Onsager倒易关系成立。例如我们将$q,\pi,\mathring{P}$引入非平衡态状态变量中,并假设$s = s(\frac{1}{\rho}, u,q,\pi,\mathring{P} )$,则通过计算可以得到熵的产生率为
\begin{equation*}
		\rho \dot{s} = - \frac{1}{T} \nabla \cdot q - \frac{1}{T} P: \nabla v - T^{-1} \mathring{P} : \mathring{D} - \alpha_1 \pi \cdot \dot{\pi} - \alpha_2 q \cdot \dot{q} - \alpha_3 \mathring{P} : ({\mathring{P} })^. .
\end{equation*}
其中$\alpha_1,\alpha_2,\alpha_3$为常数。根据下表中的热力学流和力的定义,利用Onsager关系,可以得到下面的本构关系。
\begin{subequations} \label{eq:EITconstitutive}
		\begin{align}
	\nabla T^{-1} - \alpha_2 \dot{q} = \mu_1 q - \beta^{''}\nabla \cdot \dot{P} - \beta' \nabla \pi, \\
	-T^{-1} \nabla \cdot v - \alpha_1 \dot{\pi} = \mu_0 \pi - \beta' \nabla \cdot q , \\
	-T^{-1} \mathring{D} - \alpha_3 (\mathring{P})^. = \mu_2 \mathring{P} - \beta^{''} (\mathring{\nabla q})^s. 
		\end{align}
\end{subequations}
其中$A^s = \frac{1}{2} (A+A^T)$表示张量$A$的对称部分。通过取合适的参数可以得到Maxwell-Cattaneo定律。
\begin{eqnarray*}
	\tau_1 \dot{q} + q = - \lambda \nabla T, \\
 	\tau_0 \dot{\pi } + \pi = -\zeta \nabla \cdot v, \\
 	\tau_2 (\mathring{P})^. + \mathring{P} = -2 \eta \mathring{D}.
\end{eqnarray*}
对于不可压流体,我们可以得到Maxwell模型。其中$\sigma = \pi + \mathring{P} $。EIT可以很好地用于线性粘弹性流体模型,然而对于非线性的本构关系,缺乏简单的处理方式\cite{jou1996extended}。

为了处理复杂的非线性粘弹性流体模型,Antony N. Beris和Brian J. Edwards等人通过拓展经典力学的Hamilton理论,在Possion括号之外引入耗散括号(dissipative bracket)来描述能量的耗散和不可逆过程的熵的增长\cite{edwards1990remarks,beris2013thermodynamics}。假设$X$为非平衡态系统的状态变量。系统的广义Hamilton函数设为$H=H(X)$,则$X$的演化方程为
\begin{equation*}
	\frac{dX}{dt} = L(X) \cdot \frac{\delta H(X)}{\delta X} + M(X) \cdot \frac{\delta H(X)}{\delta X} . 
\end{equation*}
$L=L(x)$和$M=M(x)$为体系的Possion矩阵和耗散矩阵。通过给出系统的能量和熵函数,选取合适的$L$和$M$即可得到描述体系的方程。 为了满足能量守恒和熵增原理,我们假设$L$与$M$可以定义Possion括号和耗散括号。
\begin{equation*}
	\{ A,B \} = \frac{\delta A(X)}{\delta X} \cdot L(X) \cdot \frac{\delta B(X)}{\delta X}, \quad  [ A,B ] = \frac{\delta A(X)}{\delta X} \cdot M(X) \cdot \frac{\delta B(X)}{\delta X}. 
\end{equation*}
其中$L$的选取需要满足经典的Possion括号定义。$M$的选取使得下面的条件成立
\begin{equation*}
	[\int \rho(x) dx, H]=0,[\int \rho(x) v(x) dx, H]=0, \quad [E,H] = 0.
\end{equation*}
从而守恒率成立,以及
\begin{equation*}
	[S,H] \ge 0,
\end{equation*}
从而热力学第二定律成立。对于不可压粘弹性流体,我们忽略温度的影响,则可以取状态变量$X 
 \{v,c\}$,其中张量$c$表示非平衡态变量(与粘弹性引起的应力能耗散有关)。取Hamilton量为
 \begin{equation*}
 	H = \int \frac{1}{2} v^2(x) + \frac{1}{2} \mbox{
 	Tr}(c(x))- \frac{1}{2} \rho \ln \det c(x) dx,
 \end{equation*}
以及Possion矩阵和耗散矩阵
\begin{equation*}
	L = \left( \begin{array}{cc} 
			u \cdot \nabla &  L_{12} \\
			L_{21}& 0
		\end{array}\right), \quad
	M =	\left( \begin{array}{cc} 
			0  & 0 \\
			0  & c \otimes I
		\end{array}\right).
\end{equation*}
其中
\begin{eqnarray*}
	(L_{12})_{l,kj} = - \frac{\partial}{\partial x_m} c_{mk} \delta_{jl} -  \frac{\partial}{\partial x_m} c_{jm} \delta_{kl}, \\
	(L_{21})_{jk,l} = -c_{mk} \delta_{jl} \frac{\partial}{\partial x_m} - c_{jm} \delta_{kl} \frac{\partial}{\partial x_m}.
\end{eqnarray*}
能量和熵函数为
\begin{equation*}
	E = \int \frac{1}{2} v^2(x)dx, \quad S = -\int \frac{1}{2} \mbox{
 	Tr}(c(x))- \frac{1}{2} \rho \ln \det c(x) dx
\end{equation*}
于是我们可以得到下面的上对流导数Maxwell模型
\begin{eqnarray*}
	v_t + v \cdot \nabla v + \nabla p  - \nabla \cdot (c - I) = 0, \\
	c_t - v \cdot \nabla v - (\nabla v)c - c (\nabla v)^T = -\frac{c-I}{\lambda}.
\end{eqnarray*}

由于Beris和Edwards理论要求的耗散括号的复杂性,如何提出满足要求的耗散矩阵是一个很有挑战性的问题。
 Hans C. \"Ottinger与Miroslav Grmela发展了GENERIC理论。GENERIC并不假设非平衡态Hamilton函数的存在,而是仅仅假设能量和熵的存在性。假设$X$为非平衡态系统的状态变量。其演化方程可以写为
\begin{equation*}
	\frac{dX}{dt} = L(X) \cdot \frac{\delta E(X)}{\delta X} + M(X) \cdot \frac{\delta S(X)}{\delta X} .
\end{equation*}
其中$E=E(X)$和$S=S(X)$分别为系统的能量和熵,$L=L(x)$和$M=M(x)$为体系的Possion矩阵和耗散矩阵。通过给出系统的能量和熵函数,选取合适的$L$和$M$即可得到描述体系的方程。 为了满足能量守恒和熵增原理,我们假设$L$与$M$可以定义Possion括号和耗散括号。
\begin{equation*}
	\{ A,B \} = \frac{\delta A(X)}{\delta X} \cdot L(X) \cdot \frac{\delta B(X)}{\delta X}, \quad  [ A,B ] = \frac{\delta A(X)}{\delta X} \cdot M(X) \cdot \frac{\delta B(X)}{\delta X}. 
\end{equation*}
其中$L$的选取需要满足经典的Possion括号定义。$M$的选取使得下面的条件成立
\begin{equation*}
	[A,B] = [B,A], \quad [A,A] \ge 0.
\end{equation*}
假设系统的能量为$E$,熵函数为$S$,我们假设$L,M$满足下面的退化条件
\begin{equation}
	L \frac{\delta S}{\delta X} = 0, \quad M \frac{\delta E}{\delta X} = 0.
\end{equation}
那么任意泛函$A$的演化方程为
 \begin{equation*}
	\frac{d A}{dt} = \frac{\delta A(x)}{\delta x} \cdot \frac{dx}{dt} = \{ A,E\} + [A,S].
\end{equation*}
于是可以得到
\begin{eqnarray*}
	\frac{d E}{dt} = \{E,E\} +[S,E] = 0, \\
	\frac{d S}{dt} = [S,S] \ge 0. 
\end{eqnarray*}
这里我们用到了退化条件$[E,S] = 0$。在GENERIC中关键的是选取合适的Possion矩阵和耗散矩阵,使之满足Possion括号和耗散括号的定义以及退化条件。GENERIC将物理量的演化分成了两个部分,一部分是由平衡态热力学引起的,而另一部分是由不可逆过程引起的,具体通过Possion括号和耗散括号实现。并且通过退化条件保证了热力学第一定律和第二定律的成立。然而在上面上对流导数Maxwell模型的例子中,却很难将可逆过程和不可逆过程的影响通过这种方式分开。如果我们采用上面的能量和熵函数$E,S$,以及矩阵$L,M$,那么可以验证$L$不在满足退化条件。  

由于EIT缺乏对客观导数的模型的考虑,以及GENERIC在选取Possion矩阵和耗散矩阵的复杂性及对可逆部分和不可逆部分耦合方式的简单性,这些理论都没有完美地解决粘弹性流体的建模问题。另一方面,这些理论建立在物理定律的基础上,缺乏对方程数学性质的分析。除上面提到的非平衡态热力学理论之外,也有很多其他的非平衡态热力学理论试图解决粘弹性流体的建模问题,可以参看\cite{stratonovich2012nonlinear,huo1993nonequilibrium,eu1992kinetic}。

雍稳安、朱毅、洪柳、杨再宝等人从双曲方程的结构出发,结合热力学的第一定律和第二定律,发展了不可逆热力学的守恒-耗散理论(CDF)\cite{zhu2014conservation}。CDF已经顺利地应用于线性粘弹性流体力学模型。然而,由于目前的CDF考虑的是守恒形式的方程,而客观导数的存在将会导致方程不在可以写成守恒形式,所以CDF在应用到非线性粘弹性流体模型时遇到了困难。本文将推广这一理论并将其应用于非线性粘弹性流体力学模型中。

\subsection{粘弹性流体的记忆衰减理论}
由于粘弹性流体可以看作连续介质,所以可以采用连续介质力学的理论对其进行建模。前文提到的模型一般都是通过小变形连续介质力学得到的。而对于粘弹性流体来说,尤其是高弹性的流体,小形变假设不在适用。基于此,Bernard Coleman和Walter Noll等人提出了利用有限形变理论对粘弹性流体进行建模的方法\cite{coleman1961foundations,coleman1964thermodynamics,coleman2012viscometric}。该理论认为物质的状态依赖于其形变的历史。而形变可以用形变张量$F$描述。这里$F$定义为流动图$x = x(\xi,t)$的梯度$F_{ij} = \frac{\partial x_i}{\partial \xi_j}$。其中$x,\xi$分别对应于流体质点的Euler和Lagrange坐标。根据定义可以得到
\begin{equation}\label{eq:Feq}
	\partial_t F + v \cdot \nabla F = (\nabla v) F.
\end{equation}
这一理论认为应力张量是$F$的历史和温度$T$的历史的函数。假设体系的自由能为
\begin{equation*}
	\psi = \mathop{\psi} \limits_{\tau=0}^{t}( C(t), C_t(\tau),\theta(t),\theta^t(\tau)).
\end{equation*}
其中$C(t) = F^T(t) F(t),C^t(\tau) = C(t - \tau)$。$\psi$依赖于从初始时刻到当前时刻$t$的应变历史。利用Gibbs关系,可以得到
\begin{equation}\label{eq:en}
	\rho \frac{d\psi}{dt} + \rho \eta \frac{d \theta}{dt} - T: \frac{dC}{dt} + w = 0.
\end{equation}
其中$T$代表第二Piola-Kirchhoff张量,与Cauchy应力张量的关系为$\sigma = \rho F T F^T$,$w$代表能量的耗散。由$\psi$的表达式可以得到
\begin{equation*}
	\frac{d\psi}{dt} = \frac{\partial \psi}{\partial C} : \frac{d C}{dt}+ \frac{\partial \psi}{\partial \theta} \frac{d\theta}{dt} + \delta \psi。
\end{equation*}
代入\eqref{eq:en}中得到
\begin{equation*}
	(\rho \frac{\partial \psi}{\partial C} - T) : \frac{dC}{dt} + \rho (\frac{\partial \psi}{\partial \theta} + \eta) d\theta + (w + \rho \delta \psi) dt  =0.
\end{equation*}
从而得到
\begin{eqnarray*}
	T = \rho \frac{\partial \psi}{\partial C} = \mathop{\mathcal{F}}\limits_{\tau=0}^{t} ( C(t), C_t(\tau),\theta(t),\theta^t(\tau)), \\
	\eta = -\frac{\partial \psi}{\partial \theta},\\
	w = -\rho \delta \psi.
\end{eqnarray*}
热力学第二定律由$w \ge 0$保证\cite{coleman1961foundations,dimitrienko2010nonlinear}。

例如经典的上对流导数Maxwell模型的熵函数可以取作(忽略温度)
\begin{equation*}
	\psi  = \psi(t) = - \frac{1}{\lambda}  \ln \det (F^T(t) F(t)) + F^T(t) F(t) : \frac{1}{\lambda}  \int_{0}^t e^{-\frac{t-s}{\tau}} F^{-1}(\tau) F^{-T}(\tau) d\tau 
\end{equation*}
于是有
\begin{equation*}
	T = \frac{1}{\lambda}  \int_{0}^t e^{-\frac{t-s}{\tau}} F^{-1}(\tau) F^{-T}(\tau) d\tau -\frac{1}{\lambda}  F^{-1} F^{-T} 
\end{equation*}
应力张量$\sigma$可以写为
\begin{equation*}
	\sigma = \frac{1}{\lambda} \int_{0}^{t} e^{-\frac{(t-s)}{\lambda} } F(t)  F^{-1}(s) F^{-T}(s) F^T(t)  dx.
\end{equation*}
注意上面的积分是沿流线的积分。于是根据\eqref{eq:Feq},可以得出$\sigma$满足上对流导数Maxwell模型。该模型的能量耗散为
\begin{equation*}
	w = - \rho \frac{1}{\lambda} \mbox{Tr}(I) = -\frac{3\rho}{\lambda} < 0.
\end{equation*}
这违背了热力学第二定律对$w>0$的要求。所以这一理论不能很好地包括上对流导数Maxwell模型。

并且这一理论得到的方程为积分-微分方程,且由于积分一般是沿着流线的积分,在实际计算中变得十分复杂。通过引入新的变量可以将积分方程变为微分方程,可以参看\cite{dimitrienko2010nonlinear}。

另外,林芳华等人在文献\cite{lin2005hydrodynamics}中提出了下面的模型。
\begin{subequations}\label{eq:linincompressible}
	\begin{align}
		\nabla \cdot v = 0, \\
		v_t + v \cdot \nabla v + \nabla p = \mu \Delta v + \nabla \cdot( F F^T), \\
		F_t + v \cdot \nabla F = \nabla v F.
	\end{align}
\end{subequations}
文献\cite{liu2008global}中发展这一方程组至可压缩形式。这里的模型实际上是Navier-Stokes方程的一个推广。利用有限形变理论将弹性形变$F F^T$包含在了应力张量之中。这一模型因为有着良好的数学结构而得到了许多数学上的好的结果\cite{lin2012some}。但是尚缺乏在实际中的应用实例。

综上所述,有限形变理论是一个描述连续介质力学的很好工具,利用有限形变理论,结合非平衡态热力学的方法发展起来的热力学理论在粘弹性流体的建模方面有一定的优势,但是也存在一些问题。

本论文将讨论基于CDF理论和有限形变理论的粘弹性流体建模方法。

\section{粘弹性流体的数学分析}
由于粘弹性流体方程的重要性和复杂性,其数学分析一直以来都吸引了很多数学家的兴趣\cite{lin2012some,fabrizio1992mathematical,renardy2008mathematical,joseph1987hyperbolicity,guillope2010regular,saut2012lectures,yong2014newtonian,qian2010well,hu2011global,lei2005global,fang2014incompressible}。目前的分析重要集中于不可压方程,主要包括Oldroyd-B模型、上对流导数Maxwell模型、FENE模型、以及林芳华等人发展的粘弹性流体力学的分析\cite{elgindi2015global,saut2012lectures,renardy2000mathematical,masmoudi2011global,lions2000global,masmoudi2008well,lei2010remarks}。近些年来,有部分工作讨论了可压粘弹性流体方程\cite{fang2014incompressible,hu2012strong,qian2010global,qian2011initial}。
\subsection{经典不可压粘弹性流体力学模型的分析}
经典线性粘弹性模型的数学分析见,
\subsection{林芳华等提出的模型的数学分析}
由于林芳华等提出的模型有着良好的数学结构,其吸引了许多数学家的兴趣\cite{lin2005hydrodynamics,lebon2008classical,qian2010well,qian2011initial,hu2012formation,hu2015global}。在\cite{lin2005hydrodynamics}中,该方程在二维的Cauchy问题的局部存在性和在平衡态附近的额整体存在性均得到了证明。证明基于力学上的适应性条件$\nabla \cdot F^T$,从而可以找到势函数$\phi$使得$F = \nabla \times \phi$。在二维时
\begin{equation*}
	F = \left( \begin{matrix}
		-\partial_{x_2} \phi_1 & -\partial_{x_2} \phi_2 \\
		\partial_{x_1} \phi_1 & \partial_{x_1} \phi_2
	\end{matrix}\right).
\end{equation*}
于是方程\eqref{eq:linincompressible}可以写为下面的形式。
\begin{eqnarray*}
	\nabla \cdot v = 0, \\
	v_t + v \cdot \nabla v + \nabla p =  \mu \Delta v - \sum_{i=1}^2 \Delta \phi_i \nabla \phi_i, \\
	\nabla \cdot v = 0.
\end{eqnarray*}
利用Galerkin逼近,\cite{lin2005hydrodynamics}中其解的局部存在性得到了证明。证明采用了Gagliardo–Nirenberg不等式


\subsection{可压粘弹性流体模型的分析}


\section{主要工作}
从当前的研究可以看出各种非平衡态热力学理论对经典的粘弹性流体力学的建模存在各种各样的问题,亦缺乏对数学上的考虑。如何考虑温度和压缩型的影响对粘弹性力学的模型十分重要。对粘弹性流体力学方程的分析工作目前仍然是具体模型具体分析,缺乏一个统一的框架。

为了解决这些问题,本论文基于守恒-耗散理论提出了建立粘弹性流体力学模型的一般框架,并利用双曲方程的一般理论对一些模型进行了数学分析。主要完成了以下工作
%本论文的主要目的,一方面是推广经典的不可压粘弹性流体力学模型,使之考虑压缩性以及温度的影响。另一方面是基于双曲方程的相关理论对粘弹性流体力学模型进行数学分析。在建模方面,我们通过推广守恒-耗散理论CDF,使之可以纳入带客观导数的非线性粘弹性模型。在分析方面,我们考察了可压上对流导数Maxwell模型的主要完成了以下工作:
\begin{enumerate}
\item 推广守恒-耗散理论使之能够纳入带客观导数的粘弹性模型,并基于推广的守恒-耗散理论发展了多种可压粘弹性流体力学模型。另外,考察将温度的影响并利用守恒-耗散理论导出了不同的非平衡热传导模型。
\item 提出了利用有限形变理论和守恒-耗散理论对粘弹性流体进行建模的理论框架。得到了方程双曲性的充分条件。
\item 利用双曲方程的相关理论分析了线性粘弹性流体力学模型的平衡态附近解的整体存在性,并考察了其同经典Navier-Stokes方程的一致性。
\item 利用双曲方程的相关理论分析了可压上对流导数Maxwell模型的平衡态附近解的存在性,并利用双曲松弛系统的相关理论证明了在松弛参数趋于0时该模型和经典Navier-Stokes方程的一致性。
\item 考察了一般弹性能函数下林芳华等人提出的无穷大Weissenberg数粘弹性流体力学模型的平衡态附近解的整体存在性。分析了力学适应性条件在方程解的存在性证明中的作用。
\end{enumerate}

\section{行文结构}
本论文共分六章。
\begin{itemize}
\item 第二章首先讨论了守恒-耗散理论的一般框架及其在线性粘弹性流体力学模型中的应用。考察了压缩性和温度的影响。并给出了守恒-耗散结构在数学上的意义。利用双曲方程的相关理论给出了线性粘弹性流体力学模型的解的全局存在性和同经典Navier-Stokes方程的一致性分析。
\item 第三章提出了推广的守恒-耗散理论。利用推广的守恒-耗散理论导出了可压缩流体的上对流导数Maxwell模型。并分析了不同客观导数下的可压缩粘弹性流体力学模型。结合有限形变理论和守恒-耗散理论,提出了一些粘弹性流体力学模型。
\item 第四章分析了可压上对流导数Maxwell模型,首先给出了平衡态附近解的全局存在性。并证明了松弛参数趋于0时与Navier-Stokes方程的一致性。
\item 第五章分析了无穷大Weissenberg数粘弹性流体力学模型,基于Kwashima卢纶给出了方程解的整体存在性的一个新的证明,并分析了力学上的适应性条件在证明中的作用。
\item 第六章对上面的工作进行了总结,并提出了未来研究的发展方向。
\end{itemize}





	\bibliography{ref}
	\bibliographystyle{plain}

 \end{document}