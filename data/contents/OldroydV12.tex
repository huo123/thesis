% draft of the paper, built on Sep.29 by Huo Xiaokai, Email: hxk12@mails.tsinghua.edu.cn
% Chang on Dec. 3
% Revised on Dec. 7
% this is a combination of Oldroydnew1 and Oldroyddecay
% \documentclass{article}
% \usepackage{ctex}
% \usepackage{amsfonts}
% \usepackage{amsmath}
% \usepackage{amsthm}
% \usepackage{cite}
% %\usepackage{showkeys}
% \theoremstyle{plain}
% \begin{document}
%  \newtheorem{theorem}{定理}
%  \newtheorem{remark}{注释}
%  \newtheorem{thm}{Theorem}
%  \newtheorem{lemma}{Lemma}
%  \theoremstyle{definition}
%  \theoremstyle{Remark}
% \newtheorem{rem}{Remark}
% \newtheorem{defn}{Definition}
% \begin{subequations} \label{eq:ECDFsecondisothermal}
% 		\begin{align}
% 			\rho_t + \nabla \cdot (\rho v) = 0 ,\\
% 			(\rho v)_t + \nabla \cdot (\rho v \otimes v) + \nabla (p + \frac{1}{2} \rho c: \rho c)  - \nabla \cdot ( (2 \rho c + 2 \rho c \cdot \rho c)) =0 ,\\
% 			(\rho c)_t +  \nabla \cdot (\rho c \otimes v) - (\nabla v) \rho c - (\rho c) (\nabla v)^T - 2 D = - \frac{\rho \dot{c}I}{\kappa} -  \frac{\rho \mathring{c}}{\xi}  .
% 		\end{align}
% \end{subequations}
% 在一维时,对应的方程为
% \begin{subequations} \label{eq:ECDFsecond1D}
% 		\begin{align}
% 			\rho_t + \partial_x (\rho v) = 0 ,\\
% 			(\rho v)_t + \partial_x (\rho v^2) + \partial_x (p)   -  (2+ 3 \rho c) \partial_x (  \rho c) =0 ,\\
% 			(\rho c)_t +  \partial_x (\rho c  v) - 2 \rho c \partial_x  v  - 2 \partial_x v = - \frac{\rho \dot{c}}{\kappa}  .
% 		\end{align}
% \end{subequations}

% 虽然模型\eqref{eq:ECDFsecond1D}的熵函数存在,但是由于方程不是守恒形式,熵函数的Hessian矩阵无法对称化这个一维方程组。我们在本小节中将会首先给出这一方程组的一个对称子。然后验证其满足Kawashima条件,从而证明平衡态附近解的整体存在性定理。最后我们讨论了源项含有小参数时的松弛极限,并给出了松弛参数趋于0时其近似一维Navier-Stokes方程的严格分析。

\subsection{熵函数和对称性}
首先由前面的分析,我们可以得到方程\eqref{eq:ECDFsecondisothermal}的比熵为
\begin{equation*}
	s (\nu, c ) =  -\int_{\rho_0}^{1/\nu} \frac{\pi(z)}{z^2} dx - \frac{1}{2 \nu}  \mbox{Tr} (c^2).
\end{equation*}
令$U = (\rho, \rho v, \rho c)^T$,我们可以得到方程的熵函数
\begin{equation} \label{eq:cucmentropy}
	\eta = \eta(U) = \frac{(\rho v)^2}{2\rho}+\rho \int_{\rho_0}^{\rho} \frac{\pi(z)}{z^2} dz + \frac{1}{2}  \mbox{Tr} \left((\rho c)^2\right).
\end{equation}
注意我们在这里采用的熵函数$\eta$的符号与之前正好相反。%实际上这里$\eta$代表的是Hemholtz自由能。
$\eta$为其变量的下凸函数,这是因为
\begin{equation*}
	\eta_{U_{i''k''l''}U_{ikl}} = \left( \begin{array}{ccc} 
		\frac{p_\rho}{\rho} + \frac{v^2}{\rho} & -\frac{v_i}{\rho} & 0 \\
		-\frac{v_{i''}}{\rho} & \frac{1}{\rho} & 0 \\
		0 & 0 & \delta_{k''k}\delta_{l''l}
	\end{array}\right)
\end{equation*}
当$\pi_\rho>0$时是正定的。而因为静压力$\pi$为$\rho$的单调增函数,$\pi_\rho>0$成立。

通过计算,我们可以得到熵的演化方程
\begin{eqnarray}\label{23}
	\eta_t &=&  - \nabla \cdot ( \eta v+ \pi v - (2\rho c + 2 \rho c \cdot \rho c) \cdot v) - \frac{1}{\kappa} \rho c: \rho c \nonumber\\
  &\equiv& -\nabla \cdot J(U) + \Delta.	
\end{eqnarray}
其中$\Delta \ge 0$。

对于含有熵函数的守恒律方程组\cite{friedrichs1971systems},我们知道熵函数的Hessian矩阵可以对称化该方程组。然而由于方程组\eqref{eq:ECDFsecondisothermal}不是守恒形式,熵函数的Hessian矩阵不能对称化这个方程组,下面我们来说明这一点。

首先我们将$U$的方程表示成下面的形式
\begin{equation*}
	U_t + \sum_{j=1}^n A_j(U) U_{x_j} = \mathcal{Q}(U).
\end{equation*}
其中$\mathcal{Q}$和$A_j$的表达式如下
\begin{eqnarray*}
	&&\mathcal{Q} (U)= \left( \begin{array}{c}
		0 \\ 0 \\-\frac{1}{\kappa} \rho c
	\end{array} \right), \quad A_j(U)_{ikl,i'k'l'} = \\
	 &&\left( \begin{array}{ccc}
		0 & \delta_{i'j} & 0 \\
		p_\rho -v_i v_j & v_i \delta_{i'j}+v_j \delta_{i'i} & (23) \\
		- c_{kl} v_j + c_{jl} v_k + c_{kj} v_l + \frac{1}{\rho}({v_k} \delta_{jl} + v_l \delta_{jk})  & (32) & v_j \delta_{kk'} \delta_{ll'}
	\end{array} \right)
\end{eqnarray*}
其中$(23)=\rho (c_{k'l'} \delta_{ij} - 2\delta_{ik'} c_{jl'} - 2\delta_{il'} c_{jk'} ) - (\delta_{ik'} \delta_{jl'} + \delta_{il'} \delta_{jk'}),(32)= c_{kl} \delta_{i'j}- \frac{1}{\rho} ( \delta_{i'k} c_{jl} + \delta_{i'l} c_{jk} + \delta_{ik} \delta_{jl} + \delta_{il} \delta_{jk})$
而熵函数的Hessian矩阵$\eta_{UU}$与$A_j$的乘积为
\begin{eqnarray*}
	\eta_{UU}  A_j(U) = \\
		 \left( \begin{array}{ccc}
		* & \frac{\pi_\rho -v_{i} v_j}{\rho}  & (13)  \\
		\frac{\pi_\rho -v_i v_j}{\rho} & * &  (23) \\
		(31) & (32) & *
	\end{array} \right)
\end{eqnarray*}
其中$(13)=- c_{k'l'} v_j + 2 c_{jl'} v_{k'} + 2 c_{k'j} v_{l'} + \frac{1}{\rho}({v_{k'}} \delta_{j{l'}} + v_{l'} \delta_{jk'}), (31)=- c_{k''l''} v_j + c_{jl''} v_{k''} + c_{k''j} v_{l''} + \frac{1}{\rho}({v_{k''}} \delta_{j{l''}} + v_{l''} \delta_{jk''}),(23)=c_{k'l'} \delta_{ij} -( \delta_{ik'} c_{jl'} + \delta_{il'} c_{jk'} +\delta_{ik'} \delta_{jl'} + \delta_{il'} \delta_{jk'}),(32) =c_{k''l''} \delta_{i'j} -( \delta_{i'k''} c_{jl''} + \delta_{i'l''} c_{jk''} + \delta_{ik''} \delta_{jl''} + \delta_{il''} \delta_{jk''}) $。
注意到该矩阵的$(13)$和$(31)$项中$c_{jl''} v_{k''}+ c_{k''j} v_{l''}$与$c_{jl'} v_{k'}+ c_{k'j} v_{l'}$前的系数不同,
%由于上对流Maxwell导数中包含非守恒形式的项$(\nabla v) \rho c$和$\rho c (\nabla v)^T$,
从而熵的Hessian矩阵无法对称化方程组\eqref{eq:ECDFsecondisothermal}。

然而,对于一维的情况,
\begin{equation*}
	A(U) = \left( \begin{array}{ccc}
		0 & 1 & 0 \\
		\pi_\rho - v^2 & 2v & -3 \rho c - 2 \\
		\frac{ (\rho c + 2) v}{\rho} & -  \frac{\rho c + 2}{\rho} & v 
	\end{array}\right), \quad
	Q(U) = \left( \begin{array}{c}
		0 \\ 0 \\-\frac{1}{\kappa} \rho c
	\end{array} \right).
\end{equation*}
我们可以找到下面的对称子
\begin{eqnarray}\label{31}
A_0(U) = \frac{1}{\rho} \left( \begin{array}{ccc}
	 \pi_\rho  +v^2 & -v & 0 \\ [2mm]
	-v & 1 & 0 \\[2mm]
	0 & 0 & \frac{3\rho c+2}{\rho c+2}\rho  \end{array} \right),
\end{eqnarray}
使得$A_0(U)$和
\begin{eqnarray*}
A_0(U) A(U) = \frac{1}{\rho} \left( \begin{array}{ccc}
		-(\pi_{\rho}-v^2)v & {p_\rho -v^2} & {(3\rho c + 2)v} \\[2mm]
		{\pi_\rho-v^2} & {v} & -{(3 \rho c+2)} \\[2mm]
		{(3 \rho c+ 2)v} & -{(3 \rho c+2)} & \frac{3\rho c+2}{\rho c + 2}\rho v
		\end{array} \right)
\end{eqnarray*}
都是对称的。并且当
\begin{eqnarray*}
\pi_\rho > 0 ,\quad \rho c > -\frac{2}{3}, \quad   \rho c < -2 
\end{eqnarray*}
时,$A_0(U)$为正定的。在这里,我们取
\begin{equation*}%\label{37}
G := \{(\rho,\rho v, \rho c): \rho>0, \quad \rho c> - \frac{2}{3}\}
% \ \mbox{or} \ > 1
\end{equation*}
作为1维模型\eqref{eq:ECDFsecond1D}的定义域。

\subsection{Yong的稳定性条件}
我们可以得到对于平衡态$U=U_e = (\rho_e,0,0)$,有
\begin{equation}\label{32}
A_0(U_e)Q_U(U_e) + Q_U^T(U_e)A_0(U_e) =-\frac{2}{\kappa}\mbox{diag}(0, 0, 1),
\end{equation}
从而\eqref{eq:ECDFsecond1D}满足Yong的稳定性条件\eqref{eq:yongstability}\cite{yong1992singular,yong1999singular}。

\subsection{Kawashima条件}
对于存在熵函数的双曲平衡律方程组,Yong发展了平衡态附近的整体存在性理论\cite{yong2004entropy,hanouzet2003global}。其整体存在性的证明是基于熵函数的存在性、Yong的稳定性条件和Kawashima条件。熵函数的存在性和稳定性条件已经验证。%对于Kawashima条件,对于非守恒形式含有熵的方程\eqref{eq:ECDFsecond1D}的整体存在性,无法采用一般的Kawashima理论直接得到。
在这里我们将验证方程组\eqref{eq:ECDFsecond1D}满足Kawashima条件并给出补偿矩阵(Compensating Matrix)$K$\cite{kawashima1985systems},然后利用Yong的整体存在性理论\cite{yong2004entropy}得到整体存在性定理的证明。

令
\begin{eqnarray}\label{33}
K=\left( \begin{array}{ccc}
	0 & 1 & 0 \\
	-1 & 0 & -\rho_e \\
	0 & \rho_e & 0
	\end{array} \right),
\end{eqnarray}
和$\bar{L} = \eta\mbox{diag}(0, 0, 1)$,计算可得
\begin{eqnarray*}
K A(U_e) + (K A(U_e))^T + \bar{L} =
	\left( \begin{array}{ccc}
	2 \pi_\rho(\rho_e) & 0 & \rho_e \pi_\rho(\rho_e) -2  \\
	0 & 2 & 0 \\
	\rho_e \pi_\rho(\rho_e) - 2 & 0 & \eta -4 	
	\end{array} \right).
\end{eqnarray*}
在下面的条件下该矩阵为正定的。
\begin{eqnarray*}
\pi_\rho  > 0, \ \eta > 4 \rho, \ 2\pi_\rho(\eta - 4)-(\pi_{\rho}-2)^2 = 2 \pi_\rho \eta  - (2 +  \rho \pi_\rho)^2 >0.
\end{eqnarray*}
第一个条件是物理上对压力的要求。后两个在
$$
\eta \ge 2 \rho_e +\frac{(2 + \rho_e \pi_\rho(\rho_e))^2}{2 \pi_\rho(\rho_e)}
$$
时成立。
从而对于足够大的$\eta$,我们有
\begin{eqnarray}\label{35}
K A(U_e) + (K A(U_e))^T + \bar{L} \ge C_s I,
\end{eqnarray}
其中$C_s$为仅依赖于$\rho_e$的常数。

\subsection{平衡态附近解的整体存在性(定理3.1)}
本小节我们考虑方程组\eqref{eq:ECDFsecond1D}在平衡态$U_e$附近解的整体存在性。我们将采用Yong在\cite{yong2004entropy,kawashima2009decay}中对于双曲平衡律系统的估计得到\eqref{eq:ECDFsecond1D}的整体解存在的证明。虽然这里的方程是非守恒形式的,然而由于熵函数的存在、Yong的稳定性条件的成立以及Kawashima条件的成立,我们仍然可以得到相关的估计及整体存在性定理。本小节的主要结果如下:
% 我们期望方程组\eqref{eq:ECDFsecond1D}在平衡点$U_e$附近存在整体光滑解。这里主要的问题是处理上对流导数中非守恒形式的项。本小节的主要结果为下面的定理。
\begin{theorem} \label{theoremglobal}
令整数$s \ge 2$。假设$U_0=U_0(x) \in H^s(\mathbf{R})$且$\|U_0 -U_e\|_{H^s}$足够小。那么方程组\eqref{eq:ECDFsecond1D}以$U_0$为初值的Cauchy问题存在唯一的整体解$U=U(x,t)$。且对于任意的$T>0$,$U$满足
$$
U-U_e \in C([0,+\infty),H^s(\mathbf{R})),
$$
与
\begin{eqnarray}\label{41}
\|U(T)-U_e\|^2_{H^s} + \int_0^T \left[ \|c(t)\|^2_{H^s} + \|\partial_x U(t)\|^2_{H^{s-1}} \right] dt \le C \|U_0 -U_e\|^2_{H^s}.
\end{eqnarray}
其中$C$为与时间$T$无关的正常数。
\end{theorem}

\begin{proof}
由于方程组\eqref{eq:ECDFsecond1D}是对称双曲组,根据相关理论\cite{majda1984compressible, kato1975cauchy},其存在唯一的局部解$U \in H^s$。为了证明解的整体存在性,我们只需要证明先验估计(a priori estimate)\eqref{41},然后解的整体存在性可以通过标准的分析得到\cite{yong2004entropy}。下面我们分三步证明\eqref{41}成立。

\emph{第一步:}我们首先利用熵函数得到$U-U_e$的$L^2$估计。

定义
\begin{eqnarray*}
E=E(U,U_e) = \eta(U)-\eta(U_e)-\eta_U(U_e)(U-U_e).
\end{eqnarray*}
因为$\eta=\eta(U)$是严格下凸的,所以存在正常数$c_1$ 和 $C_1$,使得对于充分靠近$U_e$的$U$,有
\begin{eqnarray}\label{42}
c_1 |U-U_e|^2 \le E(U,U_e) \le C_1 |U-U_e|^2.
\end{eqnarray}
下面我们计算$E$的演化方程。根据\eqref{eq:ECDFsecond1D}和\eqref{23},
\begin{eqnarray}\label{43}
E_t &= \eta_t - \eta_U(U_e)U_t = -\nabla \cdot \left(J(U) +( \int_{\rho_0}^{\rho_e} \frac{\pi(z)}{z^2} dz + \frac{\pi(\rho_e)}{\rho_e})\rho v \right) - \frac{(\rho c)^2}{\kappa} \nonumber \\
& - S_U(U_e)(\mathcal{Q}(U) - A(U)U_x).
\end{eqnarray}
这里我们用到了
\begin{eqnarray*}
\eta_U(U_e) A(U) U_x &=& ( \int_{\rho_0}^{\rho_e} \frac{\pi(z)}{z^2} dz + \frac{\pi(\rho_e)}{\rho_e}, 0, 0)
\left( \begin{array}{ccc}
		0 & 1 & 0 \\
		\pi_\rho - v^2 & 2v & -3 \rho c - 2 \\
		cv + \frac{2 v}{\rho} & -  \rho c - 2 & v 
	\end{array}\right)
\left( \begin{array}{ccc}
\rho \\ \rho v \\ \rho c \end{array} \right)_x \\
&=& \left( \int_{\rho_0}^{\rho_e} \frac{\pi(z)}{z^2} dz + \frac{\pi(\rho_e)}{\rho_e}\right) (\rho v )_x.
\end{eqnarray*}
在$(x, t)\in(-\infty, + \infty)\times[0,T]$对\eqref{43}积分可以得到
\begin{eqnarray*}
\int_\mathbf{R}E(U(x, T), U_e)dx - \int_\mathbf{R} E(U_0, U_e)dx = - \int_0^T\int_\mathbf{R} \frac{c^2(x, t)}{\kappa} dx dt.
\end{eqnarray*}
结合\eqref{42},我们最后可以得到下面的$L^2$估计
\begin{eqnarray}\label{44}
\|U(\cdot, T)-U_0\|^2_{L^2} + \int_0^T \|c(\cdot, t)\|^2_{L^2}dt \le C\|U_0 - U_e\|^2_{L^2}.
\end{eqnarray}

\emph{第二步:}下面我们对高阶导数的$L^2$范数进行估计。

首先对方程\eqref{eq:ECDFsecond1D}的两端求$l\le s$($l$为整数)阶导数
\begin{eqnarray*}
\partial^l_x U_t + A(U) \partial^l_x U_x = \partial^l_x Q(U) + [A(U),\partial^l_x]U_x,
\end{eqnarray*}
其中$[X,Y]=XY-YX$为交换子。由于$A_0(U)$和$A_0(U)A(U)$对称,对上面的式子与$A_0(U)\partial^l_x U$取$L^2$内积得到
\begin{multline}\label{45}
  (A_0(U)\partial^l_x U,\partial^l_x U)_t + \int (\partial^l_x U^T A_0(U)A(U)\partial^l_x U)_x dx \\
  = ((\partial_t A_0(U)+\partial_x(A_0(U)A(U)))\partial^l_x U,\partial^l_x U) \\
  + 2(A_0(U)[A(U),\partial^l_x]U_x,\partial^l_x U) + 2(A_0(U) \partial^l_x Q(U), \partial^l_x U).
\end{multline}
对任意的时间$t$,$U(\cdot, t)\in H^s(\mathbf{R})$,所以上式左端第二项为0。右端三项估计如下:
%It follows from the postive definite property of $A_0(U)$ that

对第一项,有
\begin{eqnarray}\label{HsEST:1}
  && ((\partial_t A_0(U) + \partial_x (A_0(U)A(U)))\partial^l_x U, \partial^l_x U) \nonumber\\
  &\le&|\partial_t A_0(U) + \partial_x (A_0(U)A(U))|_{L^\infty}\|\partial^l_x U\|_{L^2}^2 \nonumber\\
  &\le& C (|\partial_t U|_{L^\infty}+|\partial_x U|_{L^{\infty}})\|\partial^l_x U\|_{L^2}^2 \nonumber \\
  &\le& C (|\partial_x U|_{L^\infty}+|Q(U)|_{L^\infty})\|\partial^l_x U\|_{L^2}^2\nonumber\\
  &\le& C (|\partial_x U|_{L^\infty} + |c|_{L^\infty}) \|\partial^l_x U\|_{L^2}^2\nonumber \\
  &\le& C \|U-U_e\|_{H^s} \|\partial^l_x U\|_{L^2}^2. %\nonumber
\end{eqnarray}
第二项可以估计如下
\begin{eqnarray*}
  2(A_0(U) [A,\partial^l_x] U_x, \partial^l_x U) \le C |A_0(U)|_{L^\infty} \|[A(U),\partial^l_x] U_x\|_{L^2} \|\partial^l_x U\|_{L^2}.
\end{eqnarray*}
对于含交换子的项我们可以采用Sobolev空间中的演算不等式\cite{majda1984compressible}得到
\begin{eqnarray*}
  \|[A(U),\partial^l_x] U_x\|_{L^2} &\le& C (\|\partial_x^s A(U)\|_{L^2}|U_x|_{L^\infty} + |\partial_x A(U)|_{L^{\infty}} \|\partial_x^{s-1} U_x\|_{L^2})  \\
  &\le& C|U_x|_{L^\infty} \|\partial_x^s U\|_{L^2}.
\end{eqnarray*}
于是我们得到
\begin{eqnarray}\label{HsEST:2}
  2(A_0(U) [A,\partial_x^l] U_x,\partial^l_x U) \le C \|U-U_e\|_{H^s}\|\partial_x U\|_{H^{s-1}}^2.
\end{eqnarray}
对于第三项,回忆$Q(U)$和$A_0(U)$的表达式,我们得到
\begin{eqnarray}\label{HsEST:3}
&& 2(A_0(U)\partial^l_x Q(U),\partial^l_x U) \nonumber \\
&=& 2( A_0(U_e)\partial^l_x Q(U),\partial^l_x U) + 2((A_0(U)-A_0(U_e)) \partial^l_x Q(U),\partial^l_x U) \nonumber \\
&\le&  -2 \|\partial^l_x c\|_{L^2}^2 + C|U-U_e|_{L^\infty} \|\partial^l_x U\|_{L^2}^2 \nonumber \\
&\le& -2 \|\partial^l_x c\|_{L^2}^2 + C\|U-U_e\|_{H^s} \|\partial^l_x U\|_{L^2}^2.
\end{eqnarray}

由估计\eqref{HsEST:1}、\eqref{HsEST:2}和\eqref{HsEST:3},我们在$[0,T]$上对\eqref{45}积分得到
\begin{eqnarray}\label{49}
  \|\partial^l_x U(T)\|_{L^2}^2&  + & \int_0^T \|\partial^l_x c(t)\|^2_{L^2}dt \le C\|\partial^l_x U(0)\|_{L^2}^2  \nonumber\\[3mm]
 & +& C\sup_{t \in [0,T]}\|U(t)-U_e\|_{H^s} \int_0^T \|\partial_x U(t)\|^2_{H^{s-1}}dt .
\end{eqnarray}
这里我们用到了
\begin{eqnarray*}
   C^{-1} \|\partial^l_x U\|_{L^2}^2 \le  (A_0(U) \partial^l_x U,\partial^l_x U) \le C \|\partial^l_x U\|_{L^2}^2.
\end{eqnarray*}
将\eqref{44}和\eqref{49}($1 \le l \le s$)相加可以得到
  \begin{eqnarray}\label{210}
    && \|U(T)-U_e\|^2_{H^{s}}  +  \int_0^T \|c(t)\|^2_{H^s} dt \nonumber \\
    & \le & C \|U_0-U_e\|_{H^{s}}^2 + C \sup_{t \in [0,T]} \|U(t) - U_e\|_{H^s} \int_0^T \|\partial_x U\|_{H^{s-1}}^2dt.
%  \\  \le C \|\partial_x U_0\|^2_{H^{s-1}} + C M(t)D_0(t)^2
\end{eqnarray}

\emph{第三步:}下面我们利用Kawashima条件来得到上式\eqref{210}中最后一项的估计。

首先将\eqref{eq:ECDFsecond1D}写作
\begin{eqnarray*}
  U_t + A(U_e) U_x  = (A(U_e) -A(U))U_x + Q(U).
\end{eqnarray*}
对两边求$l$阶导数并用$K$乘以方程两边得到
\begin{eqnarray*}
  K \partial^l_x U_t + K A(U_e) \partial^l_x U_x  = K \partial^l_x ((A(U_e)-A(U))U_x) + K\partial^l_x Q(U).
\end{eqnarray*}
然后与$\partial^l_x U_x$ 取$L^2$内积得到
\begin{eqnarray}\label{211}
  (K \partial^l_x U_t, \partial^l_x U_x) + (KA(U_e) \partial^l_x U_x, \partial^l_x U_x) \nonumber \\ = (K \partial^l_x((A(U_e)-A(U))U_x),\partial^l_x U_x) + (\partial^l_x( K Q(U) ),\partial^l_x U_x).
\end{eqnarray}
由$K$反对称,左端第一项可以处理如下。
\begin{eqnarray}\label{212}
  (K\partial^l_x U_t, \partial^l_x U_x) &=& \frac{1}{2} \int_\mathbf{R} \left[(\partial^l_x U^T_x K \partial^l_x U )_t -  (\partial^l_x U_t^T K\partial^l_x U  )_x \right]dx  \nonumber \\
  &=& \frac{1}{2}(K \partial^l_x U,\partial^l_x U_x)_t .
\end{eqnarray}
对第二项,利用\eqref{35}得到
\begin{eqnarray}\label{213}
  2(KA(U_e) \partial^l_x U_x,\partial^l_x U_x) &=& ( (KA(U_e)+ (K A(U_e))^T +\bar{L}) \partial^l_x U_x, \partial^l_x U_x) - (\bar{L}\partial^l_x U_x, \partial^l_x U_x) \nonumber\\
  &\ge& C_s \|\partial^l_x U_x \|^2_{L^2} -\eta \|\partial^l_x c\|^2_{L^2},
\end{eqnarray}

\eqref{211}的右端可以估计如下
\begin{eqnarray}\label{214}
  (K \partial^l_x Q(U),\partial^l_x U_x) \le \epsilon \|\partial^l_x U_x\|_{L^2}^2 + \frac{C}{\epsilon} \|\partial^l_x c\|_{L^2}^2,
\end{eqnarray}
以及
\begin{eqnarray}\label{215}
  && (K\partial^l_x ((A(U_e)-A(U))U_x),\partial^l_x U_x) \le \epsilon\|\partial^l_x U_x\|_{L^2}^2 + \frac{C}{\epsilon} \|\partial^l_x( (A(U_e)-A(U))U_x)\|_{L^2}^2 \nonumber \\
  &\le& \epsilon \|\partial^l_x U_x \|_{L^2}^2 + C(\epsilon)( |A(U_e)-A(U )|_{L^\infty}^2 \|\partial^l_x U_x\|^2_{L^2}+\|\partial^l_x A(U)\|_{L^2}^2|U_x|_{L^\infty}^2) \nonumber \\
  &\le& \epsilon \|\partial^l_x U_x\|_{L^2}^2 + C(\epsilon) (|U-U_e|_{L^\infty}^2 \|\partial^l_x U_x\|_{L^2}^2 + |U_x|_{L^\infty}^2 \|\partial^l_x U\|_{L^2}^2).
\end{eqnarray}
这里我们也用到了Sobolev演算不等式。

利用估计\eqref{212}—\eqref{215},我们从\eqref{211}可以得到
\begin{multline*}
  C_s \|\partial^l_x U_x\|^2_{L^2} \le (\eta+\frac{C}{\epsilon}) \|\partial^l_x c\|_{L^2}^2-\frac{1}{2} (K \partial^l_x U,\partial^l_x U_x)_t  + 2\epsilon \|\partial^l_x U_x\|^2_{L^2} \\
  +C(\epsilon) |U-U_e|_{L^\infty}^2 \|\partial^l_x U_x\|_{L^2}^2 + C(\epsilon) |U_x|_{L^\infty}^2 \|\partial^l_x U\|_{L^2}^2.
\end{multline*}
选取足够小的$\epsilon$(例如$\epsilon = C_s/4$)。在$[0,T]$上对上述不等式积分,我们得到对于$1\le l \le s-1$,估计
\begin{multline*}
  \int_0^T \|\partial^l_x U_x(t)\|^2_{L^2} dt \le C \int_0^T \|\partial^l_x c(t)\|^2_{L^2} dt
  +C\|\partial^l_x U(T)\|_{H^1}^2 + C\|\partial^l_x U_0\|_{H^1}^2 \\
  + C \sup_{t\in [0,T]} (|U(t)-U_e|_{L^\infty}^2 + |U_x(t)|_{L^\infty}^2) \int_0^T \|\partial^l_x U(t)\|_{L^2}^2 dt
\end{multline*}
成立。对$1 \le l \le s-1$求和可以得到
\begin{multline}\label{216}
  \int_0^T \|\partial_x U(t)\|^2_{H^{s-1}} dt \le C \int_0^T \|c(t)\|^2_{H^s} dt + C\|U_0-U_e\|^2_{H^s} + C\|U(T)-U_e\|^2_{H^s} \\+C \sup_{0 \le t \le T}(|U(t)-U_e|_{L^\infty}^2 + |U_x(t)|_{L^\infty}^2) \int_0^T \|\partial_x U(t)\|_{H^{s-1}}^2 dt \\
  \le C \int_0^T \|c(t)\|_{H^s}^2 dt + C \|U_0-U_e\|_{H^s}^2 + C\|U(T) -U_e\|_{H^s}^2  \\ + C \sup_{t \in [0,T] }\|U(t)-U_e\|_{H^s}^2 \int_0^T \|\partial_x U(t)\|_{H^{s-1}}^2 dt.
\end{multline}

令$\alpha>0$,用$\alpha$乘式子\eqref{216} 并将结果与\eqref{210}相加可以得到
\begin{eqnarray*}
  \|U(T)-U_e\|_{H^s}^2 + \int_0^T \|c(t)\|_{H^s}^2 dt + \alpha \int_0^T \|\partial_x U(t)\|_{H^{s-1}}^2 dt \\
  \le C \alpha \int_0^T \|c(t)\|_{H^s}^2 dt +C(1+\alpha)\|U_0 -U_e\|_{H^s}^2 + C \alpha \|U(T)-U_e\|_{H^s}^2  \\ + C(\sup_{t \in [0,T]} \|U(t)-U_e\|_{H^s} + \alpha \sup_{t \in [0,T]} \|U(t)-U_e\|_{H^s}^2)  \int_0^T \|\partial_x U\|_{H^{s-1}}^2 dt.
\end{eqnarray*}
$\alpha$足够小时,由上述不等式可以导出
\begin{eqnarray*}
  \|U(T)-U_e\|_{H^s}^2 + \int_0^T \|c(t)\|_{H^s}^2 dt + \int_0^T \|\partial_x U(t)\|_{H^{s-1}}^2 dt \\
  \le C \|U_0-U_e\|_{H^s}^2 + C \sup_{t \in [0,T]} \|U(t)-U_e\|_{H^s} \int_0^T \|\partial_x U(t)\|_{H^{s-1}}^2 dt \\
  +C\alpha  \sup_{t \in [0,T]} \|U(t)-U_e\|_{H^s}^2 \int_0^T \|\partial_x U(t)\|_{H^{s-1}}^2 dt.
\end{eqnarray*}
由此定理\ref{theoremglobal}中的先验估计对于足够小的$\|U_0 -U_e\|_{H^s}$成立,这样我们证明了定理\ref{theoremglobal}。
\end{proof}

\subsection{与一维Navier-Stokes方程组的兼容性(定理3.2)}

当$\kappa$足够小时,我们可以认为粘弹性效应足够小,这时我们有理由相信仅考虑粘性效应足以对粘弹性流体做很好的近似描述。在本小节中,我们将说明一维的等温可压上对流Maxwell模型\eqref{eq:ECDFsecond1D}与经典的可压Navier-Stokes方程在$\kappa$很小时是兼容的,即$\kappa$很小时Navier-Stokes方程可以很好地近似等温可压上对流Maxwell模型。

首先采用Maxwell迭代可以得到方程组\eqref{eq:ECDFsecond1D}形式上的近似Navier-Stokes方程组。将\eqref{eq:ECDFsecond1D}写为
\begin{eqnarray*}
 \rho c= -\kappa(( \rho c)_t + v \partial_x (\rho c) -  \rho c \partial_x v + 2 \partial_x v).
\end{eqnarray*}
迭代一次得到
\begin{eqnarray*}
  \rho c = 2 \kappa \partial_x v + O(\kappa^2).
\end{eqnarray*}
将其带入\eqref{eq:ECDFsecond1D}中的动量方程,并忽略高阶项,可以得到下面的一维Navier-Stokes方程组
\begin{align}\label{51}
  \partial_t \rho + \partial_x (\rho v ) = 0, \nonumber \\
  \partial_t (\rho v) + \partial_x( \rho v^2 + \pi) = 4 \kappa \partial^2_x v.
\end{align}
其中$4\kappa$为粘性系数。这一形式近似的合理性由下面的定理保证:
% 由\eqref{32},方程组\eqref{eq:ECDFsecond1D}满足文献\cite{yong1999singular}中的条件。从而可以得知其一阶摄动展开方程组的解可以很好地近似原方程组的解。这样我们只需证明Navier-Stokes方程组可以很好地近似其摄动展开方程组。
% 我们的分析基于\cite{yang2015validity},但是我们需要处理\eqref{eq:ECDFsecond1D}中的非守恒项。本小节的主要结果如下。
\begin{theorem}\label{theoremCE}
令整数$s \ge 2$, ${\bar u} =({\bar \rho}(x),\bar{\rho}(x){\bar v}(x))$满足
  \begin{eqnarray*}
    \bar{u}\in H^{s+2},\ \inf_{x} \bar{\rho}(x)>0.
 \end{eqnarray*}
那么存在与松弛参数$\kappa$无关的时间$T_*>0$,使得方程组\eqref{eq:ECDFsecond1D}以$(\bar{u},0)$为初值的Cauchy问题和Navier-Stokes方程组\eqref{51}以${\bar u}$为初值的Cauchy问题分别有唯一解$(u^\kappa=(\rho^\kappa,\rho^\kappa v^\kappa, c^\kappa)(x,t),u^\kappa_p=(\rho^\kappa_p,\rho^\kappa_p v^\kappa_p)(x,t) \in C([0,T_*], H^s)$,且对于充分小的$\kappa$,它们满足
  \begin{equation}\label{52}
    \sup_{t \in [0, T_*]} \|(u^\kappa-u^\kappa_p)(\cdot,t)\|_{H^s} \le C(T_*) \kappa^2.
  \end{equation}
 其中$C(T_*)>0$不依赖于$\kappa$。
\end{theorem}

% \begin{remark}
% 简单起见,定理\ref{theoremCE}中方程组\eqref{eq:ECDFsecond1D}和\eqref{51}的初值相同。但是对于一般的初值,对\eqref{51}取初值${\tilde u}_p =({\tilde \rho}_p(x, \kappa),\tilde{\rho}_p(x,\kappa){\tilde v}_p(x, \kappa))$,对$\eqref{eq:ECDFsecond1D}$取$({\tilde u} =({\tilde \rho}(x, \kappa),\tilde{\rho}(x,\kappa){\tilde v}(x, \kappa)),\tilde{\rho} \tilde {c}(x, \kappa))$,那么只要初值只要满足
% \begin{eqnarray*}
%     \tilde{u}_p(\cdot,\kappa),\tilde{u}(\cdot,\kappa) \in H^{s+2}, \quad  \inf_{x, \kappa} \tilde{\rho}_p(x,\kappa), \inf_{x, \kappa} \tilde{\rho}(x,\kappa)>0, \quad \inf_{x, \kappa} \tilde{c}(x,\kappa)> -2/3, \nonumber  \\
%     \|\tilde{\rho}(\cdot,\kappa)-\tilde{\rho}_p(\cdot,\kappa)\|_{H^s}, \| \tilde{\rho} \tilde{v}(\cdot,\kappa)-\tilde{\rho}_p\tilde{v}_p(\cdot,\kappa)- \frac{3}{2} \kappa \bar{c}_{0x} \bar{c}_0 +2 \kappa \bar{c}_{0x}\|_{H^s}  = O(\kappa^2)
%  \end{eqnarray*}
%   其中$\bar{c}_0=\tilde{c}(x,0)$。那么定理\ref{theoremCE}仍然成立,只是估计\eqref{52}变为
% \begin{equation*}\label{eq:Result}
%     \sup_{t \in [-\kappa\ln\kappa, T_*]} \|(u^\kappa-
%     %u'_\kappa-
%     u^\kappa_p)(\cdot,t)\|_{H^s} \le C(T_*) \kappa^2,
%   \end{equation*}
%   这是因为对一般的初始值边界层可能存在。只需要对下面的证明做微小改动就可以得到一般情况的证明。
% \end{remark}

为了证明上述定理,我们注意到方程组\eqref{eq:ECDFsecond1D}满足Yong的稳定性条件\cite{yong1992singular,yong1999singular}。从而,Yong的一阶双曲偏微分方程的奇异摄动理论可以用于本方程组。于是我们得到\eqref{eq:ECDFsecond1D}以$(\bar{u},0)$为初值的解$U^\kappa = (\rho^\kappa, \rho^\kappa v^\kappa, c^\kappa)^T$和一阶摄动展开对应的方程组的解$U_\kappa^1=(\rho_\kappa^1,\rho_\kappa^1 v_\kappa^1, c^1_\kappa)^T$在$[0,T_*]$上满足
\begin{eqnarray}\label{53}
  \sup_{t \in [0, T_*]} \|U^\kappa(\cdot, t) - U_\kappa^1(\cdot, t)\|_{H^s} \le K\kappa^2,
\end{eqnarray}
其中$K$为不依赖与$\kappa$的正常数。从而$U^\kappa$在$[0,T_*]$的解亦存在。

$U^\kappa_1$的表达式可由\cite{yong1992singular,yong1999singular}中的方法得到。假设其形式为
\begin{eqnarray*}
U^1_\kappa  = U_0(x,t) + \kappa U_1(x,t) + U'_0(x,t') + \kappa U'_1(x,t'),
\end{eqnarray*}
其中$t'=t/\kappa$。
对于外展开$U_0=(u_0, (\rho c)_0)$和$U_1=(u_1, (\rho c)_1)$满足
\begin{eqnarray}
 (\rho c)_0 = 0, \quad
  \partial_t u_0+ \partial_x f(u_0,0)=0 , \nonumber \\
  (\rho c)_1 = 2\partial_x v_0, \quad
  \partial_t u_1 + \partial_x (f_u(u_0,0) u_1 + f_{\rho c}(u_0,0)(\rho c)_1)  = 0 \label{54},
	\end{eqnarray}
其中
\begin{eqnarray*}
  f(\rho,\rho v,\rho c) = \left( \begin{array}{cc} \rho v \\ \rho v^2 + \pi - \frac{3}{2} (\rho c)^2 - 2\rho c \end{array} \right);
\end{eqnarray*}
由最后一个方程可得
\begin{equation*}
	  \partial_t u_1 + \partial_x (f_u(u_0,0) u_1   = \left( \begin{matrix}
	  		0 \\ 4 \partial_x^2 v_0
	  \end{matrix} \right).
\end{equation*}
内展开$U_0'=(u_0', (\rho c)_0')$和$U_1'=(u_1', (\rho c)_1')$满足
\begin{eqnarray*}
 \partial_{t'} u'_0 = 0,  \quad \partial_{t'} (\rho c)'_0 = -(\rho c)'_0 , \\
  \partial_{t'} u'_1 = \partial_x(f(\bar u_0,0) - f(\bar u_0,(\rho c)'_0)), \\
  \partial_{t'} (\rho c)'_1 = - (\rho c)'_1 - \bar v_0 \partial_x (\rho c)_0' -2 (\rho c)_0'\partial_x \bar v_0.
\end{eqnarray*}
其初值由下面的匹配条件决定。
\begin{eqnarray*}
  \lim_{t' \to \infty} U'_0(x,t') = 0 , \quad \lim_{t' \to \infty} U'_1(x,t') = 0 .
\end{eqnarray*}
从而有
\begin{eqnarray*}
 u'_0 =0, \quad u_0(x,0) = \bar{u}(x), \quad  (\rho c)'_0 = (\rho c)'_0(x,0)e^{-t/\kappa}=0 , \\[4mm]
 u'_1(x,0) = -\partial_x\int_0^{+\infty} (f(\bar u_0,0) - f(\bar u_0,(\rho c)'_0)) dt'=0\\
% = \partial_x\int_0^{+\infty} \left( \begin{array}{cc} 0 \\  \frac{3}{2} (c_0')^2 - 2c'_0 \end{array} \right) dt' =
 %\left( \begin{array}{cc} 0 \\  \frac{3}{2} \bar c_{0x}\bar c_0 - 2\bar c_{0x} \end{array} \right), \\[4mm]
 u_1(x,0) =  - u'_1(x,0)=0, \quad (\rho c)'_1(x,0) = - 2\partial_x\bar v_0(x).
\end{eqnarray*}
这与式子\eqref{53}给出下面的估计
\begin{eqnarray*}
  \|u^\kappa(\cdot, t) - u_0 (\cdot, t) - \kappa u_1(\cdot, t) \|_{H^s} \le K \kappa^2
\end{eqnarray*}
对$t \in [0,T_*]$成立。从而为了证明定理\ref{theoremCE},只需证明下面的结果:
\begin{lemma}\label{lemmaCE}
在定理\ref{theoremCE}的条件下,方程组\eqref{51}以$\bar{u}$为初值的Cauchy问题有唯一解$u_p^\kappa \in C([0,T_*],H^s)$,且对充分小的$\kappa$其满足
\begin{eqnarray}\label{eq:wcediff}
  \sup_{t \in [0,T_*]} \| u^\kappa_p(\cdot,t) - u_0 (\cdot, t) - \kappa u_1(\cdot, t)\|_{H^s} \le C \kappa^2.
\end{eqnarray}
这里$C=C(T_*)$与$\kappa$无关。
\end{lemma}

\begin{proof}
由Navier-Stokes方程的存在性定理\cite{kawashima1984systems},方程组\eqref{51}存在唯一解$u^\kappa_p=u^\kappa_p(x,t)$,且其满足$u^\kappa_p \in C([0,T],H^s)$。对集合$G_1\subset\subset G= \{(\rho, \rho v, \rho c): \rho>0,\ \rho c > -2/3\}$,我们定义其解的最大存在时间为
\begin{eqnarray*}
  T^\kappa :=\sup \{T>0 : u_p^\kappa  \in C([0,T],H^s), u^\kappa_p(x,t) \in G_1 \}.
\end{eqnarray*}
由\cite{yong2001basic}中结果可知只要下面的估计成立就有$\kappa$趋于$0$时
$T^\kappa > T_*$
成立:
\begin{eqnarray*}
\sup_{x,t} |u_p^\kappa(x,t) - u_0 (x, t) - \kappa u_1(x,t)|=o(1), \\
\sup_t \| u_p^\kappa(\cdot ,t) - u_0 (\cdot, t) - \kappa u_1(\cdot, t) \|_{H^s} = O(1),
\end{eqnarray*}
对$t \in [0,\min\{T_*,T^\kappa\})$成立。这样我们只需证明误差估计\eqref{eq:wcediff}(其中$T_*$由$\min\{T_*,T^\kappa\}$代替)。

方便起见,我们估计$w=(\rho,v)$从而得到$u=(\rho,\rho v)$的估计。这里我们将$u^\kappa_p$记做$u^\kappa$。由Sobolev演算不等式\cite{majda1984compressible},$w$的范数与$u$的范数等价。

令
\begin{eqnarray*}
  a(w) = \left( \begin{array}{cc} v & \rho \\ \frac{\pi_\rho}{\rho} & v \end{array} \right) \ \ \mbox{与} \ \
  a_0(w) = \left( \begin{array}{cc} \frac{\pi_\rho}{\rho^2} & 0 \\ 0 & 1 \end{array} \right).
\end{eqnarray*}
这里$a_0(w)$是对称正定的且$a_0(w)a(w)=a^T(w)a_0(w)$。
方程组\eqref{51}可以写作
\begin{eqnarray}\label{57}
  \partial_t w^\kappa + a(w^\kappa) \partial_x w^\kappa = \left( \begin{array}{cc} 0 \\ \frac{4 \kappa}{\rho^\kappa} \partial^2_x v^\kappa \end{array} \right) .
\end{eqnarray}

另外,由\eqref{54}得到$u_\kappa \equiv u_0 +\kappa u_1$满足下面的方程
\begin{eqnarray*}
  &&\partial_t u_\kappa + f(u_\kappa,0)_x \nonumber \\
  &=&  (f(u_\kappa,0) - \kappa f_u(u_0,0)u_1 - f(u_0,0))_x + \left( \begin{array}{c} 0 \\ 4 \kappa \partial^2_x v_0 \end{array} \right) \nonumber \\
	&=& \left( \begin{array}{c} 0 \\ 4 \kappa \partial^2_x v_\kappa \end{array} \right)   + R,
\end{eqnarray*}
其中
$$
R=(f(u_\kappa,0) - \kappa f_u(u_0,0)u_1 - f(u_0,0))_x -\left( \begin{array}{c} 0 \\ 4 \kappa^2 \partial^2_x v_1 \end{array} \right).
$$
易知$w_\kappa$满足
\begin{eqnarray}\label{580}
  \partial_t w_\kappa + a(w_\kappa) \partial_x w_\kappa = \left( \begin{array}{cc} 0 \\ \frac{4 \kappa}{\rho_\kappa} \partial^2_x v_\kappa \end{array} \right) + \hat R,
\end{eqnarray}
其中
\begin{eqnarray*}
\hat{R} = \left( \begin{array}{cc} 1 & 0 \\ 0 & 1/\rho_\kappa \end{array} \right)R .
\end{eqnarray*}
由Sobolev演算不等式\cite{majda1984compressible}与$u_\kappa \in H^{s+2}$得到
\begin{eqnarray*}
  \|\partial^2_x v_1 \|_{H^s} &\le&  \|v_1\|_{H^{s+2}},
\end{eqnarray*}
与
\begin{eqnarray*}
  \|(f(u_\kappa,0) &-& \kappa f_u(u_0,0)u_1 - f(u_0,0))_x\|_{H^s} \\
  &=& \kappa^2 \| u_1^Tf_{uu}(u_0+\kappa \theta_2 \theta_1  u_1,0) \theta_1 u_1\|_{H^{s+1}} \\
  &\le&  C \kappa^2 \|u_1\|_{H^{s+1}}^2 \|u_0+\kappa \theta_2 \theta_1  u_1\|_{H^{s+1}},
\end{eqnarray*}
其中$\theta_1,\theta_2 \in (0,1)$,从而$\|\hat{R}\|_{H^s} =O(\|R\|_{H^s})= O(\kappa^2)$。


取$E = w^\kappa - w_\kappa$,由\eqref{57}和\eqref{580}得到
\begin{eqnarray*}
  \partial_t E + a(w^\kappa) \partial_x E = (a(w_\kappa) - a(w^\kappa)) \partial_x w_\kappa + \left( \begin{array}{cc} 0 \\ \frac{4\kappa}{\rho^\kappa} \partial^2_x v^\kappa - \frac{4\kappa}{\rho_\kappa} \partial^2_x v_\kappa \end{array} \right) - \hat{R}.
\end{eqnarray*}
这可以重写为
\begin{eqnarray*}
   \partial_t E + a(w^\kappa) \partial_x E = (a(w_\kappa) - a(w^\kappa)) \partial_x w_\kappa + \left( \begin{array}{cc} 0 \\ \frac{4\kappa}{\rho^\kappa} \partial^2_x (v^\kappa - v_\kappa) \end{array} \right) \\
   + \left( \begin{array}{cc} 0 \\ 4\kappa(\frac{1}{\rho^\kappa} -\frac{1}{ \rho_\kappa})\partial^2_x v_\kappa \end{array} \right)- \hat{R}.
\end{eqnarray*}
两边求$l$阶导数($0 \le l \le s$)得到
\begin{eqnarray*}
  \partial_t \partial^l_x E + a(w^\kappa) \partial^l_x E_x  = [a(w^\kappa),\partial^l_x] E_x  + \left( \begin{array}{cc} 0 \\4\kappa \partial^l_x(\frac{1}{\rho^\kappa} \partial^2_x (v^\kappa-v_\kappa)) \end{array} \right) \\  + \left( \begin{array}{cc} 0 \\ 4 \kappa \partial^l_x ((\frac{1}{\rho^\kappa} - \frac{1}{\rho_\kappa}) \partial^2_x v_\kappa) \end{array} \right) -\partial^l_x \hat{R} + \partial^l_x ((a(w_\kappa)-a(w^\kappa))\partial_x w_\kappa).
\end{eqnarray*}
将上述等式乘以$\partial^l_x E^Ta_0(w^\kappa)$并对空间$x$积分得到
\begin{eqnarray}\label{58}
&& (a_0(w^\kappa) \partial^l_x E,\partial^l_x E)_t + \int ( \partial^l_x E^T a_0(w^\kappa) a(w^\kappa) \partial^l_x E )_x dx \nonumber \\
&=&  ((a_0(w^\kappa)_t + (a_0(w^\kappa)a(w^\kappa))_x) \partial^l_x E,\partial^l_x E)
+ 2(a_0(w^\kappa)[a(w^\kappa),\partial^l_x] E_x,\partial^l_x E)\nonumber \\
&& + 2 (4\kappa \partial^l_x (\frac{1}{\rho^\kappa} \partial^2_x (v^\kappa-v_\kappa)),\partial^l_x (v^\kappa-v_\kappa)) + 2(4\kappa \partial^l_x ( (\frac{1}{\rho^\kappa}-\frac{1}{\rho_\kappa})\partial^2 v_\kappa),\partial^l_x (v^\kappa-v_\kappa)) \nonumber\\
&& - 2(a_0(w^\kappa)\partial^l_x\hat{R},\partial^l_x E)+ 2(a_0(w^\kappa)\partial^l_x((a(w_\kappa)-a(w^\kappa))\partial_x w_\kappa),\partial^l_x E) \nonumber \\
& \equiv& I_1 +I_2 +I_3 +I_4+I_5 + I_6.
\end{eqnarray}

下面我们估计上述方程中的项。首先我们注意到左端第二项为$0$。并且$w^\kappa$在紧集$G_1$中取值,$\|w_\kappa(\cdot, t)\|_{H^{s+2}}$对$\kappa$一致有界。 利用Sobolev演算不等式\cite{majda1984compressible},$I_2, I_4, I_5$和$I_6$可以估计如:
\begin{eqnarray*}
  I_2 &\le& 2 |a_0(w^\kappa)|_{L^\infty} \|[a(w^\kappa),\partial^l_x]E_x\|_{L^2} \|\partial^l_x E\|_{L^2} \nonumber \\
  &\le& C \|\partial^l_x E\|_{L^2} (|\partial_x a(w^\kappa)|_{L^\infty}\|\partial^{s-1}_x E_x\|_{L^2} + |E_x|_{L^\infty} \|\partial^s_x a(w^\kappa)\|_{L^2})  \nonumber \\
  &\le& C \|w^\kappa\|_{H^s} \|E\|_{H^s}^2 \le C (1+\|E\|_{H^s}) \|E\|_{H^s}^2, \nonumber \\
  I_4 &\le&  C \kappa \|\partial^l_x (v^\kappa-v_\kappa)\|_{L^2} \|\frac{1}{\rho^\kappa}-\frac{1}{\rho_\kappa}\|_{H^s} \|\partial^2_x v_\kappa\|_{H^s}
  \le C \kappa \|E\|_{H^s}^2, \nonumber\\
  I_5 &\le& C \|\partial^l_x \hat{R}\|_{L^2} \|\partial^l_x E\|_{L^2} \le C\kappa^2\|\partial^l_x E\|_{L^2} \le C \kappa^4 + C \|\partial^l_x E\|_{L^2}^2, \nonumber \\
  I_6 &\le& C\|\partial^l_x ((a(w_\kappa)-a(w^\kappa))\partial_x w_\kappa)\|_{L^2} \|\partial^l_x E\|_{L^2}\nonumber\\
      &\le&  C \|w_\kappa-w^\kappa\|_{H^s} \|w_\kappa\|_{H^{s+1}} \|\partial^l_x E\|_{L^2} \nonumber\\
      &\le&  C \|E\|_{H^s}^2. \label{515}
\end{eqnarray*}

对第一项,利用\eqref{51}得到
\begin{eqnarray*}
  I_1 &\le&  |a_0(w^\kappa)_t + (a_0(w^\kappa) a(w^\kappa))_x|_{L^\infty} \|\partial^l_x E\|_{L^2}^2 \nonumber\\
      &\le& C (|w^\kappa_t|_{L^\infty} + |w^\kappa_x|_{L^\infty}) \|\partial^l_x E\|_{L^2}^2 \nonumber\\
      &\le& C (|w^\kappa_x|_{L^\infty} + \kappa |\partial^2_x v^\kappa|_{L^\infty}) \|\partial^l_x E\|_{L^2}^2 \nonumber\\
      &\le& C (\|w^\kappa\|_{H^s} + \kappa \|\partial_x(v^\kappa-v_\kappa)\|_{H^s} + \kappa \|\partial_x v_\kappa\|_{H^s} )\|\partial^l_x E\|_{L^2}^2 \nonumber\\
      &\le& C (\|w^\kappa-w_\kappa\|_{H^s}+\|w_\kappa\|_{H^s}+\kappa\|v_\kappa\|_{H^{s+1}} + \kappa \|\partial_x(v^\kappa-v_\kappa)\|_{H^s}) \|\partial^l_x E\|_{L^2}^2 \nonumber\\
      &\le& C (1+\|E\|_{H^s} + \kappa \|\partial_x (v^\kappa-v_\kappa)\|_{H^s}) \|E\|_{H^s}^2. \label{516}
\end{eqnarray*}
第三项$I_3$的处理需要采用分部积分:
\begin{eqnarray*}
  I_3 &=& 8 \kappa ([\partial^l_x,\frac{1}{\rho^\kappa}] \partial^2_x(v^\kappa-v_\kappa) + \frac{1}{\rho^\kappa} \partial^{l+2}_x(v^\kappa-v_\kappa),\partial^l_x (v^\kappa-v_\kappa) )\\
  &=& 8 \kappa ([\partial^l_x,\frac{1}{\rho^\kappa}] \partial^2_x(v^\kappa-v_\kappa),\partial^l_x (v^\kappa-v_\kappa))  -8 \kappa(\partial_x (\frac{1}{\rho^\kappa}) \partial^l_x(v^\kappa-v_\kappa),\partial^{l+1}_x (v^\kappa-v_\kappa)) \\ && -8\kappa ( \frac{1}{\rho^\kappa} \partial^{l+1}_x(v^\kappa-v_\kappa), \partial^{l+1}_x (v^\kappa-v_\kappa)) \\
  &\le& 8\kappa \|[\partial^l_x,\frac{1}{\rho^\kappa}]\partial^2_x(v^\kappa-v_\kappa)\|_{L^2} \|\partial^l_x (v^\kappa-v_\kappa)\|_{L^2} \\ & &+8\kappa |\partial_x \frac{1}{\rho^\kappa}|_{L^\infty} \|\partial^l_x (v^\kappa-v_\kappa)\|_{L^2} \|\partial^{l+1}_x(v^\kappa-v_\kappa)\|_{L^2}  -C_0 \kappa \|\partial^{l+1}_x (v^\kappa-v_\kappa)\|_{L^2}^2\\
  &\le& C\kappa \|\partial^l_x (v^\kappa-v_\kappa)\|_{L^2} (|\partial_x \frac{1}{\rho^\kappa}|_{L^\infty} \|\partial_x^{s-1} \partial^2_x(v^\kappa-v_\kappa)\|_{L^2} + |\partial^2_x (v^\kappa-v_\kappa)|_{L^\infty} \|\partial^s_x \frac{1}{\rho^\kappa}\|_{L^2})  \\ & & + C\kappa \|\rho^\kappa\|_{H^s} \|v^\kappa-v_\kappa\|_{H^s}\|\partial_x (v^\kappa-v_\kappa)\|_{H^s} -C_0 \kappa\|\partial^{l+1}_x (v^\kappa-v_\kappa)\|_{L^2}^2\\
  &\le& C\kappa (1+ \|E\|_{H^s})\|E\|_{H^s} \|\partial_x (v^\kappa-v_\kappa)\|_{H^s}-C_0 \kappa \|\partial^{l+1}_x (v^\kappa-v_\kappa)\|_{L^2}^2 ,
\end{eqnarray*}
其中$C_0 =\frac{ 8}{{\max_{x,t,\kappa} \{\rho^\kappa(x,t) \}}}$。

结合\eqref{58}和上面$I_k$的估计,我们得到
\begin{eqnarray*}
 && \sum_{l=0}^s  (a_0(w^\kappa) \partial^l_x E,\partial^l_x E)_t + C_0 \kappa \|\partial_x (v^\kappa-v_\kappa)\|_{H^s}^2 \\
 &\le& C (1+\|E\|_{H^s})\|E\|_{H^s}^2 + C \kappa(1+\|E\|_{H^s})\|E\|_{H^s} \|\partial_x(v^\kappa-v_\kappa)\|_{H^s} + C \kappa^4  \\
 &\le& C (1+\|E\|_{H^s})\|E\|_{H^s}^2 + \frac{C_0\kappa}{2} \|\partial_x(v^\kappa-v_\kappa)\|_{H^s}^2 + C \kappa (1+\|E\|_{H^s})^2 \|E\|_{H^s}^2 + C \kappa^4.
\end{eqnarray*}
从而
\begin{eqnarray*}
  \sum_{l=0}^s  (a_0(w^\kappa) \partial^l_x E,\partial^l_x E)_t + \frac{C_0}{2} \kappa \|\partial_x (v^\kappa-v_\kappa)\|_{H^s}^2 \le   C (1+\|E\|_{H^s}^2)\|E\|_{H^s}^2 + C \kappa^4.
\end{eqnarray*}
回忆$a_0(w^\kappa)$是正定的,我们将上述不等式在$t \in [0,T],\ T \le \min \{T_*,T^\kappa\}$上积分得到
\begin{eqnarray} \label{59}
 && \|E(T)\|_{H^s}^2 + \kappa \int_0^T \|\partial_x(v^\kappa(t)-v_\kappa(t))\|_{H^s}^2 dt \nonumber \\
& \le &  C\int_0^T (1+\|E(t)\|_{H^s}^2)\|E(t)\|_{H^s}^2 dt +CT_* \kappa^4.
\end{eqnarray}
这里我们用到了$E(x, 0)=0$。

令
\begin{eqnarray*}
  \phi(T) =  \int_0^T (1+\|E(t)\|_{H^s}^2)\|E(t)\|_{H^s}^2 dt + T_* \kappa^4.
\end{eqnarray*}
由\eqref{59}得到
\begin{eqnarray*}
  \phi' \le C  \phi(1+C\phi)
\end{eqnarray*}
或
\begin{eqnarray*}
  -(\frac{1}{\phi})_t \le -C (-\frac{1}{\phi})+C^2.
\end{eqnarray*}
采用Gronwall不等式我们得到
\begin{eqnarray*}
  \phi(T) \le \frac{e^{CT} \phi_0}{1-C e^{CT}  \phi_0}\le\frac{e^{CT_*} \phi_0}{1-C e^{CT_*} \phi_0} .
\end{eqnarray*}
选取充分小的$\kappa$以使$C e^{CT_*} \phi_0 \le \frac{1}{2}$,我们最终得到
\begin{eqnarray*}
  \phi(T) \le 2T_*e^{CT_*}\kappa^4,
\end{eqnarray*}
即
\begin{eqnarray*}
  \|E(T)\|_{H^s} \le C(T_*) \kappa^2
\end{eqnarray*}
对$T \in [0,\min\{T_*,T^\kappa\})$成立。这样我们证明了引理\ref{lemmaCE},从而定理\ref{theoremCE}成立。
\end{proof}

% \bibliography{ref}
% \bibliographystyle{unsrt}

 % \end{document}





