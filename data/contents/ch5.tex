\chapter{总结与展望}
% \documentclass{article}

% \usepackage{ctex}
% \begin{document}
近些年来纳米科学和材料科学的发展为粘弹性流体力学的建模提出了更大的挑战。如何发展经典的粘弹性流体模型以包含温度和压缩性的影响是一个重要的课题。本文通过推广非平衡态热力学的守恒-耗散理论,提出了几类考虑热传导和流体压缩性的粘弹性流体模型,总结如下:
\begin{enumerate}
	\item 利用守恒-耗散理论,推广了热传导的Guyer-Krumhansl理论并将其应用于粘弹性流体力学的模型中(模型\eqref{eq:CNSTgeneral})。
	\item 发展了守恒-耗散理论以将客观性原理纳入其中。推广了经典的上对流Maxwell模型(模型\eqref{eq:genUCM})和FENE-P模型(模型\eqref{eq:generalizedFENEP}),并提出了非等温可压上对流Maxwell模型(模型\eqref{eq:ECDFsecond})。
	\item 结合有限形变理论提出了有限形变守恒-耗散理论,由此推广了Lin等人的模型而提出了有限形变Maxwell模型(模型\eqref{eq:lingen})。
\end{enumerate}
本文提出的粘弹性流体模型自然地纳入了压缩性和热传导的影响,并且有着良好的热力学理论基础和数学结构。本文的工作为粘弹性流体力学的建模提供了新的手段。

基于守恒-耗散理论的数学结构,本文对粘弹性流体力学的模型做了数学分析,主要得到了以下结果:
\begin{enumerate}
	\item 利用Yong的双曲平衡率方程组小解的整体存在性理论证明了等温可压Maxwell模型、一维等温可压上对流Maxwell模型平衡态附近解的整体存在性(定理\ref{th:Kawashima}和定理\ref{theoremglobal})。
	\item 利用Yong的双曲方程组松弛极限理论,给出了等温可压Maxwell模型、一维等温可压上对流Maxwell模型与Navier-Stokes方程组的兼容性分析(定理\ref{th:chapmanenskog}和定理\ref{theoremCE})。
	\item 推广双曲抛物组的Kawashima理论,结合力学适应性条件的分析给出了Lin等人的模型整体存在性定理的一个新的证明(定理\ref{theoremcom})。
	% \item 证明了等温可压模型平衡态附近解的整体存在性(定理\ref{th:Kawashima})和松弛参数趋于$0$时与经典Navier-Stokes方程组的兼容性(定理\ref{th:chapmanenskog})。
	% \item 证明了一维等温上对流Maxwell模型平衡态附近解的整体存在性(定理\ref{theoremglobal}),及在松弛参数趋于$0$时与一维Navier-Stokes方程的兼容性(定理\ref{theoremCE})。
	% \item 验证了林芳华等人提出的模型不满足Kawashima条件,并基于对力学适应性条件的分析给出了其平衡态附近解的整体存在性的一个新的证明(定理\ref{theoremcom})。
\end{enumerate}
与目前大多数粘弹性流体的分析工作不同,我们试图将粘弹性流体力学的分析纳入双曲方程组的一般理论框架中,另外我们推广了双曲抛物组的Kawashima理论以将力学适应性条件的约束纳入其中,期待这些工作能够为粘弹性流体力学的数学分析提供新的思路。


尽管本文对粘弹性流体的建模和分析方面做了一些工作,但还有许多问题尚未解决。在建模方面,我们所做的守恒-耗散理论的推广仅仅是一个初步的尝试,如何推广这一理论将更加复杂的粘弹性流体模型纳入其理论框架中是一个具有挑战性的研究方向。另外,推广的守恒-耗散理论和有限形变守恒-耗散理论的数学分析尚缺乏统一的框架,例如第三章讨论的等温可压上对流Maxwell模型在高维情况下平衡态附近解的整体存在性和松弛参数趋于$0$时与Navier-Stokes方程组的兼容性需要进一步讨论。未来的研究可以从下面几个方面进行:
\begin{itemize}
	\item 发展守恒-耗散理论以包含更加复杂的粘弹性流体模型,如凝胶、液晶和活性物质的模型,并推广经典的模型以考虑温度和压缩性。
	\item 讨论守恒-耗散理论的微观机理,将其应用于微观粘弹性流体模型中。
	\item 研究有限形变守恒-耗散理论中熵函数的凸性条件及其与方程组数学性质的关系。
	\item 研究有限形变Maxwell模型的牛顿极限,在数学上严格证明其与Lin等人的模型的兼容性。	
	\item 研究本文得到的非等温可压粘弹性流体力学模型的数值格式,并与实验进行比较以验证本文模型的合理性。
\end{itemize}

% \end{document}