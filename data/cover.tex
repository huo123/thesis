\thusetup{
  ctitle={粘弹性流体的数学建模和分析},
  cdegree={理学博士},
  cdepartment={高等研究院},
  cmajor={数学},
  cauthor={霍晓凯},
  csupervisor={雍稳安研究员},
  etitle={Modeling and Analysis of Viscoelastic Fluids},
  edegree={Doctor of Science},
  emajor={Mathematics},
  eauthor={Huo, Xiaokai},
  esupervisor={Researcher Yong, Wen-An}, 
   ckeywords={粘弹性流体,非平衡态热力学,双曲方程组,整体存在性,松弛极限},
  ekeywords={Viscoelastic Fluids, Nonequilibrium Thermodynamics, Hyperbolic System, Global Existence, Relaxation Limit }
}

% 定义中英文摘要和关键字
\begin{cabstract}
纳米科学的发展为粘弹性流体的建模提出了新的挑战,物质的压缩性和热传导效应在这些材料粘弹性行为的描述中变得越来越重要。因此,推广经典的不可压缩粘弹性流体力学模型以包含这些性质的影响成为当前的一个研究热点。另一方面,近年来非平衡态热力学的快速发展为这一问题的解决提供了重要的手段。然而,已有的非平衡态热力学理论似乎尚未成熟。如何完善已有的非平衡态热力学理论并将其应用于粘弹性流体的数学建模,是本文的主要研究目标。
% 尚未完善,其数学性质也没有统一的研究。如何提出物理上合理、数学上具有好的性质的非平衡态热力学理论并将其应用于粘弹性流体的数学建模,是本文研究的主要内容。

% 近年来,基于Yong提出的一阶双曲方程组的守恒-耗散结构性条件,Yong、Zhu、Hong、Yang发展了非平衡态热力学的守恒-耗散理论,并且将其应用于粘弹性流体的建模,提出了非等温可压Maxwell模型。这一理论得到的模型自动满足热力学第一、第二定律并拥有良好的数学性质。

% 雍稳安等提出的守恒-耗散理论(以下简称守恒-耗散理论)从双曲方程结构性条件出发,基于雍稳安提出的一阶双曲方程组的守恒-耗散结构性条件,发展了对非平衡体系的建模框架,并应用于粘弹性流体建模中提出了非等温可压Maxwell模型。这一理论自动满足热力学第一、第二定律并拥有良好的数学性质,是一个很有潜力的非平衡态热力学理论。

本文通过推广Yong、Zhu、Hong、Yang发展的守恒-耗散理论,提出了几类非等温可压粘弹性流体模型:1、 推广了热传导的Guyer-Krumhansl理论;2、发展了含对流导数的守恒-耗散理论,由此提出了非等温可压上对流Maxwell模型;3、结合有限形变理论和守恒-耗散理论提出了有限形变守恒—耗散理论,由此推广了Lin等人提出的模型。


% 对经典粘弹性流体力学模型进行了推广。首先,利用守恒-耗散理论推广了热传导的Guyer-Krumhansl理论并将其应用于线性粘弹性流体的建模中。然后为了纳入含有客观导数的非线性粘弹性模型,推广了守恒-耗散理论并发展了不可压上对流导数Maxwell模型和FENE-P模型至非等温可压情形,并且提出了等温可压上对流导数Maxwell模型。最后基于有限形变理论和守恒-耗散理论提出了有限形变守恒—耗散理论并应用于粘弹性流体的建模中,利用这一理论推广了林芳华等人提出的模型。

在数学分析方面,利用Yong的平衡率方程组小解整体存在性理论和双曲方程松弛极限理论,证明了等温可压Maxwell模型和一维等温可压上对流Maxwell模型平衡态附近解的整体存在性,以及松弛参数趋于$0$时同经典Navier-Stokes方程的兼容性。最后验证了Lin等人的有限形变粘弹性模型不满足双曲—抛物方程的Kawashima条件,并且通过对力学适应性条件的分析,给出了这一模型平衡态附近整体解存在性的一个新的证明。

% 由于守恒-耗散理论得到的方程组满足雍稳安提出的守恒-耗散条件,其解在平衡态附近的整体存在性和松弛极限的数学分析可以利用雍稳安发展的含熵守恒律方程组整体存在性理论和双曲松弛系统的数学理论来处理。利用雍稳安发展的理论,本文证明了等温可压Maxwell模型和一维等温可压上对流导数Maxwell模型平衡态附近解的整体存在性,以及松弛参数趋于$0$时同经典Navier-Stokes方程的一致性。%针对非线性粘弹性流体力学,本文利用雍稳安发展的理论分析了一维等温可压上对流导数Maxwell模型在平衡态附近解的存在性,及松弛参数趋于$0$时该模型和经典一维Navier-Stokes方程的一致性。%,虽然方程为非守恒形式,但是对称子的存在保证了雍稳安发展的理论同样适用。
% 最后利用双曲—抛物方程的Kawashima理论给出了林芳华、柳春、张平等发展的有限形变粘弹性模型的平衡态附近整体解存在性的一个新的证明。
%考察了由有限形变守恒耗散理论得到的模型和林芳华等人提出的无穷大Weissenberg数粘弹性流体力学模型的平衡态附近解的整体存在性,虽然经典的Kawashima条件并不成立,但是力学适应性条件的存在弥补了这一缺陷,从而可以证明整体存在性定理所需的估计。 

% 本文的研究表明,守恒-耗散理论为发展非等温可压粘弹性流体力学模型提供了理论框架,并且由这一理论得到的方程有着良好的数学结构。

\end{cabstract}

\begin{eabstract}
  \noindent New challenges occur in the modeling of viscoelastic fluids with the development of nanosciences. The compressiblity and thermodynamical behaviors have become more and more important in the description of their viscoelastic behavior. Therefore, the promotion of classical incompressible viscoelastic hydrodynamic models to include these effects has become a hotspot in the current research. On the other hand, rapid developments of non-equilibrium thermodynamics in recent years have provided important tools to solve this problem. However, the current theory of non-equilibrium thermodynamics is not yet perfect. How to improve the existing non-equilibrium thermodynamics theoris and apply them to to the mathematical modeling of viscoelastic fluid are the main goals of this paper.

  % Recently, Wen-An Yong, Yi Zhu, Liu Hong and Zaibao Yang has developed a theory called conservation-dissipation theory of irreversible thermodynamics. This theory is based on the conservation-dissipation structure for hyperbolic systems proposed by Yong. And the structure guarantees the first and second law of thermodynamics. It has been successfully applied to the linear viscoelastic models but encounters problem when applying to nonlinear models with objective derivatives.
  
 We develop severals non-isothermal compressible viscoelastic fluids models through a generalization of the conservation dissipation formalism of irreversible thermodynamics proposed by Yong, Zhu, Hong and Yang. First, a generalized Guyer-Krumhansl theory is developed. Second, a conservation dissipation theory including convective derivatives is developed and a non-isothermal compressible upper convected Maxwell model is proposed. Last, a conservation dissipation theory combing finite strain theory is developed and a generalized Lin's model is proposed.

  % the classical viscoelastic hydrodynamic models are geneneralized with the help of the conservation-dissipation theory. First, by following the conservation dissipation formalism, the Guyer-Krumhansl law of heat conduction is generalized and applied to the viscoelastic model. In order to include the nonlinear viscoelastic models with objective derivative, the conservation-dissipation formalism is extended. Following this generalized theory, the compressible versions of upper convected maxwell model and FENE-P model are derived. A isothermal compressible upper convected maxwell model is also developed by using the same method. In addition, based on the finite deformation theory and the conservation-dissipation formalism, a finite deformation conservation-dissipation theory is proposed. Using this theory, we generalize a model proposed by Fanghua Lin et al.

  In the aspect of mathematical analysis of the viscoelastic models, the existences of solutions near equilibrium states and the consistencies with Navier-Stokes systems of the isothermal compressible Maxwell model and the one dimensional isothermal compressible upper convected Maxwell model are proved using the global existence theory and singular limit theory of hyperbolic systems developed by Yong. Finally, we show that the Lin's model fails to satisfy the Kawashima condition. However, it can be compensated by the mechanically compatibility conditions, enabling us to give a new proof of the global existence near equilibriums of Lin's model.

  Their consistencis with Navier-Stokes systems are rigorously justified with the mathematical theory of Chapman-Enskog expansions developed by Yong and Yang. In addition, we give a new proof of the model proposed by Fanghua Lin, et al. The proof is based on the Kawashima theory of general hyperbolic-parabolic systems and an analysis of the compatibility conditions of mechanics.
   % due to the good mathematical structure of the conservation-dissipation theory, the existence of smooth solutions near the equilibrium state of the isothermal Maxwell model is proved using the theory of hyperbolic systems. And its consistency with the classical Navier-Stokes equations is also justified with the mathematical theory of Chapman-Enskog expansion developed by Yong and Yang. For the nonlinear viscoelastic models, the global existence of unique smooth solution near the equilibrium state of an one-dimensional isothermal compressible upper convected Maxwell model is analyzed by using the theory of hyperbolic systems. Its consistency with the classical Navier-Stokes equations is also rigorously investigated. 
  %Although the model is not in the form of conservation, the symmetric hyperbolicity still holds true in the one-dimensional case, thus the relevant analysis method is still applicable. 
  % Finally, with the Kawashima theory of general hyperbolic-parabolic systems, we give a different proof of the global existence of smooth solutions near equilibrium of the model proposed by Fanghua Lin et al.
  
  % oelastic hydrodynamic model proposed by the finite deformation conservation theory is discussed. Although the classical Kawashima condition does not hold, the mechanical adaptability condition compensate this defect, with which we can prove the required estimates for the global existence theorem.

\end{eabstract}

