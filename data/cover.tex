\thusetup{
  ctitle={粘弹性流体的数学建模和分析},
  cdegree={理学博士},
  cdepartment={高等研究院},
  cmajor={数学},
  cauthor={霍晓凯},
  csupervisor={雍稳安研究员},
  etitle={Modeling and Analysis of Viscoelastic Fluids},
  edegree={Doctor of Science},
  emajor={Mathematics},
  eauthor={Huo, Xiaokai},
  esupervisor={Researcher Yong, Wen-An}, 
   ckeywords={粘弹性流体,非平衡态热力学,双曲方程组,整体存在性,松弛极限},
  ekeywords={Viscoelastic Fluids, Nonequilibrium Thermodynamics, Hyperbolic System, Global Existence, Relaxation Limit }
}

% 定义中英文摘要和关键字
\begin{cabstract}
  纳米科学和材料科学的发展为粘弹性流体的建模提出了新的挑战。物质的压缩性和热传导在粘弹性的这些材料的粘弹性行为的描述中变得越来越重要。因此,推广经典的不可压缩粘弹性流体力学模型以包含这些性质的影响成为当前的一个研究热点。近些年来的非平衡态热力学的快速发展为这一问题的解决提供了重要的建模工具。然而,由于目前非平衡态热力学尚未完善,其数学性质也没有统一的研究。如何提出物理上合理、数学上具有好的性质的非平衡态热力学理论并将其应用于粘弹性流体的数学建模,是本文研究的主要内容。

本论文通过发展非平衡态热力学的守恒-耗散理论,对经典的粘弹性流体力学模型进行了推广。首先,基于经典的守恒-耗散理论推广了经典的线性粘弹性模型,提出了满足Laudau极限的可压粘弹性流体模型。并将热传导Cattaneo定律、热质理论和Guyer-Krumhansl理论等纳入模型中。为了包含带客观导数的非线性粘弹性模型,推广了守恒-耗散理论并发展了可压的非线性粘弹性流体力学模型。另外基于有限形变理论和守恒-耗散理论提出了有限形变守恒-耗散理论提出了可压的有限形变粘弹性流体模型,推广了林芳华等人提出的模型。

在数学分析方面,由于守恒-耗散理论的良好数学结构,采用双曲方程的相关理论证明了线性粘弹性流体力学模型的平衡态附近解的整体存在性,并严格证明了松弛参数的极限方程为经典 Navier-Stokes 方程。针对非线性粘弹性流体力学,利用双曲方程的相关理论分析了可压上对流导数 Maxwell模型在一维时的平衡态附近解的存在性,并利用双曲松弛系统的相关理论证明了一维时在松弛参数趋于 0 时该模型和经典 Navier-Stokes方程的一致,虽然该模型不是守恒形式的,但是一维情况下的双曲对称性保证了相关的分析方法仍然适用。最后考察了由有限形变守恒耗散理论得到的模型和林芳华等人提出的无穷大Weissenberg数粘弹性流体力学模型的平衡态附近解的整体存在性,虽然经典的Kawashima条件并不成立,但是力学适应性条件的存在弥补了这一缺陷,从而可以证明整体存在性定理所需的估计。 

\end{cabstract}

\begin{eabstract}
  \noindent The development of nanoscience and materials science brings new challenges in the modeling of viscoelastic fluids. The compressibility of the material and the heat transfer phenomena become more and more important in the description of the viscoelastic behavior of these materials. Therefore, the promotion of classical incompressible viscoelastic hydrodynamic models to include the effects of these properties has become a hotspot in the current research. The rapid development of non-equilibrium thermodynamics in recent years has provided an important tool for solving this problem. However, the current non-equilibrium thermodynamics is not yet perfect, and its mathematical properties are not unified investigated. How to develop a non-equilibrium thermodynamics theory which is physically reasonable and has good properties in mathematics and apply it to the mathematical modeling of viscoelastic fluid are the main contents of this paper.
  
  In this paper, the classical viscoelastic hydrodynamic model is extended by developing the conservation-dissipation theory of non-equilibrium thermodynamics. First, a classical linear viscoelastic model is generalized based on the classical conservation-dissipation theory, and a compressible viscoelastic fluid model satisfying the Laudau limit is proposed. The heat conduction Cattaneo law, thermal theory and Guyer-Krumhansl theory are also taken into consideration in the models developed. In order to include the nonlinear viscoelastic model with objective derivative, the conservation-dissipation formalism is extended and some compressible nonlinear viscoelastic hydrodynamic models is developed. In addition, based on the finite deformation theory and the conservation-dissipation formalism, a finite deformation conservation-dissipation theory is proposed. Using this theory, a compressible finite-deformation viscoelastic fluid model is proposed, whose relaxation limit is the model proposed by Fanghua Lin, et al.

  In the aspect of mathematical analysis of the viscoelastic models, due to the good mathematical structure of the conservation-dissipation theory, the existence of the solution near the equilibrium state of the linear viscoelastic hydrodynamic model is proved by with the theory of the hyperbolic equation. And the consistency with classical Navier-Stokes equations is justified with the mathematical theory of Chapman-Enskog expansion. For the nonlinear viscoelastic models, the global existence of unique smooth solution near the equilibrium state of an one-dimensional upper convected Maxwell model is analyzed by using the theory of hyperbolic systems. Its consistency with the classical Navier-Stokes equations when the relaxation parameter tends to zero is rigorously investigated. Although the model is not in the form of conservation, the symmetric hyperbolicity still holds true in the one-dimensional case, thus the relevant analysis method is still applicable. Finally, the global existence of the equilibrium solution near the equilibrium state of the infinite Weissenberg viscoelastic hydrodynamic model proposed by the finite deformation conservation theory is discussed. Although the classical Kawashima condition does not hold, the mechanical adaptability condition compensate this defect, with which we can prove the required estimates for the global existence theorem.

\end{eabstract}

